Along with the most important properties of human vision, the previous chapter discussed the basics of photometry and colorimetry with introducing the concepts of luminance and the CIE XYZ color space.
The present chapter introduces the practical color representation applied in analog and digital video technologies based on the foregoing.

\vspace{3mm}
In video applications the direct XYZ representation of color points is scarcely ever used, except for several \href{https://en.wikipedia.org/wiki/Digital_Cinema_Package}{digital cinema} and film archiving applications\footnote{
One reason for this is that the XYZ coordinate system can not be used directly for color reproduction, with the X,Y,Z primaries not being real visible colors.
On the other hand the digital representation of all the visible colors requires high bit depth unnecessarily, since the set of visible colors only fills a small part of the positive XYZ octant (many digital codes would be assigned to non-visible colors)}.
However, as the real importance of the XYZ colors space, it allows the investigation and mathematical description of the colors, reproducible in specific cameras and displays, and allows the color conversion between imaging devices.
The following section investigates these device-specific color representations, i.e. the \textbf{device dependent color spaces}.

\section{Device-dependent color spaces}

Due to the trichromaticity and near-linearity of the human vision the visible colors can be represented in a 3D vector space, allowing the addition of color vectors:
The perceived color generated by the mixture of two arbitrary colors can be obtained the addition of the two colors' position vectors (independently of the physical, spectral properties of the involved light rays).
On the $xy$ chromaticity diagram the mixture is located along the straight line, connecting the original two color points.

As a consequence, relying on the metamerism of human vision most of the visible colors can be reproduced as the combination of appropriately chosen \textbf{primary colors} (or simply \textbf{primaries}).
Generally speaking this is the basic idea behind color reproduction.
Obviously, the reproduction of all the visible colors is not a realistic goal when using a limited number of primaries:
in the chromaticity diagram the visible colors are bounded by the continuous, smooth curve of spectral colors, meaning that reproduction all the visible colors---to cover all the color points as a point on the line, connecting two primary colors---theoretically, infinite number of primary colors would be needed, realizing all the spectral colors as primaries.
This rises the question, what is the feasible number of primaries used for color reproduction.


\begin{figure}[]
	\centering
	\begin{overpic}[width = 1\columnwidth]{figures_en/Video_colorspaces/color_space_gamut.png}
	\small
	\put(0,0){(a)}
	\put(50,0){(b)}
	\put(22,22){Gamut}	
	\end{overpic}
	\caption{Gamut of the SD, HD and sRGB color spaces, all applying the same RGB primaries, illustrated on the $xy$ (a) and the $u^*v^*$ diagram.}
	\label{Fig:gamut}
\end{figure}

In the $xy$ diagram the area of reproducible colors is defined by a polygon with the apex representing the primaries.
Most of the $xy$ chromaticity diagram can be covered by a rectangle, meaning 4 primaries allows the reproduction of nearly all the visible colors.
On the other hand the $L^*u^*v^*$ chromaticity diagram reflected that the resolution of human vision for shades of green color is low and the perceptually uniform chormaticity diagram resembles to a triangle.
Therefore, imaging systems based on additive color mixing almost exclusively apply three properly chosen red, green and blue primary colors, allowing to cover most of the perceptually uniform chromaticity diagram.

The set of colors that can be reproduced with positive primary coordinates (i.e. positive RGB intensities) is termed as a \textbf{device dependent color space}, while colorimetric, absolute color spaces like CIE XYZ, $L^*u^*v^*$, $L^*a^*b^*$ are referred to as \textbf{device independent color spaces}.
The set of the reproducible colors in a device dependent RGB color space are located on the edges and inside a triangle on the $xy$ chromaticy diagram.
This triangle is referred to as the \textbf{gamut} of the color space.
A simple example for the gamut of RGB color spaces is illustrated in Figure \ref{Fig:gamut}\footnote{
Obviously, device dependent color spaces exist applying different numbers of primaries.
A commonly used device dependent space is the CMYK color space of color printing, with the four primaries denoting the color of the inks, used for reproduction.
In the following the investigation is limited to RGB color spaces.}.
Colors, located on the apexes and edges of the gamut are the most saturated colors in the RGB color space, being the closest color to spectral colors.
These color are termed the \textbf{quasi-spectral colors}, with the common property that they contain a maximum of two primaries.

\vspace{3mm}
Once an RGB color space is well-defined, for an arbitrary color $C$ the RGB primary intensities can be specified so that their additive mixture coincides with color $C$ (as long as the intensities are positive).
These RGB intensities $\mathbf{c}_{RGB} = \begin{bmatrix}
       R_c \\[0.3em] G_c \\[0.3em] B_c \end{bmatrix}$  are referred to as the \textbf{RGB coordinates} of color $C$, describing the position of the color point in the RGB color space.
By definition, the RGB coordinates may take values between 0 and 1, therefore, vectors
\begin{equation}
\mathbf{r}_{R\!G\!B} = \begin{bmatrix}
       1 \\[0.3em]
       0 \\[0.3em]
       0 \end{bmatrix}, \hspace{4mm}
\mathbf{g}_{R\!G\!B} = \begin{bmatrix}
       0 \\[0.3em]
       1 \\[0.3em]
       0 \end{bmatrix}, \hspace{4mm}
\mathbf{b}_{R\!G\!B} = \begin{bmatrix}
       0 \\[0.3em]
       0 \\[0.3em]
       1 \end{bmatrix}.
\end{equation}
denote the red, green and blue primary colors with $100~\%$ intensities.

The following section discusses the practical definition of RGB color spaces.

\subsection{Definition of device-dependent color spaces}

Assume a device dependent color space applying RGB primaries.
The three primary colors (red, green, blue) can be represented as three vectors in the $XYZ$ pointing into the actual perceived color of the primaries.
The chromaticities ($xy$ coordinates) of the primaries are found at the intersection of the primary vectors and the unit plane.
The constellation is illustrated in Figure \ref{Fig:device_dep}.
The $XYZ$-coordinates of the RGB primaries are denoted by
\begin{equation}
\mathbf{r}_{X\!Y\!Z} = \begin{bmatrix}
       X_r \\[0.3em]
       Y_r \\[0.3em]
       Z_r \end{bmatrix}, \hspace{4mm}
\mathbf{g}_{X\!Y\!Z} = \begin{bmatrix}
       X_g \\[0.3em]
       Y_g \\[0.3em]
       Z_g \end{bmatrix}, \hspace{4mm}
\mathbf{b}_{X\!Y\!Z} = \begin{bmatrix}
       X_b \\[0.3em]
       Y_b \\[0.3em]
       Z_b \end{bmatrix}.
\end{equation}
%
\begin{figure}[]
	\centering
	\begin{minipage}[c]{0.65\textwidth}
	\begin{overpic}[width = 1\columnwidth ]{figures_en/device_dep.png}
	\small
	\put(89,19){$X$}
	\put(12,96){$Y$}
	\put(0,4){$Z$}
	\put(36,64){$(X_g,Y_g,Z_g)$}
	\put(10,8){$(X_b,Y_b,Z_b)$}
	\put(39,33){$(X_r,Y_r,Z_r)$}
	\end{overpic}\end{minipage}\hfill
	\begin{minipage}[c]{0.35\textwidth}
	\caption{RGB primaries and their chromaticities, illustrated in the $XYZ$ color space.}
	\label{Fig:device_dep}  \end{minipage}
\end{figure}
As long as the $XYZ$-coordinates of the RGB primaries are known, the RGB color space is fully defined:
The RGB primary intensities $\mathbf{c}_{RGB}$ can be defined for an arbitrary color point described by its $XYZ$ coordinates $\mathbf{c}_{X\!Y\!Z}$.
The RGB coordinates $\mathbf{c}_{RGB}$, giving the weight of the RGB primaries in the given color point read as
\begin{equation} 
\underbrace{\begin{bmatrix}[c]
       R_c \\[0.3em]
       G_c \\[0.3em]
       B_c \end{bmatrix}}_{\mathbf{c}_{RGB}}
       =
     \mathbf{M}_{X\!Y\!Z \rightarrow R\!G\!B}
      \underbrace{\begin{bmatrix}[c]
       X_c \\[0.3em]
       Y_c \\[0.3em]
       Z_c \end{bmatrix}}_{\mathbf{c}_{X\!Y\!Z}},
\end{equation}
where $ \mathbf{M}_{X\!Y\!Z \rightarrow R\!G\!B}$ is a linear transform matrix. 
Vice versa, if a color point is given by its RGB coordinates, the $XYZ$ vector coordinates can be defined by
\begin{equation}
      \underbrace{\begin{bmatrix}[c]
       X_c \\[0.3em]
       Y_c \\[0.3em]
       Z_c \end{bmatrix}}_{\mathbf{c}_{X\!Y\!Z}} = 
     \mathbf{M}_{R\!G\!B \rightarrow X\!Y\!Z}
\underbrace{\begin{bmatrix}[c]
       R_c \\[0.3em]
       G_c \\[0.3em]
       B_c \end{bmatrix}}_{\mathbf{c}_{RGB}}.
\end{equation}
Obviously, for the involved transform matrices $\mathbf{M}_{R\!G\!B \rightarrow X\!Y\!Z} = \mathbf{M}_{X\!Y\!Z \rightarrow R\!G\!B}^{-1}$ holds.

The latter transform matrix can be simply defined by utilizing the rule for change of basis from elementary linear algebra:
The columns of the transform matrix $\mathbf{M}_{R\!G\!B \rightarrow X\!Y\!Z}$ are simply containing the bases of the RGB color space in the $XYZ$ space, i.e. the $XYZ$-coordinates of the RGB primaries\footnote{
The validity of the formula can be simply admitted by substituting e.g. $\mathbf{c}_{RGB} = \begin{bmatrix}[c]
       1 \\[0.3em]
       0 \\[0.3em]
       0 \end{bmatrix}$, describing the red primary in the RGB color space, and substitution to \eqref{Eq:CS_transform} returns the first column of the transform matrix.}
\begin{equation}
\begin{bmatrix}[c]
       X_c \\[0.3em]
       Y_c \\[0.3em]
       Z_c \end{bmatrix}
       = 
       \underbrace{
  \begin{bmatrix}[c|c|c]
   X_r & X_g & X_b  \\
   Y_r & Y_g & Y_b \\
   Z_r & Z_g & Z_b  \\
\end{bmatrix}}_{\mathbf{M}_{R\!G\!B \rightarrow X\!Y\!Z}}
\cdot
\begin{bmatrix}[c]
       R_c \\[0.3em]
       G_c \\[0.3em]
       B_c \end{bmatrix}.
\label{Eq:CS_transform}
\end{equation}

%TODO : POynoton 250.oldal
These transform matrices allow the conversion between RGB color spaces of specific devices.
On the other hand coordinate $Y_c$ of and arbitrary color point gives the luminance, being proportional to the perceived brightness of the given color.
Hence, the second row of the transform matrix $\mathbf{M}_{R\!G\!B \rightarrow X\!Y\!Z}$ defines how the luminance content can be calculated of an arbitrary color, given in the RGB space.

\vspace{3mm}
The set of all reproducible colors in an RGB color space is illustrated in Figure \ref{Fig:device_dep_2}.
Clearly, the set of all vectors that can be written as the linear combination of three primary vectors with positive weights distends a parallelepiped, i.e. an RGB color space can be illustrated in the XYZ coordinate system by a parallelepiped.

\begin{figure}[]
	\centering	
	\small
	(a)
	\begin{overpic}[width = 0.45\columnwidth ]{figures_en/Video_colorspaces/device_dep_2.png}
	\small
	\put(-2,5){$Z$}
	\put(89,17){$X$}
	\put(11,97){$Y$}
	\end{overpic}
	(b)
	\begin{overpic}[width = 0.45\columnwidth ]{figures_en/Video_colorspaces/The-RGB-colour-cube.png}
	\end{overpic}
	\caption{Illustration of an RGB color space in the $XYZ$ coordinate system (a) and in the RGB cube (b).
	The color of the vectors in (a) denotes the color in the endpoints.
}
	\label{Fig:device_dep_2}
\end{figure}

Obviously, since the RGB coefficients can take values between 0 and 1, the set of all reproducible colors in the RGB coordinate system fills the unit cube in the positive space octant.\footnote{Hence RGB color spaces are sometimes referred to as RGB cube}, with the edges passing through the origin containing the RGB primary colors with different intensities.
Therefore, the above transform matrices perform linear transforms, transforming a cube into a parallelepiped and a parallelepiped into a cube.

\paragraph{A relatív fénysűrűség bevezetése:\\}
Egy RGB színtér tehát teljes egészében adott, amennyiben az alapszín-vektorok $XYZ$ koordinátái ismertek.
A gyakorlatban azonban egy RGB színtér definiálása során az $XYZ$ koordináták helyett az RGB alapszínek és a fehérpontjának színezetét, azaz $xy$ színkoordinátáit adják meg.
Definíció szerint egy adott színtér \textbf{fehérpontja} az adott térben elérhető legvilágosabb pont, amelyet az alapszínek egyenlő arányú keverékével érhetünk el.
Az adott eszközfüggő színtérben a 100\%-os ez alapján (hasonlóan az $XYZ$-beli fehérhez), definíció szerint 
\begin{equation}
\mathbf{w}_{RGB} = \begin{bmatrix}[c]
       1 \\[0.3em]
       1 \\[0.3em]
       1 \end{bmatrix}, \hspace{5mm} \text{és} \hspace{5mm} 
Y_w = 1,
\end{equation}
ahol $Y_w$ a színpont \textbf{relatív fénysűrűsége}, amely tehát 0 és 1 között vehet fel értékeket.
 A \ref{Fig:device_dep} ábrán látható példában a fehér szín vektora a paralelepipedon szürkével jelölt főátlója, ezen vonal mentén helyezkednek el a különböző világosságértékű (árnyalatú) fehér színek.
A fehér szín színezete, azaz $x_w$ és $y_w$ koordinátái ezen vektor az egységsíkkal vett döféspontja határozza meg.

A színteret tehát úgy definiáljuk, hogy a három alapszínvektor $xy$ koordinátája (azaz az iránya) mellett megadjuk az alapszínek egyenlő energiájú keverékének a színezetét, (azaz a három bázisvektor összegének irányát), és rögzítjük, hogy az összegvektor $Y$ koordinátája egységnyi.
Ebből a 9 adatból meghatározhatók az RGB bázisvektorok tényleges hossza, és így a szükséges transzformációs mátrixok felírhatók.

\vspace{3mm}
Az RGB színterek ilyen módú definíciója mögött a motíváció a következő:
Láthattuk, hogy az $XYZ$ koordináták a színérzetet létrehozó spektrummal szorosan összefüggnek, az $Y$ koordináta pl. a fényinger fénysűrűségét adja meg ([$\mathrm{cd}/\mathrm{m}^2$]-ben, vagy nit-ben).
A gyakorlati alkalmazások során azonban nem szempont egy RGB színtér alapszíneinek---pl. egy RGB kijelző LCD alapszíneinek---fizikai jellemzőinek pontos ismerete (azaz pl. hány nit fénysűrűséget hoz létre az R, G, vagy B pixel-elem).
Ennek oka, hogy képi reprodukció során a tényleges, fotometriai abszolút fénysűrűséget szinte soha nem célunk visszaállítani (nem is tudnánk, ha a képernyő maximális létrehozható fénysűrűsége kisebb, mint az eredeti mért fénysűrűség).
Ehelyett az adott megjelenítő eszköz által létrehozható legvilágosabb színhez képest reprodukáljuk az adott képpontok relatív fénysűrűségét.
Az, hogy ez a legvilágosabb pont ténylegesen hány nit fénysűrűséget hoz létre eszközről eszközre változhat, és a megjelenítők fontos paramétere (ez az általában [$\mathrm{cd}/\mathrm{m}^2$]-ben megadott maximális fényerő paraméter).
Az eszközfüggő színterek fenti definíciója tehát azt biztosítja, hogy az $Y$ koordináta az RGB alapszínek fizikai jellemzőitől függetlenül a relatív fénysűrűséget írja le.

\paragraph{A fehér színről általában:\\}
Látható tehát, hogy a fehér szín önmagában szubjektív fogalom: adott környezetben a leginkább akromatikus fényingert nevezzük fehérnek, amelynek spektrális sűrűségfüggvénye minél inkább egységnyi (azaz minél több spektrális komponenst tartalmaz), és ezzel analóg módon RGB színtérben ábrázolva minél közelebb van a csupa-egy vektorhoz.
A fehér fogalom egységesítéséhez bevezettek ún. szabványos megvilágításokat (standard illuminants), amelyet szabványosított RGB  színterek esetén előírnak, mint fehérpont.
Ezeknek a szabványos megvilágításoknak a spektrális sűrűségfüggvénye (és persze az általa keltett színinger $xy$-koordinátái) adott, jól-definiált.
Ilyen szabványos megvilágítások a következők:
\begin{figure}[]
	\centering
	\begin{minipage}[c]{0.6\textwidth}
	\begin{overpic}[width = 0.9\columnwidth ]{figures_en/Video_colorspaces/PlanckianLocus.png}
	\end{overpic} \end{minipage}\hfill
	\begin{minipage}[c]{0.4\textwidth}
	\caption{Különböző hőmérsékletű feketetest sugárzók által keltett színek összessége, azaz a Planck görbe.}
	\label{Fig:planck}  \end{minipage}
\end{figure}
\begin{itemize}
\item E fehér: egyenlő energiájú fehér, a CIE XYZ színtér fehérpontja. Kolorimetria szempontjából jelentős, videótechnikában kevésbé fontos a szerepe, mivel a gyakorlatban nem fordul elő olyan fényforrás, amely minden hullámhosszon azonos energiával sugároz.
\item A fehér: a CIE által szabványosított, egyszerű háztartási wolfram-szálas izzó fényét (azzal azonos színérzetet keltő) fényforrás spektruma és színe, $T_{\mathrm{C}} = 2856~\mathrm{K}$ korrelált színhőmérséklettel \footnote{A korrelált színhőmérséklet (correlated color temperature, CCT, $T_{\mathrm{C}}$) azon feketetest sugárzó hőmérsékletét jelzi, amely az emberi szemben a minősítendő fényforrással azonos színérzetet kelt.
A feketetest (hőmérsékleti) sugárzó által keltett színingerek az $xy$ színdiagramon az ún. Planck-görbét járják be, amely a \ref{Fig:planck} ábrán látható.}.
\item B és C fehér: Az A fehérből egyszerű szűréssel nyerhető, napfényt szimuláló megvilágítások.
A B fehér a déli napsütést modellezi $4874~\mathrm{K}$ színhőmérséklettel, míg a C fehér a teljes napra vett átlagos fény színét (spektrumát) modellezi $6774~\mathrm{K}$ színhőmérséklettel.
\item D fehér: szintén a napfény közelítésére alkalmazott megvilágítások sora.
Videótechnika szempontjából a legfontosabb a D65 fehér, amely jelenleg is az UHD formátumok színterének szabványos fehérpontja.
\end{itemize}

\subsection{Color space conversions}
Az eddigiekben látható volt, hogyan definiálható egy eszközfüggő színtér az alapszíneivel.
Ahogy az elnevezés is mutatja, ezek a színterek jellegzetesen adott eszközre érvényesek, pl. egy kamera a beépített RGB szenzorok, egy kijelző az alkalmazott RGB kristályok által meghatározott RGB színtérben dolgoznak.
Emellett léteznek szabványos RGB színterek amelyek a képi tartalom tárolására, továbbítására szolgálnak egységesített, szabványos módon.
A következő szakasz ezeket a szabványos videószíntereket tárgyalja részletesebben.
%TODO Lab, luv spaces: conversion
Felmerül tehát a természetes igény az egyes színterek közti átjárásra, amelyet \textbf{színtér konverziónak} nevezünk.

A színtérkonverziót az $XYZ$ színtér teszi lehetővé, amely egy eszközfüggetlen, abszolút színtér:
egyes színterek közti konverzió a forrás által létrehozott jelek $XYZ$ színtérbe való transzformációjával, majd ezen reprezentáció a nyelő színterébe való transzformációval történik.
Az $XYZ$ színtér így tehát színterek közti átjárást biztosít, ún. Profile Connection Space-ként működik (hasonlóan pl. a gyakran azonos célra alkalmazott $Lab$ színtérhez).

\begin{figure}[]
	\centering
	\begin{overpic}[width = 1\columnwidth]{figures_en/Video_colorspaces/cs_conversion.png}
	\small
	\put(1,37){$RGB_{\mathrm{cam}}$}
	\put(35,37.5){$XYZ$}
	\put(67,39){$RGB_{\mathrm{ITU}-709}$}
	\put(13,18){$RGB_{\mathrm{ITU}-709}$}
	\scriptsize
	\put(15,29.25){$\mathbf{M}_{\!R\!G\!B_{\mathrm{c\!a\!m}} \!\!\rightarrow \!\!X\!Y\!Z}$}
	\scriptsize
	\put(49,29.25){$\mathbf{M}_{\!X\!Y\!Z \!\rightarrow \!R\!G\!B_{7\!0\!9}} $}
	\small
	\put(87,29){\parbox{.86in}{MPEG kódolás, műsorszórás, tárolás}}
	\put(52,18){$XYZ$}
	\put(87,17){$RGB_{\mathrm{TV}}$}
	\scriptsize
	\put(32.5,9.5){$\mathbf{M}_{\!R\!G\!B_{\mathrm{7\!0\!9}} \!\!\rightarrow \!\!X\!Y\!Z}$}
	\scriptsize
	\put(66.5,9.6){$\mathbf{M}_{\!X\!Y\!Z \!\rightarrow \!R\!G\!B_{7\!0\!9}} $}	
	\end{overpic} 	
	\caption{Színtér-konverzió folyamatábrája.}
	\label{Fig:cs_conversion}
\end{figure}
Egy tipikus színtér konverziós folyamatot az \ref{Fig:cs_conversion} ábra mutat.
Tegyük fel, hogy adott egy HD kamera által rögzített képanyag, ahol a kamera színterét $RGB_{\mathrm{cam}}$ jelöli.
A HD formátum szabványos színteret alkalmaz, amelyet az ITU-709 ajánlásban rögzítettek (lásd később).
A kamera RGB jeleit tehát az esetleges kódolás és tárolás előtt ebbe a HD színtérbe kell konvertálni.
Ez a konverzió a kamerajelek $XYZ$ térbe, majd innen az ITU-709 színtérbe való transzformációval oldható meg, amely a megfelelő transzformációs-mátrixszal való szorzással valósítható meg:
\begin{equation} 
\begin{bmatrix}[c]
       R_{\mathrm{ITU}-709} \\[0.3em]
       G_{\mathrm{ITU}-709} \\[0.3em]
       B_{\mathrm{ITU}-709} \end{bmatrix}
       =
       \mathbf{M}_{ X\!Y\!Z \rightarrow R\!G\!B_{709} } \cdot 
\left(     \mathbf{M}_{R\!G\!B_{\mathrm{cam}} \rightarrow X\!Y\!Z } \cdot
\begin{bmatrix}[c]
       R_{\mathrm{cam}} \\[0.3em]
       G_{\mathrm{cam}} \\[0.3em]
       B_{\mathrm{cam}} \end{bmatrix} \right)
\end{equation}
Természetesen az egymás utáni két mátrixszorzás összevonható, így a két $RGB$ színtér között közvetlen lineáris leképzés határozható meg.
Ez a transzformáció jellegzetesen már a kamerán belül megvalósul.
%
Hasonlóképp, megjelenítőoldalon a
\begin{equation} 
\begin{bmatrix}[c]
       R_{\mathrm{cam}} \\[0.3em]
       G_{\mathrm{cam}} \\[0.3em]
       B_{\mathrm{cam}} \end{bmatrix}
       =
       \mathbf{M}_{ X\!Y\!Z \rightarrow R\!G\!B_{\mathrm{TV}} } \cdot 
\left(     \mathbf{M}_{R\!G\!B_{709}  \rightarrow X\!Y\!Z } \cdot
\begin{bmatrix}[c]
       R_{\mathrm{ITU}-709} \\[0.3em]
       G_{\mathrm{ITU}-709} \\[0.3em]
       B_{\mathrm{ITU}-709} \end{bmatrix}
 \right)
\end{equation}
transzformációt kell elvégezni.

Ez az egyszerű transzformációs módszer lehetővé teszi egy adott színtérben mért színpontok másik színtérben való ábrázolását.
Ugyanakkor felmerül a probléma, hogy a nagyobb gamuttal rendelkező színtérből kisebbe való áttérés esetén az új színtérben nem ábrázolható, gamuton kívüli színek negatív és egynél nagyobb RGB koordinátákkal jelennek meg, míg a kisebb gamutú térből való áttérés esetén a nagyobb gamutú tér egy része kihasználatlan marad.
A probléma megoldására a fenti transzformációk mellett az egyes színterek gamutját valamilyen nemlineáris leképzés segítségével lehet egymásra illeszteni (expandálással, kompresszálással).
Ezek az ún. gamut-mapping technikák.

A következőekben az egyes SD, HD és UHD videóformátumok tárolására és továbbítására alkalmazott eszközfüggő színtereket tárgyaljuk.

\subsection{Color spaces of video technology}

% http://www.displaymate.com/crtvslcd.html
\paragraph{Az NTSC színmérőrendszere:\\}
Az első kodifikált színmérő rendszer az NTSC (National Television System Committee) által 1953-ban szabványosított színes-televíziós műsorszóráshoz alkalmazott NTSC szabvány volt.
A színteret a korabeli foszfortechnológiával létrehozható CRT kijelzők (TV vevők) alapszíneik megfelelően írták elő, így színtérkorrekció vevő oldalon nem volt szükség.
A színmérő rendszer C fehérponttal dolgozott, alapszíneit pedig a \ref{tab:ntsc_colorimetry} táblázat mutatja.
Az így kapott gamut az \ref{Fig:gamut} ábrán látható.
\begin{table}[h!]
\caption{Az NTSC szabvány színmérőrendszere}
\renewcommand*{\arraystretch}{1}
\label{tab:ntsc_colorimetry}
\begin{center}
\small\addtolength{\tabcolsep}{15pt}
    \begin{tabular}[h!]{ @{}c | | l | l @{} }%\toprule
		&   x  	&    y \\ \hline
    R   &  0.67 &	0.33 \\
    G   &  0.21 &   0.71  \\
    B   & 0.14   &	0.08\\
    C fehér     &  0.310 &	0.316  \\
    \end{tabular}
\end{center}
\end{table}
Az alapszínekből és a fehérpontból meghatározható az $RGB_{\mathrm{NTSC}} \rightarrow XYZ$ transzformációs mátrix, amely alakja általánosan
\begin{equation}
\begin{bmatrix}[c]
       X \\[0.3em]
       Y \\[0.3em]
       Z \end{bmatrix}
       = 
  \begin{bmatrix}[c c c]
   0.60 & 0.17 & 0.2  \\
   0.30 & 0.59 & 0.11 \\
   0 & 0.07 & 1.11
\end{bmatrix}
\cdot
\begin{bmatrix}[c]
       R \\[0.3em]
       G \\[0.3em]
       B \end{bmatrix}_{\mathrm{NTSC}}
\label{Eq:NTSC_transform}
\end{equation}
Az egyenlet második sora kitüntetett szereppel bír: meghatározza, hogy az NTSC színtérben hogyan számítható adott $RGB$ színpont relatív fénysűrűsége (világossága):
\begin{equation}Y_{\mathrm{NTSC}} = 
   0.30R + 0.59G + 0.11 B. 
\label{Eq:NTSC_luminance}
\end{equation}
A világosságjel számítása egészen a HD formátum megjelenése (azaz közel 50 éven keresztül) a fenti egyenlet szerint történt.

\vspace{3mm}
Az foszfortechnológia fejlődésével az újabb megjelenítők egyre inkább feláldozták a széles gamutot (azaz a minél telítettebb alapszínek használatát) a minél nagyobb fényerő érdekében: 
Az alkalmazott foszforok a nagyobb érzékelt világosság (fénysűrűség) érdekében egyre nagyobb sávszélességben sugároztak, így az alapszínek egyre kevésbé telítettek lettek, a gamut tehát csökkent (más szóval: az alapszínek spektruma a Dirac-impulzus helyett---amely teljesen telített spektrálszín lenne---szélesebb görbe lett, így a görbe alatti terület---és ezzel a szín világossága nőtt---de telítettsége csökkent).

Mivel így a megjelenítő gamutja jelentősen eltért az NTSC szabványtól, ezért ez a képernyőn látható színek torzulását eredményezte.
Ennek megoldásául a TV vevőkbe analóg színtérkonverziós áramköröket ültettek, amelyek az NTSC és a megjelenítő saját színtere közti konverziót valósította meg\footnote{Ahogy látni fogjuk a későbbiekben: a vevőkbe már csak a nem-lineárisan Gamma-előtorzított $RGB$ jelek jutottak, ahol az inverz torzítást maga a kijelző hajtotta végre. Emiatt a színtérkonverziót csak Gamma-torzított $R'G'B'$ jeleken tudták végrehajtani, ami azonban a telített színeknél ismét látható színezet és fénysűrűség-hibát okozott.}.
Ettől a ponttól tehát a műsorszórás szabványos színtere és a megjelenítők színtere különváltak.

\begin{figure}[]
	\centering
	\begin{overpic}[width = 0.7\columnwidth ]{figures_en/Video_colorspaces/gamuts.png}
	\end{overpic}
	\caption{Az NTSC, PAL/SD/HD/sRGB és UHD szabványok gamutja az $xy$-színpatkóban.
	Az NTSC jóval nagyobb gamuttal dolgozott, mint a ma is használt HD és sRGB formátumok. Ennek oka, hogy a korai CRT megjelenítők ugyan telítettebb, de ugyanakkor kisebb fénysűrűségű és nagy időállandójú foszforokkal dolgoztak, amivel bár nagy színtartományt tudtak megjeleníteni, de kis fényerővel, és mozgó objektumoknál a képernyőn akaratlanul is nyomokat hagyva.}
	\label{Fig:gamut}
\end{figure}
\paragraph{A PAL és az SD színmérőrendszere:\\}
Az európai színes műsorszórás bevezetéséhez az EBU (European Broadcasting Union) 1963-ban szabványosította a PAL (Phase Alternating Line) rendszert, újradefiniálva a színmérőrendszert, új alapszíneket és D65 fehéret alkalmazva:
\begin{table}[h!]
\caption{A PAL szabvány színmérőrendszere}
\renewcommand*{\arraystretch}{1}
\label{tab:pal_colorimetry}
\begin{center}
\small\addtolength{\tabcolsep}{15pt}
    \begin{tabular}[h!]{ @{}c | | l | l @{} }%\toprule
		&   x  	&    y \\ \hline
    R   &  0.64 &  0.33 \\
    G   &  0.29 &  0.60  \\
    B   & 0.15 & 0.06\\
    D65 fehér     &  0.3127 & 0.3290 	  \\
    \end{tabular}
\end{center}
\end{table}
%
Ez matematikailag helyesen a transzformációs mátrix és a világosságjel számításának módjának megváltozását jelentené.
Praktikussági szempontokból azonban a PAL rendszer az NTSC-vel azonos módon, \eqref{Eq:NTSC_luminance} alapján állítja elő a világosságjelet, mivel a gyakorlatban a különbség alig volt látható \footnote{Ennek oka, hogy a világosságjel átviteltechnológia szempontjából fontos: a kamera és a kijelző is $RGB$ jeleket használ, a világosságjelet, ahogy a következőekben látjuk csak a képanyag átviteléhez számítjuk ki.}.
Az PAL alapszíneit és a világosságjel számításának módját átvette az első digitális videóformátum, az ITU (International Telecommunication Union) által szabványosított ITU-601-es SD formátum is 1982-ben.

\paragraph{A HD és UHD formátumok színmérőrendszere:\\}
A HD formátumot az 1990-ben szabványosították az ITU-709-es ajánlás formájában.
Az ajánlás átvette az PAL rendszer alapszíneit, azonban immáron matematikailag precízen, újraszámította a transzformációs mátrixot és a világosságjel együtthatókat, amely tehát HD esetén
\begin{equation}Y_{\mathrm{ITU}-709} = 
   0.2126\,R + 0.7152\,G + 0.0722\,B. 
\label{Eq:NTSC_luminance}
\end{equation}
alapján számítható.
Fontos megjegyezni, hogy az ITU-709 szabvány színmérőrendszerét átvette az sRGB szabvány is, ami a mai napig a számítógépes alkalmazások (és operációs rendszerek) alapértelmezett színteréül szolgál.

Az alkalmazott alapszíneket végül számottevően csak az UHD formátum változtatta meg az ITU-2020 számú ajánlásában 2012-ben.
Az UHD alkalmazásokra a szabvány egy széles gamutú, spektrál-alapszíneket alkalmazó színteret ajánl a \ref{tab:UHDTV_colorimetry} táblázatban látható paraméterekkel. 
\begin{table}[h!]
\caption{Az ITU-2020 szabvány színmérőrendszere}
\renewcommand*{\arraystretch}{1}
\label{tab:UHDTV_colorimetry}
\begin{center}
\small\addtolength{\tabcolsep}{15pt}
    \begin{tabular}[h!]{ @{}c | | l | l @{} }%\toprule
		&   x  	&    y \\ \hline
    R   &  0.708 &	0.292  \\
    G   &  0.17 &	0.797  \\
    B   & 0.131 &	0.046 \\
    D65 fehér     &  0.3127 & 0.3290 	  \\
    \end{tabular}
\end{center}
\end{table}
A szabvány természetesen újradefiniálta a világosság komponens számításának a módját is, amely tehát UHD esetben
\begin{equation}Y_{\mathrm{ITU}-2020} = 
   0.2627\,R + 0.678 \,G + 0.0593\,B 
\label{Eq:UHD_luminance}
\end{equation}
alapján számítható.
A szabvány természetesen nem igényli, hogy az UHD megjelenítők spektrálszíneket legyenek képesek alapszínekként realizálni, a minél szélesebb gamut inkább a jövőbeli technológiák szempontjából ad ajánlást.
A mai konzumer megjelenítők az UHD képanyagot megjelenítés előtt a saját színterükben konvertálják, amely jellegzetesen jóval kisebb a szabvány színterénél.

\subsection{Example for device-dependent color space}
\label{sec:CRT}

Egyszerű példaként az eddig leírtakra vizsgáljuk, hogyan számítható és illusztrálható egy CRT kijelző által megjelenített színek tartománya, röviden rávilágítva a CRT technológia működési elvére is \footnote{Természetesen az itt leírtak változtatás nélkül alkalmazhatók más technológia alapján működő kijelzőkre is, pl. LCD.}.
Bár a CRT technológia kezd egyre inkább eltűnni, néhány évvel ezelőttig a stúdiómonitorok jelentős része még mindig CRT alapon működött köszönhetően a színhű megjelenítésüknek, és a mai LCD megjelenítőkhöz képest is jóval nagyobb statikus kontrasztjuknak.

\begin{figure}[]

	\centering
	\begin{overpic}[width = 0.5\columnwidth ]{figures_en/Video_colorspaces/1024px-CRT_color_enhanced.png}
	\end{overpic}
	\caption{CRT megjelenítő felépítése.}
	\label{Fig:crt}
\end{figure}

A katódsugárcsöves (CRT) kijelzők sematikus ábrája az \ref{Fig:crt} ábrán látható.
A CRT-k kijelzők működésének alapja három ún. elektronágyú volt, amelyek egy fűtőtt katódból (1) és egy nagyfeszültségre helyezett anódból állt.
A melegítés hatására a katód környezetébe szabad elektronok léptek ki, így egy elektronfelhőt képezve a katód körül.
A katód közelébe helyezett nagyfeszültségű (néhány száz Volt) gyorsítóanód hatására a szabad elektronok az anód felé kezdtek mozogni, egy szabad elektronáramot (2) indítva a vákuumban (ugyanezen az elven működtek a vákuum-diódák, triódák, pentódák, stb. is).
Elegendően nagy anódfeszültség (és további anódok jelenléte) esetén az elektronok jelentős része nem csapódott be a gyorsítóanódra, hanem továbbhaladt.
Ezt az elektronnyalábot elektrosztatikusan és mágnesesen (3) fókuszálták, majd egy vezérelt mágneses eltérítő (4) sorról sorra végigfuttatta azt egy anódfeszültségű-ernyőn (5), azaz a képernyőn.
Színes kijelző esetén természetesen három elektronágyú üzemelt párhuzamosan.
A képernyő felszínét pixelekre bontva képpontonként három különböző foszforral borították (7-8), amely gerjesztés (becsapódó elektronok) hatására bizonyos ideig adott spektrális sűrűségfüggvényű fényt bocsájtott ki\footnote{Ellentétben a fluoreszkáló anyagok csak a gerjesztés fennállásának idején bocsájtanak ki fényt. 
A foszforeszkálás időállandója előnyös, hiszen megfelelően megválasztott foszforok épp egy képidőig bocsájtanak ki fényt, így a kijelzett kép nem fog villogni.
Ugyanakkor a korai kijelzők ezen időállandója túl nagy volt, ezért a gyors mozgások elmosódtak a kijelzett képen.}, realizálva ezzel az RGB alapszíneket.

\begin{figure}[]
	\centering
	\begin{overpic}[width = 0.54\columnwidth]{figures_en/Video_colorspaces/sony.png}
	\small
	\put(0,0){(a)}
	\end{overpic}
	\begin{overpic}[width = 0.39\columnwidth]{figures_en/Video_colorspaces/sony_gamut.png}
	\small
	\put(0,0){(b)}
	\end{overpic}
	\begin{overpic}[width = 0.014\columnwidth]{figures_en/Video_colorspaces/sony_gamut_2.png}
	\end{overpic}
	\caption{CRT megjelenítő foszforai által kibocsájtott sugárzás spektrális sűrűségfüggvénye (a) a megjelenítő gamutja és az adott spektrumok/alapszínek által keltett színérzet, valamint a színtér fehérpontja (b).
	A jobb oldali oszlop bal fele a Sony monitor alapszíneit és fehérpontját, a jobb fele az sRGB színtér alapszíneit és fehérpontját szemlélteti.}
	\label{Fig:sony}
\end{figure}

Tekintsünk példaként egy Sony F520 CRT kijelzőt: 
A kijelző RGB foszforjai gerjesztés hatására a \ref{Fig:sony} (a) ábrán látható spektrális sűrűségfüggvényű (sugársűrűségű) fényt bocsájtanak ki egységnyi felületről, egységnyi térszögbe, azaz rendelkezésre állnak a mért $L_{e}^R(\lambda)$, $L_{e}^G(\lambda)$ és $L_{e}^B(\lambda)$ függvények.
Fejezzük ki ezek segítségével a kijelző működéséhez szükséges RGB vezérlőjeleket, illetve vizsgáljuk a megjeleníthető színek tartományát!

A $\overline{x}(\lambda)$, $\overline{y}(\lambda)$, $\overline{z}(\lambda)$ szabványos $XYZ$ spektrális színösszetevő függvények alkalmazásával a piros (és persze a zöld és kék) alapszín abszolút $XYZ$ színkoordinátái rendre a
\begin{align}
\begin{split}
\overline{X}_R &= K_m \int_{380~\mathrm{nm}}^{780~\mathrm{nm}} L_{e}^R(\lambda) \cdot \overline{x}(\lambda) \mathrm{d} \lambda = 45.3, \hspace{5mm} \overline{X}_G = 21.4,\hspace{5mm}  \overline{X}_B = 16.6 \\
\overline{Y}_R &= K_m \int_{380~\mathrm{nm}}^{780~\mathrm{nm}} L_{e}^R(\lambda) \cdot \overline{y}(\lambda) \mathrm{d} \lambda = 25.5
, \hspace{5mm} \overline{Y}_G = 48,\hspace{5mm}  \overline{Y}_B = 6.7 \\
\overline{Z}_R &= K_m \int_{380~\mathrm{nm}}^{780~\mathrm{nm}} L_{e}^R(\lambda) \cdot \overline{z}(\lambda) \mathrm{d} \lambda  = 2.4, \hspace{5mm} \overline{Z}_G = 11.6,\hspace{5mm}  \overline{Z}_B =84.6\\
\end{split}
\end{align}
integrálok numerikus kiértékelésével számítható, ahol $K_m = 683~\mathrm{lm/W}$ fényhasznosítási tényező.
A színtérben előállítható fehér szín definíció szerint az alapszínvektorok egyenlő súlyú összegeként áll elő, azaz
\begin{equation}
\overline{X}_W = \overline{X}_R + \overline{X}_G + \overline{X}_B, \hspace{6mm} 
\overline{Y}_W = \overline{Y}_R + \overline{Y}_G + \overline{Y}_B, \hspace{6mm} 
\overline{Z}_W = \overline{Z}_R + \overline{Z}_G + \overline{Z}_B,
\end{equation}
azaz pl. a fehér szín abszolút fénysűrűsége $80.2~\mathrm{cd/m^2}$.
Ez egészen pontosan megegyezik az sRGB szabvány által előírt referenciamonitor fénysűrűségével ($80~\mathrm{cd/m^2}$).

Természetesen az alapszíneknek nem az abszolút XYZ koordinátái a fontosak, hanem a relatív koordináták, amelyekre teljesül, hogy $Y_W=1$, és így $Y$ a relatív fénysűrűség.
A fenti alapszínvektorok tehát $\overline{Y}_W$ értékével normálandók.
Az így kapott relatív alapszínvektorokból már összeállíthatók a színtér alkalmazásához szükséges transzformáció mátrixok:
\begin{align}
\begin{split}
\begin{bmatrix}[c]
       X \\[0.3em]
       Y \\[0.3em]
       Z \end{bmatrix} &= 
     \underbrace{ \begin{bmatrix}[c|c|c]
       0.5646 &  0.2665 &  0.2068 \\[0.3em]
       0.3174 &  0.5992 &  0.0834 \\[0.3em]
       0.0302 &  0.1443 &  1.0539 \end{bmatrix} }_{\mathbf{M}_{R\!G\!B \rightarrow X\!Y\!Z}}
\begin{bmatrix}[c]
       R \\[0.3em]
       G \\[0.3em]
       B \end{bmatrix}_{\mathrm{F}520}
\\ \vspace{1mm} \\
&\mathbf{M}_{X\!Y\!Z \rightarrow   R\!G\!B} = \mathbf{M}_{R\!G\!B \rightarrow X\!Y\!Z}^{-1}
\end{split}
\end{align}
Az alapszínek és a fehérpont színezete ezután
\begin{equation}
x_R = \frac{X_R}{X_R + Y_R + Z_R}, \hspace{1cm} y_R = \frac{Y_R}{X_R + Y_R + Z_R}
\end{equation}
alapján számolható.
Az így meghatározott színtér gamutja a \ref{Fig:sony} ábrán látható, az alapértelmezett számítógépes sRGB színtérrel együtt.

Jelen dokumentum sRGB színtérben kerül tárolásra (és megjelenítéskor az sRGB színtér az olvasó kijelzőjének saját színterébe transzformálva), így jelen dokumentumban az $XYZ$ koordinátáival adott alapszínek az sRGB térbe való konverzió után kerülhetnek megjelenítésre (ahogy \ref{Fig:sony} ábrán látható), amely pl. a vörös alapszínre
\begin{equation}
\begin{bmatrix}[c]
       R_R \\[0.3em]
       G_R \\[0.3em]
       B_R \end{bmatrix}_{\mathrm{sRGB}}
       =
     \mathbf{M}_{X\!Y\!Z \rightarrow R\!G\!B_{\mathrm{sRGB}}}
      \begin{bmatrix}[c]
       X_R \\[0.3em]
       Y_R \\[0.3em]
       Z_R \end{bmatrix} =      
       \begin{bmatrix}[c]
       1.13 \\[0.3em]
       0.25 \\[0.3em]
       -0.02 \end{bmatrix} 
\end{equation}
alakú.
A Sony megjelenítő alapszíneinek sRGB koordinátáira negatív és 1-nél nagyobb $RGB$ értékek is adódnak.
Ez a \ref{Fig:sony} ábrán is látható gamutok közti eltérést tükrözi.

\section{The $Y,\,R-Y,\,B-Y$ representation}

Az előző szakasz bemutatta egy színes képpont ábrázolásának módját adott RGB eszközfüggő színtérben.
Láthattuk, hogy egy színinger leírására a fő érzeti jellemzők a színpont világossága, színezete és telítettsége volt.
Felmerül tehát a kérdés, hogy létezik-e hatékonyabb reprezentációja az egyes színpontoknak, ami jobban leírja a fent említett szubjektív jellemzőket, így kevesebb redundáns információt tartalmaz az RGB reprezentációnál.

\subsection{The color difference signals}
A fő oka, hogy a korai TV- és videójeleket nem közvetlenül az RGB jeleknek választották (bár manapság már gyakori a közvetlen RGB ábrázolás) az NTSC bevezetésének idejében a visszafelé kompatibilitás biztosítása volt:
A színes műsorszórás kezdetén a korabeli háztartásokban szinte kizárólag fekete-fehér TV-vevők voltak találhatók.
Természetes volt az igény a már kiépített fekete-fehér műsorszóró rendszerrel való visszafelé kompatibilitásra színes kép-továbbítás esetén, amelyet a fekete-fehér kép és a színinformáció külön kezelésével volt elérhető.
Természetesen manapság már ez a tradicionális ok nem szempont videójelek megválasztása esetén.
Azonban látni a színinformáció külön kezelése lehetővé teszi a színek csökkentett felbontással való tárolását, amely jelentős adattömörítést (analóg esetben sávszélesség-csökkentést) tesz lehetővé.

A fekete-fehér kép egy színes kép világosságinformációjának fogható fel, amely a színpont relatív fénysűrűségével arányos, és így az $RGB$ koordináták lineáris kombinációjaként számítható.
Az együtthatók az adott eszközfüggő színtértől függnek, az NTSC alapszínei esetén pl. \eqref{Eq:NTSC_luminance} alapján adottak.
Ebből kifolyólag színes TV esetén is az változatlanul továbbítandó jelnek a \textbf{világosságjelet (luminance)} választották, amely tehát a relatív fénysűrűséggel megegyezik, és így pl. NTSC esetén az RGB jelekből
\begin{equation}Y_{\mathrm{NTSC}} = 
   0.30R + 0.59G + 0.11 B. 
   \label{Eq:NTSC_luminance}
\end{equation}
alapján számítható \footnote{Fontos ismét kihangsúlyozni, hogy a világosság-számítás módja színtérfüggő, az alapszínektől és a fehérponttól függ a már bemutatott módon.}.

Egy színes képpont leírásához 3 komponens szükséges, egy lehetséges és hatékony leírás pl. a képpont világossága, színezete és telítettsége.
A világosságjel mellé tehát két független információ kell, amelyek egyértelműen meghatározzák az adott színpont színezetét és telítettségét\footnote{A visszafelé-kompatibilitás biztosításához ezt a két színezetet leíró jelet kellett az NTSC rendszerben a változatlan fekete-fehér jelhez úgy hozzáadni, hogy a meglévő fekete-fehér vevők a világosságjelet demodulálni tudják, és a hozzáadott többletinformáció minimális látható hatással legyen a megjelenített képre.}.
Ugyanakkor fontos szempont volt ezen világosságinformáció-mentes, pusztán színinformációt leíró jelek könnyű számíthatósága az RGB komponensekből az egyszerű analóg áramköri megvalósíthatóság érdekében.

A színinformáció/világosságinformáció-szétválasztás legegyszerűbb (de jól működő) megoldásaként egyszerűen vonjuk ki a világosságot az RGB jelekből!
Mivel az $Y$ együtthatóinak összege definíció szerint (tetszőleges színtérben) egységnyi, így pl. NTSC esetén \eqref{Eq:NTSC_luminance} mindkét oldalából $Y$-t kivonva igaz a 
\begin{equation} 
   0.30 ( R - Y ) + 0.59 ( G - Y )  + 0.11 ( B - Y )  = 0 
   \label{eq:chrominances}
\end{equation}
egyenlőség.
Az $ ( R - Y ) $, $ ( G - Y ) $ és $ ( B - Y ) $ a TV-technika ún. \textbf{színkülönbségi jelei}, és a következő tulajdonságokkal bírnak:
\begin{itemize}
\item Nem függetlenek egymástól, kettőből számítható a harmadik.
\item Előjeles mennyiségek.
\item Ha két színkülönbségi jel zérus, akkor a harmadik is az.
Ekkor $R = G = B = Y$, így tehát a színtér fehérpontjában vagyunk.
A fehér színre kapott zérus színkülönbségi jelek azt mutatják, hogy a színinformációt valóban a színkülönbségi jelek jelzik, a fénysűrűség (világosság) pedig tőlük független mennyiség.
\item Az adott színkülönbségi jel értéke maximális ha a hozzá tartozó alapszín maximális intenzitású, és vice versa.
NTSC rendszerben vörös színkülönbségi jelre $R = 1$, $G = B= 0$ esetén
\begin{equation}
Y = 0.30 \cdot 1 + 0.59 \cdot 0 + 0.11 \cdot 0 \hspace{3mm }\rightarrow \hspace{3mm } R - Y  = 0.7,
\end{equation}
és hasonlóan $R=0$, $G = B = 1$ esetén
\begin{equation}
Y = 0.30 \cdot 0+ 0.59 \cdot 1 + 0.11 \cdot 1 \hspace{3mm }\rightarrow \hspace{3mm } R - Y  = -0.7.
\end{equation}
\item A fenti megfontolások alapján a színkülönbségi jelek dinamikatartománya:
\begin{align}
\begin{split}
-0.7 \leq R-Y \leq& 0.7 , \hspace{2cm} -0.89 \leq G-Y \leq 0.89, \\
 &-0.41 \leq B-Y \leq 0.41
\end{split}
\end{align}
\end{itemize}
A három színkülönbségi jelből kettő elegendő a színpont színinformációjának leírásához.
Mivel jel/zaj-viszony szempontjából ökölszabályszerűen mindig a nagyobb dinamikatartományú jelet célszerű továbbítani, így a választás a vörös és zöld színkülönbségi jelekre esett.

A videótechnikában tehát egy adott színpont ábrázolása a
\begin{align*}
Y&: \text{Luminance }\\
 	\left.\begin{array}{lr}
        R-Y\\
        B-Y
        \end{array}\right\}&: \text{Chrominance}
\end{align*}
ún. \textbf{luminance-chrominance térben} történik, amely felfogható egy új színmérőrendszernek/színtérnek is az $RGB$ színtérhez képest.

\subsection{The luminance-chrominance color space}
Vizsgáljuk most, hol helyezkednek el az adott RGB eszközfüggő színtérben ábrázolható színek ebben az új, $Y,\, R-Y,\, B-Y$ térben!
Az előzőekben láthattuk, hogy az $XYZ$ térben ez a színhalmaz egy paralelepipedont, az RGB térben egy egységnyi oldalú kockát jelent (lásd \ref{Fig:device_dep} ábra).
Vegyük észre, hogy a $Y,\, R-Y,\, B-Y$ koordinátákat akár az $XYZ$, akár az RGB komponensekből egy lineáris transzformációval előállíthatjuk:
Jelöljük adott RGB alapszínek esetén a relatív fénysűrűség RGB együtthatóit $k_r, k_g, k_b$-vel.
Ekkor általánosan a színkülönbségi jelek a 
\begin{align}
\begin{bmatrix}[c]
       Y \\[0.3em]
       B - Y \\[0.3em]
       R - Y\end{bmatrix} &= 
\begin{bmatrix}[c c c]
      k_r &  k_g&  k_b  \\[0.3em]
      -k_r &  -k_g&  1-k_b  \\[0.3em]
      1-k_r &  -k_g&  -k_b \end{bmatrix} 
\begin{bmatrix}[c]
       R \\[0.3em]
       G \\[0.3em]
       B \end{bmatrix}
\end{align}
transzformációval számíthatók.
Példaképp maradva az NTSC rendszer világosság-együtthatóinál (kiindulva abból, hogy $Y = 0.3R + 0.59G + 0.11B$) a transzformáció alakja
\begin{align}
\begin{bmatrix}[c]
       Y \\[0.3em]
       B - Y \\[0.3em]
       R - Y \end{bmatrix} &= 
\begin{bmatrix}[c c c]
      0.3 &  0.59&  0.11  \\[0.3em]
       -0.3 &  -0.59 & 0.89  \\[0.3em]
      0.7 &  -0.59&  -0.11  \end{bmatrix} 
\begin{bmatrix}[c]
       R \\[0.3em]
       G \\[0.3em]
       B \end{bmatrix}_{\mathrm{NTSC}}.
\end{align}
A lineáris transzformációt az RGB kockára végrehajtva megkaphatjuk az ábrázolható színek halmazát.
Az így kapott test az \ref{Fig:YCbCr_space} (a) ábrán látható.
Láthatjuk, hogy az RGB egységkocka egy paralelepipedonba transzformálódott, ahol a paralelepipedon főátlója az $Y$ világosság tengely.
Ennek mentén, az $R-Y = B-Y = 0$ tengelyen helyezkednek el a különböző szürke árnyalatok. 
\begin{figure}[htp]
	\centering
	\begin{overpic}[width = 0.45\columnwidth ]{figures_en/Video_colorspaces/LC_space_1.png}
	\small
	\put(0,0){(a)}
	\put(45,90){$Y$}
	\put(48,2){$R\!-\!Y$}
	\put(87,26){$B\!-\!Y$}
	\end{overpic}
	\hspace{6mm}
	\begin{overpic}[width = 0.48\columnwidth ]{figures_en/Video_colorspaces/LC_space_2.png}
	\small
	\put(0,0){(b)}
	\scriptsize
	\put(39,82){$R$}
	\put(25,24){$G$}
	\put(89,44){$B$}
	\put(12,58){$Y\!e$}
	\put(65,19){$C\!y$}
	\put(78,77){$M\!g$}
	\end{overpic}
	\caption{Az $Y, R-Y, B-Y$ színtér ábrázolható színeinek halmaza oldalnézetből (a) és felülnézetből (b).}
	\label{Fig:YCbCr_space}
\end{figure}

Az eredeti RGB kockához hasonlóan, paralelepipedon főátlón kívüli csúcsaiban (amelyben az $Y=0$ fekete és az $R=G=B=Y=1$ fehér található) az eszközfüggő színtér egy, vagy két $100~\%$-os intenzitású alapszínnel kikeverhető
\begin{equation}
R = \begin{bmatrix}[c] 1\\[0.3em] 0\\[0.3em] 0\end{bmatrix} \hspace{2mm}
G = \begin{bmatrix}[c] 0\\[0.3em] 1\\[0.3em] 0\end{bmatrix}\hspace{2mm}
B = \begin{bmatrix}[c] 0\\[0.3em] 0\\[0.3em] 1\end{bmatrix}\hspace{2mm}
Cy = \begin{bmatrix}[c] 0\\[0.3em] 1\\[0.3em] 1\end{bmatrix}\hspace{2mm}
Mg = \begin{bmatrix}[c] 1\\[0.3em] 1\\[0.3em] 0\end{bmatrix}\hspace{2mm}
Ye = \begin{bmatrix}[c] 1\\[0.3em] 1\\[0.3em] 0\end{bmatrix}
\end{equation}
vörös, zöld, kék alap- és cián, magenta, sárga ún. komplementer színek találhatók \footnote{Ezen komplementer színek tulajdonsága, hogy az egyes RGB alapszínekkel RGB kockában átlósan helyezkednek el, így a színtérben a lehető legmesszebb elhelyezkedő színpárokat alkotják.
Ennek megfelelően egymás mellé vetítve a komplementer színpárok (vörös-cián, sárga-kék, zöld-magenta) váltják ki a legnagyobb érzékelt kontrasztot.}.

A paralelepipedonra az $Y$-tengely irányából ránézve (\ref{Fig:YCbCr_space} (b) ábra) láthatjuk a világosságjeltől függetlenül, adott színtérben kikeverhető színek összességét.
Az $R-Y, B-Y, Y$ térben gyakori adott $Y$ világosság mellett a színek ezen $R-Y, B-Y$ síkon való ábrázolása.
Minthogy az $R-Y, B-Y$ jelek meghatározzák adott színpont színezetét és telítettségét, így az ábra azt jelzi, hogy a különböző színezetű és telítettségű színek egy szabályos hatszöget töltenek ki.
A hatszög csúcsai a színtér alap- és komplementerszínei.
Természetesen adott $Y$ érték mellett az ábrázolható színek nem tölti ki teljesen ezt a hatszöget:
adott világosságérték mellett az ábrázolható színek halmaza a $Y, R-Y, B-Y$ paralelepipedon egy adott $Y$ magasságban húzott síkkal vett metszeteként képzelhető el, azaz tetszőleges $0 \leq Y \leq1$ esetén rajzolható egy $R-Y, B-Y$ diagram.
Az így rajzolható diagramokra példákat a \ref{Fig:YCbCr_sect} ábra mutat.
\begin{figure}[htp]
	\centering
	\begin{overpic}[width = 1\columnwidth ]{figures_en/Video_colorspaces/YCbCr_2_11.png}
	\small
	\put(0,3){(a)}
	\put(0,37){$Y = 0.11$}
	\end{overpic}
	\vspace{2mm}
	\begin{overpic}[width = 1\columnwidth]{figures_en/Video_colorspaces/YCbCr_2_30.png}
	\small
	\put(0,37){$Y = 0.3$}
	\put(0,3){(b)}
	\end{overpic}
	\vspace{2mm}
	\begin{overpic}[width = 1\columnwidth]{figures_en/Video_colorspaces/YCbCr_2_59.png}
	\small
	\put(0,37){$Y = 0.59$}
	\put(0,3){(c)}
	\end{overpic}
	\caption{Különböző $Y$ értékek mellett rajzolható $B-Y, R-Y$ diagramok.}
	\label{Fig:YCbCr_sect}
\end{figure}
Nyilván rögzített $Y$ mellett nem biztos, hogy minden szín $100~\%$-os telítettséggel van jelen a $B-Y,R-Y$ diagramon. 
Például: teljesen telített kékre $Y=0.11$, azaz a $100~\%$ intenzitású kék alapszín ezen magasságban vett diagramon található.
Más magasságban vett  $B-Y, R-Y$ diagramon csak fehérrel higított kék található, azaz nem teljesen telített kék található.

A vizsgált diagramokból leszűrhető, hogy a világosságjel valóban független a színinformációtól, adott színpont színezetét és telítettségét pusztán az $R-Y$ és $B-Y$ diagramokon vett helye meghatározza.
Vizsgáljuk most, hogyan definiálhatóak ezen érzeti jellemzők, azaz a színezet és telítettség a TV technika $Y, R-Y, B-Y$ színterében!

\subsection{Hue and saturation in device-dependent color spaces}
A könnyebb elképzelhetőség kedvéért ábrázoljuk az $R-Y, B-Y$ koordinátákhoz tartozó színeket, az adott színponthoz tartozó olyan világosságérték mellett, amely esetén pontonként teljesül, hogy $X \!+\!Y\!+\!Z = 1$: 
ezzel gyakorlatilag az adott RGB színtér $xy$-színpatkón felvett színét képezzük le az $R-Y, B-Y$ diagramra.
\begin{figure}[htp]
	\centering
	\begin{minipage}[c]{0.6\textwidth}
	\begin{overpic}[width = 1\columnwidth ]{figures_en/Video_colorspaces/YCbCr_gamut.png}
	\small
	\put(56,46){$\alpha$}
	\end{overpic} \end{minipage}\hfill
	\begin{minipage}[c]{0.4\textwidth}
	\caption{Adott $Y, R-Y, B-Y$ térben ábrázolható színek gamutja.}
	\label{Fig:ycbcr_gamut}  \end{minipage}
\end{figure}
Az így kapott színhalmaz, amely felfogható az adott alapszínek mellett a luminance-chrominance tér gamutjának is, a \ref{Fig:ycbcr_gamut} ábrán látható.

\paragraph{Színezet:}
Megfigyelhető, hogy a diagramon az origóból kiinduló félegyenesen azok a színek vannak, amelyek egymásból kinyerhetők fehér szín hozzáadásával.
Tehát az origóból kiinduló félegyenesen az azonos színezetű, de eltérő telítettségű színek vannak. 
Azaz tetszőleges színpontot vizsgálva, a $B-Y,R-Y$ diagramon a színpontba mutató helyvektor iránya egyértelműen meghatározza az adott pont színezetét.
Ennek megfelelően a TV technikában a színezetet a $B-Y, R-Y$ diagramon a színpont helyvektorának irányszögeként definiáljuk:
\begin{equation}
\text{színezet}_{\mathrm{TV}} = \alpha  = \arctan \frac{R-Y}{B-Y}
\label{eq:hue}
\end{equation}
a \ref{Fig:ycbcr_gamut} ábrán látható jelölés alkalmazásával.

\paragraph{Telítettség:}
A telítettség kifejezése már kevésbé egyértelmű, több definíció bevezethető rá.
Általánosan, a telítettség azt fejezi ki, mennyi fehér hozzáadásával keverhető ki egy adott szín a színezetét meghatározó teljesen telített alapszínből.
Az $XYZ$-térben bevezettük a telítettségre a színtartalmat, illetve színsűrűséget.
Felmerül a kérdés, hogyan terjeszthető ki a telítettség fogalma eszközfüggő RGB színterekre.
Láthattuk, hogy az adott RGB színtérben előállítható legtelítettebb színek a gamut határán elhelyezkedő kvázi-spektrál színek, amelyek a legközelebb vannak az azonos színezetű valódi spektrálszínhez.
A bevezetendő telítettség-mennyiség célszerűen a kvázi-spektrál színekre tehát maximális, egységnyi értékű.

A telítettség ezek után a következő módokon definiálható.
\begin{itemize}
\item  Minthogy egy tetszőleges színnek a fehér színtől, azaz az origótól vett távolsága arányos a szín fehér-tartalmával, így legegyszerűbb módon a telítettség közelíthető a
\begin{equation}
\text{telítettség}_{\mathrm{TV},1} = \sqrt{ (R-Y)^2 +(B-Y)^2}
\label{eq:saturation_1}
\end{equation}
távolsággal.
Később tárgyalt okok miatt az analóg időkben TV technikusok körében ez a definíció volt érvényben.
Az így számolt telítettség valóban $0$ a fehér színre, azonban a kvázi-spektrálszínek telítettsége így nem egységnyi.
%
\item A matematikailag korrekt telítettség-definíció bevezetéséhez kiterjeszthetjük a korábban megismert színsűrűséget eszközfüggő színterekre\footnote{Ismétlésként: az $XYZ$ térben adott pont színsűrűsége $p_c = \frac{Y_d}{Y}$, ahol $Y_d$ az adott színhez tartozó domináns hullámhosszú szín fénysűrűsége, $Y$ a vizsgált szín saját fénysűrűsége.}.
Ennek egyszerűbb értelmezéséhez ábrázoljuk adott színpont paramétereit ún. területdiagramon!
%
\begin{figure}[htp]
	\centering
	\begin{minipage}[c]{0.6\textwidth}
	\begin{overpic}[width = 1\columnwidth ]{figures_en/Video_colorspaces/area_chart.png}
	\end{overpic} \end{minipage}\hfill
	\begin{minipage}[c]{0.4\textwidth}
	\caption{Tetszőlegesen választott $R,G,B$ koordináták esetén rajzolható területdiagram.}
	\label{Fig:area_diagram}  \end{minipage}
\end{figure}
%
A területdiagram a következő módon rajzolható fel egy tetszőleges RGB koordinátáival adott szín esetén: 
A vízszintes tengelyt osszuk fel az $Y$ fénysűrűség RGB együtthatóinak megfelelően, majd az egyes RGB komponenseket ábrázoljuk az intenzitásuknak megfelelő magasságú oszlopokkal.
Ekkor egy $Y$ magasságban húzott vonal alatt és fölött a színkülönbségi jeleknek megfelelő magasságú oszlopok alakulnak ki, amely oszlopok előjelesen vett területeinek összege \eqref{eq:chrominances} alapján zérus.
\begin{figure}[b!]
	\centering
	\begin{overpic}[width = 1\columnwidth ]{figures_en/Video_colorspaces/YCbCr_saturation.png}
	\small
	\put(0,0){(a)}
	\put(50,0){(b)}
	\end{overpic}
	\caption{Az $R-Y,B-Y$ térben ábrázolt színek telítettsége \eqref{eq:saturation_1} (a) és \eqref{eq:saturation_2} (b) alapján számolva}
	\label{Fig:saturations}  
\end{figure}

Válasszuk ki ezután a legkisebb $RGB$ komponenst (a \ref{Fig:area_diagram} ábrán látható példában az $R$) és húzzunk egy vízszintes vonalat ennek magasságában!
Ekkor a vizsgált színt két részre osztottuk: egy fehér színre (amelyre $R=G=B$) és egy kvázi-spektrálszínre, amelynek az egyik $RGB$ komponense zérus, és amelynek fénysűrűsége $Y_d = \min (R,G,B) - Y$.
A domináns hullámhosszú spektrálszín szerepét erre a kvázi-spektrálszínre cserélve kiterjeszthetjük a színsűrűséget az adott eszközfüggő színtérre, amely alapján a telítettség definíciója
\begin{equation}
\text{telítettség}_{\mathrm{TV},2} = \frac{| \min(R,G,B) - Y |}{Y}.
\label{eq:saturation_2}
\end{equation}
Könnyen belátható, hogy az $R = G=B=Y$ fehérpontokra a telítettség definíció szerint 0, míg kvázi-spektrálszínekre ($\min(R,G,B) = 0$) a telítettség azonosan 1.
\end{itemize}
A fent tárgyalt két telítettség-definíció alkalmazásával a \ref{Fig:ycbcr_gamut} ábrán látható színek telítettségét az \ref{Fig:saturations} ábra szemlélteti, megerősítve az eddig elmondottakat.
%
\section{The $Y', R'-Y', B'-Y'$ components}

Az előző szakasz bemutatta, hogyan választható legegyszerűbben szét a világosság és színezet/telítettség információ.
A tényleges videójelek ezen $Y, R-Y, B-Y$ jelekkel rokonmennyiségek, azonban történelmi okokból a feldolgozási lánc egy nem-lineáris transzformációt is tartalmaz, az ún. \textbf{gamma-korrekciót}.

\subsection{The role of Gamma-correction}
A gamma-korrekció bevezetése történeti okokra vezethető vissza.
A CRT megjelenítők elektron-ágyúja erős nem-lineáris karakterisztikával rendelkezik, azaz a képernyő pontjain létrehozott fénysűrűség az anódfeszültség nemlineáris függvénye\footnote{Ez a nemlinearitás az anód-katód feszültség-áram karakterisztikájából származik főleg.
A megjelenítésért felelős foszforok már jó közelítéssel lineárisan viselkednek, azaz a gerjesztéssel egyenesen arányos a létrehozott fénysűrűségük.}.
Ez a karakterisztika jól közelíthető egy 
\begin{equation}
L_{R,G,B} \sim U^{\gamma}
\end{equation} 
hatványfüggvénnyel, ahol a legtöbb korabeli kijelzőre az exponens $\gamma \approx 2.5$, $L_{R,G,B}$ az egyes RGB pixelek fénysűrűsége és $U$ a pixelek vezérlőfeszültsége.
Ez a nemlineáris átvitel természetesen jól látható hatással lenne a megjelenített képre:
Az alacsony RGB szintek kompresszálódnak, míg a világos árnyalatok expandálódnak, ennek hatására a telített színek túltelítődnek, illetve a sötét árnyalatok még sötétebbé válnak.
A nem-kívánatos torzulás az \ref{Fig:gamma} ábrán figyelhető meg.

\begin{figure}[]
	\centering
	\begin{overpic}[width = 1\columnwidth ]{figures_en/Video_colorspaces/Gamma.png}
	\small
	\put(0,0){(a)}
	\put(52,0){(b)}
	\end{overpic}
	\caption{RGB kép megjelenítése Gamma-korrekcióval (a) és Gamma-korrekció hiányában (b).
	Utóbbi esetben az $R,G,B$ komponensek egy 2.4 exponensű hatványfüggvénnyel előtorzítottak.}
	\label{Fig:gamma}  
\end{figure}
\vspace{3mm}
A torzítás korrekciója kézenfekvő: 
Az RGB komponensek megjelenítés előtti inverz hatványfüggvénnyel való előtorzítása esetén az előtorzítás és a CRT kijelző torzítása együttesen az $RGB$ jelek lineáris megjelenítését teszi lehetővé $\left(U^{\gamma}\right)^{\frac{1}{\gamma}} = U$ alapján.
Ez a nemlineáris előtorzítás az ún. \textbf{gamma-korrekció}.
\begin{figure}[b!]
	\centering
	\begin{minipage}[c]{0.65\textwidth}
	\begin{overpic}[width = 0.95\columnwidth ]{figures_en/Video_colorspaces/gamma2.png}
	\end{overpic} \end{minipage}\hfill
	\begin{minipage}[c]{0.33\textwidth}
	\caption{A Gamma-korrekció alapelve az RGB jelek előtorzításával.}
	\label{Fig:gamma2}  \end{minipage} 
\end{figure}

A korrekció természetesen a megjelenítés előtt bárhol elvégezhető a videófeldolgozási lánc során, azonban a lehető legegyszerűbb felépítésű TV vevők érdekében az előtorzítást az RGB forrás-oldalon célszerű elvégezni\footnote{Természetesen ez a korai TV vevők esetén volt fontos szempont, amikor a gamma-korrekciót drága/komplex analóg áramkörökkel kellett megvalósítani}.
Ennek megfelelően a gamma-korrekció már kamera oldalon megvalósul (akár analóg, akár digitális módon) az RGB jelek közvetlen gamma-korrigálásával.
A következőkben tehát
\begin{align*}
\begin{split}
R' = R^{\frac{1}{\gamma}}, \hspace{10mm} 
G' = G^{\frac{1}{\gamma}}, \hspace{10mm}
B' = B^{\frac{1}{\gamma}}
\end{split}
\end{align*}
a Gamma-előtorzított RGB összetevőket jelölik, ahol $\frac{1}{\gamma} \approx 0.4-0.6$ szabványtól függően (ld. később).

\hspace{3mm}
Fontos leszögezni, hogy ugyan a Gamma-korrekciót a CRT képernyők nemlinearitásának kompenzációjára vezették be, a gamma-korrekció rendszertechnikája manapság is változatlan annak ellenére, hogy a CRT kijelzők alkalmazását szinte teljesen felváltotta az LCD és LED technológia.
A gamma-korrekció fennmaradásának oka, hogy a videójel digitalizálása során perceptuális kvantálást valósít meg, ahogyan az a következő fejezetben láthatjuk.

\subsection{The luma and chroma components}
A Gamma-korrekció ismeretében bevezethetjük a mai videórendszerekben is alkalmazott tárolt és továbbított videójel-komponenseket:
\begin{figure}[]
	\centering
	\begin{overpic}[width = 0.53\columnwidth ]{figures_en/Video_colorspaces/video_signals.png}
	\end{overpic}
	\hspace{2mm}
	\begin{overpic}[width = 0.44\columnwidth ]{figures_en/Video_colorspaces/video_signals_2.png}
	\end{overpic}
	\caption{A Gamma-korrekció rendszertechnikája és a videójel-komponensek.}
	\label{Fig:gamma_system}  
\end{figure}
A videókomponensek előállításának rendszertechnikája a \ref{Fig:gamma_system} ábrán látható, az egyszerűség kedvéért most a kamerából ITU szabványba, ITU szabványból megjelenítő saját színterébe való színtérkonverziókat figyelmen kívül hagyva.
\begin{itemize}
\item A gamma-korrekció a kamera RGB-jelein hajtódik végre, SD, illetve HD esetében egy kb. 0.5 kitevőjű hatványfüggvény szerint.
A pontos gamma-korrekciós görbéket a következőekben fogjuk tárgyalni.
\item Az gamma-torzított $R',G',B'$ jelekből ezután az adott színtér előírt világosság-együtthatói alapján előállíthatók az $Y', R'-Y', B'-Y'$ jelek.
Továbbra is példaként az NTSC rendszer együtthatóinál maradva ezek alakja
\begin{align}
\begin{split}
Y' &= 0.3 \, R' + 0.59 \, G' + 0.11 \, B' \\
R'-Y' &= 0.7 \, R' - 0.59 \, G' - 0.11 \, B' \\
B'-Y' &= -0.3 \, R' - 0.59 \, G' - 0.89 \, B' \\
\end{split}
\end{align}
Ezek tehát az alapvető videójel-komponensek, amelyek végül ténylegesen tárolásra, tömörítésre, továbbításra (pl. műsorszórás) kerülnek.
\item Megjelenítő oldalon a fenti videójelekből a megfelelő inverz-mátrixolással az $R', G', B'$ jelek visszaszámíthatóak.
Megjelenítés során a megjelenítő gamma-torzításának hatására a kameraoldalon mért RGB komponensekkel lineárisan arányos fénysűrűségű RGB pixelek jelennek meg a kijelzőn.
\end{itemize}
Az így létrehozott $Y', R'-Y', B'-Y'$ jelek kitüntetett szereppel bírnak a videótechnikában.
Az eddigieket összegezve: ezek adják meg egy színes képpont ábrázolásának módját.
A komponensek neve:
\begin{itemize}
\item $Y'$: \textbf{luma jel}
\item $R'-Y'$, $B'-Y'$: \textbf{chroma jel}.
\end{itemize}

\paragraph{A luma és chroma jelek fizikai tartalma:\\}
Fontos észrevenni, hogy a luma jel nem egyszerűen a gamma-korrigált relatív fénysűrűség, hanem a gamma-korrigált RGB jelekből az eredeti $Y$ együtthatókkal számított videójel, azaz
\begin{equation}
Y' = 0.3R^{\frac{1}{\gamma}} + 0.11G^{\frac{1}{\gamma}} + 0.59B^{\frac{1}{\gamma}} \neq Y^{\frac{1}{\gamma}} = \left( 0.3R + 0.59G + 0.11B\right)^{\frac{1}{\gamma}}
\end{equation}
\begin{figure}[]
	\centering
	\begin{overpic}[width = 0.32\columnwidth ]{figures_en/Video_colorspaces/luma_chroma_0_11.png}
\small
\put(0,0){(a)}
	\end{overpic}
	\begin{overpic}[width = 0.32\columnwidth ]{figures_en/Video_colorspaces/luma_chroma_0_30.png}
\small
\put(0,0){(b)}
	\end{overpic}
	\begin{overpic}[width = 0.32\columnwidth ]{figures_en/Video_colorspaces/luma_chroma_0_59.png}
\small
\put(0,0){(c)}
	\end{overpic}
	\caption{A chroma térben ábrázolható színek halmaza fix $Y'$ értékek mellett vizsgálva.}
	\label{Fig:luma_chroma_space}  
\end{figure}
A luma jel fizikai tartalma emiatt nehezen kezelhető: 
Legszorosabban az adott színpont világosságával függ össze, fehér szín speciális esetén pl. ahol $R=G=B=Y_W$
\begin{equation}
Y'_W = \left( 0.3 + 0.59 +0.11 \right)Y_W^{\frac{1}{\gamma}} = Y_W^{\frac{1}{\gamma}}
\end{equation}
az egyenlőtlenség egyenlőségbe megy át, azaz a luma megegyezik a gamma-korrigált világosságjellel.
Általánosan azonban a luma jel színinformációt is hordoz magában.
Hasonlóan, a chroma jelek nem szimplán a gamma-korrigált színkülönbségi jelek (de hasonlóan, fehér esetében azonosan nullák), és így világosságinformációt is hordoznak magukban.
	
Adott luma értékek mellett az ábrázolható színek halmaza a \ref{Fig:luma_chroma_space} ábrán látható.
Megfigyelhető, hogy a luminance-chrominance térrel azonosan az ábrázolható színek egy hatszöget feszítenek ki, és a 100\%-osan telített színek helye nem változik (hiszen a 0 és 1 értékeken nem változtat a gamma-korrekció), ennek megfelelően az egyes pontok színezete a chroma térben változatlan.
Az ábrákon azonban egyértelműen látható, hogy adott $Y'$ értékek mellett is az ábrázolt színek világossága változik, tehát a chroma jelek világosságinformációt is tartalmaznak.
Látható, hogy a gamma-torzítás hatására---ahogy \ref{Fig:gamma} ábrán is megfigyelhető---adott $Y'$ mellett a telítetlen (fehérhez közeli) színek sötétebbé válnak, míg a telítettebb színek még telítettebbé válnak. 

%Ez az eddig elmondottak alapján nem kell, hogy problémát okozzon, hiszen pusztán annyit jelent, hogy a világosság és színinformációt nem teljesen szeparáltan kezeljük átvitel tárolás és átvitel során.
%Ugyanakkor látni fogjuk, hogy az emberi látás tulajdonságait kihasználva a színjeleket---azaz a chroma jeleket---csökkentett sávszélességgel, vagy digitális esetben kisebb felbontással továbbítjuk.
%Minthogy a fentiek alapján így kis részben a világosságjel sávszélessége/felbontása is csökken, amelynek már látható hatása lehet a megjelenített képen.-

\section{The \ypbpr color space}

A luma és chroma központi szerepet játszanak videótechnikában, a leggyakrabban ezek a jelek a színes képpont ábrázolásának alapja mind komponens, mind kompozit (több komponens kombinációjaként létrehozott videó) formátumok esetén.
Utóbbi formátum létrehozásával a következő fejezet foglalkozik részletesen.
Analóg, komponens videótechnikában egy színes képpont luma-chroma térben való leírását az \textbf{\ypbpr színtérben} való ábrázolásnak nevezzük (az ezekből képzett \ypbpr videójeleket a következő fejezet részletezi).

Az \ypbpr színtér $Y'$ jele maga a luma komponens, míg a $P'_{\mathrm{B}}, P'_{\mathrm{R}}$ jelek szimplán az átskálázott chroma komponensek, a skálafaktort úgy megválasztva, hogy dinamikatartományuk $\pm 0.5$ legyen.

Jelölje az adott RGB színtérben a relatív fénysűrűség együtthatóit $k_r, k_g$ és $k_b$.
Minthogy az $R'-Y$ és $B'-Y'$ komponensek dinamikatartományra rendre $1 - k_r$ és $1 - k_b$, ezért általános az \ypbpr jelek az luma-chroma jelekből a
\begin{align}
\begin{split}
Y' &= k_r \, R' + k_g \, G' + k_b \, B' ,\\
P_R &= k_1 \, \left( R' - Y' \right) = \frac{1}{2} \frac{1}{1 - k_r} \, \left( R' - Y' \right)\\
P_B &=  k_2 \, \left( B' - Y' \right) = \frac{1}{2} \frac{1}{1 - k_b} \, \left( B' - Y' \right)
\end{split}
\end{align}
összefüggés alapján számítható.
Az egyenletekben $R', G', B'  \in \lbrace 0, 1 \rbrace$ a Gamma-korrigált színkoordinátái az ábrázolt színpontnak adott eszközfüggő

Hasonlóan meghatározhatjuk általános $R',G',B'$ komponensekre az \ypbpr jelek kiszámításához szükséges transzformációs mátrixot
\begin{align}
\begin{bmatrix}[c]
       Y' \\[0.3em]
       P_{\mathrm{B}} \\[0.3em]
       P_{\mathrm{R}} \end{bmatrix}
       =& 
  \begin{bmatrix}[c c c]
   k_r & k_g & k_b  \\
   -\frac{1}{2}\frac{k_r}{1-k_b} & -\frac{1}{2}\frac{k_g}{1-k_b} & \frac{1}{2} \\
   \frac{1}{2}& -\frac{1}{2}\frac{k_g}{1-k_r} & -\frac{1}{2}\frac{k_b}{1-k_r} \\
\end{bmatrix}
\cdot
\begin{bmatrix}[c]
       R' \\[0.3em]
       G' \\[0.3em]
       B' \end{bmatrix},
\end{align}
míg az inverz-transzformációt 
\begin{align}
\begin{bmatrix}[c]
       R' \\[0.3em]
       G' \\[0.3em]
       B' \end{bmatrix}
       =& 
  \begin{bmatrix}[c c c]
   1 & 0 & 2 - 2 \cdot k_r  \\
   1 & -\frac{k_b}{k_g} \cdot (2-2k_b) &  -\frac{k_r}{k_g} \cdot (2-2k_r)  \\
   1 & 2 - 2 \cdot k_b & 0 \\
\end{bmatrix}
\cdot       \begin{bmatrix}[c]
       Y' \\[0.3em]
       P_{\mathrm{B}} \\[0.3em]
       P_{\mathrm{R}} \end{bmatrix}
\end{align}
írja le.

\begin{figure}[]
	\centering
	\begin{minipage}[c]{0.6\textwidth}
	\begin{overpic}[width = 1\columnwidth ]{figures_en/Video_colorspaces/YPbPr.png}
	\end{overpic} \end{minipage}\hfill
	\begin{minipage}[c]{0.4\textwidth}
	\caption{Az \ypbpr térben ábrázolható színek gamutja.}
	\label{Fig:ypbpr_gamut}  \end{minipage}
\end{figure}
Egyszerű példaként a HD szabvány színterében 
\begin{equation}
k_r = 0.2126, \hspace{7mm}
k_g = 0.7152, \hspace{7mm}
k_b = 0.0722
\end{equation}
Az adott alapszínek mellett az ábrázolható színek tartománya a \ref{Fig:ypbpr_gamut} ábrán látható.

\section{Digital representation of color information}

So far, the current chapter has introduced the color representation of video technologies by assuming continuous RGB, luma and chroma values.
The digital representation of color pixels can be obtained by the direct digitization of the $R'G'B'$, or more often the \ypbpr components.
The digital representation of the \ypbpr signals have its own terminology: it is termed as the \ycbcr color space \footnote{
The \ycbcr signals are sometimes incorrectly referred to as $Y'U'VI$ signals (e.g. in VLC player), which term was originally used for the components of the PAL composite video format.}.

The \ycbcr digital color space can be obtained by the quantization of the \ypbpr components, with representing the originally continuous values at discrete levels.
It is therefore obvious that \ycbcr is a device dependent representation, depending on the RGB primaries and its gamut coincides with the gamut of the color gamut of the \ypbpr color space, depicted in Figure \ref{Fig:ypbpr_gamut} (with of course only discrete number of the reproducible colors due to digitization).
In the following the current chapter deals with the questions arising at the quantization of the \ypbpr color space.

\subsection{Perceptual quantization and bit depth}
First the optimal quantizer transfer characteristics is investigated, in order to achieve bit-efficient digital representation.
As a result, the real role of gamma correction is highlighted.

For the sake of simplicity first it is assumed that the signal-to-quantize is the $Y$ component, i.e. the linear relative luminance signal (without gamma correction).
The starting point for defining an appropriate quantizer transfer characteristics is given by the perceptual properties of the human visual system:
As a rule of thumb it can be stated that in case of image reproduction, the HVS can not discern luminance levels below $1~\%$ of the maximal luminance on the given scene.
Loosely speaking relative luminances below $\frac{1}{100}$ just appear black for the human observer, therefore, the dynamic range of luminance levels to be reproduced is 100:1.

Within this dynamic range the lightness perception of the HVS is approximately logarithmic function of luminance with the contrast sensitivity being $1~\%$.
This means that two luminance levels can be distinguished only if their relative difference is larger than 1.01.
Later the relative luminance-percepted lightness characteristics ($L(Y)$) was given more accurately by the CIE $L^*$ function, describing the lightness as the power function of luminance with the exponent being approximately 0.4.

These properties of human vision establishes the following requirements for quantizing the luminance signal without visible quantization noise:
\begin{itemize}
\item The ratio of the largest and the smallest quantized luminance levels should be at least 100:1
\item The ratio of the adjacent quantized luminance levels should be at most 1.01, i.e. their relative difference should be less than or equal to $1~\%$
\end{itemize}

In the following, as a counterexample for the appropriate quantization strategy the problem with linear quantization is investigated.
In this case the digital signal levels are assigned to the relative luminance levels within the dynamic range of $Y \in \lbrace Y_0, 100 Y_0 \rbrace$ linearly. 
The quantization can be, therefore, performed by simple rounding to the nearest integer.
In case of representing the digital samples at $N$ bits the mapping is given by 
\begin{equation}
q =  \nint{ \left( 2^N - 1 \right) \cdot  \frac{Y - Y_0 }{Y_1 - Y_0 } },
\end{equation}
where $\nint{}$ is the rounding operation, and $Y_1 = 100Y_0$ is the maximal quantized luminance value.
Similarly, the inverse mapping, i.e. the luminance levels of the $q$-th digital code is given by
\begin{equation}
Y^q = q \cdot \frac{Y_1 - Y_0}{2^N - 1} + Y_0.
\end{equation}

\begin{figure}[]
	\centering
	\begin{overpic}[width = 1\columnwidth ]{figures_en/linear_vs_perc_quant.png}
	\end{overpic}
	\caption{
Relative difference of adjacent digital codes in case of linear (a) and perceptual (b) quantization.}
	\label{Fig:linear_vs_perc_quant}
\end{figure}

The relative difference of the adjacent digital codes is then given as
\begin{equation}
\frac{Y^{q+1}-Y^q}{Y^q} = \frac{1 }{q + \frac{2^N - 1}{99}}.
\label{Eq:rel_dif}
\end{equation}
As an example of quantization with $N = 8$ bits, the relative difference between the 101-st and 100-th code (with $q = 100$) the relative difference is
\begin{equation*}
\frac{Y^{101}-Y^{100}}{Y^{100}} \approx 0.01 = 1~\%,
\end{equation*}
meaning that the luminance levels of the adjacent codes are just noticeable.
For smaller, or larger codes (e.g. 20 and 21, or 200 and 201) the relative difference is given by
\begin{equation*}
\frac{Y^{21}-Y^{20}}{Y^{20}} \approx 0.05 = 5~\%, \hspace{1cm} \frac{Y^{201}-Y^{200}}{Y^{200}} \approx 0.005 = 0.5~\%.
\label{eq:code_100}
\end{equation*}
Obviously, the luminance levels of codes under 100 are easily distinguishable, meaning that quantization noise at these code levels is clearly visible.
As a consequence, in case of the linear quantization of the luminance signal for dark shades the boundary of the different quantization levels would be easily noticeable, leading to so-called banding artifact.

Straightforwardly, by increasing the bit depth ($N$) banding could be avoided:
It is clear that the largest relative difference is between codes 0 and 1.
%
\begin{figure}[]
	\centering
	\begin{overpic}[width = 0.8\columnwidth ]{figures_en/linear_vs_perc_quant_2.png}
	\end{overpic}
	\caption{Quantized luminance ramp by applying linear (a) and perceptual (b) quantization with $N=7$ bits.
	In case of linear quantization the just noticeable quantization is found at $q \approx 99$ (based on $1 / q + \frac{2^7 - 1}{99} = 0.01$).}
	\label{Fig:linear_vs_perc_quant_2}
\end{figure}
%
By setting $q = 0$ the smallest bit depth for which \eqref{Eq:rel_dif} is larger than $1~\%$, according to
\begin{equation}
\frac{99}{2^N - 1} \leq 0.01 \hspace{1cm} N \geq 13.27 
\end{equation}
is given by $N = 14~\mathrm{bits}$.
This means that linear quantization ensures unnoticeable qunatization noise by applying the bit depth of 14\footnote{
As a consequence digital cameras often digitize the pixel levels by using 14 bits linear quantization, which is requantized to the final bit depth after digital gamma correction.}.
On the other hand \eqref{eq:code_100} reflects that codes above 100 have decreasing perceptual utility: luminance at these regimes is quantized with ineffectively fine resolution.

\vspace{3mm}
A straightforward strategy in order to avoid the problem with linear quantization would be to quantize the percepted lightness ($L$) instead of the luminance information, resulting in uniform perceptual difference between adjacent codes.
By employing the lightness definition of the CIE ($L^*$) this \textbf{perceptual quantization} can be achieved by the distortion of the relative luminance with the power function of 0.4 before quantization.

The quantization mapping and the inverse mapping in this perceptual case are given by
\begin{equation}
q =  \nint{ \left( 2^N - 1 \right) \cdot  \frac{Y^{0.4} - Y_0^{0.4} }{Y_1^{0.4} - Y_0^{0.4}} }
\hspace{1cm}
Y^q = \left( q \cdot \frac{Y_1^{0.4} - Y_0^{0.4}}{\left( 2^N - 1 \right) } + Y_0^{0.4} \right)^{\frac{1}{0.4}} .
\end{equation}
Based on these formulae the relative difference of adjacent codes can be expressed for perceptual quantization.
The result is depicted in Figure \ref{Fig:linear_vs_perc_quant} (b).
It is verified that the relative difference is approximately constant over the entire dynamic range, and even in case of $N = 10$ quantization, it is only slightly higher than $1~\%$.

As a consequence, in the field of video technologies in studio standards the luminance is quantized perceptually, representing the luminance in 10 bits, while in most consumer electronics (e.g. JPEG image compression, MPEG video compression and video broadcasting) representation with $N=8$ is satisfactory\footnote{
As a third option logarithmic quantization could be performed by setting the ratio of the luminance levels of the adjacent codes to 1.01.
In this case according to $1.01^q \geq 100$ the number of required codes is $q = 463$, which can be represented in $N = 9$ bits.
Due to historical reasons (due to gamma correction) the presented, power function-based perceptual quantization was introduced in the video standards.}.
Therefore, SD and HD studio standards (Recommendations ITU-601 and ITU-709) include the digital representation applying $N = 8$ or $N = 10$ bits, with a rigorous definition of the implementation of quantization and the non-linear pre-distortion curve.
This curve is investigated in thew following section in details.

\subsection{Gamma correction: goal and implementation}
The previous section introduced the basic principle of perceptual quantization: pre-distortioning the luminance values by a power function with the exponent being approximately 0.4 the quantization noise can be uniformly distributed over the entire dynamic range.
As a result the visible banding of dark shades can be avoided, as it is illustrated in Figure \ref{Fig:linear_vs_perc_quant_2}.
In the following the actual implementation of perceptual quantization within the image/video processing chain is discussed.

The optimal signal processing scheme would be the following, as shown on Figure \ref{Fig:gamma_flow} (a):
At the source of the RGB signals perceptual quantization is achieved by the direct quantization of the perceived lightness ($L^* \sim Y^{0.4}$), obtained from the luminance $Y$, that can be calculated as the linear combination of the RGB coordinates.
At the receiver (e.g. display, TV receiver) following D/A conversion the original luminance value is regained by the inverse distorting the perceived lightness.
Finally, from the luminance inforamtion the RGB values to be displayed are obtained by the corresponding inverse transformation.

Note that so far linear RGB values and luminance levels were assumed, without taking the gamma correction into consideration:
As a consequence, although perceptual quantization could be achieved, the non-linear transfer characteristics of the CRT display would still result in distorted dynamics of the displayed image.
The required neutralization of the CRT distortion would introduce a further non-linear transfer function---as depicted in Figure \ref{Fig:gamma_flow} (b)---overcomplicating the signal processing chain (as well as making it more expensive).

However, as a lucky coincidence the luminance-lightness characteristics of the human vision exactly coincides with the required CRT gamma compensation function, both described by $\sim x^{0.4}$, allowing the simplification of the signal processing.
As an engineering approximation, both at the source and the receiver side the order of the non-linear mapping and the linear transformation $P: RGB \rightarrow Y, R-Y, B-Y$ is interchanged.
This interchange of operations will have two consequences:

%
\begin{figure}[]
	\centering
	\begin{overpic}[width = 1\columnwidth ]{figures_en/gamma_flow_1.png}
	\small	
	\put(0,0){(a)}
	 \vspace{5mm}
 	\end{overpic}
	\begin{overpic}[width = 1\columnwidth ]{figures_en/gamma_flow_2.png}
	\small	
	\put(0,0){(b)}
	\end{overpic} \vspace{5mm}
	\begin{overpic}[width = 0.98\columnwidth ]{figures_en/gamma_flow_3.png}
	\small	
	\put(0,0){(c)}
	\end{overpic}
	\caption{Signal processing scheme of gamma correction:
	Figure (a) realizes true perceptual quantization, however, with the CRT gamma distortion still leading to visible artifacts in the reproduced image.
	Figure (b) shows a possible, but over-complicated solution for this problem.
	As an engineering approximation the order of linear and non-linear operations are interchanged on Figure (c), giving a mathematically imprecise solution, but allowing the considerable simplification of the block diagram.}
	\label{Fig:gamma_flow}
\end{figure}
%
\begin{itemize}
\item By leaving the principle of strict perceptual quantization, on the source side the quantized signal is not the perceived lightness $L^* = Y^{0.4}$, but the \textbf{luma signal} $Y'$, obtained from the gamma corrected color-coordinates, $R^{0.4},G^{0.4},B^{0.4}$.
As it was already declared, for white shades ($R=G=B$) the luma signal is the gamma corrected luminance signal, i.e. $Y_W^{0.4} = Y_W' = L_W^*$ holds, while for other colors the luma signal carries color information as well.
Therefore, the presented signal processing chain realizes perceptual quantization of white shades, and only approximates it for any other color.
%
\item On the receiver side the CRT correction function exactly neautralizes the inverse quantization function $L^* \rightarrow Y$, resulting in linear resultant transfer characteristics.
\end{itemize}
The signal processing chain obtained consists only a single non-linear transfer block, with the entire system resulting in the gaama correction technology, discussed already in the previous sections.

Hence, the actual present role of gamma correction is highlighted:
Although the role of CRT imaging systems has been almost entirely superseded by LCD and LED based displays, gamma correction is applied in the video processing chain in a completely unchanged manner.
However, its real role nowadays is \textbf{not} the CRT transfer compensation, but it achieves and approximation of perceptual quantization, adapting the quantization characteristics to the properties of human vision.


\vspace{3mm}
Obviously, nowadays the implementation of non-linear transfer functions is computationally inexpensive.
Similarly to CRTs, the currently used displays also exhibit highly non-linear driving voltage-display luminance characteristics, which has to be compensated prior to reproducing the input image.
Therefore, in current display first the gamma distortion (due to perceptual quantization) has to be neutralized, afterwards the actual display characteristics has to be compensated.
These non-linear operations are usually implemented before D/A conversion, based on lookup tables (LUTs).

\vspace{3mm}
The actual form of gamma correction that has to be implemented before A/D conversion is rigorously codified in SD and HD recommendations.
Within these standards the non-linear distortion is referred to as the \textbf{opto-electronic transfer function (OETF)}.
The actual choice of the OETF is depends on two aspects:
\begin{itemize}
\item to achieve approximately perceptual quantization
\item to compensate the effects of the viewing environment
\end{itemize}
On the other hand the standards also codify the non-linear transfer function at the receiver side, which neutralizes the effect of OETF.
This receiver side non-linear compensation is termed as the \textbf{electro-optical transfer function (EOTF)}.

\paragraph{Compensation of the viewing environment:\\}
So far it was inherently assumed that the exponent of the gamma correction curve is 0.4.
However, in actual reproduction systems the exponent of the standardized OETF is usually a higher value, with the actual choice depending on the supposed average illumination level in the reproduction environment.

It was illustrated in Figure \ref{Fig:gamma} that the non-linear distortion of the RGB components with the exponent being larger than 1 will result in the compression of deep shades (increase in contrast) and in the saturation of colors.
This fact allows the compensation of loss of contrast and saturation due to dim viewing environments:
According to the Stevens (Bartleson-Breneman) and Hunt effects in a dark viewing environment the ability of discerning dark shades, the perceived contrast of the image and the perceived saturation of colors decreases.
%
\begin{figure}[]
	\centering
	\begin{overpic}[width = 1\columnwidth ]{figures_en/stevens.png}
	\end{overpic}
	\caption{Illustration of the Bartleson-Breneman effect.}
	\label{Fig:stevens_effect}
\end{figure}

The Bartleson-Breneman effect is illustrated in Figure \ref{Fig:stevens_effect}.
It is verified that within a displayed image, the image contrast  increases with the luminance of surround lighting:
\begin{itemize}
\item with bright surround luminance the image contrast (the perceived ratio of the brightest and darkest shades) is larger, than in case of a dark background.
\item also the entire luminance-lightness characteristic curve changes depending on the surround luminance: in case of bright surround the perceived contrast between dark shades increases, while bright shades can be distinguished less efficiently, and vice versa.
\end{itemize}
This means that instead of the well-known $L^* \sim Y^{0.4}$ characteristics, the exponent slightly increases for bright surround luminance levels, and decreases in dark backgrounds.

As a consequence if the environment in image reproduction is dark (e.g. a cinema), in order to avoid the loss of contrast and saturation the overall non-linear transfer function of the signal processing chain should have an exponent of about 1.2-1.5 instead of the linear transfer.
This can be achieved with choosing the source gamma correction, i.e. the OETF to be higher than the original value of 0.4, which was originally chosen to compensate the effect of CRT distortion.

Based on these considerations as an example, the ITU-709 HDTV recommendation defines the following OETF
\begin{equation}
E = 
\begin{cases}
4.500 L, \hspace{20mm} \mathrm{ha}\, L < 0.018 \\
1.099 L^{0.45} - 0.099, \hspace{3mm} \mathrm{ha}\, L \geq 0.018,
\end{cases}
\end{equation}
where $L \in \{ R, G, B \}$.
The entire curve consists of a power function and a linear segment.
This linear segment is required in order to avoid the infinite slope of the power function around the origin, that would result in infinite gain of noise around the black level.
The entire curve can be well-approximated with a power function of 0.5 (i.e. a square root function), as depicted in \ref{Fig:itu709}.

\begin{figure}[]
	\centering
	\begin{overpic}[width = 0.7\columnwidth ]{figures_en/itu709.png}
	\end{overpic}
	\caption{The opto-electronic transfer function of the ITU-709 standard and the NTSC standard.}
	\label{Fig:itu709}
\end{figure}
The HD standard assumes the receiver gamma distortion to be $\gamma_D \approx 2.5$ (i.e. the EOTF is assumed to be a power function with the exponent being 0.4).
Along with the prescribed gamma correction this results in a resultant transfer function of $0.5 \cdot 2.5 = 1.25$, which ensures the desired contrast and saturation in an average living room at daylight.

As an other example: the current most widespread digital cinema standard, called the DCI-P3, with the correctly chosen OETF and EOTF achieves a resultant non-linear transfer with the exponent being 1.5.
\footnote{
Actually, in case of e.g. the HD standard, each element of the production and reproduction chain is rigorously defined.
The content is produced in such a manner that it would be reproduced with the desired aesthetic properties in a standardized viewing environment (defined by ITU-R BT.2035) displayed on a standardized display apparatus (standardized by ITU-R BT.1886).}.
In practice, in case of current computer and TV displays the actual value of the EOTF, i.e. the display gamma can be freely adjusted in order to achieve the desired contrast.

\subsection{Dynamic range of \ycbcr representation}

In the foregoing it was presented how the gamma correction allows the representation of SD and HD image content as low as 8 (for consumer) or 10 (for studio and professional processing) bits.
It seems to be straightforward to utilize the entire possible dynamic range in order to represent the video data, e.g. to let the $Y'$ luma data to cover the $\lbrace 0, 255 \rbrace$ code range in case of 8 bit representation.
This so called \textbf{full range} approach is however only applied in the JPEG encoder, and several image editor softwares, operating directly in the RGB color space.

%https://books.google.hu/books?id=hOu5DQAAQBAJ&pg=PA427&lpg=PA427&dq=RGB+headroom+footroom&source=bl&ots=NsT6C3TiLr&sig=ACfU3U1HLa7oM0fxBZLHjs6PFfuyi2kflg&hl=en&sa=X&ved=2ahUKEwjSlf-Uw-PoAhWkw4sKHebNCl0Q6AEwC3oECA0QLw#v=onepage&q=RGB%20headroom%20footroom&f=false 

\begin{figure}[]
	\centering
	\begin{overpic}[width = 0.7\columnwidth ]{figures_en/ycbcr_dyn_range.png}
	\end{overpic}
	\caption{The dynamic range of the \ycbcr codes.}
	\label{Fig:ycbcr_dyn_range}
\end{figure}
In the field of video technologies in case of \ycbcr representation the image/video data is usually stored and transmitted with a \textbf{narrow range} approach.
In this case the valid video data is allowed to fill only the part of the available digital dynamic range:
In high-quality video, it is necessary to preserve transient signal undershoots below black, and overshoots above white, that are liable to result from processing by digital and analog filters without clipping the video data. 
Studio video standards provide \textbf{footroom} below reference black, and \textbf{headroom} above reference white. 
This code ranges are only containing video data during video processing, and their content is discarded during video storing and transmission.

For the case of 8 bit representation the luma ($Y'$) and (in case of RGB representation) and the $R',G',B'$ components have a headroom of 15 and a footroom of 19 codes, thus the dynamic range of the components is $16 \leq Y' \leq 235$ (codes 0 and 255 are reserved for synchronization in digital interfaces).
The asymmetry of the headroom and footroom has no important reason.

For the $C'_\mathrm{B}$ and $C'_\mathrm{R}$ chroma components the zero level is the middle of the dynamic range, (i.e. code 128 digitally), while the headroom and footroom are symmetrically 15 codes, i.e. $16 \leq C'_\mathrm{B}, C'_\mathrm{R} \leq 240$ holds.

For higher bit depths the width of headroom and footroom increases proportionally.
As a summary the \ycbcr digital levels can be obtained from the \ypbpr analog signal levels according to
\begin{equation}
\begin{bmatrix}[c]
       Y' \\[0.3em]
       C'_{\mathrm{B}} \\[0.3em]
       C'_{\mathrm{R}} \end{bmatrix}
       =
D\cdot
\begin{bmatrix}[c]
       16 \\[0.3em]
       128 \\[0.3em]
       128 \end{bmatrix}
+
D\cdot
\begin{bmatrix}[c]
       219 Y' \\[0.3em]
       224 P'_\mathrm{B} \\[0.3em]
       224 P'_{\mathrm{R}} \end{bmatrix},
\end{equation}
where $D = 2^{N-8}$ holds, with $N$ denoting the bit depth.

\subsection{Chroma subsampling}
%
The luma-chroma representation of image and video content has two great advantages over the direct RGB representation:
On one the transmission of the separated luminance information and the color information allowed the color TV transmission system to be fully backward compatible with the then-existing black-and-white TV receivers, as it is discussed in the next chapter.
On the other hand it is more adapted to the properties of human vision\footnote{
The conversion of light to stimulus is ensured by the frequency-selective photoreceptors of the retina, with the three types of cone cells being sensitive to mainly red, green and blue lights.
However, the transmission of the stimulus towards the brain on three types of optic nerves, one carrying black and white and two carrying color information.}: 
The perceptual spatial resolution of the HVS (i.e. the visual acuity) is much lower for spatial change of color information than for that of the luminance.
Therefore, the separate transmission of color information allows the bandwidth reduction of the chroma signals, i.e. their transmission with lower spatial resolution.
In case of digital representation the reduction of chroma resolution is performed by \textbf{chroma subsampling}.

\begin{figure}[]
	\centering
	\begin{overpic}[width = 1\columnwidth ]{figures_en/umbrella.png}
 	\end{overpic}
	\caption{The content of the \ycbcr components for a simple test image.}
	\label{Fig:umbrella}
\end{figure}

\subsection*{Notation of subsampling schemes}
The content of the \ycbcr components in case of a simple test image is illustrated in Figure \ref{Fig:umbrella}.
Clearly, the color information carries minor high frequency details---and even where details are present they are hardly noticeable for the human vision---thus the transmission of color information with lower resolution is satisfactory.
The resolution of chroma components can be decreased both in the horizontal and vertical dimensions, usually to the half, or quarter of the luma resolution, therefore, different types of chroma subsampling schemes can be implemented.
The rate of decimation along the horizontal/vertical dimension are usually denoted by using the following notation:
\begin{equation}
J : a : b : \alpha,
\end{equation}
where the digits denote the following properties:
\begin{itemize}
\item $J$: the first digit stands for the horizontal sampling reference of the luma component.
Originally (in the NTSC and SD systems) the first digit indicated the sampling frequency of the luma signal a the multiple of the color subcarrier frequency, via $f^{Y'}_s = J \cdot 3\,\frac{3}{8}~\mathrm{MHz}$.
In case of HD formats based on this interpretation often $J=22$ and higher values should be used, so for the sake of simplicity the first digit is usually fixed to the value 4, serving as a reference number for the following digits.
\item $a$: is the $C_{\mathrm{B}}$ and $C_{\mathrm{R}}$ horizontal decimation factor, given relatively to the first digit.
more informally it gives  the number of chroma samples in the first row of $J$ pixels.
As an example: $J=a = 4:2$ means that the horizontal resolution of the color information is half of the luma resolution.
\item $b$: is the number of changes of chroma samples between first and second row of $J$ pixels.
If $b=a$, then no vertical subsampling of the chroma samples is performed.
If $b=0$, then the vertical resolution of chroma is half of that of the luma samples.
\item $\alpha$: indicates the presence of alpha channel (e.g. chroma keying).
May be omitted, if alpha component is not present, and is equal to $J$ when present.
\end{itemize}
\begin{figure}[]
	\centering
	\begin{overpic}[width = 1\columnwidth ]{figures_en/chroma_subsampling.png}
 	\end{overpic}
	\caption{Illustration of frequently used chroma subsampling schemes.}
	\label{Fig:chroma_subsampling}
\end{figure}

The chroma subsampling process can be interpreted as a simple lossy compression technique, allowing to reduce the amount of video or image data during storing or transmission.
Before reproducing the original RGB signals on the display side, obviously, the discarded chroma samples has to be reconstructed by some interpolation technique.

\subsubsection*{Frequently used subsampling schemes}

The most common chroma subsampling schemes are summarized in Figure \ref{Fig:chroma_subsampling}:
\begin{itemize}
\item \textbf{4:4:4}: In this case no subsampling is performed neither in the horizontal, nor in the vertical dimension.
The spatial resolution of the chroma data is the same as the luma's.
If no subsampling is applied, the direct $R'G'B'$ representation of color pixels is often used instead of the \ycbcr space.
The scheme is, however, scarcely applied in consumer electronics, but mainly used in the field of film archivation, CGI and movie production.
The first video studio-standard to support 4:4:4 sampling was the UHD format, published in the ITU-2020 recommendation.
The representation of one pixel requires $3 \cdot 8 ~\mathrm{bit} = 24~\mathrm{bits}$ in case of the bit depth of 8.
%
\item \textbf{4:2:2}: The horizontal resolution of the chroma samples is the half of the luma resolution, while in the vertical dimension no subsampling is performed.
The chroma samples are \textbf{cosited} with every second luma samples horizontally, meaning that the horizontal center of the luma samples coincide with the center of every second luma samples.
4:2:2 is the default chroma sampling scheme of SD, HD and UHD studio standards, with also many high-end digital video formats and interfaces using this scheme.
Since two pixels consists of two luma, one $C_{\mathrm{B}}$ and one $C_{\mathrm{R}}$ sample, therefore the representation of one pixel requires$(\frac{2 + 1 + 1}{2})\cdot 8~\mathrm{bit} = 16~\mathrm{bit}$, meaning that the compression factor from 4:4:4 to 4:2:2 is $\frac{2}{3}$.
%
\item \textbf{4:1:1}: The horizontal resolution of chroma is quarter of that of the luma samples, and in the vertical dimension no subsampling is performed.
Originally it was the default sampling scheme of low-end consumer digital electronics, e.g. handheld DVCAMs, but nowadays it is rarely used.
Since every 4 pixels contains 4 luma and 1-1 chroma samples, therefore the representation of one pixel requires $( \frac{4 + 1 + 1}{4})\cdot 8~\mathrm{bit} = 12~\mathrm{bits}$ and the compression factor from 4:4:4 to 4:1:1 is $\frac{1}{2}$.
%
\item \textbf{4:2:0}: Both the horizontal and the vertical resolution of the chroma samples are half of the luma resolution.
This is the most commonly used subsampling scheme, used for digital video storing, local playback and broadcasting.
Similarly to 4:1:1, 4 pixels contain 4 luma and 1-1 chroma samples, therefore the representation of one pixel requires $( \frac{4 + 1 + 1}{4})\cdot 8~\mathrm{bit} = 12~\mathrm{bits}$ and the compression factor from 4:4:4 to 4:2:0 is $\frac{1}{2}$.

There are two main variants of 4:2:0 schemes, having different horizontal siting (different position of the subsampled chroma samples)
\begin{itemize}
\item In JPEG, H.261, and MPEG-1, $C_{\mathrm{B}}$ and $C_{\mathrm{R}}$ are sited \textbf{interstitially}, halfway between alternate luma samples.
\item In MPEG-2, $C_{\mathrm{B}}$ and $C_{\mathrm{R}}$ are \textbf{cosited} horizontally and are sited between pixels in the vertical direction (interstitially).
\end{itemize}
\end{itemize}
The question may arise, how the siting of the chroma samples can be interpreted.
In order to answer this the basic steps of chroma subsampling process has to be investigated in more details.


\subsubsection*{Signal processing questions of chroma subsampling}

The practical realization of the chroma subsampling concept requires two basic digital signal processing steps:
\begin{itemize}
\item on the source side (e.g. camera) the chroma samples---obtained from the $R'G'B'$ components---has to be sampled with a reduced sampling frequency. 
Digitally speaking, if the digital chroma samples are directly available then the chroma samples has to be \textbf{decimated} (resampled with reduced sampling frequency) according to the current subsampling scheme.
The subsampled chroma samples can be stored, transmitted between digital equipment, or broadcasted. 
\item On the receiver side (e.g. at a display) the $R'G'B'$ components has to be calculated from the luma-chroma representation.
In order to do so the discarded chroma samples has to be approximated based on the received chroma data, i.e. the missing chroma samples have to be \textbf{interpolated} (resampled with increased sampling frequency) .
\end{itemize}

\paragraph{Decimation of the chroma samples:}
As a well-known fact, the frequency content of a discrete signal contains the spectrum of the underlying continuous signal, repeating on the multiples of the sampling frequency.
Therefore if the bandwidth of the underlying continous signal is larger than the half of the sampling frequency (the so-called \textbf{Nyquist frequency}), after discretization the repeating spectra will overlap, leading to \textbf{aliasing} artifacts, and the original continuous signal can not be reproduced from its discrete samples.
In order to avoid the spectral overlaping, the signal has to be bandlimited to the half of the sampling frequency with an \textbf{antialiasing filter}, being a low pass filter with the cut-off frequency being the Nyquist frequency.

In the field of image reproduction appropriate antialiasing filtering is crucial: 
aliasing manifests in clearly visible low-frequency patterns on the sampled image.
As these aliasing patterns, termed as \textbf{Moiré patterns} are the result of the spatial sampling of continuous images the type of aliasing is termed as \textbf{spatial aliasing}.
\begin{figure}[]
	\centering
	\begin{overpic}[width = 0.8\columnwidth ]{figures_en/decimation.png}
 	\end{overpic}
	\caption{Signal processing chain of decimation (subsampling): 
	(1) is the input signal and $F(1)$ the input spectrum.
	(2) is the output of the antialiasing filter and $F(2)$ is its spectrum.
	(3) is the output signal.}
	\label{Fig:decimation}
\end{figure}

Generally speaking, decimation is the reduction of the sampling frequency of an arbitrary discrete input signal.
The new sampling frequency is lower than the initial one, and after decimation the spectrum of the input signal will repeat on the multiples of the new sampling frequency.
Therefore, prior to setting the new sampling frequency by discarding e.g. every second input samples, the original input signal has to be antialiasing filtered below the half of the new sampling frequency.
The process is illustrated in Figure \ref{Fig:decimation}.

\vspace{3mm}
Subsampling from e.g. 4:4:4 to 4:2:2 means the reduction of the chroma samples' horizontal sampling frequency by discarding every second samples in a line of samples, i.e. decimating it horizontally with the ratio of 2:1.
Therefore, without antialiasing filtering prior to decimation, the spectrum of the resampled chroma lines would overlap and Moiré patterns would appear in the displayed image.
In the present example the horizontal frequency content of the chroma signals has to be bandlimited to the half of the original bandwidth  by applying an appropriate \textbf{horizontal (spatial) low pass filter}, or horizontal antialiasing filter.
In case of converting from 4:4:4 $\rightarrow$ 4:2:0 also a further, \textbf{vertical low pass filtering} is required.

\begin{figure}[h!]
	\centering
	\begin{overpic}[width = 0.75\columnwidth ]{figures_en/aliasing2.png}
 	\end{overpic}
	\caption{
	Example for the aliasing of the chroma samples.
	The original 4:4:4 test image (a) contains a sine signal with its frequency increasing along both the horizontal and vertical dimensions, oscillating between the red and green primaries.
	Figure (b) shows the result of chroma subsampling conversion 4:4:4 $\rightarrow$ 4:1:1, with the interpolation performed by simply repeating the nearest sample ($\mathbf{h}_H= \left[ 1\,1 \,1 \,1 \right]^{\mathrm{T}}$).
	Figure (c) shows the result of ideal low pass filtering for both antialiasing and interpolation filtering.}
	\label{Fig:chroma_subsampling}
\end{figure}
\vspace{3mm}
The effect of omitting the antialiasing filter prior to decimation is illustrated in Figure \ref{Fig:chroma_subsampling} via the example of 4:1 decimation along the horizontal direction, i.e. in the case of the subsampling scheme conversion from 4:4:4 to 4:1:1.
Figure \ref{Fig:chroma_subsampling} (a) depicts the 4:4:4 representation of the image.
In Figure \ref{Fig:chroma_subsampling} the chroma samples are subsampled and reconstructed without any antialiasing filter applied.
The resulting aliasing Moiré patterns are cleary visible, seriously degrading the image quality.
Loosely speaking, the color information of the image contains high-frequency components (small details) that can not be represented in the reduced sampling grid, therefore the Nyquist sampling condition is violated, resulting in the visible patterns\footnote{Aliasing images/Moiré patterns are even more enhanced in case of spatially periodical images, since the aliasing components are also periodical, like the one in the present example.}.
Hence, the application of appropriate antialiasing filtering is crucial.
Obviously, since high frequency components in the spectrum represent small details in the image, therefore, spatial low pass filtering can be interpreted as ,,smoothing'' the image by blurring the small details.

\vspace{3mm}
Without going deep into signal processing details: the blurring of the small details in a signal can be most easily performed by weighted averaging of the adjacent samples (pixels), i.e. with FIR filtering.
Assume that the image is filtered in both the horizontal and vertical dimensions, meaning that both the horizontally and vertically adjacent samples are averaged.
The weights of the horizontal samples in the calculated sample is given by vector $\mathbf{h}_H = h_H(n)$.
Similarly, the vertical weight factors are given by $\mathbf{h}_V = h_V(n)$.
Let $x(m,n)$ denote the intensity of the input sample in the $m$-th row and $n$-th column, e.g. in this case being either $C_{\mathrm{B}}$ or  $C_{\mathrm{R}}$ samples.
The intensity of the filtered (or averaged) output sample can be expressed as
\begin{equation}
y(m,n) = \sum_{k = -\infty}^{\infty} \sum_{l = -\infty}^{\infty} x(k,l)\, h_V(m-k) \, h_H(n-l),
\end{equation}
describing consequent horizontal and vertical 1D convolutions.
Vectors $h_H(n)$ and $h_V(n)$ are the horizontal and vertical filter coefficients, or filter kernels (impulse responses).

As the simplest filtering approach the output is generated as the average of two adjacent samples both horizontally and vertically.
This simple averaging process is described by convolution with the filter coefficients
\begin{equation}
\mathbf{h}_H =
\mathbf{h}_V =
\begin{bmatrix}[c]
       1/2 \\[0.3em]
       1/2\end{bmatrix}.
\end{equation}
As a result, the output sample is obtained as the simple sum of 4 adjacent samples, therefore, the average sample is located between the input samples both horizontally and vertically, i.e. it is sited interstitially.
This is the antialiasing filtering approach, applied by the JPEG and MPEG-1 encoders during conversion to 4:2:0 subsampling scheme.

With increasing the computational complexity of filtering---i.e. by applying filters with longer impulse response/involving more input samples to the averaging process---the accuracy of the filtering can be improved, with higher achieved attenuation factor above the filter's cutoff frequency.
As an example: MPEG-2 encoder applies the same vertical filter coefficients as the MEPG-1, but in the horizontal direction every output sample is obtained from 3 adjacent input samples, resulting in improved antialiasing performance.
The filter coefficients for MPEG-2 are given by
\begin{equation}
\mathbf{h}_V =
\begin{bmatrix}[c]
       1/2 \\[0.3em]
       1/2\end{bmatrix}
,
\hspace{1cm}
\mathbf{h}_H =
\begin{bmatrix}[c]
       1/4 \\[0.3em]
       1/2 \\[0.3em]
       1/4\end{bmatrix}
\end{equation}
Since along the horizontal dimension 3 adjacent samples are symmetrically averaged, therefore, the output sample will be located coinciding with the input sample at the center position, i.e. they are cosited.
\begin{figure}[]
	\centering
	\begin{overpic}[width = 0.8\columnwidth]{figures_en/interpolation.png}
 	\end{overpic}
	\caption{Interpolation process, realized by simple linear filtering:
	(1) is the input signal, and $F(1)$ illustrating its spectrum.
	(2) zero stuffed input signal and its spectrum $F(2)$.
	(3) output of the interpolation filter.}
	\label{Fig:interpolation}
\end{figure}

\paragraph{Interpolation of the chroma samples:}
The inverse operation of decimation is the increasing of the sampling frequency, by approximating the samples of the input signal in intermediate sampling positions.
Interpolation of the missing chroma samples has to be performed in the receiver of the video signal in order to restore the chroma resolution prior to calculating the RGB signal to be displayed.

The signal processing chain of interpolation is illustrated in Figure \ref{Fig:interpolation}:
In order to increase the sampling frequency the original signal is stuffed with zeroes in the new sampling positions.
This zero stuffing leaves the input spectrum unchanged, since a zero-only spectrum is added to the input spectrum, but with an increased sampling frequency.
The approximation of the intermediate samples can be interpreted as filtering out the image spectrum, i.e. the repeating baseband spectrum on the original sampling frequency.
Hence, interpolation max be performed by simple low pass filtering of the zero-stuffed signal with an appropriate \textbf{reconstruction filter}.

\begin{figure}[t!]
	\centering
	\begin{overpic}[width = 0.8\columnwidth]{figures_en/subsampling_Example.png}
 	\end{overpic}
	\caption{Simple example for the chroma subsampling of natural images.
	The upper row presents the chroma subsampled image after reconstruction, the lower row presents the chroma information only.}
	\label{Fig:chroma_subsampling_Ex}
\end{figure}

Similarly to decimation, low pass filtering for interpolation can be performed by calculating the weighted average of the adjacent samples.
For chroma interpolation in case of 4:2:0 scheme interpolation must be performed both in the horizontal and vertical dimensions.
The simplest horizontal and vertical reconstruction filters are given by the coefficients
\begin{equation}
\mathbf{h}_H =
\mathbf{h}_V =
\begin{bmatrix}[c]
       1 \\[0.3em]
       1\end{bmatrix}.
\end{equation}
Obviously, the filter coefficients above realize the repetition of the nearest input samples in the interpolation position.
By increasing the computational cost, the missing samples can be approximated more accurately.
As an example: Filtering with the horizontal and vertical coefficients
\begin{equation}
\mathbf{h}_H =
\mathbf{h}_V =
\begin{bmatrix}[c]
       1/2 \\[0.3em]
       1 \\[0.3em]
       1/2 \end{bmatrix}
\end{equation}
realize linear interpolation (in case of the interpolation ratio 2:1). 

Also, interpolation can be carried out by more sophisticated methods, besides linear filtering, e.g. based on fitting higher order polynomials to the input data (bicubic interpolation).

\vspace{3mm}
Clearly, the introduced antialiasing and reconstruction filter coefficients coincide up to a constant.
However, the sum of the antialiasing filtering coefficients has to be equal to 1, otherwise the averaging process would change the intensity of the input signal (e.g. the saturation of the colors).
On the other hand, the reconstruction filter in case of $N:1$ interpolation should have the total energy $N$, in order to keep the signal energy unchanged.

Figure \ref{Fig:interpolation} (c) illustrates the effect of chroma subsampling and reconstruction by applying ideal low pass filters for both antialiasing and reconstruction (meaning that its frequency transfer characteristics is a rectangular window).
It is clearly shown that on those parts of the image where the rapid change of color information can not be represented in the subsampled grid, the blurring the color information results in an average yellow hue instead of the oscillation between red and green.
In practice, in case of natural images the chroma component rarely contains alternating high frequency components, which would result in such clearly visible artifacts after filtering.

Figure \ref{Fig:chroma_subsampling_Ex} depicts the chroma subsampling and reconstruction process of a more natural test image.
As it is demonstrated, even in the case of a theoretical 4:1:0 subsampling scheme---in which case the chroma resolution is the quarter of the luma information both in the horizontal and vertical dimensions---the final quality of the reproduced image is only slightly degraded.
The subsampling scheme 4:2:0, on the other hand, ensures the image quality, approximately indistinguishable from the original.

\vspace{2cm}
\noindent\rule{12cm}{0.4pt}

\subsection*{End-of-Chapter Questions}

\begin{itemize}
\item What are the most commonly used chroma subsampling schemes? 
What is the compression factor of the 4:2:0 scheme, compared to the 4:2:2 scheme?
\item What is the difference between the subsampling scheme of MPEG-1 and MPEG-2?
\end{itemize}