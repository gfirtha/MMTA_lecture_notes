\section{Videojel formátumok}

A következő szakasz a különböző analóg, valamint digitális videoformátumok paraméterválasztásának kérdésével foglalkozik.
Láthatjuk, milyen irányelvek mentén került megválasztásra az egyes formátumok képmérete, térbeli felbontása (pixelszáma), képfrissítési frekvenciája. 

\subsection{SD formátumok}

Elsőként a korai, alacsony felbontású NTSC és PAL analóg televíziós rendszerek képformátumát és paramétereinek megválasztását tárgyaljuk.
Bár ezen analóg rendszerek már csak elvétve vannak világszerte használatban, mégis, történelmi jelentőségük mellett a jelenlegi digitális műsorszórásban legelterjedtebb SDTV (Standard Definition) digitális formátumokat közvetlenül az NTSC és PAL videojelek digitalizálásával kapjuk meg.\footnote{Pontosabban az NTSC és PAL kompozit jeleket alkotó chroma és luma jelek digitalizálásával.}

\paragraph{Raszterformátum:}

\subsection{HD formátumok}

Whitaker 592. oldal

\subsection{UHD formátumok}

