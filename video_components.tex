\section{Videóformátumok}

A következő szakasz a különböző analóg, valamint digitális videoformátumok paraméterválasztásának kérdéseivel foglalkozik.
Láthatjuk, milyen irányelvek mentén került megválasztásra az egyes formátumok képmérete, térbeli felbontása (pixelszáma), képfrissítési frekvenciája. 

\subsection{SD formátumok}

Elsőként a korai, normál felbontású NTSC és PAL analóg televíziós rendszerek képformátumát és paramétereinek megválasztását tárgyaljuk.
Bár ezen analóg rendszerek már csak elvétve vannak használatban világszerte---Magyarországon például több éves digitális átállásra való előkészülés után 2013-ban szűnt meg az analóg műsorszórás---, mégis fontos tárgyalni főbb jellemzőit.
Ennek oka történelmi jelentőségük mellett az, hogy a jelenlegi digitális műsorszórásban (a HD adás mellett) legelterjedtebb \textbf{normál felbontású (Standard Definition, SD)} digitális formátumokat közvetlenül az NTSC és PAL videojelek digitalizálásával kapjuk meg.\footnote{Pontosabban az NTSC és PAL kompozit jeleket alkotó chroma és luma komponensek digitalizálásával.}

\paragraph{Képarány és képméret:\\}
Elsőként fontos leszögezni, mekkora képméretre kell optimális formátum-paramétereket választani.
Az \ref{sec:HVS} fejezetben látható volt, hogy az emberi szemben a színlátás helye a sárgafolt, ezen belül is az éleslátásért a látógödörben (fovea centralis) elhelyezkedő receptorok felelnek.
A látógödör mérete alapján az éleslátásunk a teljes $\approx200$ fokos látószögünkből kb. 10-15 fokot fed le a horizontális irányban.\footnote{http://hyperphysics.phy-astr.gsu.edu/hbase/vision/retina.html}
A normál felbontású televíziós szabvány megalkotása során a cél ezen fő látószög tartalommal való kitöltése volt, vagyis a normál felbontású televízió kb. a látótérből 10 fokot kell, hogy kitöltsön (azaz a periférikus látásnak a képalkotásban nem volt szerepe).
Természetesen a konkrét képméret ezek után a nézőtávolság függvénye.
Adott pixelméret/sortávolság mellett az optimális nézőtávolság megválasztásával a későbbiekben foglalkozunk.

A kép mérete mellett fontos térbeli jellemző a kijelző horizontális és vertikális dimenziójának aránya, azaz az ún. \textbf{képarány}.
Az SD formátum alapjául szolgáló NTSC szabvány létrehozása az 1940-es évekig nyúlik vissza, és kidolgozása során nyilvánvaló törekvés volt a korabeli mozifilmek megjelenítésével való kompatibilitás biztosítása.
A mozi korai korszaka, így a teljes némafilm korszak (az anamorf lencsék megjelenése előtt) kizárólag 4:3 képarányt alkalmazott, azaz a horizontális és vertikális képhosszak aránya $1.3\dot{3}$ volt\footnote{A 4:3 képarány létrejötte egészen Thomas Alva Edison munkájáig vezethető vissza, aki az általa használt 35 mm széles filmen egy képkockát 4 perforációnyi magasságúra (19 mm) definiált. 
A perforációk közötti kihasználható szélességből (25.375 mm) így a hasznos terület épp 4:3-hoz képarányúra adódik. 
A 35 mm-es filmen 4 perforációnyi képméretet 1909-ben fogadták el általános szabványnak ("4-perf negative pulldown"), lehetővé téve a szabványos mozikamerák, mozigépek és így a mozi térhódítását.}.
Habár az 50-es években megjelentek az első szélesvásznú mozis formátumok, az NTSC szabvány ezt a 4:3 képarányt fogadta el a televízió szabványos képarányának.
%TODO anamorphic lenses?
% Forrás: https://www.shutterstock.com/blog/4-3-aspect-ratio
% https://www.cinematographers.nl/FORMATS1.html

\paragraph{Képfrissítési frekvencia:\\}

Következő kérdésként vizsgáljuk a mozgókép temporális mintavételi frekvenciájának, azaz a másodpercenként felvillantott képelemek számának megválasztási szempontjait.
A továbbiakban ezt a frekvenciát \textbf{képfrissítési frekvenciának} nevezzük.
Ennek meghatározáséhoz két szempontot szükséges figyelembe vennünk.
Egyrészt mozgó objektumok képi reprodukciója során fontos, hogy elegendő mozgási fázist tároljunk ahhoz, hogy a megfigyelő folytonosnak érzékelje a képtartalom változását.
Emellett elegendően magas képfrissítési frekvenciát kell választanunk a \textbf{villogás (flickering)} elkerüléséhez, azaz a képfrissítési frekvenciának a \textbf{fúziós frekvencia (flicker fusion threshold)} fölé kell, hogy essen.

Mint látni fogjuk, az utóbbi igény támaszt szigorúbb követelményt a képfrissítési frekvencia megválasztásánál.
Ennek oka az ún. béta mozgás (beta movement) nevű optikai illúzió, amely a látás azon jellemzője, hogy egymás után vetített statikus képek sorozatát $~10-12~\mathrm{kép/másodperc}$ (vagy frame-per-sec, fps) változás fölött az emberi szem már folytonos, látszólagos mozgásként érzékeli.
A béta mozgás magyarázata máig sem teljesen tisztázott, leggyakrabban a látóidegen terjedő ingerület létrejöttének gyakoriságával, terjedési tulajdonságaival magyarázzák.
A béta mozgás miatt tehát a folytonos mozgás biztosításához $~20 \mathrm{Hz}$ képfrissítési frekvencia már elegendő lenne 
\footnote{Érdemes megjegyezni, hogy ez a képfrissítési frekvencia csak ahhoz elegendő, hogy ténylegesen mozgásnak érzékeljük a képsorozatot, ettől még a mozgás gyakran ,,darabos'': a nagyobb---pl. 60 kép/másodperccel rögzített és vetített képek folytonosabbnak, ,,simábbnak'' fognak tűnni. 
Épp ezért számos modern kijelző, illetve számítógépes szoftver képes időbeli interpolációra, amely során az MPEG kódolóban is használatos mozgásbecslés alkalmazásával megpróbálják ''kitalálni'' az egyes képkockák közötti tartalmat.
Érdekes tény azonban, hogy a néző szeme már kellően hozzászokott a mozis 24 fps rögzítési frekvenciához, emiatt a magasabb fps-el rögzített, vagy interpolált videó természetellenesen hat.
Ennek a hatásnak a neve a szappanopera effektus (soap opera effect), amely elnevezés onnan származik, hogy a TV-s szappanoperákat---a klasszikus filmhez képest olcsón---közvetlenül digitális videóra rögzítették jellemzően $60 \mathrm{fps}$-el.},
ezt a képfrissítési frekvenciát azonban az átlagos néző még villogónak érzékelné.

Ahhoz, hogy ezt elkerüljük, a képfrissítési frekvenciának tehát magasabbnak kell lennie a fúziós frekvenciánál.
A fúziós frekvencia fényingerek változásának azon frekvenciája, amely fölött a fényinger változását az emberi szem már nem képes követni.
Különböző világosságú felületek váltakozása esetén a gyakorlatban efölött a megfigyelő csak egy ,,összeolvadt" átlagos világosságot érzékelni.
A fúziós frekvencia értéke számtalan tényezőtől függ. 
Többek között emberről emberre változik, függ az átlagos megvilágítási szinttől és színhőmérséklettől, az adaptációs állapottól, a váltakozó fényinger színétől (a frekvencia növelésével jellemzően 15-20 Hz környékén a színezetbeli fluktuáció megszűnik, és csak a világosságszintek közötti vibrálás érzékelhető) amplitúdójától, és a gerjesztés helytől a retinán: azaz, hogy a villogást a fő látóterünkben, vagy a periférikus látásunkkal érzékeljük-e.

Általánosan elmondható, hogy a fúziós frekvencia az embereknél 50-90 Hz közé esik: a fő látótérben, amelyet a csapok dominálnak a látás lassabb, itt a fúziós frekvencia ~50 Hz, míg a periférikus látás jóval gyorsabb, itt a fúziós frekvencia magasabb.
Mivel az NTSC bevezetésénél a cél a fő látótér tartalommal való kitöltése volt, így célszerűen a képfrissítési frekvenciát 50-60 Hz környékére kellett választani.
A konkrét érték megválasztását azonban már a katódsugárcsöves TV technológia egy hátránya határozta meg: a katódsugárcső tápfeszültségére rákerülő hálózati ,,brumm''.

\begin{figure}[]
	\centering
	\begin{overpic}[width = 1\columnwidth ]{figures/ripple.png}
	\end{overpic}
	\caption{Periodikus hálózati brumm megjelenése az egyenirányított tápfeszültségen egyutas egyenirányítás esetén}
	\label{Fig:ripple}
\end{figure}

\vspace{3mm}
A brumm (angolul ripple) a hálózati váltófeszültség egyenirányításának tökéletlenségéből származó periodikus zavarjel, ahogy az a \ref{Fig:ripple} ábrán látható.
A zavarjel frekvenciája a hálózati frekvenciával egyezik meg (egyutas egyenirányítás), vagy annak kétszerese (kétutas egyenirányítás esetén).
Magyar elnevezése a hangerősítők kimenetén megszólaló jellemzően 50 Hz-es mélyfrekvenciás zugásból származik.
Televízió esetében mivel ez a zavarjel közvetlenül hozzáadódik a katódsugárcső vezérlőjeléhez, ezért a zavarjel kirajzolódik a kijelzőn, így látható hibát okoz.

Vizsgáljuk meg, mi rajzolódik ki a képernyőn, ha a katódsugárcső vezérlőjele, azaz maga a videó jel (az egyszerűség kedvéért fekete-fehér esetben, azaz a jel a kirajzolandó fénysűrűség) periodikus, legegyszerűbb esetben 0 és 1 között oszcilláló szinuszos, azaz
\begin{equation}
Y(t) = \frac{1}{2} \sin 2 \pi f t + \frac{1}{2},
\end{equation}
ahol $f$ a vezérlőjel frekvenciája.
A képernyőre ekkora sorról sorra kirajzolódik ez a szinuszos vezérlőjel.
A kérdés, hogy mi a képernyő tartalma a $f$ vezérlőfrekvencia függvényében.

Jelölje a képfrekvenciát $f_V$ (mint vertikális frekvencia), és a sorfrekvenciát $f_H$ (mint horizontális frekvencia), köztünk természetesen fennáll az
\begin{equation}
f_H = N_V \cdot f_V
\end{equation}
összefüggés, ahol $N_V$ a képernyő sorainak száma.
Továbbá a későbbiekben jelölj $T_H = \frac{1}{f_H}$ a soridőt és $T_V = \frac{1}{f_V}$ a képidőt.
Könnyen belátható, hogy 
\begin{itemize}
\item $f = f_H$ választással minden sor tartalma ugyanazon szinuszhullám, a hullám kezdőfázisa minden sor elején és minden kép elején azonos, így egy álló, horizontális hullámforma jelenik meg a képernyőn, ahogy az a \ref{Fig:ripple_display} (a) ábrán látható.
\item $f > f_H$ választással a szinuszos jel fázisa sorról sorra lassan növekszik (mivel a periódushossza rövidebb, mint egy TV sor), így a hullámforma a horizontálishoz képest enyhe dőlést mutat.
Emellett már az első sorban is a hullám kezdőfázisa képről képre változik, így a teljes képtartalom lassan balra mozog.
Hasonlóképp a sorfrekvencia alatti választással lassan jobbra mozgó képet kapunk.
\item $f = f_V$ választással a teljes szinuszhullám egy teljes kép kirajzolásának ideje alatt rajzolódik ki.
Mivel egy sor ideje alatt (megfelelően nagy $N_V$ sorszám esetén) a jel értéke alig változik, ezért soronként állandónak tekintheő a tartalom.
Így tehát a teljes képidő alatt egy álló, vertikális szinuszhullám jelenik meg a kijelzőn, ahogy az az \ref{Fig:ripple_display} (b) ábrán látszik.
\item $f > F_V$ választással a jelalak kezdőfázisa képről képre nő, így a hullámalak lassan felfelé mozdul.
Hasonlóképp $f < F_V$ esetén a hullámalak lefelé mozog.
\end{itemize}
\begin{figure}[]
	\centering
	\begin{overpic}[width = 0.45\columnwidth ]{figures/vertical_sine.png}
	\small
	\put(0,0){(a)}
	\end{overpic}
	\hspace{5mm}
	\begin{overpic}[width = 0.45\columnwidth ]{figures/horizontal_sine.png}
	\small
	\put(0,0){(b)}
	\end{overpic}
	\caption{Periodikus jel képernyőn megjelenítve $f = f_V$ (a) és $f = f_H$ (b) választással}
	\label{Fig:ripple_display}
\end{figure}

Az eszmefuttatás eredményeképp beláttuk, hogy periodikus jelek megjelenítése során megfelelő választással álló rajzolatot jeleníthetünk meg a kijelzőn.
Márpedig a hálózati brumm épp ilyen periodikus zavarjelként jelenik meg a képernyőn, frekvenciája pedig az adott régió hálózati frekvenciája.
A korai, fekete-fehér televíziós rendszer megalkotása során végzett megfigyelési tesztek egyértelműen kimutatták, hogy elektromos zavar esetén az álló zavarkép jóval kevésbé zavarja a nézőt, mintha a zavar mozgó rajzolatként jelenne meg.
Ennek megfelelően mind az amerikai, mind később, az európai rendszer esetében a képfrekvenciát a hálózati frekvenciának választották meg, így biztosítva, hogy az esetleges hálózati brumm a kijelzőn egy vertikális állóképként jelenik meg, amely a nézők számára alig észrevehető.
Így tehát az amerikai rendszerben a képfrekvencia értéke $f_{\mathrm{V,USA}} = 60~\mathrm{Hz}$, az európai rendszerben $f_{\mathrm{V,Eu}} = 50~\mathrm{Hz}$ lett \footnote{A helyzet a színes TV bevezetésével, azaz az NTSC megjelenésével Amerikában bonyolódott, mivel a színsegédvivő frekvenciáját nem lehetett megfelelően megválasztani.
Részletek nélkül: ennek eredményeképp mind a képfrekvenciát, mind a sorfrekvenciát $0.1~\%$-al csökkentették, így az amerikai rendszer képfrekvenciája $f_V = 60\cdot \frac{1000}{1001} = 59.94~\mathrm{Hz}$ lett végül. 
Ezt a változás szerencsére a megfelelő szinkronjeleknek köszönhetően a már létező TV vevőkészüléket nem befolyásolta.}.

\paragraph{Progresszív és váltott soros letapogatás:\\}
A következőekben a különböző képernyő-letapogatási (scanning) módok vizsgáljuk.
Az elnevezés a CRT kijelzőkhöz köthetők, ahol a katódsugár ténylegesen végigpásztázta valamilyen trajektória mentén.

\vspace{3mm}
A legkézenfekvőbb képernyő bejárási mód az ún. \textbf{progresszív letapogatás (progessive scanning)}, amely során a katódsugár egy képidő alatt sorról-sorra bejárja a képernyő összes sorát.
\begin{figure}[]
	\centering
	\begin{minipage}[c]{0.6\textwidth}
	\begin{overpic}[width = 1\columnwidth ]{Figures/progressive_scan.png}
	\end{overpic}   \end{minipage}\hfill
		\begin{minipage}[c]{0.3\textwidth}
	\caption{Progresszív letapogatás szemléltetése (az egyszerűség kedvéért 11 sorral ábrázolva), az aktív sortartalommal (1), sorvisszafutással (2) és képvisszafutással (3)}
	\label{Fig:progressive}  \end{minipage}
\end{figure}
A letapogatás módját a \ref{Fig:progressive} ábra szemlélteti.
Természetesen a jelenlegi LCD kijelzők esetében értelmetlen letapogatásról beszélni, ezek progresszív megjelenítési módban egyszerre változtatják az összes pixelsor tartalmát.
Átviteltechnika szempontjából hasonlóan, ez azt jelenti, hogy az adott interface-en (konzumer berendezések esetében jellegzetesen HDMI-n keresztül) a kijelzőn megjelenítendő adat sorról sorra érkezik, és természetesen a teljes kép adatait egy soridő alatt továbbítani kell.
A progresszív formátumot az alkalmazott sorszám utáni ,,p'' jelölés mutatja, lásd HD esetében 1080p.

Bár a progresszív letapogatás tűnik a legegyértelműbb, legkézenfekvőbb megoldásnak, mégis, egészen az UHDTV szabvány megjelenéséig nem ez volt az általánosan elfogadott megoldás.
Ennek okait a következőekben tárgyaljuk.

\vspace{3mm}
Az előző fejezetben láthattuk, hogy a folytonos mozgás biztosításához már $20-25~\mathrm{Hz}$ képfrekvencia elegendő lenne, míg a villogás elkerüléséhez legalább $50-60~\mathrm{Hz}$ képfrissítési frekvencia szükséges.
Ez már bizonyos szintű tömörítést tesz lehetővé, hiszen a teljes képtartalom elegendő, ha lassabban változik, mint kijelző rajzolási frekvenciája.

\begin{figure}  
\small
  \begin{minipage}[c]{0.64\textwidth}
	\begin{overpic}[width = 1\columnwidth ]{Figures/triple_blade_shutter.png}
	\end{overpic}   \end{minipage}\hfill
	\begin{minipage}[c]{0.3\textwidth}
    \caption{Triple blade shutter működése: \url{https://www.youtube.com/watch?v=jrSzRAch930} }
\label{fig:triple_blade_shutter}  \end{minipage}
\end{figure}

Ez a tömörítés már a korai mozitechnikában is megjelent: 
A korai, némafilmes korszakban számos képfrekvencia volt használatban $16-24~\mathrm{Hz}$ között.
Manapság mozitechnikában a szabványos rögzítési frekvenciát $24~\mathrm{fps}$-re rögzítették.
A villogás elkerüléséhez (tehát a képfrissítési frekvencia növeléséhez) speciális rekesszel látták el a vetítőgépet.
A fénynyaláb útjában forgó rekesz, amelyen kettő, vagy három rés volt található (az ún. ''two'', vagy ''three blade shutter'') egy képkocka megjelenítése során tett meg egy teljes fordulatot, így a vetítőgép ugyanazt a képkockát kétszer, vagy háromszor villantja fel, mielőtt továbbhúzza a mozigép a szalagot.
Ezzel az egyszerű trükkel a $24~\mathrm{fps}$-en rögzített tartalmat $48\mathrm{fps}$, illetve manapság jellemzően $72~\mathrm{fps}$-en lehet megjeleníteni a mozikban.

Hasonlóan elven, a modern megjelenítők esetében a kijelző képfrissítési frekvenciája (pl. amivel egy LCD kijelző esetében a háttérvilágítás villog $200~\mathrm{Hz}$ körül) jóval a tényleges képtartalom frissítési frekvenciája fölött van.
A TV műsorszórás bevezetésének idején azonban a vevőkészülékek nem voltak képesek a képtartalom tárolására, a vett jel közvetlenül, valós időben rajzolódott ki a kijelzőre.
A feladat megoldásául, azaz a másodpercenként átvivendő képek számának csökkentésére, és így sávszélesség-takarékosságra az ún. \textbf{váltott-soros letapogatást (interlaced scanning)} vezették be.

\begin{figure}[]
	\centering
	\begin{overpic}[width = 0.85 \columnwidth ]{Figures/interlaced_scan.png}
	\end{overpic}
	\caption{Váltott-soros képbontás (TV sorok közbeszövése), a jobb áttekinthetőség kedvéért 21 sorral.
	A teljes képernyő pásztázásához (feltéve, hogy az elektronnyaláb az első félkép első sorának elejéről indul) az első félképnek fél sorban kell végződnie, míg a második félképnek félsorral kell kezdődnie.
	Ez csak páratlan teljes sorszám esetén teljesül (mindkét félkép $N_{\frac{V}{2}} + \frac{1}{2}$ sorból áll, a teljes sorszám $2 N_{\frac{V}{2}} + 1$, ami szükségszerűen páratlan) }
	\label{Fig:interlaced}
\end{figure}

A megoldás alapötlete--ahogy a \ref{Fig:interlaced} ábrán is látható---a következő:
Ahelyett, hogy a kijelző egy teljes képidő alatt az összes sort egymás után végigpásztázná, bontsuk a képernyőt páros és páratlan sorszámú sorokra, amelyek így egy páratlan és egy páros félképet alkotnak.
A teljes képet (angolul \textbf{frame}) tehát két \textbf{félképre} (angolul \textbf{field}) bontjuk.
A kijelző ezután a teljes képidő első felében a páratlan, a második felében a páros sorokat pásztázza végig.
A váltottsoros formátum jelzése a sorszám mögé illesztett ,,i'' jelzés (pl. 1080i).

Természetesen a képernyő tartalma a félképek frissítési frekvenciájával, az ún. \textbf{félképfrekvenciával} frissül, tehát ahhoz, hogy elkerüljük a villogást a félképfrekvenciának kell a fúziós frekvencia fölé esnie. 
Így váltottsoros letapogatás esetén a félképfrekvencia lett az európai rendszerben $50$, valamint az amerikaiban $60~\mathrm{Hz}$-re (pontosabban $59.94~\mathrm{Hz}$-re) választva.
A teljes, effektív képfrekvencia pedig ezek felére, tehát $25~\mathrm{Hz}$, illetve $30~\mathrm{Hz}$-re ($29.97~\mathrm{Hz}$-re) adódik
A technikával tehát a mozis technikához hasonlóan, a képfrissítési frekvenciát elegendően magasra emelték, míg a tényleges, teljes ,,felbontású'' képtartalom ehhez képest fele sebességgel érkezik.

Fontos megjegyezni, hogy a félképek (field-ek) különböző időpillanatokban készülnek, azaz nem ugyanazon teljes képhez tartoznak (nem állítható elő egy teljes kép páros-páratlan sorra való felbontásával). 
Ez eltérés a mozis rendszerhez képest, amely ugyanazt a képkockát mutatta be többször.
Ennek eredményeként a következő módhatóak el a váltottsoros videóról:
\begin{itemize}
\item A váltott-soros letapogatás a progresszívhez képest 2:1 arányú tömörítést valósít meg, azaz a továbbítandó adatmennyiséget (és így a szükséges sávszélességet) lefelezi
\item Álló képtartalomnál a progresszív letapogatással megegyező vertikális felbontást valósít meg (hiszen a páros és páratlan félképek ugyanazt a képet egészítik ki)
\item Gyorsan mozgó képtartalom mellett a függőleges felbontás gyakorlatilag a progresszív formátum fele (hiszen a félkép tartalma folyamatosan változik)
\end{itemize}
Általánosan elmondhatjuk, hogy lassan változó képtartalom esetén (pl. filmek) a váltott-soros letapogatás megfelelően nagy vertikális felbontást és a progresszívnál folytonosabb mozgásreprodukciót biztosít, megfelelő tömörítés (sávszélesség-hatékonyság) mellett.
Gyors kameramozgások esetén, pl. sporttartalom már láthatóvá válhatnak a felezett vertikális felbontásból származó hatások.
\vspace{3mm}

Az elmondottak alapján a normálfelbontású SD formátum kizárólag váltottsoros letapogatási módot alkalmaz.
A HD szabvány bevezetésével már mind interlaced, mind progesszív formátumok léteznek, míg UHDTV esetén a szabványok már kizárólag progresszív formátumokat definiálnak.

\vspace{3mm}
Érdekességképp elmondható, hogy az interlaced technika számos kérdést, nehézséget is felvet egyszerűsége mellett.

Egyik példaképp: korábban (előadáson) láthattuk, hogy a térbeli mintavételi frekvencia megsértése térbeli átlapolódási jelenségekhez vezet, amelyek jellegzetesen térben periodikus képek esetében (pl. téglafal, ,,kockás'' ing) jól látható Moiré ábrák megjelenését okozza.
Mivel interlaced esetben a vertikális mintavételi frekvenciát lefelezzük, ezért félképeken ezek a Moiré ábrák erőteljesen megjelenhetnek, az egymás utáni átlapolódó félképek váltakozása pedig igen zavaró átlapolódási jelenségekhez, ún. interline twitter jelenséghez vezet már állókép megjelenítése esetén is.
A jelenségre egy szemléltető példa \href{https://en.wikipedia.org/wiki/File:Indian_Head_interlace.gif}{itt} található.
Minthogy az összes SD formátum interlaced letapogatást alkalmazott, épp az interline twitter jelensége volt a fő oka a TV felvételek során a négyzetrácsos, csíkos öltözékek elkerülésének.

\begin{figure}  
\small
  \begin{minipage}[c]{0.64\textwidth}
	\begin{overpic}[width = 1\columnwidth ]{Figures/Interlaced_video_frame_(car_wheel).jpg}
	\end{overpic}   \end{minipage}\hfill
	\begin{minipage}[c]{0.3\textwidth}
    \caption{Megfelelő deinterlacing technika nélkül váltottsoros formátum megjelenítése progresszív kijelzőn.}
\label{fig:deinterlacing}  \end{minipage}
\end{figure}

További érdekes kérdést vet fel az interlaced és progresszív formátum közötti konverzió.
Progresszívről interlaced formátumba a feladat viszonylag egyértelmű, a teljes kép páros és páratlan sorokra bontásával megoldható.
A váltottsoros formátumról progresszívre történő konverzió konzumer felhasználási szempontból gyakoribb, gondoljunk csak egy jellegzetesen váltottsoros formátumban rögzített DVD lemez jellemzően progresszív számítógép monitoron történő megjelenítésére.
Legegyszerűbb stratégiaként a monitor a szomszédos félképeket összeszőve alakít ki egy teljes felbontású képet.
Ez azonban gyors mozgások esetén ún. fésűsödési jelenségekhez vezet, amelyet az \ref{fig:deinterlacing} ábra szemléltet.
Épp ezért, a konverzióhoz kifinomultabb \textbf{deinterlacing} eljárás szükséges a félképek sorai közötti adatok interpolációjához.
A feladat létjogosultsága manapság is nagy, hiszen a jelenlegi LCD TV és számítógép monitorok már nem támogatnak natív interlaced megjelenítést, míg a HD műsorszórás még napjainkban is váltottsoros formátumot alkalmaz (jellemzően 1080i-t).

\paragraph{Analóg SD formátumok, az analóg videójel:\\}

Az előzőek alapján bevezethetjük a normál felbontású analóg televíziós műsorszórás képformátumát:

Ahogy azt már korábban láthattuk két analóg képformátum terjedt el a világon a színes műsorszórás kezdetével:
\begin{itemize}
\item Az Egyesült Államokban és Japánban alkalmazott NTSC képformátumot az FCC vezette be 1953-ban.
Az NTSC formátum a korabeli technológiának megfelelően 525 TV-sorból áll, és váltott-soros letapogatást alkalmaz.
A korábban tárgyalt okokból kifolyólag a rendszer képfrissítési frekvenciája, azaz a félképfrekvencia $60~\mathrm{Hz}$ ($59.94~\mathrm{Hz}$), amelyből természetesen a képfrekvencia $30~\mathrm{Hz}$-re ($29.97~\mathrm{Hz}$-re) adódik.
\item Európában, Ausztráliában és Ázsiában a PAL rendszer került bevezetésre 1967-ben.
A PAL formátum sorszáma $N_V = 625$, váltott-soros letapogatással, míg a helyi hálózati frekvenciának megfelelően a félképfrekvencia $50~\mathrm{Hz}$ és így a képfrekvencia $25~\mathrm{Hz}$.
\end{itemize}
Az NTSC és PAL formátum fő jellemzőit a \ref{tab:sd_formats} táblázat tartalmazza.

\begin{table}[h!]
\caption{}
\renewcommand*{\arraystretch}{2.25}
\label{tab:sd_formats}
\begin{center}
    \begin{tabular}[h!]{ @{}c | | l | l @{} }%\toprule
				         &   NTSC  							       & PAL \\ \hline
    Összes sorok száma:	 &  525   								   &  625 \\
    Aktív sorok száma:   &  480   								   &  576 \\
    Képfrekvencia:       &  $30~\mathrm{Hz}$ ($29.97~\mathrm{Hz}$) & $25~\mathrm{Hz}$ \\
    Félképfrekvencia     &  $60~\mathrm{Hz}$ ($59.94~\mathrm{Hz}$) & $30~\mathrm{Hz}$ \\
    Sorfrekvencia: 		 &  $525 \cdot 30 = 15750~\mathrm{Hz}$ ($15734~\mathrm{Hz}$) & $15625~\mathrm{Hz}$ \\
    Soridő:              &  $63.49~\mathrm{\mu s}$ ($63.55~\mathrm{\mu s}$) & $64~\mathrm{\mu s}$ \\
    \end{tabular}
\end{center}
\end{table}

Fontos megjegyezni, hogy a CRT kijelző letapogatása során mind az egyes sorok végén, mind a félképek végén az elektronnyaláb kioltásra került, míg visszatérítették a következő sor, illetve félkép elejére.
Ez a visszatérítés természetesen véges időbe telik.
Ennek eredményeképp minden soridő, illetve félképidő tartalmaz inaktív, kioltási időintervallumokat (\textbf{blanking interval}), amelyben hasznos videójel nem található.
Ezeket az időintervallumokat nevezzük \textbf{sorkioltási időnek} (\textbf{horizontal blanking}) valamint \textbf{félképkioltási időnek} (\textbf{vertical blanking}).

Ennek megfelelően a teljes soridő felbontható ún. \textbf{aktív soridőre}, amely a tényleges videojelet tartalmazza és a sorkioltási időre, amely a sorszinkron (horizontális szinkron) jeleket, és egyéb jelzéseket tartalmaz.

Hasonlóan, a teljes félképidő felbontható aktív sorokra, amelyek a megjelenítésre kerülő TV sorok, valamint a félképkoltási időre, amely vertikális szinkronjeleket és egyéb járulékos adatokat tartalmaz.
Az aktív sorok számát szintén a \ref{tab:sd_formats} táblázat tartalmazza.
 
\begin{figure}[t!]
\captionsetup{singlelinecheck=off}
\small
  \begin{minipage}[c]{0.64\textwidth}
	\begin{overpic}[width = 1\columnwidth ]{Figures/Timing_PAL_FrameSignal.png}
	\end{overpic}
\label{fig:PAL_frame}
    \end{minipage} \hfill
	  \begin{minipage}[c]{0.3\textwidth}
    \caption[]{ Egy teljes kép felépítése váltottsoros letapogatás esetén egyetlen videokomponensre ábrázolva:
    \begin{itemize}
    \item aktív soridő: szürke
    \item sorkioltási idő: magenta, cián és sárga
    \item félképkioltási idő: zöld, narancssárga, fehér
    \end{itemize}
    }
    \end{minipage}
\end{figure}
Egy teljes TV kép (azaz két egymás utáni félkép) felépítése a PAL rendszerben a \ref{fig:PAL_frame} ábrán látható.
Az ábra természetesen csak egyetlen video komponens felépítését szemlélteti.
Az ábrán jól megfigyelhetőek a a félkép és képkioltási idők, bennük pedig az ún. \textbf{félképszinkron jelek} (\textbf{VSYNC}) (a zöld tartományban) és \textbf{sorszinkron jelek} (\textbf{HSYNC}) (sárga tartomány).
Ezek a jelek a TV vevő (vagy általában a megjelenítőeszköz) szinkronizációját biztosítják a megfelelő megjelenítés érdekében.
A szinkronjelek hibás vétele esetén a kép vertikálisan (félképszinkron hiányában), vagy horizontális (sorszinkron hiányába) elmozdul.
Ezeket a jelenségeket ,,jitter''-nek, illetve ,,rolling''-nak nevezzük.

%\begin{figure}  
%\small
%  \begin{minipage}[c]{0.65\textwidth}
%      \includegraphics[width= 1\columnwidth  ]{Figures/Vertical_blank_2.png}\\ (a) \\
%    \includegraphics[width= 1\columnwidth  ]{Figures/Horizonta_blank.png}\\ (b) \\
%    \end{minipage} \hfill
%	  \begin{minipage}[c]{0.3\textwidth}
%    \caption{PAL jel félképkioltási intervalluma (a) és sorkioltási intervalluma (b).}
%    \end{minipage}
%\label{fig:PAL}
%\end{figure}

A jelenleg elterjedt megjelenítők esetén ezek a kioltási idők természetesen okafogyottá váltak:
a modern, főként stúdió célú CRT megjelenítők már jóval kisebb kioltási idő mellett is működőképesek, míg LCD megjelenítők esetén egyáltalán nincs szükség kioltási időre.
Ennek ellenére a kioltási idők a jelenlegi digitális szabványok esetén is ugyanúgy jelen vannak, így pl. a HDMI szabvány esetében is.
Ennek egyik, természetes oka az, hogy a technika fejlődésével megjelenő újabb és újabb szabványok mind a már létező, korábbi szabványokra épülnek.
Másrészt a kioltási idők lehetővé tették egyéb, kiegészítő adatok tárolását is ezekben az időszegmensekben.
Így a kioltási időkben továbbítható pl. a teletext adat, feliratok, és digitális esetben a video kísérő audio adat is.
Ezen adatok helyét az ITU-R BT.1364 és az SMPTE 291M szabványok definiálják. 
Megjegyezhető, hogy a szabványok a digitális hang átvitelét (pl. a HDMI szabvány esetén is) a sorkioltási időben írják elő \footnote{
Egy egyszerű példaként HDMI audio átvitelre:
1080p HD formátum esetén (összes sor: 1125, ld. később) $60~\mathrm{Hz}$-es képfrekvencia mellett a sorfrekvencia $f_V = 67.5~\mathrm{kHz}$.
$192~\mathrm{kHz}$ mintavételi frekvenciájú 8 csatornás audioanyag átvitele esetén az egy kép alatt átvivendő audiominták száma: $\frac{8 \cdot 192 000 }{60} = 25600~\mathrm{minta}$, azaz soronként kb. 23 minta átvitele szükséges.
Ez a HDMI 1.0 szabvány által megengedett audiosebesség. (Példa folytatása \href{https://www.sciencedirect.com/science/article/pii/B9780128016305000049}{itt} található.)
}.

% https://www.sciencedirect.com/topics/computer-science/blanking-interval
% https://www.sciencedirect.com/topics/computer-science/horizontal-blanking

\paragraph{Térbeli felbontás és az SD formátum:\\}

Az analóg videojel tárgyalása után a továbbiakban rátérhetünk a videojel digitális reprezentációjának tárgyalására.
Az első digitális videoformátumot a normálfelbontású, SD videót az ITU (akkoriban CCIR) alkotta meg 1982-ben a ITU-601 szabvány formájában \footnote{Munkájáért a CCIR 1983-ban tehcnikai Emmy díjat is kapott}.

Az SD formátum gyakorlatilag az eddig tárgyalt videojel komponensek digitális reprezentációjának tekinthető, azaz a \ref{fig:PAL_frame} látható videojel teljes egészében digitalizációra került kioltási intervallumokkal együtt, mind a luma és chroma komponensekre (más szóval az $Y'P_b'P_r'$ jelek közvetlen digitalizációjával kaphatjuk).
A digitalizált videojelek neve--ahogy arról már szó volt--$Y'C_b'C_r'$ jelek.
A jelek elvi előállítása az \ref{Fig:SD_production} ábrán látható.
\begin{figure}[]
	\centering
	\begin{overpic}[width = 0.8 \columnwidth ]{Figures/YCbCr_production.png}
	\end{overpic}
	\caption{Digitális SD jel előállítása a videokomponensek digitalizálásával}
	\label{Fig:SD_production}
\end{figure}


A digitalizáció egyes kérdéseit már a korábbiakban érintettük.
Nyitott kérdés még a soronkénti mintaszám meghatározása, amely a sorok számával együtt megadja az SD formátum felbontását (pixelszámát).
A feladat tehát az analóg videójel mintavételi frekvenciájának meghatározása.

A mintavételi frekvencia megválasztásánál a következő szempontokat vették figyelembe:
\begin{itemize}
\item Természetes törekvés volt, hogy a több évtizede egymás mellett létező NTSC és PAL rendszerre egyszerre alkalmazható legyen, azaz mind PAL, mind NTSC video digitális ábrázolását lehetővé tegye.
Emellett nyilvánvalóan a mintavételezést úgy kell végrehajtani, hogy mindkét rendszerben egy sorba egész számú mintavételi periódus (azaz pixel) férjen bele.
Ebből következik, hogy a mintavételi frekvencia a sorfrekvencia egész számú többszöröse kell, hogy legyen mind az NTSC, mind a PAL rendszerben, azaz
\begin{equation}
f_s = n \cdot f_H^{\mathrm{PAL}} = m \cdot f_H^{\mathrm{NTSC}},
\end{equation}
ahol $n, m$ egész számok.
Minthogy a sorfrekvenciák 
\begin{align}
f_H^{\mathrm{PAL}} &= 25 \cdot 625 = 15625~\mathrm{Hz} \\
f_H^{\mathrm{NTSC}} &= 30 \cdot \frac{1000}{1001} \cdot 525 = 15734.2~\mathrm{Hz} ,
\end{align}
ezek legkisebb közös többszöröse
\begin{equation}
144 \cdot f_H^{\mathrm{PAL}} = 143 \cdot f_H^{\mathrm{NTSC}} = 2.25~\mathrm{MHz}.
\end{equation}
A mintavételi frekvencia tehát $2.25~\mathrm{MHz}$ egész számú többszöröse.
\item Emellett a mintavételi tétel értelmében a mintavételi frekvencia az átlapolódás elkerülésének érdekében legalább a mintavett jel sávszélességének kétszerese kell, hogy legyen.
Korábban láttuk, hogy a luma jel sávszélessége $6~\mathrm{MHz}$, a chroma jeleké pedig ennek a fele.
\end{itemize}
A legkisebb frekvencia amire a két előbbi feltétel teljesül $13.5~\mathrm{MHz}$.
Ezt választották tehát a világosságjel mintavételi frekvenciájának, miközben a színkülönbségi jelek számára, figyelembe véve az emberi látás tulajdonságait, felezett mintavételi frekvenciát ($6.5~\mathrm{MHz}$) választottak.
Ez az európai rendszerben 1 sorra 864, az amerikaiban 858 teljes mintaszámot eredményez, amely a sorkioltási időt is tartalmazza.

\begin{figure}[]
	\centering
	\begin{overpic}[width = 0.65 \columnwidth ]{Figures/SD_formats.png}
	\end{overpic}
	\caption{Az SD formátum képmérete az amerikai és az európai rendszerben}
	\label{Fig:SD_format}
\end{figure}

A két rendszer további egységesítésének érdekében egy soron belül az aktív pixelek számát közösen 720 pixelre választották (amelyből csak 704 pixel tartalmaz tényleges képi adatot, a digitalizálás előtti analóg jel kezdetének bizonytalansága, szélekhez közeli torzításai, elmosódásai miatt).
Ezzel tehát megkaptuk az SD formátum tényleges képméretét, ahogy az az \ref{Fig:SD_format} ábrán látható.
Az aktív sorok száma alapján, és mivel mindkét rendszerben kizárólag interlaced videó definiált, a két formátum megjelölése \textbf{480i} és \textbf{576i}.

\begin{figure}[]
	\centering
	\begin{overpic}[width = 0.4 \columnwidth ]{Figures/sd_gamut.png}
	\small
	\put(0,0){(a)}
	\end{overpic}
	\hspace{2mm}
	\begin{overpic}[width = 0.55 \columnwidth ]{Figures/sd_OETF.png}
	\small
	\put(0,0){(b)}
	\end{overpic}
	\caption{Az SD formátum gamutja(a) és Gamma-függvénye (b)}
	\label{Fig:SD_gamut}
\end{figure}

Röviden összefoglalva a jelen, és előző fejezetet a két SD formátum létrehozásának lépései és főbb tulajdonságai:
\begin{itemize}
\item A formátum primary színei és a színtér gamutja a \ref{Fig:SD_gamut} (a) ábrán látható.
A színtér fehérpontja D65 fehér.
A luma komponens számításának módja (amely természetesen az előző adatok ismeretében levezethető)
\begin{equation}
Y' = 0.299 R' + 0.587 G' + 0.112 B'
\end{equation}
\item A forrás RGB jelei a perceptuális kvantálás megvalósításának érdekében Gamma-torzításon mennek keresztül, ahol a Gamma-függvény, vagy Optoelectronic Transfer Function:
\begin{equation}
E = 
\begin{cases}
4.500 L, \hspace{20mm} \mathrm{ha}\, L < 0.018 \\
1.099 L^{0.45} - 0.099, \hspace{3mm} \mathrm{ha}\, L \geq 0.018,
\end{cases}
\end{equation}
ahol $L \in \{ R, G, B \}$.
A teljes görbe jól közelíthető egy $L^{0.5}$ függvénnyel
\item A formátum képaránya 4:3. A későbbiekben a HD megjelenése után ezt kiegészítették 16:9 képarányú formátummal is.
\item Kizárólag interlaced formátum definiált
\item A videókomponensek megfelelő sávkorlátozás után mintavételezésen és kvantáláson esnek át.
A kvantálás 8, vagy 10 biten történik.
\item A világosságjel mintavételi frekvenciája $f_s = 13.5~\mathrm{MHz}$.
Az 576i (625 soros, 50 félkép/s) rendszerben az aktív felbontás így 720x576 pixel, a 480i (525 soros, 60 félkép/s) rendszerben 720x480 px.
\item A színkülönbségi jel az eredeti, stúdióformátumban 4:2:2, tehát a horizontális színfelbontás a világosságjelének a fele.
Ezt később kiegészítették 4:2:0 struktúrával is konzumer célokra.
\end{itemize}

\subsection{HD formátumok}

%Whitaker 592. oldal
Az előzőekben részletesen tárgyaltuk as SD digitális videoformátum megalkotásának alapelveit.
A részletes vizsgálat oka, hogy ugyanezek az alapelvek, jelfeldolgozási lépések érvényesek a jelenlegi HD és UHD formátumok esetén is, valamint a jelenlegi SD műsorszórás Magyarországon is 576p (azaz már progresszív) formátumban történik.

Láthattuk, hogy az SD formátum megalkotásánál az egyes paramétereket úgy választották meg, hogy a kitűzött kb. 10 fokos látószögben minél élethűbb képi reprodukciót lehessen megvalósítani.
A HD és UHD formátumok tárgyalása előtt vizsgáljuk meg, hogy adott felbontás (pixelméret) mellett mekkora távolságból kell az adott kijelzőt megfigyelni, rávilágítva ezzel a HD formátum létrehozásának fő motivációjára.

\paragraph{Optimális nézőtávolság:\\}

\begin{figure}[]
	\centering
	\begin{overpic}[width = 0.67 \columnwidth ]{Figures/hd_pixel_angle_mod.png}
	\small
	\end{overpic}
	\caption{Geometria az optimális nézőtávolság származtatásához}
	\label{Fig:optimal_vd}
\end{figure}

Általánosan elmondható, hogy pixel alapú képi reprodukció során a fő szempont, hogy a szomszédos pixelekből érkező fénysugarak által bezárt szög az emberi szem felbontóképessége alá essen.
Ezzel biztosítva van, hogy a kijelző pixelstruktúrája nem látható (a kép nem ,,pixeles''), valamint az RGB alapszíneket alkalmazó reprodukció is lehetővé válik, hiszen az egyes alapszínek érzékelése helyett az additív színkeverés a szemben megvalósul.

Korábban láthattuk, hogy az emberi szem felbontása 1 szögperc (azaz $\frac{1}{60}^{\circ}$) (legalábbis a világosságjelre véve, segítségünkre van, hogy színezetre ennél is rosszabb).
Adott pixelméretre természetesen ebből már meghatározható az a minimális nézőtávolság, amelyre az előbbi feltétel teljesül.
Mivel jellemzően a kijelzőknek nem a pixelmérete van megadva, hanem a kijelző mérete és a vertikális, ill. horizontális pixelszám, ezért célszerű a fenti minimális nézőtávolságot ezek függvényében kifejezni.

Vizsgáljuk az \ref{Fig:optimal_vd} ábrán látható geometriát adott $H$ magasságú, $N_V$ sorszámú kijelző esetén.
A pixelméret ekkor természetesen $\frac{H}{N_V}$.
A kijelző a megfigyelőtől $D$ távolságra helyezkedik el.
A szomszédos (szemközti) pixelekből a megfigyelő szemébe érkező fénysugarakra felírható ekkor a 
\begin{equation}
\tan \frac{\Phi}{2} = \frac{H}{2 N_V D}
\end{equation}
egyenlőség.
Alkalmazzuk a tangens függvény kisargumentumú lineáris közelítését, azaz $\tan x \approx x$, ha $x \ll 1$.
Ekkor
\begin{equation}
\Phi = \frac{H}{N_V D} \hspace{3mm} \rightarrow \hspace{3mm} D = \frac{H}{N_V \Phi}
\end{equation}
érvényes.
Az emberi szem felbontását $\frac{1}{60}\cdot \frac{\pi}{180}~\mathrm{rad} = 2.9 \cdot 10^{-4}$ behelyettesítve adott felbontású és méretű kijelző esetén az optimális (minimális) nézőtávolságra
\begin{equation}
D = H \frac{1}{N_V \,  2.9 \cdot 10^{-4}}
\end{equation}
adódik. 
Ez a távolság az ún. Lechner-távolság, amely tehát megadja, hogy a tervezés során figyelembe vett képméret és felbontás mellett mekkora az optimális nézőtávolság adott képformátum esetén.
\begin{table}[h!]
\caption{Fontosabb SD és HD formátumok ideális nézőtávolsága és az így kitöltött horizontális látószög}
\renewcommand*{\arraystretch}{2.25}
\label{tab:viewing_dist}
\begin{center}
    \begin{tabular}[h!]{ @{}c | | l | l | l @{} }%\toprule
				         &   Amerikai 		   & Európai 				&	 HDTV \\ \hline
    TV-sor/képmagasság:	 &     480 	  		   &   576   				&	 1080\\
    Nézőtávolság:   &  7-szeres képmagasság    &  6-szoros képmagasság & 3-szoros képmagasság \\
    Nézőtávolság:       &  4.25-szörös képátló &  3.6-szoros képátló	& 1.5-szörös képátló \\
    Vízszintes látószög &  kb. 11 fok 		&    kb. 13 fok & kb. 32 fok\\
    \end{tabular}
\end{center}
\end{table}

Amennyiben az $a_r$ képátló ismert (SD esetén 4:3, HD esetén 16:9), az összefüggés kifejezhető a képszélesség függvényében is.
SD esetén ez
\begin{equation}
D = \frac{W}{a_r} \frac{1}{N_V \,2.9 \cdot 10^{-4}},
\end{equation}
ahol $W$ a kép szélessége.
Ekkor ha a kijelzőt az így kapott optimális távolságról nézzük, meghatározható a képernyő által bezárt vízszintes látószög:
\begin{equation}
\tan \frac{\Phi_H}{2} = \frac{W}{2 D} \hspace{1cm} \rightarrow \hspace{1cm} D = \frac{W}{2 \tan \frac{\Phi_H}{2}}, 
\end{equation}
és így adott felbontás mellett a horizontális látószög:
\begin{equation}
\Phi_H = 2\arctan \left( \frac{a_r \, N_V \, 2.9\cdot 10^{-4}}{2} \right).
\end{equation}

Az eredményeket az eddig bemutatott SD és a következőekben tárgyalt HD formátumokra kiszámítva a \ref{tab:viewing_dist} táblázat foglalja össze.
Láthatjuk, hogy az SD felbontást ideálisan a képmagasság 6-7-szereséről célszerű nézni.
Ekkor valóban, a formátum tervezésének kiindulási pontjába érünk vissza, azaz a kijelző a fő látóterünket, kb. 10-13 fokot tölti ki horizontálisan.

Ez már előrevetíti a HD formátum megalkotásának fő célját: a vizuális élmény fokozását nagyobb kitöltött látószög alkalmazásával.

%\begin{figure}[]
%	\centering
%	\begin{overpic}[width = 0.75 \columnwidth ]{Figures/optimal-viewing-distance-television-graph-size.png}
%	\small
%	\end{overpic}
%	\caption{Optimális nézőtávolság a kijelzőátmérő függvényében}
%	\label{Fig:optimal_vd_2}
%\end{figure}


\subsection{UHD formátumok}

Periférikus látás:
https://www.quora.com/What-is-the-aspect-ratio-of-human-vision