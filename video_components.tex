Az előző fejezet bemutatta az emberi látás képi reprodukció szempontjából legfontosabb tulajdonságait és részletesen tárgyalta a fény- és színmérés alapjait, bevezetve a világosság fogalmát és a CIE XYZ színteret.
Ez a fejezet ezekre az ismeretekre építve bemutatja a televíziós-technikában használt színes-képpont ábrázolás módját, 
ez alapján bevezetve a jelenleg is alkalmazott analóg és digitális videójel komponenseket.

Videótechnika szempontjából az XYZ színteret ritkán alkalmazzák képpontok színkoordinátáinak tárolására, kivétel ez alól a digitális mozi és mozifilm-archiválási alkalmazások \footnote{Ennek oka, hogy egyrészt reprodukcióra közvetlenül nem használható, hiszen az XYZ alapszínek nem valós színek (az X,Y,Z bázisvektorok helyén nem található látható szín), másrészt a teljes látható színek tartománya igen nagy bitmélységet igényel, ráadásul feleslegesen:
Az XYZ tér pozitív térnyolcadát a látható színek csak részben töltik ki (sok olyan kód lenne, amihez nem tartozik látható szín), ráadásul a ezen belül is a megjelenítők a látható színeknek csak egy részét képesek reprodukálni.}.
Ugyanakkor az XYZ tér lehetővé teszi a különböző megjelenítők és kamerák által reprodukálható színek halmazának egyszerű vizsgálatát.
A következő szakasz ezeket a konkrét videóeszközökre jellemző, ún. \textbf{eszközfüggő színtereket} mutatja be.

\section{Eszközfüggő színterek}

Az előző fejezetben láthattuk, hogy az emberi látás trikromatikus jellegének, valamint linearitásának (illetve az egyszerű lineáris modellnek) köszönhetően a látható színek egy lineáris 3D vektortérben ábrázolhatóak, amelyben a vektorok összegzési szabálya érvényes: 
Két tetszőleges szín keverékéből származó eredő színinger meghatározható a két színbe mutató helyvektorok összegeként (függetlenül az eredeti színingereket létrehozó fény spektrumától).
Az $xy$-színpatkón ennek megfelelően két szín összege a két színpontot összekötő szakasz mentén fog elhelyezkedni.

Ebből következik, hogy az emberi látás metamerizmusát kihasználva, a látható színek nagy része előállítható mesterségesen, megfelelően megválasztott alapszínek összegeként.
Ez általánosan véve a színes képreprodukció alapja.
Természetesen nem lehet célunk az összes látható szín visszaállítása: 
Minthogy a színpatkót a spektrálszínek határolják, így elvben végtelen számú spektrálszínt kéne alapszínként alkalmazni az összes látható szín kikeveréséhez.
Felmerül tehát a kérdés, hány alapszín szükséges a színpatkó megfelelő lefedéséhez.

\begin{figure}[]
	\centering
	\begin{overpic}[width = 1\columnwidth ]{figures/color_space_gamut.png}
	\end{overpic}
	\caption{Az azonos alapszínekkel dolgozó SD formátum, HD formátum és az sRGB színtér gamutja $xy$ és $uv$ diagramon ábrázolva.}
	\label{Fig:gamut}
\end{figure}

A színdiagramban könnyen felvehető 4 színpont úgy, hogy a négy szín keverékeit lefedő négyszög (azaz a reprodukálható színek területe) csaknem azonos területű legyen a látható színek területével.
Ugyanakkor az $Luv$ színtér színpatkójából láthattuk, hogy az emberi felbontás zöld árnyalatokra vonatkozó felbontása rossz, és az perceptuálisan egyenletes színdiagram inkább háromszög alakú.
Ez azt jelenti, hogy három megfelelően megválasztott alapszínnel---amelynek különböző arányú keverékeinek színezete egy háromszögön belül helyezkedik el---az egyenletes színezetű ($uv$) színpatkó jelentős része lefedhető.
Ebből kifolyólag az additív színkeverésen alapuló képreprodukciós eszközök szinte kizárólag három megfelelően megválasztott piros, zöld és kék alapszínnel dolgozik.
Az ezekből a színekből pozitív együtthatókkal (RGB intenzitásokkal) kikeverhető színek összességét egy adott \textbf{eszközfüggő színtérnek} nevezzük.
Ezzel ellentétben a kolorimetrikus, abszolút színterek, mint pl. a CIE XYZ, vagy Luv, Lab színterek ún. \textbf{eszközfüggetlen színtereknek}.
Továbbá az adott eszközfüggő színtérben reprodukálható különböző színezetű színek az $xy$-színpatkóban felvett háromszögét a színtér \textbf{gamutjának} nevezzük.
Egy egyszerű példa színterek gamutjára a \ref{Fig:gamut} ábrán látható.

Amennyiben egy RGB színtér teljesen ismert\footnote{Természetesen nem csak RGB színterek léteznek, nyomdatechnikában pl a CMYK eszközfüggő színterek elterjedtek, amelyek esetében a négy alapszín a nyomdában alkalmazott tinták színét jelzi.
A következőekben a vizsgálatunkat kizárólag RGB színterekre végezzük el.}, tetszőleges $C$ színre meghatározhatóak azok az RGB intenzitások, amelyekkel az RGB alapszíneket súlyozva megkaphatjuk a $C$ színt (amennyiben az RGB értékek pozitívak).
Ezek az adott $C$ szín RGB koordinátái.

Vizsgáljuk most, hogyan szokás egy adott eszközfüggő (RGB) színteret definiálni a gyakorlatban, azaz hogyan kell megadni a színteret ahhoz, hogy ezután tetszőleges szín RGB koordinátái számíthatók legyenek.

\paragraph{Eszközfüggő színterek definíciója:\\}

Tekintsünk egy három alapszínt alkalmazó RGB színteret.
Az R, G és B alapszínek természetesen egy-egy vektorként találhatóak meg az $XYZ$ koordinátarendszerben, és vetületük/metszéspontjuk az egységsíkkal adja meg a színpatkón vett $xy$-koordinátáikat.
Ezt illusztrálja a \ref{Fig:device_dep} ábra.
Az alapszín-vektorok $XYZ$ koordinátáit jelölje rendre 
\begin{equation}
\mathbf{r}_{XYZ} = \begin{bmatrix}
       X_r \\[0.3em]
       Y_r \\[0.3em]
       Z_r \end{bmatrix}, \hspace{4mm}
\mathbf{g}_{XYZ} = \begin{bmatrix}
       X_g \\[0.3em]
       Y_g \\[0.3em]
       Z_g \end{bmatrix}, \hspace{4mm}
\mathbf{b}_{XYZ} = \begin{bmatrix}
       X_b \\[0.3em]
       Y_b \\[0.3em]
       Z_b \end{bmatrix}
\end{equation}

\begin{figure}[]
	\centering
	\begin{overpic}[width = 0.75\columnwidth ]{figures/device_dep.png}
	\small
	\put(89,19){$X$}
	\put(12,96){$Y$}
	\put(0,4){$Z$}
	\put(36,64){$(X_g,Y_g,Z_g)$}
	\put(10,8){$(X_b,Y_b,Z_b)$}
	\put(39,33){$(X_r,Y_r,Z_r)$}
	\end{overpic}
	\caption{RGB színtér alapszíneinek helye, és metszéspontja az egységsíkkal az XYZ színtérben.}
	\label{Fig:device_dep}
\end{figure}
Amennyiben a három alapszín $XYZ$ koordinátái ismertek, úgy a színtér teljesen definiálva van:
tetszőleges $\mathbf{c}_{XYZ}$ színvektor koordinátái meghatározhatóak az adott eszközfüggő $RGB$ térben, amely $\mathbf{c}_{RGB}$ vektor tehát azt írja le, milyen súlyozással keverhető ki az adott $C$ szín az RGB alapszínekből:
\begin{equation} 
\underbrace{\begin{bmatrix}[c]
       R_c \\[0.3em]
       G_c \\[0.3em]
       B_c \end{bmatrix}}_{\mathbf{c}_{RGB}}
       =
     \mathbf{M}_{X\!Y\!Z \rightarrow R\!G\!B}
      \underbrace{\begin{bmatrix}[c]
       X_c \\[0.3em]
       Y_c \\[0.3em]
       Z_c \end{bmatrix}}_{\mathbf{c}_{X\!Y\!Z}},
\end{equation}
ahol $ \mathbf{M}_{X\!Y\!Z \rightarrow R\!G\!B}$ egy bázistranszformációs mátrix. 
Vice versa, az $RGB$ színtérben adott szín $XYZ$ koordinátái meghatározhatóak 
\begin{equation}
      \underbrace{\begin{bmatrix}[c]
       X_c \\[0.3em]
       Y_c \\[0.3em]
       Z_c \end{bmatrix}}_{\mathbf{c}_{X\!Y\!Z}} = 
     \mathbf{M}_{R\!G\!B \rightarrow X\!Y\!Z}
\underbrace{\begin{bmatrix}[c]
       R_c \\[0.3em]
       G_c \\[0.3em]
       B_c \end{bmatrix}}_{\mathbf{c}_{RGB}}
\end{equation}
egyenletből.
Természetesen fennáll a $\mathbf{M}_{R\!G\!B \rightarrow X\!Y\!Z} = \mathbf{M}_{X\!Y\!Z \rightarrow R\!G\!B}^{-1}$ összefüggés.

Utóbbi transzformációs mátrix egyszerűen meghatározható elemi lineáris algebra ismereteinkkel:
$\mathbf{M}_{R\!G\!B \rightarrow X\!Y\!Z}$  mátrix oszlopai egyszerűen az $RGB$ színtér bázisainak $XYZ$-ben vett reprezentációja, azaz általánosan igaz
\begin{equation}
\begin{bmatrix}[c]
       X_c \\[0.3em]
       Y_c \\[0.3em]
       Z_c \end{bmatrix}
       = 
       \underbrace{
  \begin{bmatrix}[c|c|c]
   X_r & X_g & X_b  \\
   Y_r & Y_g & Y_b \\
   Z_r & Z_g & Z_b  \\
\end{bmatrix}}_{\mathbf{M}_{R\!G\!B \rightarrow X\!Y\!Z}}
\cdot
\begin{bmatrix}[c]
       R_c \\[0.3em]
       G_c \\[0.3em]
       B_c \end{bmatrix}
\label{Eq:CS_transform}
\end{equation}
összefüggés\footnote{Az összefüggés érvényessége könnyen tesztelhető pl. $\mathbf{c}_{RGB} = \begin{bmatrix}[c]
       1 \\[0.3em]
       0 \\[0.3em]
       0 \end{bmatrix}$ helyettesítéssel, amely az $R$ alapszín $RGB$-ben vett reprezentációja, és \eqref{Eq:CS_transform} egyenletben a transzformációs mátrix első oszlopát választja ki.}.

A transzformációs mátrixok több szempontból jelentősek: 
egyrészt lehetővé teszik a különböző színtérkonverziókat (lásd köv. bekezdés), valamint egy adott $RGB$ térben ábrázolt képpont $c_Y$ koordinátája megadja az adott szín relatív fénysűrűségét, azaz világosságát.

Itt jegyezzük meg, hogy az $XYZ$ térben vizsgálva adott $RGB$ bázisvektorokkal a pozitív együtthatókkal kikeverhető színek halmaza egy paralelepipedont feszít ki, azaz adott eszközfüggő $RGB$ színtér az $XYZ$ egy paralelepipedonként ábrázolható.
\begin{figure}[]
	\centering
	\small
	(a)
	\begin{overpic}[width = 0.45\columnwidth ]{figures/device_dep_2.png}
	\small
	\put(-2,5){$Z$}
	\put(89,17){$X$}
	\put(11,97){$Y$}
	\end{overpic}
	(b)
	\begin{overpic}[width = 0.45\columnwidth ]{figures/The-RGB-colour-cube.png}
	\end{overpic}
	\caption{Egy adott $RGB$ színtér ábrázolása az $XYZ$ térben (a) és az RGB kockában (b). Az (a) ábrán szereplő vektorok színe a végpontjukban található szín határozza meg.}
	\label{Fig:device_dep}
\end{figure}
Az $RGB$ együtthatók definíció szerint 0 és 1 között vehetnek fel értékeket.
Ennek megfelelően egy adott $RGB$ térben az ebben a színtérben reprodukálható színek egy kockában helyezkednek el, ahol a kocka 3 origóból induló éle mentén az alkalmazott $RGB$ alapszínek helyezkednek el.
Emiatt az $RGB$ színtereket gyakran RGB kockaként említik.
A transzformációs mátrixok tehát gyakorlatilag olyan lineáris transzformációt valósítanak meg, amelyek a paralelepipedont kockába, és a kockát paralelepipedonba viszik.

\vspace{3mm}
Egy $RGB$ színtér tehát teljes egészében adott, amennyiben az alapszín-vektorok $XYZ$ koordinátái ismertek (ez tehát 9 koordináta ismeretét jelenti).
A gyakorlatban az ilyen definíció helyett az alkalmazott alapszínek színezetét, azaz $xy$-koordinátáit adják meg.
Ennek az oka egyrészt a színtér gamutjának egyszerű ábrázolása (lásd \ref{Fig:gamut} ábra), másrészt a fehér szín konzisztens, $RGB$ színtértől független relatív fénysűrűsége ($Y$ koordinátája).

Definíció szerint egy adott színtér ún. \textbf{fehérpontja} az adott térben elérhető legvilágosabb (legnagyobb fénysűrűségű) pontja, amelyet az alapszínek egyenlő arányú keverékével érhetünk el.
Mivel adott térben a 100\%-os fehér a legvilágosabb elérhető szín, ezért definíció szerint a relatív fénysűrűsége ($Y$ koordinátája) egységnyi.
A 100\%-os fehér tehát hasonlóan az $XYZ$-hez, definíció szerint 
\begin{equation}
\mathbf{w}_{RGB} = \begin{bmatrix}[c]
       1 \\[0.3em]
       1 \\[0.3em]
       1 \end{bmatrix}, \hspace{5mm} \text{és} \hspace{5mm} 
Y_w = 1.
\end{equation}
A \ref{Fig:device_dep} ábrán látható példában a fehér szín vektora a paralelepipedon szürkével jelölt főátlója, ezen vonal mentén helyezkednek el a különböző világosságértékű (árnyalatú) fehér színek.
A fehér szín színezete, azaz $x_w$ és $y_w$ koordinátái ezen vektor egységsíkkal vett döféspontja határozza meg.
Általánosan tehát, definiáltuk az adott $RGB$ tér fehérpontját, amelynek érzékelt színezetét az adott alapszínek határozzák meg.
Ez más szóval a szín akromatikus pontja, amely kijelzőről kijelzőre változhat, az alkalmazott pl. LCD elemek függvényében.

A három alapszín $xy$-koordinátái mellett a fehérpont $x_w$ és $y_w$ koordinátái és $Y_w = 1$ fénysűrűsége már elegendő információ szükség esetén a transzformációs mátrixok meghatározásához.

\paragraph{A fehér színről általában:\\}
Látható tehát, hogy a fehér szín önmagában szubjektív fogalom: adott környezetben a leginkább akromatikus fényingert nevezzük fehérnek, amelynek spektrális sűrűségfüggvénye minél inkább egységnyi (azaz minél több spektrális komponenst tartalmaz), és ezzel analóg módon $RGB$ színtér esetén a színvektora minél közelebb van a csupa-egy vektorhoz.
A fehér fogalom egységesítéséhez bevezettek ún. szabványos megvilágításokat (standard illuminants), amelyet szabványosított $RGB$ színterek esetén előírnak, mint fehérpont.
Ezeknek a szabványos megvilágításoknak a spektrális sűrűségfüggvénye adott (és persze az általa keltett színinger $xy$-koordinátái).
Ilyen szabványos megvilágítások a következők:
\begin{figure}[]
	\centering
	\begin{minipage}[c]{0.6\textwidth}
	\begin{overpic}[width = 0.9\columnwidth ]{figures/1200px-PlanckianLocus.png}
	\end{overpic} \end{minipage}\hfill
	\begin{minipage}[c]{0.4\textwidth}
	\caption{Különböző hőmérsékletű feketetest sugárzók által keltett színek összessége, azaz a Planck görbe.}
	\label{Fig:planck}  \end{minipage}
\end{figure}
\begin{itemize}
\item E fehér: egyenlő energiájú fehér, a CIE XYZ színtér fehérpontja. Kolorimetria szempontjából jelentős, videótechnikában kevésbé fontos a szerepe, mivel a gyakorlatban nem fordul elő olyan fényforrás, amely minden hullámhosszon azonos energiával sugároz.
\item A fehér: a CIE által szabványosított, egyszerű háztartási wolfram-szálas izzó fényét (azzal azonos színérzetet keltő) fényforrás spektruma és színe, $2856~\mathrm{K}$ korrelált színhőmérséklette\footnote{A korrelált színhőmérséklet (correlated color temperature, CCT, $T_{\mathrm{C}}$) azon feketetest sugárzó hőmérsékletét jelzi, amely az emberi szemben a minősítendő fényforrással azonos színérzetet kelt.
A feketetest (hőmérsékleti) sugárzó által keltett színingerek az $xy$ színdiagramon az ún. Planck-görbét járják be, amely a \ref{Fig:planck} ábrán látható.}.
\item B és C fehér: Az A fehérből egyszerű szűréssel nyerhető, napfényt szimuláló megvilágítások.
A B fehér a déli napsütést modellezi $4874~\mathrm{K}$ színhőmérséklettel, míg a C fehér a teljes napra vett átlagos fény színét (spektrumát) modellezi $6774~\mathrm{K}$ színhőmérséklettel.
\item D fehér: szintén a napfény közelítésére alkalmazott megvilágítások sora.
Videótechnika szempontjából a legfontosabb a D65 fehér, amely jelenleg is az UHD formátumok színterének szabványos fehérpontja.
\end{itemize}

\paragraph{Színtér konverziók:\\}
Az eddigiekben látható volt, hogyan definiálható egy eszközfüggő színtér az alapszíneivel.
Ahogy az elnevezés is mutatja, ezek a színterek jellegzetesen adott eszközre érvényesek, pl. egy kamera a beépített $RGB$ szenzorok, egy kijelző az alkalmazott $RGB$ kristályok által meghatározott $RGB$ színtérben dolgoznak.
Emellett léteznek szabványos $RGB$ színterek amely a képi és videotartalom tárolására, továbbítására szolgálnak egységesített, szabványos módon.
A következő bekezdés ezeket a szabványos videószíntereket tárgyalja részletesebben.
%TODO Lab, luv spaces: conversion
Felmerül tehát a természetes igény az egyes színterek közti átjárásra, amelyet \textbf{színtér konverziónak} nevezünk.

A színtérkonverziót az $XYZ$ színtér teszi lehetővé, amely egy eszközfüggetlen, abszolút színtér:
egyes színterek közti konverzió a forrás által létrehozott jelek $XYZ$ színtérbe való transzformációjával, majd ezen reprezentáció a nyelő színterébe való transzformációval történik.
Az $XYZ$ színtér így tehát színterek közti átjárást biztosít, ún. Profile Connection Space-ként működik (hasonlóan pl. a gyakran azonos célra alkalmazott $Lab$ színtérhez).

\begin{figure}[]
	\centering
	\begin{overpic}[width = 1\columnwidth]{figures/cs_conversion.png}
	\small
	\put(1,37){$RGB_{\mathrm{cam}}$}
	\put(35,37.5){$XYZ$}
	\put(67,39){$RGB_{\mathrm{ITU}-709}$}
	\put(13,18){$RGB_{\mathrm{ITU}-709}$}
	\scriptsize
	\put(15,29.25){$\mathbf{M}_{\!R\!G\!B_{\mathrm{c\!a\!m}} \!\!\rightarrow \!\!X\!Y\!Z}$}
	\scriptsize
	\put(49,29.25){$\mathbf{M}_{\!X\!Y\!Z \!\rightarrow \!R\!G\!B_{7\!0\!9}} $}
	\small
	\put(87,29){\parbox{.86in}{MPEG kódolás, műsorszórás, tárolás}}
	\put(52,18){$XYZ$}
	\put(87,17){$RGB_{\mathrm{TV}}$}
	\scriptsize
	\put(32.5,9.5){$\mathbf{M}_{\!R\!G\!B_{\mathrm{7\!0\!9}} \!\!\rightarrow \!\!X\!Y\!Z}$}
	\scriptsize
	\put(66.5,9.6){$\mathbf{M}_{\!X\!Y\!Z \!\rightarrow \!R\!G\!B_{7\!0\!9}} $}	
	\end{overpic} 	
	\caption{Színtér-konverzió folyamatábrája.}
	\label{Fig:cs_conversion}
\end{figure}
Egy tipikus színtér konverziós folyamatot az \ref{Fig:cs_conversion} ábra mutat.
Tegyük fel, hogy adott egy HD kamera által rögzített képanyag, ahol a kamera színterét $RGB_{\mathrm{cam}}$ jelöli.
A HD formátum szabványos színteret alkalmaz, amelyet az ITU-709 ajánlásban rögzítettek (lásd később).
A kamera $RGB$ jeleit tehát az esetleges kódolás és tárolás előtt ebbe a HD színtérbe kell konvertálni.
Ez a konverzió a kamerajelek $XYZ$ térbe, majd innen az ITU-709 színtérbe való konverzióval oldható meg, amely konverziók a megfelelő transzformációsmátrixszal való szorzással valósítható meg:
\begin{equation} 
\begin{bmatrix}[c]
       R_{\mathrm{ITU}-709} \\[0.3em]
       G_{\mathrm{ITU}-709} \\[0.3em]
       B_{\mathrm{ITU}-709} \end{bmatrix}
       =
       \mathbf{M}_{ X\!Y\!Z \rightarrow R\!G\!B_{709} } \cdot 
\left(     \mathbf{M}_{R\!G\!B_{\mathrm{cam}} \rightarrow X\!Y\!Z } \cdot
\begin{bmatrix}[c]
       R_{\mathrm{cam}} \\[0.3em]
       G_{\mathrm{cam}} \\[0.3em]
       B_{\mathrm{cam}} \end{bmatrix} \right)
\end{equation}
Természetesen az egymás utáni két mátrixszorzás összevonható, így a két $RGB$ színtér között közvetlen lineáris leképzés határozható meg.
Ez a transzformáció jellegzetesen már a kamerán belül megvalósul.
%
Hasonlóképp, megjelenítőoldalon a
\begin{equation} 
\begin{bmatrix}[c]
       R_{\mathrm{cam}} \\[0.3em]
       G_{\mathrm{cam}} \\[0.3em]
       B_{\mathrm{cam}} \end{bmatrix}
       =
       \mathbf{M}_{ X\!Y\!Z \rightarrow R\!G\!B_{\mathrm{TV}} } \cdot 
\left(     \mathbf{M}_{R\!G\!B_{709}  \rightarrow X\!Y\!Z } \cdot
\begin{bmatrix}[c]
       R_{\mathrm{ITU}-709} \\[0.3em]
       G_{\mathrm{ITU}-709} \\[0.3em]
       B_{\mathrm{ITU}-709} \end{bmatrix}
 \right)
\end{equation}
transzformációt kell elvégezni.

Ez az egyszerű transzformációs módszer lehetővé teszi egy adott színtérben mért színpontok másik térbe való ábrázolását.
Ugyanakkor felmerül a probléma, hogy nagyobb gamuttal rendelkező színtérből kisebbe való áttérés esetén az új színtérben nem ábrázolható, gamuton kívüli színek negatív, és egynél nagyobb $RGB$ koordinátákkal jelennek meg, míg kisebb gamutú térből való áttérés esetén a nagyobb gamutú tér egy része kihasználatlan marad.
A probléma megoldására a fenti transzformációk mellett az egyes színterek gamutját valamilyen nemlineáris leképzés segítségével lehet egymásra illeszteni (expandálással, kompresszálással).
Ezek az ún. gamut-mapping technikák.

A következőekben az egyes SD, HD és UHD videóformátumok tárolására és továbbítására alkalmazott eszközfüggő színtereket vizsgáljuk.

\paragraph{A videótechnika színterei:\\}

% http://www.displaymate.com/crtvslcd.html
Az első kodifikált színmérő rendszer az NTSC (National Television System Committee) által 1953-ban szabványosított színes-televíziós műsorszóráshoz alkalmazott, az azt létrehozó bizottság után elnevezett NTSC szabvány volt.
A színteret a korabeli foszfortechnológiával létrehozható CRT kijelzők (TV vevők) alapszíneik megfelelően írták elő, így színtérkorrekció vevő oldalon nem volt szükség.
Egy egyszerű példa CRT kijelző alapszíneinek meghatározása a következőekben lesz látható.
A színmérő rendszer C fehérponttal dolgozott, alapszíneit pedig a \ref{tab:ntsc_colorimetry} táblázat mutatja.
Az így kapott gamut az xy ábrán látható.
\begin{table}[h!]
\caption{Az NTSC szabvány színmérőrendszere}
\renewcommand*{\arraystretch}{1}
\label{tab:ntsc_colorimetry}
\begin{center}
\small\addtolength{\tabcolsep}{15pt}
    \begin{tabular}[h!]{ @{}c | | l | l @{} }%\toprule
		&   x  	&    y \\ \hline
    R   &  0.67 &	0.33 \\
    G   &  0.21 &   0.71  \\
    B   & 0.14   &	0.08\\
    C fehér     &  0.310 &	0.316  \\
    \end{tabular}
\end{center}
\end{table}
Az alapszínekből és a fehérpontból meghatározható az $RGB_{\mathrm{NTSC}} \rightarrow XYZ$ transzformációs mátrix, amely alakja általánosan
\begin{equation}
\begin{bmatrix}[c]
       X \\[0.3em]
       Y \\[0.3em]
       Z \end{bmatrix}
       = 
  \begin{bmatrix}[c c c]
   0.60 & 0.17 & 0.2  \\
   0.30 & 0.59 & 0.11 \\
   0 & 0.07 & 1.11
\end{bmatrix}
\cdot
\begin{bmatrix}[c]
       R \\[0.3em]
       G \\[0.3em]
       B \end{bmatrix}_{\mathrm{NTSC}}
\label{Eq:NTSC_transform}
\end{equation}
Az egyenlet második sora kitüntetett szereppel bír: meghatározza, hogy az NTSC színtérben hogyan számítható adott $RGB$ színpont világossága:
\begin{equation}Y_{\mathrm{NTSC}} = 
   0.30R + 0.59G + 0.11 B. 
\label{Eq:NTSC_luminance}
\end{equation}
A világosságjel számítása egészen a HD formátum megjelenése (azaz közel 50 éven keresztül) a fenti egyenlet szerint történt.

\vspace{3mm}
Az foszfortechnológia fejlődésével az újabb megjelenítők egyre inkább feláldozták a széles gamutot (azaz a minél telítettebb alapszínek használatát) a minél nagyobb fényerő érdekében: 
Az alkalmazott foszforok a nagyobb érzékelt világosság (fénysűrűség) érdekében egyre nagyobb sávszélességben sugároztak, így az alapszínek egyre kevésbé telítettek lettek, a gamut tehát csökkent (más szóval: az alapszínek spektruma a Dirac-impulzus helyett---amely teljesen telített spektrálszín lenne---szélesebb görbe lett, így a görbe alatti terület---és ezzel a szín világossága nőtt---de telítettsége csökkent).
Mivel így a megjelenítő gamutja jelentősen eltért az NTSC szabványtól, ezért ez a képernyőn látható színek torzulását eredményezte.
Ennek megoldásául a TV vevőkbe analóg színtérkonverziós áramköröket ültettek, amelyek az NTSC és a megjelenítő saját színtere közti konverziót valósította meg\footnote{Ahogy látni fogjuk a későbbiekben: a vevőkbe már csak a nem-lineárisan Gamma-előtorzított $RGB$ jelek jutottak, ahol az inverz torzítást maga a kijelző hajtotta végre. Emiatt a színtérkonverziót csak Gamma-torzított $R'G'B'$ jeleken tudták végrehajtani, ami azonban a telített színeknél ismét látható színezet és fénysűrűség-hibát okozott.}
Ettől a ponttól tehát a műsorszórás szabványos színtere és a megjelenítők színtere különválnak.

Az európai színes műsorszórásra az EBU (European Broadcasting Union) a PAL (Phase Alternating Line) rendszert vezette be 1963-ban, újradefiniálva a színmérőrendszert, új alapszínekre és D65 fehéret alkalmazva:
\begin{table}[h!]
\caption{A PAL szabvány színmérőrendszere}
\renewcommand*{\arraystretch}{1}
\label{tab:pal_colorimetry}
\begin{center}
\small\addtolength{\tabcolsep}{15pt}
    \begin{tabular}[h!]{ @{}c | | l | l @{} }%\toprule
		&   x  	&    y \\ \hline
    R   &  0.64 &  0.33 \\
    G   &  0.29 &  0.60  \\
    B   & 0.15 & 0.06\\
    D65 fehér     &  0.3127 & 0.3290 	  \\
    \end{tabular}
\end{center}
\end{table}
%
Ez matematikailag helyesen a transzformációs mátrix és a világosságjel számításának módjának megváltozását jelentené.
Praktikussági szempontokból azonban a PAL rendszer az NTSC-vel azonos módon, \eqref{Eq:NTSC_luminance} alapján állítja elő a világosságjelet, mivel a gyakorlatban a különbség alig volt látható \footnote{Ennek oka, hogy a világosságjel átviteltechnológia szempontjából fontos: a kamera és a kijelző is $RGB$ jeleket használ, a világosságjelet, ahogy a következőekben látjuk csak a képanyag átviteléhez számítjuk ki.}.
Az PAL alapszíneit és a világosságjel számításának módját átvette az első digitális videóformátum, az ITU-601-es SD formátum is 1982-ben.

\begin{figure}[]
	\centering
	\begin{overpic}[width = 0.7\columnwidth ]{figures/gamuts.png}
	\end{overpic}
	\caption{Az NTSC, PAL/SD/HD/sRGB és UHD szabványok gamutja az $xy$-színpatkóban.
	Az NTSC jóval nagyobb gamuttal dolgozott, mint a ma is használt HD és sRGB formátumok. Ennek oka, hogy a korai CRT megjelenítők ugyan telítettebb, de ugyanakkor kisebb fénysűrűségű és nagy időállandójú foszforokkal dolgoztak, amivel bár nagy színtartományt tudtak megjeleníteni, de kis fényerővel, és mozgó objektumoknál a képernyőn akaratlanul is nyomokat hagyva.}
	\label{Fig:gamut}
\end{figure}

A HD formátumot az 1990-ben szabványosították az ITU-709-es ajánlás formájában.
Ez ugyanúgy átvette az PAL rendszer alapszíneit, azonban immáron matematikailag precízen, újraszámította a transzformációs mátrixot és a világosságjel együtthatókat, amely tehát HD esetén
\begin{equation}Y_{\mathrm{ITU}-709} = 
   0.2126\,R + 0.7152\,G + 0.0722\,B. 
\label{Eq:NTSC_luminance}
\end{equation}
alapján számítható.
Fontos megjegyezni, hogy az ITU-709 szabvány színmérőrendszerét átvette az sRGB szabvány is, ami a mai napig a számítógépes alkalmazások (és operációs rendszerek) alapértelmezett színteréül szolgál.

Az alkalmazott alapszíneket végül számottevően csak az UHD formátum változtatta meg az ITU-2020 számú ajánlásában 2012-ben.
Az UHD alkalmazásokra a szabvány egy széles gamutú, spektrál-alapszíneket alkalmazó színteret ajánl a \ref{tab:UHDTV_colorimetry} táblázatban látható paraméterekkel. 
\begin{table}[h!]
\caption{Az ITU-2020 szabvány színmérőrendszere}
\renewcommand*{\arraystretch}{1}
\label{tab:UHDTV_colorimetry}
\begin{center}
\small\addtolength{\tabcolsep}{15pt}
    \begin{tabular}[h!]{ @{}c | | l | l @{} }%\toprule
		&   x  	&    y \\ \hline
    R   &  0.708 &	0.292  \\
    G   &  0.17 &	0.797  \\
    B   & 0.131 &	0.046 \\
    D65 fehér     &  0.3127 & 0.3290 	  \\
    \end{tabular}
\end{center}
\end{table}
A szabvány természetesen újradefiniálta a világosság komponens számításának a módját is, amely tehát UHD esetben
\begin{equation}Y_{\mathrm{ITU}-2020} = 
   0.2627\,R + 0.678 \,G + 0.0593\,B 
\label{Eq:UHD_luminance}
\end{equation}
alapján számítható.
A szabvány természetesen nem igényli, hogy az UHD megjelenítők spektrálszíneket legyenek képesek alapszínekként realizálni, a minél szélesebb gamut inkább a jövőbeli technológiák szempontjából ad ajánlást.
A mai konzumer megjelenítők az UHD képanyagot megjelenítés előtt a saját színterükben konvertálják, amely jellegzetesen jóval kisebb a szabvány színterénél.

\paragraph{Példa CRT kijelző eszközfüggő színterére:\\}

Egyszerű példaként az eddig leírtakra vizsgáljuk, hogyan számítható és illusztrálható egy CRT kijelző által megjelenített színek tartománya, röviden rávilágítva a CRT technológia működési elvére is \footnote{Természetesen az itt leírtak változtatás nélkül alkalmazhatók más technológia alapján működő kijelzőkre is, pl. LCD.}.
Bár a CRT technológia kezd egyre inkább eltűnni, néhány évvel ezelőttig a stúdiómonitorok jelentős része még mindig CRT alapon működött köszönhetően a színhű megjelenítésüknek, és a mai LCD megjelenítőkhöz képest is jóval nagyobb statikus kontrasztjuknak.

\begin{figure}[]

	\centering
	\begin{overpic}[width = 0.5\columnwidth ]{figures/1024px-CRT_color_enhanced.png}
	\end{overpic}
	\caption{CRT megjelenítő felépítése.}
	\label{Fig:crt}
\end{figure}

A katódsugárcsöves (CRT) kijelzők sematikus ábrája az \ref{Fig:crt} ábrán látható.
A CRT-k kijelzők működésének alapja három ún. elektronágyú volt, amelyek egy fűtőtt katódból (1) és egy nagyfeszültségre helyezett anódból állt.
A melegítés hatására a katód környezetébe szabad elektronok léptek ki, így egy elektronfelhőt képezve a katód körül.
A katód közelébe helyezett nagyfeszültségű (néhány száz Volt) gyorsítóanód hatására a szabad elektronok az anód felé kezdtek mozogni, egy szabad elektronáramot (2) indítva a vákuumban (ugyanezen az elven működtek a vákuum-diódák, triódák, pentódák, stb. is).
Elegendően nagy anódfeszültség (és további anódok jelenléte) esetén az elektronok jelentős része nem csapódott be a gyorsítóanódra, hanem továbbhaladt.
Ezt az elektronnyalábot elektrosztatikusan és mágnesesen (3) fókuszálták, majd egy vezérelt mágneses eltérítő (4) sorról sorra végigfuttatta azt egy anódfeszültségű-ernyőn (5), azaz a képernyőn.
Színes kijelző esetén természetesen három elektronágyú üzemelt párhuzamosan.
A képernyő felszínét pixelekre bontva képpontonként három különböző foszforral borították (7-8), amely gerjesztés (becsapódó elektronok) hatására bizonyos ideig adott spektrális sűrűségfüggvényű fényt bocsájtott ki\footnote{Ellentétben a fluoreszkáló anyagok csak a gerjesztés fennállásának idején bocsájtanak ki fényt. 
A foszforeszkálás időállandója előnyös, hiszen megfelelően megválasztott foszforok épp egy képidőig bocsájtanak ki fényt, így a kijelzett kép nem fog villogni.
Ugyanakkor a korai kijelzők ezen időállandója túl nagy volt, ezért a gyors mozgások elmosódtak a kijelzett képen.}, realizálva ezzel az $RGB$ alapszíneket.

\begin{figure}[]
	\centering
	\begin{overpic}[width = 0.54\columnwidth]{figures/sony.png}
	\small
	\put(0,0){(a)}
	\end{overpic}
	\begin{overpic}[width = 0.39\columnwidth]{figures/sony_gamut.png}
	\small
	\put(0,0){(b)}
	\end{overpic}
	\begin{overpic}[width = 0.014\columnwidth]{figures/sony_gamut_2.png}
	\end{overpic}
	\caption{CRT megjelenítő foszforai által kibocsájtott sugárzás spektrális sűrűségfüggvénye (a) a megjelenítő gamutja és az adott spektrumok/alapszínek által keltett színérzet, valamint a színtér fehérpontja (b).
	A jobb oldali oszlop bal fele a Sony monitor alapszíneit és fehérpontját, a jobb fele az sRGB színtér alapszíneit és fehérpontját szemlélteti.}
	\label{Fig:sony}
\end{figure}

Tekintsünk példaként egy Sony F520 CRT kijelzőt: 
A kijelző $RGB$ foszforjai gerjesztés hatására a \ref{Fig:sony} (a) ábrán látható spektrális sűrűségfüggvényű (sugársűrűségű) fényt bocsájtanak ki magukban egységnyi felületről, egységnyi térszögbe, azaz rendelkezésre állnak a $L_{e}^R(\lambda)$, $L_{e}^G(\lambda)$ és $L_{e}^B(\lambda)$ függvények.
Ekkor a $\overline{x}(\lambda)$, $\overline{y}(\lambda)$, $\overline{z}(\lambda)$ szabványos $XYZ$ spektrális színösszetevő függvények alkalmazásával a piros alapszín színkoordinátái rendre
\begin{align}
\begin{split}
\overline{X}_R &= \int_{380~\mathrm{nm}}^{780~\mathrm{nm}} L_{e}^R(\lambda) \cdot \overline{x}(\lambda) \mathrm{d} \lambda = 0.0633 \\
\overline{Y}_R &= \int_{380~\mathrm{nm}}^{780~\mathrm{nm}} L_{e}^R(\lambda) \cdot \overline{y}(\lambda) \mathrm{d} \lambda = 0.0373\\
\overline{Z}_R &= \int_{380~\mathrm{nm}}^{780~\mathrm{nm}} L_{e}^R(\lambda) \cdot \overline{z}(\lambda) \mathrm{d} \lambda  = 0.0035 \\
\end{split}
\end{align}
és persze hasonlóan számíthatóak az zöld és kék alapszínek $XYZ$-koordinátái, az integrálok numerikus kiértékelésével.
A színtér fehérpontja definíció szerint az alapszínvektorok egyenlő súlyú összegeként áll elő, azaz
\begin{equation}
\overline{X}_W = \overline{X}_R + \overline{X}_G + \overline{X}_B, \hspace{6mm} 
\overline{Y}_W = \overline{Y}_R + \overline{Y}_G + \overline{Y}_B, \hspace{6mm} 
\overline{Z}_W = \overline{Z}_R + \overline{Z}_G + \overline{Z}_B,
\end{equation}
amelyből az alapszínvektorok pontos hossza meghatározható, hiszen definíció szerint $Y_W = 1$ érvényes (az alapszínvektorok eszerint normálandók).
Így az alapszín-vektorok, és így a színtér alkalmazásához szükséges transzformáció mátrixok a következők:
\begin{align}
\begin{split}
\begin{bmatrix}[c]
       X \\[0.3em]
       Y \\[0.3em]
       Z \end{bmatrix} &= 
     \underbrace{ \begin{bmatrix}[c|c|c]
       0.5646 &  0.2665 &  0.2068 \\[0.3em]
       0.3174 &  0.5992 &  0.0834 \\[0.3em]
       0.0302 &  0.1443 &  1.0539 \end{bmatrix} }_{\mathbf{M}_{R\!G\!B \rightarrow X\!Y\!Z}}
\begin{bmatrix}[c]
       R \\[0.3em]
       G \\[0.3em]
       B \end{bmatrix}_{\mathrm{F}520}
\\ \vspace{1mm} \\
&\mathbf{M}_{X\!Y\!Z \rightarrow   R\!G\!B} = \mathbf{M}_{R\!G\!B \rightarrow X\!Y\!Z}^{-1}
\end{split}
\end{align}
Az alapszínek és a fehérpont színezete ezután
\begin{equation}
x_R = \frac{X_R}{X_R + Y_R + Z_R}, \hspace{1cm} y_R = \frac{Y_R}{X_R + Y_R + Z_R}
\end{equation}
alapján számolható.
Az így meghatározott színtér gamutja a \ref{Fig:sony} ábrán látható, az alapértelmezett számítógépes sRGB színtérrel együtt.

Természetesen jelen esetben a színtér a megjelenítő színterének alapszíneinek ábrázolásához az $XYZ$ térben adott alapszíneket az $RGB$ térbe kell konvertálni.
Jelen dokumentum sRGB színtérben kerül tárolásra (és megjelenítéskor az sRGB színtér az olvasó kijelzőjének saját színterébe transzformálva), így jelen dokumentumban az $XYZ$ koordinátáival adott alapszínek az sRGB térbe való konverzió után kerülhetnek megjelenítésre, amely pl. a vörös alapszínre
\begin{equation}
\begin{bmatrix}[c]
       R_R \\[0.3em]
       G_R \\[0.3em]
       B_R \end{bmatrix}_{\mathrm{sRGB}}
       =
     \mathbf{M}_{X\!Y\!Z \rightarrow R\!G\!B_{\mathrm{sRGB}}}
      \begin{bmatrix}[c]
       X_R \\[0.3em]
       Y_R \\[0.3em]
       Z_R \end{bmatrix} =      
       \begin{bmatrix}[c]
       1.13 \\[0.3em]
       0.25 \\[0.3em]
       -0.02 \end{bmatrix} 
\end{equation}
alakú.
A Sony megjelenítő alapszíneinek sRGB koordinátáira negatív és 1-nél nagyobb $RGB$ értékek is adódnak.
Ez a \ref{Fig:sony} ábrán is látható gamutok közti eltérést tükrözi.

\section{A TV-technika színkülönbségi színtere}

Az előző szakasz bemutatta egy színes képpont ábrázolásának módját adott $RGB$ eszközfüggő színtérben.
A következő felmerülő kérdés, hogy ezekből az RGB jelekből---amelyek tehát egy pontja egy megjelenítendő képelem RGB koordinátáit írja le---hogyan hozhatóak létre a ténylegesen rögzített és továbbított videójelek.

\paragraph{A világosság és színkülönbségi jelek:\\}
A fő oka, hogy a videójeleket nem közvetlenül az RGB jeleknek választották (bár manapság már gyakori a közvetlen RGB ábrázolás) az NTSC bevezetésének idejében a visszafelé kompatibilitás biztosítása volt:
A színes műsorszórás kezdetén a korabeli háztartásokban szinte kizárólag fekete-fehér TV-vevők voltak találhatók.
Természetes volt az igény a már kiépített fekete-fehér műsorszóró rendszerrel visszafelé kompatibilitása színes kép-továbbításra.

A fekete-fehér kép gyakorlatilag egy színes kép világosságinformációjának fogható fel, amely az $RGB$ koordinátákból azok lineáris kombinációjaként számítható.
Az együtthatók az adott eszközfüggő színtértől függenek, az NTSC alapszínei esetén pl. \ref{Eq:NTSC_luminance} alapján adottak.
Mivel a fekete-fehér TV vevők közvetlenül ezt a világosságjelet jelenítették meg, ezért a színes TV esetén is az egyik, változatlanul továbbítandó jelet a \textbf{világosságjelnek (luminance)} választották, amely tehát például NTSC esetén az $RGB$ jelekből
\begin{equation}Y_{\mathrm{NTSC}} = 
   0.30R + 0.59G + 0.11 B. 
   \label{Eq:NTSC_luminance}
\end{equation}
alapján számítható \footnote{Fontos ismét kihangsúlyozni, hogy a világosság-számítás módja színtérfüggő, az alapszínektől és a fehérponttól függ a már bemutatott módon.}.

Egy színes képpont leírásához 3 komponens szükséges, egy lehetséges és hatékony leírás pl. a képpont világossága, színezete és telítettsége.
A világosságjel mellé tehát két független információ kell, amelyek egyértelműen meghatározzák az adott színpont színezetét és telítettségét\footnote{A visszafelé-kompatibilitás biztosításához ezt a két színezetet leíró jelet kellett az NTSC rendszerben a változatlan fekete-fehér jelhez úgy hozzáadni, hogy a meglévő fekete-fehér vevők a világosságjelet demodulálni tudják, és a hozzáadott többletinformáció minimális látható hatással legyen a megjelenített képre.}.
Ugyanakkor fontos szempont volt ezen világosságinformáció-mentes, pusztán színinformációt leíró jelek könnyű számíthatósága az RGB komponensekből az egyszerű analóg áramköri megvalósíthatóság érdekében.

A színinformáció/világosságinformáció-szétválasztás legegyszerűbb (de jól működő) megoldásaként egyszerűen vonjuk ki a világosságot az RGB jelekből.
Mivel az $Y$ együtthatók összege definíció szerint (tetszőleges színtérben) egységnyi, így pl. NTSC esetén \eqref{Eq:NTSC_luminance} mindkét oldalából $Y$-t kivonva igaz a 
\begin{equation} 
   0.30 ( R - Y ) + 0.59 ( G - Y )  + 0.11 ( B - Y )  = 0 
   \label{eq:chrominances}
\end{equation}
egyenlőség.
Az $ ( R - Y ) $, $ ( G - Y ) $ és $ ( B - Y ) $ a TV-technika ún. \textbf{színkülönbségi jelei}, és a következő tulajdonságokkal bírnak:
\begin{itemize}
\item Nem függetlenek egymástól, kettőből számítható a harmadik.
\item Előjeles mennyiségek.
\item Ha két színkülönbségi jel zérus, akkor a harmadik is az.
Ekkor $R = G = B = Y$, így tehát a színtér fehérpontjában vagyunk.
A fehér színre kapott zérus színkülönbségi jelek azt mutatják, hogy a színinformációt valóban a színkülönbségi jelek jelzik, a fénysűrűség (világosság) pedig tőlük független mennyiség.
\item Az adott színkülönbségi jel értéke maximális ha a hozzá tartozó alapszín maximális, és vice versa.
NTSC rendszerben vörös színkülönbségi jelre $R = 1$, $G = B= 0$ esetén
\begin{equation}
Y = 0.30 \cdot 1 + 0.59 \cdot 0 + 0.11 \cdot 0 \hspace{3mm }\rightarrow \hspace{3mm } R - Y  = 0.7,
\end{equation}
és hasonlóan $R=0$, $G = B = 1$ esetén
\begin{equation}
Y = 0.30 \cdot 0+ 0.59 \cdot 1 + 0.11 \cdot 1 \hspace{3mm }\rightarrow \hspace{3mm } R - Y  = -0.7.
\end{equation}
\item A fenti megfontolások alapján a színkülönbségi jelek dinamikatartománya:
\begin{align}
\begin{split}
-0.7 \leq R-Y \leq& 0.7 , \hspace{2cm} -0.89 \leq G-Y \leq 0.89, \\
 &-0.41 \leq B-Y \leq 0.41
\end{split}
\end{align}
\end{itemize}
A három színkülönbségi jelből kettőt kell választani a színpont színinformációjának leírásához.
Mivel jel/zaj-viszony szempontjából ökölszabályszerűen mindig a nagyobb dinamikatartományú jelet célszerű továbbítani, így a választás a vörös és zöld színkülönbségi jelekre esett.

A videótechnikában tehát egy adott színpont ábrázolása a
\begin{align*}
Y&: \text{Luminance }\\
 	\left.\begin{array}{lr}
        R-Y\\
        B-Y
        \end{array}\right\}&: \text{Chrominance}
\end{align*}
ún. \textbf{luminance-chrominance térben} történik, amely gyakorlatilag felfogható egy új színmérőrendszernek is az $RGB$ színtérhez képest.

\paragraph{Az $Y,\,R-Y,\,B-Y$ színtér:\\}
Vizsgáljuk most, hol helyezkednek el az adott $RGB$ eszközfüggő színtérben ábrázolható színek ebben az új, $Y,\, R-Y,\, B-Y$ térben!
Az előzőekben láthattuk, hogy az $XYZ$ térben ez a színhalmaz egy paralelepipedont, az $RGB$ térben egy egységnyi oldalú kockát jelent (lásd \ref{Fig:device_dep} ábra).
Vegyük észre, hogy a $Y,\, R-Y,\, B-Y$ koordinátákat akár az $XYZ$, akár az $RGB$ komponensekből egy lineáris transzformációval előállíthatjuk:
Maradva az NTSC rendszer világosság-együtthatóinál (kiindulva abból, hogy $Y = 0.3R + 0.59G + 0.11B$) a transzformáció alakja
\begin{align}
\begin{bmatrix}[c]
       R- Y \\[0.3em]
       B - Y \\[0.3em]
       Y \end{bmatrix} &= 
\begin{bmatrix}[c c c]
      0.7 &  -0.59&  -0.11  \\[0.3em]
       -0.3 &  -0.59 & 0.89  \\[0.3em]
      0.3 &  0.59&  0.11 \end{bmatrix} 
\begin{bmatrix}[c]
       R \\[0.3em]
       G \\[0.3em]
       B \end{bmatrix}_{\mathrm{NTSC}}.
\end{align}
A lineáris transzformációt az RGB kockán elvégezve megkaphatjuk az ábrázolható színek halmazát.
Az így kapott test az \ref{Fig:YCbCr_space} (a) ábrán látható.
Láthatjuk, hogy az $RGB$ egységkocka egy paralelepipedonba transzformálódott, ahol a paralelepipedon főátlója az $Y$ világosság tengely.
Ennek mentén, az $R-Y = B-Y = 0$ tengelyen helyezkednek el a különböző szürke árnyalatok. 
\begin{figure}[]
	\centering
	\begin{overpic}[width = 0.45\columnwidth ]{figures/YC_space_1.png}
	\small
	\put(0,0){(a)}
	\put(45,90){$Y$}
	\put(48,2){$R\!-\!Y$}
	\put(87,26){$B\!-\!Y$}
	\end{overpic}
	\hspace{6mm}
	\begin{overpic}[width = 0.48\columnwidth ]{figures/YC_space_2.png}
	\small
	\put(0,0){(b)}
	\scriptsize
	\put(39,82){$R$}
	\put(25,24){$G$}
	\put(89,44){$B$}
	\put(12,58){$Y\!e$}
	\put(65,19){$C\!y$}
	\put(78,77){$M\!g$}
	\end{overpic}
	\caption{Az $Y, R-Y, B-Y$ színtér ábrázolható színeinek halmaza oldalnézetből (a) és felülnézetből (b).}
	\label{Fig:YCbCr_space}
\end{figure}

Az eredeti RGB kockához hasonlóan, paralelepipedon főátlón kívüli csúcsaiban (amelyben az $Y=0$ fekete és az $R=G=B=Y=1$ fehér található) az eszközfüggő színtér egy, vagy két $100~\%$-os intenzitású alapszínnel kikeverhető
\begin{equation}
R = \begin{bmatrix}[c] 1\\[0.3em] 0\\[0.3em] 0\end{bmatrix} \hspace{2mm}
G = \begin{bmatrix}[c] 0\\[0.3em] 1\\[0.3em] 0\end{bmatrix}\hspace{2mm}
B = \begin{bmatrix}[c] 0\\[0.3em] 0\\[0.3em] 1\end{bmatrix}\hspace{2mm}
Cy = \begin{bmatrix}[c] 0\\[0.3em] 1\\[0.3em] 1\end{bmatrix}\hspace{2mm}
Mg = \begin{bmatrix}[c] 1\\[0.3em] 1\\[0.3em] 0\end{bmatrix}\hspace{2mm}
Ye = \begin{bmatrix}[c] 1\\[0.3em] 1\\[0.3em] 0\end{bmatrix}
\end{equation}
vörös, zöld, kék alap- és cián, magenta, sárga ún. komplementer színek találhatóak.

Ezen komplementer színek tulajdonsága, hogy az egyes RGB alapszínekkel RGB kockában átlósan helyezkednek el, így a színtérben a lehető legmesszebb elhelyezkedő színpárokat alkotják.
Ennek megfelelően egymás mellé vetítve a komplementer színpárok (vörös-cián, sárga-kék, zöld-magenta) váltják ki a legnagyobb érzékelt kontrasztot.

A paralelepipedonra az $Y$-tengely irányából ránézve (\ref{Fig:YCbCr_space} (b) ábra) láthatjuk a világosságjeltől függetlenül, adott színtérben kikeverhető színek összességét.
Az $R-Y, B-Y, Y$ térben gyakori adott $Y$ világosság mellett a színek ezen $R-Y, B-Y$ síkon való ábrázolása.
Minthogy az $R-Y, B-Y$ jelek meghatározzák adott színpont színezetét és telítettségét, így az ábra azt jelzi, hogy a különböző színezetű és telítettségű színek egy szabályos hatszöget töltenek ki.
A hatszög csúcsai a színtér alap- és komplementerszínei.
Természetesen adott $Y$ érték mellett az ábrázolható színek nem tölti ki teljesen ezt a hatszöget:
adott világosságérték mellett az ábrázolható színek halmaza a $Y, R-Y, B-Y$ paralelepipedon egy adott $Y$ magasságban húzott síkkal vett metszeteként képzelhető el, azaz tetszőleges $0 \leq Y \leq1$ esetén rajzolható egy $R-Y, B-Y$ diagram.
Az így rajzolható diagramokra példákat a \ref{Fig:YCbCr_sect} ábra mutat.
\begin{figure}[]
	\centering
	\begin{overpic}[width = 1\columnwidth ]{figures/YCbCr_2_11.png}
	\small
	\put(0,3){(a)}
	\put(0,37){$Y = 0.11$}
	\end{overpic}
	\vspace{2mm}
	\begin{overpic}[width = 1\columnwidth]{figures/YCbCr_2_30.png}
	\small
	\put(0,37){$Y = 0.3$}
	\put(0,3){(b)}
	\end{overpic}
	\vspace{2mm}
	\begin{overpic}[width = 1\columnwidth]{figures/YCbCr_2_59.png}
	\small
	\put(0,37){$Y = 0.59$}
	\put(0,3){(c)}
	\end{overpic}
	\caption{Különböző $Y$ értékek mellett rajzolható $B-Y, R-Y$ diagramok.}
	\label{Fig:YCbCr_sect}
\end{figure}
Nyilván rögzített $Y$ mellett nem biztos, hogy minden szín $100~\%$-os telítettséggel van jelen a $B-Y,R-Y$ diagramon. 
Például: teljesen telített kékre ($\begin{bmatrix}[c] 0\\[0.3em] 0\\[0.3em] 1\end{bmatrix}$) $Y=0.11$, azaz a $100~\%$ intenzitású kék alapszín ezen magasságban vett diagramon található.
Más magasságban vett  $B-Y, R-Y$ diagramon csak fehérrel higított kék található, azaz nem teljesen telített kék található.

A vizsgált diagramokból leszűrhető, hogy valóban, a világosságjel független a színinformációtól, adott színpont színezetét és telítettségét pusztán az $R-Y$ és $B-Y$ diagramokon vett helye meghatározza.
Vizsgáljuk most, hogyan definiálhatóak ezen érzeti jellemzők, azaz a színezet és telítettség a TV technika $Y, R-Y, B-Y$ színterében!

A könnyebb elképzelhetőség kedvéért ábrázoljuk az $R-Y, B-Y$ koordinátákhoz tartozó színeket, az adott színponthoz tartozó olyan világosságérték mellett, amely esetén minden pontra teljesül, hogy $X+Y+Z = 1$, azaz ezzel gyakorlatilag az adott $RGB$ színtér $xy$-színpatkón vett színét képezzük le az $R-Y, B-Y$ diagramra.
\begin{figure}[]
	\centering
	\begin{minipage}[c]{0.6\textwidth}
	\begin{overpic}[width = 1\columnwidth ]{figures/YCbCr_gamut.png}
	\small
	\put(56,46){$\alpha$}
	\end{overpic} \end{minipage}\hfill
	\begin{minipage}[c]{0.4\textwidth}
	\caption{Adott $Y, R-Y, B-Y$ térben ábrázolható színek gamutja.}
	\label{Fig:ycbcr_gamut}  \end{minipage}
\end{figure}
Az így kapott színhalmaz, amely felfogható az adott alapszínek mellett a luminance-chrominance tér gamutjának is, a \ref{Fig:ycbcr_gamut} ábrán látható.

Megfigyelhető, hogy a diagramon az origóból kiinduló félegyenesen azok a színek vannak, amelyek egymásból kinyerhetők fehér szín hozzáadásával.
Tehát az origóból kiinduló félegyenesen az azonos színezetű, de eltérő telítettségű színek vannak. 
Azaz tetszőleges színpontot vizsgálva, a $B-Y,R-Y$ diagramon a színpontba mutató helyvektor iránya egyértelműen meghatározza az adott pont színezetét.
Ennek megfelelően a TV technikában a színezetet a $B-Y, R-Y$ diagramon a színpont helyvektorának irányszögeként definiáljuk:
\begin{equation}
\text{színezet}_{\mathrm{TV}} = \alpha  = \arctan \frac{R-Y}{B-Y}
\end{equation}

A telítettség kifejezése már kevésbé egyértelmű, több definíció bevezethető rá.
Mindkét esetben a telítettség természetesen azt fejezi ki, mennyi fehér hozzáadásával keverhető ki egy adott szín a színezetét meghatározó teljesen telített alapszínből.
Az $XYZ$-térben bevezettük a telítettségre a színtartalmat, illetve színsűrűséget.
Mindkét telítettségdefiníció zérus értékű volt a $C$-fehérre, és egységnyi a színpatkót határoló spektrálszínekre.
Felmerül a kérdés, hogyan terjeszthető ki a telítettség fogalma eszközfüggő $RGB$-színterekre.

Ehhez bevezethetjük az \textbf{kávzi-spektrálszínek} fogalmát, amelyek adott $RGB$ színtérben előállítható legtelítettebb színek (a legközelebb vannak az azonos színezetű valódi spektrálszínhez).
Ennek megfelelően a kvázi-spektrálszínek az $xy$-diagramon az adott $RGB$ színtér gamutjának határán helyezkednek el, tehát kikeverhetőek legfeljebb két alapszínből.
Hasonlóképp, \ref{Fig:ycbcr_gamut} diagramon a színteret határoló hatszög csúcsaiban és oldalin találhatóak.

A telítettség ezek után a következő módokon definiálható.
\begin{itemize}
\item  Minthogy egy tetszőleges színnek a fehér színtől, azaz az origótól vett távolsága arányos a szín fehér-tartalmával, így legegyszerűbb módon a telítettség közelíthető a
\begin{equation}
\text{telítettség}_{\mathrm{TV},1} = \sqrt{ (R-Y)^2 +(B-Y)^2}
\label{eq:saturation_1}
\end{equation}
távolsággal.
Később tárgyalt okok miatt az analóg időkben TV technikusok körében ez a definíció volt érvényben.
Az így számolt telítettség valóban $0$ a fehér színre, azonban a kvázi-spektrálszínek telítettsége így nem egységnyi.
%
\item A matematikailag korrekt telítettség-definíció bevezetéséhez kiterjeszthetjük a korábban megismert színsűrűséget eszközfüggő színterekre \footnote{Ismétlésként: az $XYZ$ térben adott pont színsűrűsége $p_c = \frac{Y_d}{Y}$, ahol $Y_d$ az adott színhez tartozó domináns hullámhosszú szín fénysűrűsége, $Y$ a vizsgált szín saját fénysűrűsége.}.
Ennek egyszerűbb értelmezéséhez ábrázoljuk adott színpont paramétereit ún. területdiagramon!
%
\begin{figure}[]
	\centering
	\begin{minipage}[c]{0.6\textwidth}
	\begin{overpic}[width = 1\columnwidth ]{figures/area_chart.png}
	\end{overpic} \end{minipage}\hfill
	\begin{minipage}[c]{0.4\textwidth}
	\caption{Tetszőlegesen választott $R,G,B$ koordináták esetén rajzolható területdiagram.}
	\label{Fig:area_diagram}  \end{minipage}
\end{figure}
%
A területdiagram a következő módon rajzolható fel egy tetszőleges $RGB$ koordinátáival adott szín esetén: 
A vízszintes tengelyt osszuk fel az $Y$ fénysűrűség $RGB$ együtthatóinak megfelelően, majd az egyes $RGB$ komponenseket ábrázoljuk az intenzitásuknak megfelelő magasságú oszlopokkal.
Ekkor egy $Y$ magasságban húzott vonal alatt és fölött a színkülönbségi jeleknek megfelelő magasságú oszlopok alakulnak ki, amely oszlopok előjelesen vett területeinek összege \eqref{eq:chrominances} alapján zérus.

Válasszuk ki ezután a legkisebb $RGB$ komponenst (a \ref{Fig:area_diagram} ábrán látható példában az $R$) és húzzunk egy vízszintes vonalat ennek magasságában!
Ekkor a vizsgált színt két részre osztottuk: egy fehér színre (amelyre $R=G=B$) és egy kvázi-spektrálszínre, amelynek az egyik $RGB$ komponense zérus, és amelynek fénysűrűsége $Y_d = \min (R,G,B) - Y$.
A domináns hullámhosszú spektrálszín szerepét erre a kvázi-spektrálszínre cserélve kiterjeszthetjük a színsűrűséget az adott eszközfüggő színtérre, amely alapján a telítettség definíciója
\begin{equation}
\text{telítettség}_{\mathrm{TV},2} = \frac{| \min(R,G,B) - Y |}{Y}.
\label{eq:saturation_2}
\end{equation}
Könnyen belátható, hogy az $R = G=B=Y$ fehérpontokra a telítettség definíció szerint 0, míg kvázi-spektrálszínekre ($\min(R,G,B) = 0$) a telítettség azonosan 1.
\end{itemize}
A fent tárgylat két telítettség-definíció alkalmazásával a \ref{Fig:ycbcr_gamut} ábrán látható színek telítettségét az xy ábra szemlélteti, megerősítve az eddig elmondottakat.
\begin{figure}[]
	\centering
	\begin{overpic}[width = 1\columnwidth ]{figures/YCbCr_saturation.png}
	\small
	\put(0,0){(a)}
	\put(50,0){(b)}
	\end{overpic}
	\caption{Az $R-Y,B-Y$ térben ábrázolt színek telítettsége \eqref{eq:saturation_1} (a) és \eqref{eq:saturation_2} (b) alapján számolva}
	\label{Fig:area_diagram}  
\end{figure}
%

\section{Videójel komponensek}

\section{A digitalizálás kérdései}