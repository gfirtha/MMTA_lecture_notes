Az előző fejezet bemutatta az emberi látás képi reprodukció szempontjából legfontosabb tulajdonságait és részletesen tárgyalta a fény- és színmérés alapjait, bevezetve a világosság fogalmát és a CIE XYZ színteret.
Ez a fejezet ezekre az ismeretekre építve bemutatja a televíziós-technikában használt színes-képpont ábrázolás módját, 
ez alapján bevezetve a jelenleg is alkalmazott analóg és digitális videójel komponenseket.

Videótechnika szempontjából az XYZ színteret ritkán alkalmazzák képpontok színkoordinátáinak tárolására, kivétel ez alól a digitális mozi és mozifilm-archiválási alkalmazások \footnote{Ennek oka, hogy egyrészt reprodukcióra közvetlenül nem használható, hiszen az XYZ alapszínek nem valós színek (az X,Y,Z bázisvektorok helyén nem található látható szín), másrészt a teljes látható színek tartománya igen nagy bitmélységet igényel, ráadásul feleslegesen:
Az XYZ tér pozitív térnyolcadát a látható színek csak részben töltik ki (sok olyan kód lenne, amihez nem tartozik látható szín), ráadásul a ezen belül is a megjelenítők a látható színeknek csak egy részét képesek reprodukálni.}.
Ugyanakkor az XYZ tér lehetővé teszi a különböző megjelenítők és kamerák által reprodukálható színek halmazának egyszerű vizsgálatát.
A következő szakasz ezeket a konkrét videóeszközökre jellemző, ún. \textbf{eszközfüggő színtereket} mutatja be.

\section{Eszközfüggő színterek}

Az előző fejezetben láthattuk, hogy az emberi látás trikromatikus jellegének, valamint linearitásának (illetve az egyszerű lineáris modellnek) köszönhetően a látható színek egy lineáris 3D vektortérben ábrázolhatóak, amelyben a vektorok összegzési szabálya érvényes: 
Két tetszőleges szín keverékéből származó eredő színinger meghatározható a két színbe mutató helyvektorok összegeként (függetlenül az eredeti színingereket létrehozó fény spektrumától).
Az $xy$-színpatkón ennek megfelelően két szín összege a két színpontot összekötő szakasz mentén fog elhelyezkedni.

Ebből következik, hogy az emberi látás metamerizmusát kihasználva, a látható színek nagy része előállítható mesterségesen, megfelelően megválasztott alapszínek összegeként.
Ez általánosan véve a színes képreprodukció alapja.
Természetesen nem lehet célunk az összes látható szín visszaállítása: 
Minthogy a színpatkót a spektrálszínek határolják, így elvben végtelen számú spektrálszínt kéne alapszínként alkalmazni az összes látható szín kikeveréséhez.
Felmerül tehát a kérdés, hány alapszín szükséges a színpatkó megfelelő lefedéséhez.

\begin{figure}[]
	\centering
	\begin{overpic}[width = 1\columnwidth ]{figures/color_space_gamut.png}
	\end{overpic}
	\caption{Az azonos alapszínekkel dolgozó SD formátum, HD formátum és az sRGB színtér gamutja $xy$ és $uv$ diagramon ábrázolva.}
	\label{Fig:gamut}
\end{figure}

A színdiagramban könnyen felvehető 4 színpont úgy, hogy a négy szín keverékeit lefedő négyszög (azaz a reprodukálható színek területe) csaknem azonos területű legyen a látható színek területével.
Ugyanakkor az $Luv$ színtér színpatkójából láthattuk, hogy az emberi felbontás zöld árnyalatokra vonatkozó felbontása rossz, és az perceptuálisan egyenletes színdiagram inkább háromszög alakú.
Ez azt jelenti, hogy három megfelelően megválasztott alapszínnel---amelynek különböző arányú keverékeinek színezete egy háromszögön belül helyezkedik el---az egyenletes színezetű ($uv$) színpatkó jelentős része lefedhető.
Ebből kifolyólag az additív színkeverésen alapuló képreprodukciós eszközök szinte kizárólag három megfelelően megválasztott piros, zöld és kék alapszínnel dolgozik.
Az ezekből a színekből pozitív együtthatókkal (RGB intenzitásokkal) kikeverhető színek összességét egy adott \textbf{eszközfüggő színtérnek} nevezzük.
Ezzel ellentétben a kolorimetrikus, abszolút színterek, mint pl. a CIE XYZ, vagy Luv, Lab színterek ún. \textbf{eszközfüggetlen színtereknek}.
Továbbá az adott eszközfüggő színtérben reprodukálható különböző színezetű színek az $xy$-színpatkóban felvett háromszögét a színtér \textbf{gamutjának} nevezzük.
Egy egyszerű példa színterek gamutjára a \ref{Fig:gamut} ábrán látható.

Amennyiben egy RGB színtér teljesen ismert\footnote{Természetesen nem csak RGB színterek léteznek, nyomdatechnikában pl a CMYK eszközfüggő színterek elterjedtek, amelyek esetében a négy alapszín a nyomdában alkalmazott tinták színét jelzi.
A következőekben a vizsgálatunkat kizárólag RGB színterekre végezzük el.}, tetszőleges $C$ színre meghatározhatóak azok az RGB intenzitások, amelyekkel az RGB alapszíneket súlyozva megkaphatjuk a $C$ színt (amennyiben az RGB értékek pozitívak).
Ezek az adott $C$ szín RGB koordinátái.

Vizsgáljuk most, hogyan szokás egy adott eszközfüggő (RGB) színteret definiálni a gyakorlatban, azaz hogyan kell megadni a színteret ahhoz, hogy ezután tetszőleges szín RGB koordinátái számíthatók legyenek.

\paragraph{Eszközfüggő színterek definíciója:\\}

Tekintsünk egy három alapszínt alkalmazó RGB színteret.
Az R, G és B alapszínek természetesen egy-egy vektorként találhatóak meg az $XYZ$ koordinátarendszerben, és vetületük/metszéspontjuk az egységsíkkal adja meg a színpatkón vett $xy$-koordinátáikat.
Ezt illusztrálja a \ref{Fig:device_dep} ábra.
Az alapszín-vektorok $XYZ$ koordinátáit jelölje rendre 
\begin{equation}
\mathbf{r}_{XYZ} = \begin{bmatrix}
       X_r \\[0.3em]
       Y_r \\[0.3em]
       Z_r \end{bmatrix}, \hspace{4mm}
\mathbf{g}_{XYZ} = \begin{bmatrix}
       X_g \\[0.3em]
       Y_g \\[0.3em]
       Z_g \end{bmatrix}, \hspace{4mm}
\mathbf{b}_{XYZ} = \begin{bmatrix}
       X_b \\[0.3em]
       Y_b \\[0.3em]
       Z_b \end{bmatrix}
\end{equation}

\begin{figure}[]
	\centering
	\begin{overpic}[width = 0.75\columnwidth ]{figures/device_dep.png}
	\small
	\put(89,19){$X$}
	\put(12,96){$Y$}
	\put(0,4){$Z$}
	\put(36,64){$(X_g,Y_g,Z_g)$}
	\put(10,8){$(X_b,Y_b,Z_b)$}
	\put(39,33){$(X_r,Y_r,Z_r)$}
	\end{overpic}
	\caption{RGB színtér alapszíneinek helye, és metszéspontja az egységsíkkal az XYZ színtérben.}
	\label{Fig:device_dep}
\end{figure}
Amennyiben a három alapszín $XYZ$ koordinátái ismertek, úgy a színtér teljesen definiálva van:
tetszőleges $\mathbf{c}_{XYZ}$ színvektor koordinátái meghatározhatóak az adott eszközfüggő $RGB$ térben, amely $\mathbf{c}_{RGB}$ vektor tehát azt írja le, milyen súlyozással keverhető ki az adott $C$ szín az RGB alapszínekből:
\begin{equation} 
\underbrace{\begin{bmatrix}[c]
       R_c \\[0.3em]
       B_c \\[0.3em]
       G_c \end{bmatrix}}_{\mathbf{c}_{RGB}}
       =
     \mathbf{M}_{X\!Y\!Z \rightarrow R\!G\!B}
      \underbrace{\begin{bmatrix}[c]
       X_c \\[0.3em]
       Y_c \\[0.3em]
       Z_c \end{bmatrix}}_{\mathbf{c}_{X\!Y\!Z}},
\hspace{3mm}
\end{equation}
ahol $ \mathbf{M}_{X\!Y\!Z \rightarrow R\!G\!B}$ egy bázistranszformációs mátrix. 
Vice versa, az $RGB$ színtérben adott szín $XYZ$ koordinátái meghatározhatóak 
\begin{equation}
      \underbrace{\begin{bmatrix}[c]
       X_c \\[0.3em]
       Y_c \\[0.3em]
       Z_c \end{bmatrix}}_{\mathbf{c}_{X\!Y\!Z}} = 
     \mathbf{M}_{R\!G\!B \rightarrow X\!Y\!Z}
\underbrace{\begin{bmatrix}[c]
       R_c \\[0.3em]
       B_c \\[0.3em]
       G_c \end{bmatrix}}_{\mathbf{c}_{RGB}}
\end{equation}
egyenletből.
Természetesen fennáll a $\mathbf{M}_{R\!G\!B \rightarrow X\!Y\!Z} = \mathbf{M}_{X\!Y\!Z \rightarrow R\!G\!B}^{-1}$ összefüggés.

Utóbbi transzformációs mátrix egyszerűen meghatározható elemi lineáris algebra ismereteinkkel:
$\mathbf{M}_{R\!G\!B \rightarrow X\!Y\!Z}$  mátrix oszlopai egyszerűen az $RGB$ színtér bázisainak $XYZ$-ben vett reprezentációja, azaz általánosan igaz
\begin{equation}
\begin{bmatrix}[c]
       X_c \\[0.3em]
       Y_c \\[0.3em]
       Z_c \end{bmatrix}
       = 
       \underbrace{
  \begin{bmatrix}[c|c|c]
   X_r & X_g & X_b  \\
   Y_r & Y_g & Y_b \\
   Z_r & Z_g & Z_b  \\
\end{bmatrix}}_{\mathbf{M}_{R\!G\!B \rightarrow X\!Y\!Z}}
\cdot
\begin{bmatrix}[c]
       R_c \\[0.3em]
       G_c \\[0.3em]
       B_c \end{bmatrix}
\label{Eq:CS_transform}
\end{equation}
összefüggés\footnote{Az összefüggés érvényessége könnyen tesztelhető pl. $\mathbf{c}_{RGB} = \begin{bmatrix}[c]
       1 \\[0.3em]
       0 \\[0.3em]
       0 \end{bmatrix}$ helyettesítéssel, amely az $R$ alapszín $RGB$-ben vett reprezentációja, és \eqref{Eq:CS_transform} egyenletben a transzformációs mátrix első oszlopát választja ki.}.

A transzformációs mátrixok több szempontból jelentősek: 
egyrészt lehetővé teszik a különböző színtérkonverziókat (lásd köv. bekezdés), valamint egy adott $RGB$ térben ábrázolt képpont $c_Y$ koordinátája megadja az adott szín relatív fénysűrűségét, azaz világosságát.

Itt jegyezzük meg, hogy az $XYZ$ térben vizsgálva adott $RGB$ bázisvektorokkal a pozitív együtthatókkal kikeverhető színek halmaza egy paralelepipedont feszít ki, azaz adott eszközfüggő $RGB$ színtér az $XYZ$ egy paralelepipedonként ábrázolható.
\begin{figure}[]
	\centering
	\small
	(a)
	\begin{overpic}[width = 0.45\columnwidth ]{figures/device_dep_2.png}
	\small
	\put(-2,5){$Z$}
	\put(89,17){$X$}
	\put(11,97){$Y$}
	\end{overpic}
	(b)
	\begin{overpic}[width = 0.45\columnwidth ]{figures/The-RGB-colour-cube.png}
	\end{overpic}
	\caption{Egy adott $RGB$ színtér ábrázolása az $XYZ$ térben (a) és az RGB kockában (b). Az (a) ábrán szereplő vektorok színe a végpontjukban található szín határozza meg.}
	\label{Fig:device_dep}
\end{figure}
Az $RGB$ együtthatók definíció szerint 0 és 1 között vehetnek fel értékeket.
Ennek megfelelően egy adott $RGB$ térben az ebben a színtérben reprodukálható színek egy kockában helyezkednek el, ahol a kocka 3 origóból induló éle mentén az alkalmazott $RGB$ alapszínek helyezkednek el.
Emiatt az $RGB$ színtereket gyakran RGB kockaként említik.
A transzformációs mátrixok tehát gyakorlatilag olyan lineáris transzformációt valósítanak meg, amelyek a paralelepipedont kockába, és a kockát paralelepipedonba viszik.

\vspace{3mm}
Egy $RGB$ színtér tehát teljes egészében adott, amennyiben az alapszín-vektorok $XYZ$ koordinátái ismertek (ez tehát 9 koordináta ismeretét jelenti).
A gyakorlatban az ilyen definíció helyett az alkalmazott alapszínek színezetét, azaz $xy$-koordinátáit adják meg.
Ennek az oka egyrészt a színtér gamutjának egyszerű ábrázolása (lásd \ref{Fig:gamut} ábra), másrészt a fehér szín konzisztens, $RGB$ színtértől független relatív fénysűrűsége ($Y$ koordinátája).

Definíció szerint egy adott színtér ún. \textbf{fehérpontja} az adott térben elérhető legvilágosabb (legnagyobb fénysűrűségű) pontja, amelyet az alapszínek egyenlő arányú keverékével érhetünk el.
Mivel adott térben a 100\%-os fehér a legvilágosabb elérhető szín, ezért definíció szerint a relatív fénysűrűsége ($Y$ koordinátája) egységnyi.
A 100\%-os fehér tehát hasonlóan az $XYZ$-hez, definíció szerint 
\begin{equation}
\mathbf{w}_{RGB} = \begin{bmatrix}[c]
       1 \\[0.3em]
       1 \\[0.3em]
       1 \end{bmatrix}, \hspace{5mm} \text{és} \hspace{5mm} 
Y_w = 1.
\end{equation}
A \ref{Fig:device_dep} ábrán látható példában a fehér szín vektora a paralelepipedon szürkével jelölt főátlója, ezen vonal mentén helyezkednek el a különböző világosságértékű (árnyalatú) fehér színek.
A fehér szín színezete, azaz $x_w$ és $y_w$ koordinátái ezen vektor egységsíkkal vett döféspontja határozza meg.
Általánosan tehát, definiáltuk az adott $RGB$ tér fehérpontját, amelynek érzékelt színezetét az adott alapszínek határozzák meg.
Ez más szóval a szín akromatikus pontja, amely kijelzőről kijelzőre változhat, az alkalmazott pl. LCD elemek függvényében.

A három alapszín $xy$-koordinátái mellett a fehérpont $x_w$ és $y_w$ koordinátái és $Y_w = 1$ fénysűrűsége már elegendő információ szükség esetén a transzformációs mátrixok meghatározásához.

\paragraph{A fehér színről általában:\\}
Látható tehát, hogy a fehér szín önmagában szubjektív fogalom: adott környezetben a leginkább akromatikus fényingert nevezzük fehérnek, amelynek spektrális sűrűségfüggvénye minél inkább egységnyi (azaz minél több spektrális komponenst tartalmaz), és ezzel analóg módon $RGB$ színtér esetén a színvektora minél közelebb van a csupa-egy vektorhoz.
A fehér fogalom egységesítéséhez bevezettek ún. szabványos megvilágításokat (standard illuminants), amelyet szabványosított $RGB$ színterek esetén előírnak, mint fehérpont.
Ezeknek a szabványos megvilágításoknak a spektrális sűrűségfüggvénye adott (és persze az általa keltett színinger $xy$-koordinátái).
Ilyen szabványos megvilágítások a következők:
\begin{figure}[]
	\centering
	\begin{minipage}[c]{0.6\textwidth}
	\begin{overpic}[width = 0.9\columnwidth ]{figures/1200px-PlanckianLocus.png}
	\end{overpic} \end{minipage}\hfill
	\begin{minipage}[c]{0.4\textwidth}
	\caption{Különböző hőmérsékletű feketetest sugárzók által keltett színek összessége, azaz a Planck görbe.}
	\label{Fig:planck}  \end{minipage}
\end{figure}
\begin{itemize}
\item E fehér: egyenlő energiájú fehér, a CIE XYZ színtér fehérpontja. Kolorimetria szempontjából jelentős, videótechnikában kevésbé fontos a szerepe, mivel a gyakorlatban nem fordul elő olyan fényforrás, amely minden hullámhosszon azonos energiával sugároz.
\item A fehér: a CIE által szabványosított, egyszerű háztartási wolfram-szálas izzó fényét (azzal azonos színérzetet keltő) fényforrás spektruma és színe, $2856~\mathrm{K}$ korrelált színhőmérséklette\footnote{A korrelált színhőmérséklet (correlated color temperature, CCT, $T_{\mathrm{C}}$) azon feketetest sugárzó hőmérsékletét jelzi, amely az emberi szemben a minősítendő fényforrással azonos színérzetet kelt.
A feketetest (hőmérsékleti) sugárzó által keltett színingerek az $xy$ színdiagramon az ún. Planck-görbét járják be, amely a \ref{Fig:planck} ábrán látható.}.
\item B és C fehér: Az A fehérből egyszerű szűréssel nyerhető, napfényt szimuláló megvilágítások.
A B fehér a déli napsütést modellezi $4874~\mathrm{K}$ színhőmérséklettel, míg a C fehér a teljes napra vett átlagos fény színét (spektrumát) modellezi $6774~\mathrm{K}$ színhőmérséklettel.
\item D fehér: szintén a napfény közelítésére alkalmazott megvilágítások sora.
Videótechnika szempontjából a legfontosabb a D65 fehér, amely jelenleg is az UHD formátumok színterének szabványos fehérpontja.
\end{itemize}

\paragraph{Színtér konverziók:\\}
Az eddigiekben látható volt, hogyan definiálható egy eszközfüggő színtér az alapszíneivel.
Ahogy az elnevezés is mutatja, ezek a színterek jellegzetesen adott eszközre érvényesek, pl. egy kamera a beépített $RGB$ szenzorok, egy kijelző az alkalmazott $RGB$ kristályok által meghatározott $RGB$ színtérben dolgoznak.
Emellett léteznek szabványos $RGB$ színterek amely a képi és videotartalom tárolására, továbbítására szolgálnak egységesített, szabványos módon.
A következő bekezdés ezeket a szabványos videószíntereket tárgyalja részletesebben.
%TODO Lab, luv spaces: conversion
Felmerül tehát a természetes igény az egyes színterek közti átjárásra, amelyet \textbf{színtér konverziónak} nevezünk.

A színtérkonverziót az $XYZ$ színtér teszi lehetővé, amely egy eszközfüggetlen, abszolút színtér:
egyes színterek közti konverzió a forrás által létrehozott jelek $XYZ$ színtérbe való transzformációjával, majd ezen reprezentáció a nyelő színterébe való transzformációval történik.
Az $XYZ$ színtér így tehát színterek közti átjárást biztosít, ún. Profile Connection Space-ként működik (hasonlóan pl. a gyakran azonos célra alkalmazott $Lab$ színtérhez).

\begin{figure}[]
	\centering
	\begin{overpic}[width = 1\columnwidth]{figures/cs_conversion.png}
	\small
	\put(1,37){$RGB_{\mathrm{cam}}$}
	\put(35,37.5){$XYZ$}
	\put(67,39){$RGB_{\mathrm{ITU}-709}$}
	\put(13,18){$RGB_{\mathrm{ITU}-709}$}
	\scriptsize
	\put(15,29.25){$\mathbf{M}_{\!R\!G\!B_{\mathrm{c\!a\!m}} \!\!\rightarrow \!\!X\!Y\!Z}$}
	\scriptsize
	\put(49,29.25){$\mathbf{M}_{\!X\!Y\!Z \!\rightarrow \!R\!G\!B_{7\!0\!9}} $}
	\small
	\put(87,29){\parbox{.86in}{MPEG kódolás, műsorszórás, tárolás}}
	\put(52,18){$XYZ$}
	\put(87,17){$RGB_{\mathrm{TV}}$}
	\scriptsize
	\put(32.5,9.5){$\mathbf{M}_{\!R\!G\!B_{\mathrm{7\!0\!9}} \!\!\rightarrow \!\!X\!Y\!Z}$}
	\scriptsize
	\put(66.5,9.6){$\mathbf{M}_{\!X\!Y\!Z \!\rightarrow \!R\!G\!B_{7\!0\!9}} $}	
	\end{overpic} 	
	\caption{Színtér-konverzió folyamatábrája.}
	\label{Fig:cs_conversion}
\end{figure}
Egy tipikus színtér konverziós folyamatot az \ref{Fig:cs_conversion} ábra mutat.
Tegyük fel, hogy adott egy HD kamera által rögzített képanyag, ahol a kamera színterét $RGB_{\mathrm{cam}}$ jelöli.
A HD formátum szabványos színteret alkalmaz, amelyet az ITU-709 ajánlásban rögzítettek (lásd később).
A kamera $RGB$ jeleit tehát az esetleges kódolás és tárolás előtt ebbe a HD színtérbe kell konvertálni.
Ez a konverzió a kamerajelek $XYZ$ térbe, majd innen az ITU-709 színtérbe való konverzióval oldható meg, amely konverziók a megfelelő transzformációsmátrixszal való szorzással valósítható meg:
\begin{equation} 
\begin{bmatrix}[c]
       R_{\mathrm{ITU}-709} \\[0.3em]
       B_{\mathrm{ITU}-709} \\[0.3em]
       G_{\mathrm{ITU}-709} \end{bmatrix}
       =
       \mathbf{M}_{ X\!Y\!Z \rightarrow R\!G\!B_{709} } \cdot 
\left(     \mathbf{M}_{R\!G\!B_{\mathrm{cam}} \rightarrow X\!Y\!Z } \cdot
\begin{bmatrix}[c]
       R_{\mathrm{cam}} \\[0.3em]
       B_{\mathrm{cam}} \\[0.3em]
       G_{\mathrm{cam}} \end{bmatrix} \right)
\end{equation}
Természetesen az egymás utáni két mátrixszorzás összevonható, így a két $RGB$ színtér között közvetlen lineáris leképzés határozható meg.
Ez a transzformáció jellegzetesen már a kamerán belül megvalósul.
%
Hasonlóképp, megjelenítőoldalon a
\begin{equation} 
\begin{bmatrix}[c]
       R_{\mathrm{cam}} \\[0.3em]
       B_{\mathrm{cam}} \\[0.3em]
       G_{\mathrm{cam}} \end{bmatrix}
       =
       \mathbf{M}_{ X\!Y\!Z \rightarrow R\!G\!B_{\mathrm{TV}} } \cdot 
\left(     \mathbf{M}_{R\!G\!B_{709}  \rightarrow X\!Y\!Z } \cdot
\begin{bmatrix}[c]
       R_{\mathrm{ITU}-709} \\[0.3em]
       B_{\mathrm{ITU}-709} \\[0.3em]
       G_{\mathrm{ITU}-709} \end{bmatrix}
 \right)
\end{equation}
transzformációt kell elvégezni.

Ez az egyszerű transzformációs módszer lehetővé teszi egy adott színtérben mért színpontok másik térbe való ábrázolását.
Ugyanakkor felmerül a probléma, hogy nagyobb gamuttal rendelkező színtérből kisebbe való áttérés esetén az új színtérben nem ábrázolható, gamuton kívüli színek negatív $RGB$ koordinátákkal jelennek meg, míg kisebb gamutú térből való áttérés esetén a nagyobb gamutú tér egy része kihasználatlan marad.
A probléma megoldására a fenti transzformációk mellett az egyes színterek gamutját valamilyen nemlineáris leképzés segítségével lehet egymásrailleszteni (expandálással, kompresszálással).
Ezek az ún. gamut-mapping technikák.

A következőekben a konkrét kamerák, illetve megjelenítők által alkalmazott színtereket vizsgálatától eltekintve, a szabványos SD, HD és UHD szabványokban alkalmazott színtereket vizsgáljuk.

\paragraph{A videótechnika színterei:\\}


\section{A TV-technika luma és chroma komponensei}

\section{Videójel komponensek}

\section{A digitalizálás kérdései}