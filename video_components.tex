Az előző fejezet bemutatta az emberi látás képi reprodukció szempontjából legfontosabb tulajdonságait és részletesen tárgyalta a fény- és színmérés alapjait, bevezetve a világosság fogalmát és a CIE XYZ színteret.
Ez a fejezet ezekre az ismeretekre építve bemutatja a televíziós-technikában használt színes-képpont ábrázolás módját, 
ez alapján bevezetve a jelenleg is alkalmazott analóg és digitális videójel komponenseket.

Videótechnika szempontjából az XYZ színteret ritkán alkalmazzák képpontok színkoordinátáinak tárolására, kivétel ez alól a digitális mozi és mozifilm-archiválási alkalmazások \footnote{Ennek oka, hogy egyrészt reprodukcióra közvetlenül nem használható, hiszen az XYZ alapszínek nem valós színek (az X,Y,Z bázisvektorok helyén nem található látható szín), másrészt a teljes látható színek tartománya igen nagy bitmélységet igényel, ráadásul feleslegesen:
Az XYZ tér pozitív térnyolcadát a látható színek csak részben töltik ki (sok olyan kód lenne, amihez nem tartozik látható szín), ráadásul a ezen belül is a megjelenítők a látható színeknek csak egy részét képesek reprodukálni.}.
Ugyanakkor az XYZ tér lehetővé teszi a különböző megjelenítők és kamerák által reprodukálható színek halmazának egyszerű vizsgálatát.
A következő szakasz ezeket a konkrét videóeszközökre jellemző, ún. \textbf{eszközfüggő színtereket} mutatja be.

\section{Eszközfüggő színterek}

Az előző fejezetben láthattuk, hogy az emberi látás trikromatikus jellegének, valamint linearitásának (illetve az egyszerű lineáris modellnek) köszönhetően a látható színek egy lineáris 3D vektortérben ábrázolhatóak, amelyben a vektorok összegzési szabálya érvényes: 
Két tetszőleges szín keverékéből származó eredő színinger meghatározható a két színbe mutató helyvektorok összegeként (függetlenül az eredeti színingereket létrehozó fény spektrumától).
Az $xy$-színpatkón ennek megfelelően két szín összege a két színpontot összekötő szakasz mentén fog elhelyezkedni.

Ebből következik, hogy az emberi látás metamerizmusát kihasználva, a látható színek nagy része előállítható mesterségesen, megfelelően megválasztott alapszínek összegeként.
Ez általánosan véve a színes képreprodukció alapja.
Természetesen nem lehet célunk az összes látható szín visszaállítása: 
Minthogy a színpatkót a spektrálszínek határolják, így elvben végtelen számú spektrálszínt kéne alapszínként alkalmazni az összes látható szín kikeveréséhez.
Felmerül tehát a kérdés, hány alapszín szükséges a színpatkó megfelelő lefedéséhez.

\begin{figure}[]
	\centering
	\begin{overpic}[width = 1\columnwidth ]{figures/color_space_gamut.png}
	\end{overpic}
	\caption{Az azonos alapszínekkel dolgozó SD formátum, HD formátum és az sRGB színtér gamutja $xy$ és $uv$ diagramon ábrázolva.}
	\label{Fig:gamut}
\end{figure}

A színdiagramban könnyen felvehető 4 színpont úgy, hogy a négy szín keverékeit lefedő négyszög (azaz a reprodukálható színek területe) csaknem azonos területű legyen a látható színek területével.
Ugyanakkor az $Luv$ színtér színpatkójából láthattuk, hogy az emberi felbontás zöld árnyalatokra vonatkozó felbontása rossz, és az perceptuálisan egyenletes színdiagram inkább háromszög alakú.
Ez azt jelenti, hogy három megfelelően megválasztott alapszínnel---amelynek különböző arányú keverékeinek színezete egy háromszögön belül helyezkedik el---az egyenletes színezetű ($uv$) színpatkó jelentős része lefedhető.
Ebből kifolyólag az additív színkeverésen alapuló képreprodukciós eszközök szinte kizárólag három megfelelően megválasztott piros, zöld és kék alapszínnel dolgozik.
Az ezekből a színekből pozitív együtthatókkal (RGB intenzitásokkal) kikeverhető színek összességét egy adott \textbf{eszközfüggő színtérnek} nevezzük.
Ezzel ellentétben a kolorimetrikus, abszolút színterek, mint pl. a CIE XYZ, vagy Luv, Lab színterek ún. \textbf{eszközfüggetlen színtereknek}.
Továbbá az adott eszközfüggő színtérben reprodukálható különböző színezetű színek az $xy$-színpatkóban felvett háromszögét a színtér \textbf{gamutjának} nevezzük.
Egy egyszerű példa színterek gamutjára a \ref{Fig:gamut} ábrán látható.

Amennyiben egy RGB színtér teljesen ismert\footnote{Természetesen nem csak RGB színterek léteznek, nyomdatechnikában pl a CMYK eszközfüggő színterek elterjedtek, amelyek esetében a négy alapszín a nyomdában alkalmazott tinták színét jelzi.
A következőekben a vizsgálatunkat kizárólag RGB színterekre végezzük el.}, tetszőleges $C$ színre meghatározhatóak azok az RGB intenzitások, amelyekkel az RGB alapszíneket súlyozva megkaphatjuk a $C$ színt (amennyiben az RGB értékek pozitívak).
Ezek az adott $C$ szín RGB koordinátái.

Vizsgáljuk most, hogyan szokás egy adott eszközfüggő (RGB) színteret definiálni a gyakorlatban, azaz hogyan kell megadni a színteret ahhoz, hogy ezután tetszőleges szín RGB koordinátái számíthatók legyenek.

\paragraph{Eszközfüggő színterek definíciója:\\}

Tekintsünk egy három alapszínt alkalmazó RGB színteret.
Az R, G és B alapszínek természetesen egy-egy vektorként találhatóak meg az $XYZ$ koordinátarendszerben, és vetületük/metszéspontjuk az egységsíkkal adja meg a színpatkón vett $xy$-koordinátáikat.
Ezt illusztrálja a \ref{Fig:device_dep} ábra.
Az alapszín-vektorok $XYZ$ koordinátáit jelölje rendre 
\begin{equation}
\mathbf{r}_{XYZ} = \begin{bmatrix}
       X_r \\[0.3em]
       Y_r \\[0.3em]
       Z_r \end{bmatrix}, \hspace{4mm}
\mathbf{g}_{XYZ} = \begin{bmatrix}
       X_g \\[0.3em]
       Y_g \\[0.3em]
       Z_g \end{bmatrix}, \hspace{4mm}
\mathbf{b}_{XYZ} = \begin{bmatrix}
       X_b \\[0.3em]
       Y_b \\[0.3em]
       Z_b \end{bmatrix}
\end{equation}

\begin{figure}[]
	\centering
	\begin{overpic}[width = 0.75\columnwidth ]{figures/device_dep.png}
	\end{overpic}
	\caption{RGB színtér alapszíneinek helye, és metszéspontja az egységsíkkal az XYZ színtérben.}
	\label{Fig:device_dep}
\end{figure}
Amennyiben a három alapszín $XYZ$ koordinátái ismertek, úgy a színtér teljesen definiálva van:
tetszőleges $\mathbf{c}_{XYZ}$ színvektor koordinátái meghatározhatóak az adott eszközfüggő $RGB$ térben, amely $\mathbf{c}_{RGB}$ vektor tehát azt írja le, milyen súlyozással keverhető ki az adott $C$ szín az RGB alapszínekből:
\begin{equation} 
\underbrace{\begin{bmatrix}[c]
       R_c \\[0.3em]
       B_c \\[0.3em]
       G_c \end{bmatrix}}_{\mathbf{c}_{RGB}}
       =
     \mathbf{M}_{X\!Y\!Z \rightarrow R\!G\!B}
      \underbrace{\begin{bmatrix}[c]
       X_c \\[0.3em]
       Y_c \\[0.3em]
       Z_c \end{bmatrix}}_{\mathbf{c}_{X\!Y\!Z}},
\hspace{3mm}
\end{equation}
ahol $ \mathbf{M}_{X\!Y\!Z \rightarrow R\!G\!B}$ egy bázistranszformációs mátrix. 
Vice versa, az $RGB$ színtérben adott szín $XYZ$ koordinátái meghatározhatóak 
\begin{equation}
      \underbrace{\begin{bmatrix}[c]
       X_c \\[0.3em]
       Y_c \\[0.3em]
       Z_c \end{bmatrix}}_{\mathbf{c}_{X\!Y\!Z}} = 
     \mathbf{M}_{R\!G\!B \rightarrow X\!Y\!Z}
\underbrace{\begin{bmatrix}[c]
       R_c \\[0.3em]
       B_c \\[0.3em]
       G_c \end{bmatrix}}_{\mathbf{c}_{RGB}}
\end{equation}
egyenletből.
Természetesen fennáll a $\mathbf{M}_{R\!G\!B \rightarrow X\!Y\!Z} = \mathbf{M}_{X\!Y\!Z \rightarrow R\!G\!B}^{-1}$ összefüggés.

Utóbbi transzformációs mátrix egyszerűen meghatározható elemi lineáris algebra ismereteinkkel:
$\mathbf{M}_{R\!G\!B \rightarrow X\!Y\!Z}$  mátrix oszlopai egyszerűen az $RGB$ színtér bázisainak $XYZ$-ben vett reprezentációja, azaz általánosan igaz
\begin{equation}
\begin{bmatrix}[c]
       X_c \\[0.3em]
       Y_c \\[0.3em]
       Z_c \end{bmatrix}
       = 
       \underbrace{
  \begin{bmatrix}[c|c|c]
   X_r & X_g & X_b  \\
   Y_r & Y_g & Y_b \\
   Z_r & Z_g & Z_b  \\
\end{bmatrix}}_{\mathbf{M}_{R\!G\!B \rightarrow X\!Y\!Z}}
\cdot
\begin{bmatrix}[c]
       R_c \\[0.3em]
       G_c \\[0.3em]
       B_c \end{bmatrix}
\label{Eq:CS_transform}
\end{equation}
összefüggés\footnote{Az összefüggés érvényessége könnyen tesztelhető pl. $\mathbf{c}_{RGB} = \begin{bmatrix}[c]
       1 \\[0.3em]
       0 \\[0.3em]
       0 \end{bmatrix}$ helyettesítéssel, amely az $R$ alapszín $RGB$-ben vett reprezentációja, és \eqref{Eq:CS_transform} egyenletben a transzformációs mátrix első oszlopát választja ki.}.

A transzformációs mátrixok több szempontból jelentősek: 
egyrészt lehetővé teszik a különböző színtérkonverziókat (lásd köv. bekezdés), valamint egy adott $RGB$ térben ábrázolt képpont $c_Y$ koordinátája megadja az adott szín relatív fénysűrűségét, azaz világosságát.

Itt jegyezzük meg, hogy az $XYZ$ térben vizsgálva adott $RGB$ bázisvektorokkal a pozitív együtthatókkal kikeverhető színek halmaza egy paralelepipedont feszít ki, azaz adott eszközfüggő $RGB$ színtér az $XYZ$ egy paralelepipedonként ábrázolható.
\begin{figure}[]
	\centering
	\small
	(a)
	\begin{overpic}[width = 0.45\columnwidth ]{figures/device_dep_2.png}
	\end{overpic}
	(b)
	\begin{overpic}[width = 0.45\columnwidth ]{figures/The-RGB-colour-cube.png}
	\end{overpic}
	\caption{Egy adott $RGB$ színtér ábrázolása az $XYZ$ térben (a) és az RGB kockában (b). Az (a) ábrán szereplő vektorok színe a végpontjukban található szín határozza meg.}
	\label{Fig:device_dep}
\end{figure}
Az $RGB$ együtthatók definíció szerint 0 és 1 között vehetnek fel értékeket.
Ennek megfelelően egy adott $RGB$ térben az ebben a színtérben reprodukálható színek egy kockában helyezkednek el, ahol a kocka 3 origóból induló éle mentén az alkalmazott $RGB$ alapszínek helyezkednek el.
Emiatt az $RGB$ színtereket gyakran RGB kockaként említik.
A transzformációs mátrixok tehát gyakorlatilag olyan lineáris transzformációt valósítanak meg, amelyek a paralelepipedont kockába, és a kockát paralelepipedonba viszik.

\vspace{3mm}
Egy $RGB$ színtér tehát teljes egészében adott, amennyiben az alapszín-vektorok $XYZ$ koordinátái ismertek (ez tehát 9 koordináta ismeretét jelenti).
A gyakorlatban az ilyen definíció helyett az alkalmazott alapszínek színezetét, azaz $xy$-koordinátáit adják meg.
Ennek az oka egyrészt a színtér gamutjának egyszerű ábrázolása (lásd \ref{Fig:gamut} ábra), másrészt a fehér szín konzisztens, $RGB$ színtértől független relatív fénysűrűsége ($Y$ koordinátája).

Definíció szerint egy adott színtér ún. \textbf{fehérpontja} az adott térben elérhető legvilágosabb (legnagyobb fénysűrűségű) pontja, amelyet az alapszínek egyenlő arányú keverékével érhetünk el.
Mivel adott térben a 100\%-os fehér a legvilágosabb elérhető szín, ezért definíció szerint a relatív fénysűrűsége ($Y$ koordinátája) egységnyi.
A 100\%-os fehér tehát hasonlóan az $XYZ$-hez, definíció szerint 
\begin{equation}
\mathbf{w}_{RGB} = \begin{bmatrix}[c]
       1 \\[0.3em]
       1 \\[0.3em]
       1 \end{bmatrix}, \hspace{5mm} \text{és} \hspace{5mm} 
Y_w = 1.
\end{equation}
A \ref{Fig:device_dep} ábrán látható példában a fehér szín vektora a paralelepipedon szürkével jelölt főátlója, ezen vonal mentén helyezkednek el a különböző világosságértékű (árnyalatú) fehér színek.
A fehér szín színezete, azaz $x_w$ és $y_w$ koordinátái ezen vektor egységsíkkal vett döféspontja határozza meg.
Általánosan tehát, definiáltuk az adott $RGB$ tér fehérpontját, amelynek érzékelt színezetét az adott alapszínek határozzák meg.
Ez más szóval a szín akromatikus pontja, amely kijelzőről kijelzőre változhat, az alkalmazott pl. LCD elemek függvényében.

A három alapszín $xy$-koordinátái mellett a fehérpont $x_w$ és $y_w$ koordinátái és $Y_w = 1$ fénysűrűsége már elegendő információ szükség esetén a transzformációs mátrixok meghatározásához.

\paragraph{Színtér konverziók:\\}

\paragraph{A TV technika eszközfüggő színterei:\\}


\section{A TV-technika luma és chroma komponensei}

\section{Videójel komponensek}

\section{A digitalizálás kérdései}