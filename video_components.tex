Az előző fejezet bemutatta az emberi látás képi reprodukció szempontjából legfontosabb tulajdonságait és részletesen tárgyalta a fény- és színmérés alapjait, bevezetve a világosság fogalmát és a CIE XYZ színteret.
Ez a fejezet ezekre az ismeretekre építve bemutatja a televíziós-technikában használt színes-képpont ábrázolás módját, 
ez alapján bevezetve a jelenleg is alkalmazott analóg és digitális videójel komponenseket.

Videótechnika szempontjából az XYZ színteret ritkán alkalmazzák képpontok színkoordinátáinak tárolására, kivétel ez alól a digitális mozi és mozifilm-archiválási alkalmazások \footnote{Ennek oka, hogy egyrészt reprodukcióra közvetlenül nem használható, hiszen az XYZ alapszínek nem valós színek (az X,Y,Z bázisvektorok helyén nem található látható szín), másrészt a teljes látható színek tartománya igen nagy bitmélységet igényel, ráadásul feleslegesen:
Az XYZ tér pozitív térnyolcadát a látható színek csak részben töltik ki (sok olyan kód lenne, amihez nem tartozik látható szín), ráadásul a ezen belül is a megjelenítők a látható színeknek csak egy részét képesek reprodukálni.}.
Ugyanakkor az XYZ tér lehetővé teszi a különböző megjelenítők és kamerák által reprodukálható színek halmazának egyszerű vizsgálatát.
A következő szakasz ezeket a konkrét videóeszközökre jellemző, ún. \textbf{eszközfüggő színtereket} mutatja be.

\section{Eszközfüggő színterek}

Az előző fejezetben láthattuk, hogy az emberi látás trikromatikus jellegének, valamint linearitásának (illetve az egyszerű lineáris modellnek) köszönhetően a látható színek egy lineáris 3D vektortérben ábrázolhatóak, amelyben a vektorok összegzési szabálya érvényes: 
Két tetszőleges szín keverékéből származó eredő színinger meghatározható a két színbe mutató helyvektorok összegeként (függetlenül az eredeti színingereket létrehozó fény spektrumától).
Az $xy$-színpatkón ennek megfelelően két szín összege a két színpontot összekötő szakasz mentén fog elhelyezkedni.

Ebből következik, hogy az emberi látás metamerizmusát kihasználva, a látható színek nagy része előállítható mesterségesen, megfelelően megválasztott alapszínek összegeként.
Ez általánosan véve a színes képreprodukció alapja.
Természetesen nem lehet célunk az összes látható szín visszaállítása: 
Minthogy a színpatkót a spektrálszínek határolják, így elvben végtelen számú spektrálszínt kéne alapszínként alkalmazni az összes látható szín kikeveréséhez.
Felmerül tehát a kérdés, hány alapszín szükséges a színpatkó megfelelő lefedéséhez.

\begin{figure}[]
	\centering
	\begin{overpic}[width = 1\columnwidth ]{figures/color_space_gamut.png}
	\end{overpic}
	\caption{Az azonos alapszínekkel dolgozó SD formátum, HD formátum és az sRGB színtér gamutja $xy$ és $uv$ diagramon ábrázolva.}
	\label{Fig:gamut}
\end{figure}

A színdiagramban könnyen felvehető 4 színpont úgy, hogy a négy szín keverékeit lefedő négyszög (azaz a reprodukálható színek területe) csaknem azonos területű legyen a látható színek területével.
Ugyanakkor az $Luv$ színtér színpatkójából láthattuk, hogy az emberi felbontás zöld árnyalatokra vonatkozó felbontása rossz, és az perceptuálisan egyenletes színdiagram inkább háromszög alakú.
Ez azt jelenti, hogy három megfelelően megválasztott alapszínnel---amelynek különböző arányú keverékeinek színezete egy háromszögön belül helyezkedik el---az egyenletes színezetű ($uv$) színpatkó jelentős része lefedhető.
Ebből kifolyólag az additív színkeverésen alapuló képreprodukciós eszközök szinte kizárólag három megfelelően megválasztott piros, zöld és kék alapszínnel dolgozik.
Az ezekből a színekből pozitív együtthatókkal (RGB intenzitásokkal) kikeverhető színek összességét egy adott \textbf{eszközfüggő színtérnek} nevezzük.
Ezzel ellentétben a kolorimetrikus, abszolút színterek, mint pl. a CIE XYZ, vagy Luv, Lab színterek ún. \textbf{eszközfüggetlen színtereknek}.
Továbbá az adott eszközfüggő színtérben reprodukálható különböző színezetű színek az $xy$-színpatkóban felvett háromszögét a színtér \textbf{gamutjának} nevezzük.
Egy egyszerű példa színterek gamutjára a \ref{Fig:gamut} ábrán látható.

Amennyiben egy RGB színtér teljesen ismert\footnote{Természetesen nem csak RGB színterek léteznek, nyomdatechnikában pl a CMYK eszközfüggő színterek elterjedtek, amelyek esetében a négy alapszín a nyomdában alkalmazott tinták színét jelzi.
A következőekben a vizsgálatunkat kizárólag RGB színterekre végezzük el.}, tetszőleges $C$ színre meghatározhatóak azok az RGB intenzitások, amelyekkel az RGB alapszíneket súlyozva megkaphatjuk a $C$ színt (amennyiben az RGB értékek pozitívak).
Ezek az adott $C$ szín RGB koordinátái.

Vizsgáljuk most, hogyan szokás egy adott eszközfüggő (RGB) színteret definiálni a gyakorlatban, azaz hogyan kell megadni a színteret ahhoz, hogy ezután tetszőleges szín RGB koordinátái számíthatók legyenek.

\paragraph{Eszközfüggő színterek definíciója:\\}

Tekintsünk egy három alapszínt alkalmazó RGB színteret.
Az R, G és B alapszínek természetesen egy-egy vektorként találhatóak meg az $XYZ$ koordinátarendszerben, és vetületük/metszéspontjuk az egységsíkkal adja meg a színpatkón vett $xy$-koordinátáikat.
Ezt illusztrálja a \ref{Fig:device_dep} ábra.
Az alapszín-vektorok $XYZ$ koordinátáit jelölje rendre 
\begin{equation}
\mathbf{r}_{XYZ} = \begin{bmatrix}
       X_r \\[0.3em]
       Y_r \\[0.3em]
       Z_r \end{bmatrix}, \hspace{4mm}
\mathbf{g}_{XYZ} = \begin{bmatrix}
       X_g \\[0.3em]
       Y_g \\[0.3em]
       Z_g \end{bmatrix}, \hspace{4mm}
\mathbf{b}_{XYZ} = \begin{bmatrix}
       X_b \\[0.3em]
       Y_b \\[0.3em]
       Z_b \end{bmatrix}
\end{equation}

\begin{figure}[]
	\centering
	\begin{overpic}[width = 0.75\columnwidth ]{figures/device_dep.png}
	\small
	\put(89,19){$X$}
	\put(12,96){$Y$}
	\put(0,4){$Z$}
	\put(36,64){$(X_g,Y_g,Z_g)$}
	\put(10,8){$(X_b,Y_b,Z_b)$}
	\put(39,33){$(X_r,Y_r,Z_r)$}
	\end{overpic}
	\caption{RGB színtér alapszíneinek helye, és metszéspontja az egységsíkkal az XYZ színtérben.}
	\label{Fig:device_dep}
\end{figure}
Amennyiben a három alapszín $XYZ$ koordinátái ismertek, úgy a színtér teljesen definiálva van:
tetszőleges $\mathbf{c}_{XYZ}$ színvektor koordinátái meghatározhatóak az adott eszközfüggő $RGB$ térben, amely $\mathbf{c}_{RGB}$ vektor tehát azt írja le, milyen súlyozással keverhető ki az adott $C$ szín az RGB alapszínekből:
\begin{equation} 
\underbrace{\begin{bmatrix}[c]
       R_c \\[0.3em]
       G_c \\[0.3em]
       B_c \end{bmatrix}}_{\mathbf{c}_{RGB}}
       =
     \mathbf{M}_{X\!Y\!Z \rightarrow R\!G\!B}
      \underbrace{\begin{bmatrix}[c]
       X_c \\[0.3em]
       Y_c \\[0.3em]
       Z_c \end{bmatrix}}_{\mathbf{c}_{X\!Y\!Z}},
\end{equation}
ahol $ \mathbf{M}_{X\!Y\!Z \rightarrow R\!G\!B}$ egy bázistranszformációs mátrix. 
Vice versa, az $RGB$ színtérben adott szín $XYZ$ koordinátái meghatározhatóak 
\begin{equation}
      \underbrace{\begin{bmatrix}[c]
       X_c \\[0.3em]
       Y_c \\[0.3em]
       Z_c \end{bmatrix}}_{\mathbf{c}_{X\!Y\!Z}} = 
     \mathbf{M}_{R\!G\!B \rightarrow X\!Y\!Z}
\underbrace{\begin{bmatrix}[c]
       R_c \\[0.3em]
       G_c \\[0.3em]
       B_c \end{bmatrix}}_{\mathbf{c}_{RGB}}
\end{equation}
egyenletből.
Természetesen fennáll a $\mathbf{M}_{R\!G\!B \rightarrow X\!Y\!Z} = \mathbf{M}_{X\!Y\!Z \rightarrow R\!G\!B}^{-1}$ összefüggés.

Utóbbi transzformációs mátrix egyszerűen meghatározható elemi lineáris algebra ismereteinkkel:
$\mathbf{M}_{R\!G\!B \rightarrow X\!Y\!Z}$  mátrix oszlopai egyszerűen az $RGB$ színtér bázisainak $XYZ$-ben vett reprezentációja, azaz általánosan igaz
\begin{equation}
\begin{bmatrix}[c]
       X_c \\[0.3em]
       Y_c \\[0.3em]
       Z_c \end{bmatrix}
       = 
       \underbrace{
  \begin{bmatrix}[c|c|c]
   X_r & X_g & X_b  \\
   Y_r & Y_g & Y_b \\
   Z_r & Z_g & Z_b  \\
\end{bmatrix}}_{\mathbf{M}_{R\!G\!B \rightarrow X\!Y\!Z}}
\cdot
\begin{bmatrix}[c]
       R_c \\[0.3em]
       G_c \\[0.3em]
       B_c \end{bmatrix}
\label{Eq:CS_transform}
\end{equation}
összefüggés\footnote{Az összefüggés érvényessége könnyen tesztelhető pl. $\mathbf{c}_{RGB} = \begin{bmatrix}[c]
       1 \\[0.3em]
       0 \\[0.3em]
       0 \end{bmatrix}$ helyettesítéssel, amely az $R$ alapszín $RGB$-ben vett reprezentációja, és \eqref{Eq:CS_transform} egyenletben a transzformációs mátrix első oszlopát választja ki.}.
% POynoton 250.oldal
A transzformációs mátrixok több szempontból jelentősek: 
egyrészt lehetővé teszik a különböző színtérkonverziókat (lásd köv. bekezdés), valamint egy adott $RGB$ térben ábrázolt képpont $c_Y$ koordinátája megadja az adott szín relatív fénysűrűségét, azaz világosságát.

Itt jegyezzük meg, hogy az $XYZ$ térben vizsgálva adott $RGB$ bázisvektorokkal a pozitív együtthatókkal kikeverhető színek halmaza egy paralelepipedont feszít ki, azaz adott eszközfüggő $RGB$ színtér az $XYZ$ egy paralelepipedonként ábrázolható.
\begin{figure}[]
	\centering
	\small
	(a)
	\begin{overpic}[width = 0.45\columnwidth ]{figures/device_dep_2.png}
	\small
	\put(-2,5){$Z$}
	\put(89,17){$X$}
	\put(11,97){$Y$}
	\end{overpic}
	(b)
	\begin{overpic}[width = 0.45\columnwidth ]{figures/The-RGB-colour-cube.png}
	\end{overpic}
	\caption{Egy adott $RGB$ színtér ábrázolása az $XYZ$ térben (a) és az RGB kockában (b). Az (a) ábrán szereplő vektorok színe a végpontjukban található szín határozza meg.}
	\label{Fig:device_dep}
\end{figure}
Az $RGB$ együtthatók definíció szerint 0 és 1 között vehetnek fel értékeket.
Ennek megfelelően egy adott $RGB$ térben az ebben a színtérben reprodukálható színek egy kockában helyezkednek el, ahol a kocka 3 origóból induló éle mentén az alkalmazott $RGB$ alapszínek helyezkednek el.
Emiatt az $RGB$ színtereket gyakran RGB kockaként említik.
A transzformációs mátrixok tehát gyakorlatilag olyan lineáris transzformációt valósítanak meg, amelyek a paralelepipedont kockába, és a kockát paralelepipedonba viszik.

\vspace{3mm}
Egy $RGB$ színtér tehát teljes egészében adott, amennyiben az alapszín-vektorok $XYZ$ koordinátái ismertek (ez tehát 9 koordináta ismeretét jelenti).
A gyakorlatban az ilyen definíció helyett az alkalmazott alapszínek színezetét, azaz $xy$ színkoordinátáit adják meg.
Ez egyrészt lehetővé teszi a színtér gamutjának egyszerű ábrázolását (lásd \ref{Fig:gamut} ábra).
Másrészt, ami még fontosabb: a gyakorlatban nem szempont egy adott eszközfüggő színtér alapszíneinek---pl. egy RGB kijelző LCD alapszíneinek---fénysűrűségeinek pontos ismerete (azaz pl. hány nit fénysűrűséget hoz létre az R, G, vagy B pixel-elem).
Ennek oka, hogy képi reprodukció során---beleértve a fotográfiát, mozit, videót, nyomtatott reprodukciót---a tényleges, fotometriai szempontból mért abszolút fénysűrűséget szinte soha nem célunk mérni, vagy visszaállítani.
Ehelyett az adott megjelenítő/képrögzítő eszköz által létrehozható/mérhető legvilágosabb színhez képest reprodukáljuk/rögzítjük az adott képpont \textbf{relatív világosságát}.
Ennek megfelelően az eszközfüggő színterek következő módon való definíciójával azt biztosítjuk, hogy az $Y$ koordináta a relatív fénysűrűséget határozza meg.

Definíció szerint egy adott színtér ún. \textbf{fehérpontja} az adott térben elérhető legvilágosabb (legnagyobb fénysűrűségű) pontja, amelyet az alapszínek egyenlő arányú keverékével érhetünk el.
Mivel adott térben a 100\%-os fehér a legvilágosabb elérhető szín, ezért definíció szerint a relatív fénysűrűsége ($Y$ koordinátája) egységnyi.
Az adott eszközfüggő színtérben a 100\%-os fehér tehát hasonlóan az $XYZ$-hez, definíció szerint 
\begin{equation}
\mathbf{w}_{RGB} = \begin{bmatrix}[c]
       1 \\[0.3em]
       1 \\[0.3em]
       1 \end{bmatrix}, \hspace{5mm} \text{és} \hspace{5mm} 
Y_w = 1.
\end{equation}
Míg az $XYZ$ térben tehát általánosan az $Y$-koordináta adott szín abszolút fénysűrűségét adja meg (ami egy alapvető, mérhető fotometriai mennyiség), addig egy eszközfüggő $RGB$ színtérben az $Y$ komponens 0 és 1 között vehet fel értékeket, és az adott színtérben létrehozható "legvilágosabb" színre egységnyi:
Eszközfüggő $RGB$ színtér $Y$ komponense tehát az adott színpont \textbf{relatív fénysűrűségét} határozza meg.
 
A \ref{Fig:device_dep} ábrán látható példában a fehér szín vektora a paralelepipedon szürkével jelölt főátlója, ezen vonal mentén helyezkednek el a különböző világosságértékű (árnyalatú) fehér színek.
A fehér szín színezete, azaz $x_w$ és $y_w$ koordinátái ezen vektor egységsíkkal vett döféspontja határozza meg.
Általánosan tehát, definiáltuk az adott $RGB$ tér fehérpontját, amelynek érzékelt színezetét az adott alapszínek határozzák meg.
Ez más szóval a szín akromatikus pontja, amely kijelzőről kijelzőre változhat, az alkalmazott pl. LCD elemek függvényében.

A három alapszín $xy$-koordinátái mellett a fehérpont $x_w$ és $y_w$ koordinátái és $Y_w = 1$ fénysűrűsége már elegendő információ szükség esetén a transzformációs mátrixok meghatározásához.

\paragraph{A fehér színről általában:\\}
Látható tehát, hogy a fehér szín önmagában szubjektív fogalom: adott környezetben a leginkább akromatikus fényingert nevezzük fehérnek, amelynek spektrális sűrűségfüggvénye minél inkább egységnyi (azaz minél több spektrális komponenst tartalmaz), és ezzel analóg módon $RGB$ színtér esetén a színvektora minél közelebb van a csupa-egy vektorhoz.
A fehér fogalom egységesítéséhez bevezettek ún. szabványos megvilágításokat (standard illuminants), amelyet szabványosított $RGB$ színterek esetén előírnak, mint fehérpont.
Ezeknek a szabványos megvilágításoknak a spektrális sűrűségfüggvénye adott (és persze az általa keltett színinger $xy$-koordinátái).
Ilyen szabványos megvilágítások a következők:
\begin{figure}[]
	\centering
	\begin{minipage}[c]{0.6\textwidth}
	\begin{overpic}[width = 0.9\columnwidth ]{figures/1200px-PlanckianLocus.png}
	\end{overpic} \end{minipage}\hfill
	\begin{minipage}[c]{0.4\textwidth}
	\caption{Különböző hőmérsékletű feketetest sugárzók által keltett színek összessége, azaz a Planck görbe.}
	\label{Fig:planck}  \end{minipage}
\end{figure}
\begin{itemize}
\item E fehér: egyenlő energiájú fehér, a CIE XYZ színtér fehérpontja. Kolorimetria szempontjából jelentős, videótechnikában kevésbé fontos a szerepe, mivel a gyakorlatban nem fordul elő olyan fényforrás, amely minden hullámhosszon azonos energiával sugároz.
\item A fehér: a CIE által szabványosított, egyszerű háztartási wolfram-szálas izzó fényét (azzal azonos színérzetet keltő) fényforrás spektruma és színe, $2856~\mathrm{K}$ korrelált színhőmérséklette\footnote{A korrelált színhőmérséklet (correlated color temperature, CCT, $T_{\mathrm{C}}$) azon feketetest sugárzó hőmérsékletét jelzi, amely az emberi szemben a minősítendő fényforrással azonos színérzetet kelt.
A feketetest (hőmérsékleti) sugárzó által keltett színingerek az $xy$ színdiagramon az ún. Planck-görbét járják be, amely a \ref{Fig:planck} ábrán látható.}.
\item B és C fehér: Az A fehérből egyszerű szűréssel nyerhető, napfényt szimuláló megvilágítások.
A B fehér a déli napsütést modellezi $4874~\mathrm{K}$ színhőmérséklettel, míg a C fehér a teljes napra vett átlagos fény színét (spektrumát) modellezi $6774~\mathrm{K}$ színhőmérséklettel.
\item D fehér: szintén a napfény közelítésére alkalmazott megvilágítások sora.
Videótechnika szempontjából a legfontosabb a D65 fehér, amely jelenleg is az UHD formátumok színterének szabványos fehérpontja.
\end{itemize}

\paragraph{Színtér konverziók:\\}
Az eddigiekben látható volt, hogyan definiálható egy eszközfüggő színtér az alapszíneivel.
Ahogy az elnevezés is mutatja, ezek a színterek jellegzetesen adott eszközre érvényesek, pl. egy kamera a beépített $RGB$ szenzorok, egy kijelző az alkalmazott $RGB$ kristályok által meghatározott $RGB$ színtérben dolgoznak.
Emellett léteznek szabványos $RGB$ színterek amely a képi és videotartalom tárolására, továbbítására szolgálnak egységesített, szabványos módon.
A következő bekezdés ezeket a szabványos videószíntereket tárgyalja részletesebben.
%TODO Lab, luv spaces: conversion
Felmerül tehát a természetes igény az egyes színterek közti átjárásra, amelyet \textbf{színtér konverziónak} nevezünk.

A színtérkonverziót az $XYZ$ színtér teszi lehetővé, amely egy eszközfüggetlen, abszolút színtér:
egyes színterek közti konverzió a forrás által létrehozott jelek $XYZ$ színtérbe való transzformációjával, majd ezen reprezentáció a nyelő színterébe való transzformációval történik.
Az $XYZ$ színtér így tehát színterek közti átjárást biztosít, ún. Profile Connection Space-ként működik (hasonlóan pl. a gyakran azonos célra alkalmazott $Lab$ színtérhez).

\begin{figure}[]
	\centering
	\begin{overpic}[width = 1\columnwidth]{figures/cs_conversion.png}
	\small
	\put(1,37){$RGB_{\mathrm{cam}}$}
	\put(35,37.5){$XYZ$}
	\put(67,39){$RGB_{\mathrm{ITU}-709}$}
	\put(13,18){$RGB_{\mathrm{ITU}-709}$}
	\scriptsize
	\put(15,29.25){$\mathbf{M}_{\!R\!G\!B_{\mathrm{c\!a\!m}} \!\!\rightarrow \!\!X\!Y\!Z}$}
	\scriptsize
	\put(49,29.25){$\mathbf{M}_{\!X\!Y\!Z \!\rightarrow \!R\!G\!B_{7\!0\!9}} $}
	\small
	\put(87,29){\parbox{.86in}{MPEG kódolás, műsorszórás, tárolás}}
	\put(52,18){$XYZ$}
	\put(87,17){$RGB_{\mathrm{TV}}$}
	\scriptsize
	\put(32.5,9.5){$\mathbf{M}_{\!R\!G\!B_{\mathrm{7\!0\!9}} \!\!\rightarrow \!\!X\!Y\!Z}$}
	\scriptsize
	\put(66.5,9.6){$\mathbf{M}_{\!X\!Y\!Z \!\rightarrow \!R\!G\!B_{7\!0\!9}} $}	
	\end{overpic} 	
	\caption{Színtér-konverzió folyamatábrája.}
	\label{Fig:cs_conversion}
\end{figure}
Egy tipikus színtér konverziós folyamatot az \ref{Fig:cs_conversion} ábra mutat.
Tegyük fel, hogy adott egy HD kamera által rögzített képanyag, ahol a kamera színterét $RGB_{\mathrm{cam}}$ jelöli.
A HD formátum szabványos színteret alkalmaz, amelyet az ITU-709 ajánlásban rögzítettek (lásd később).
A kamera $RGB$ jeleit tehát az esetleges kódolás és tárolás előtt ebbe a HD színtérbe kell konvertálni.
Ez a konverzió a kamerajelek $XYZ$ térbe, majd innen az ITU-709 színtérbe való konverzióval oldható meg, amely konverziók a megfelelő transzformációsmátrixszal való szorzással valósítható meg:
\begin{equation} 
\begin{bmatrix}[c]
       R_{\mathrm{ITU}-709} \\[0.3em]
       G_{\mathrm{ITU}-709} \\[0.3em]
       B_{\mathrm{ITU}-709} \end{bmatrix}
       =
       \mathbf{M}_{ X\!Y\!Z \rightarrow R\!G\!B_{709} } \cdot 
\left(     \mathbf{M}_{R\!G\!B_{\mathrm{cam}} \rightarrow X\!Y\!Z } \cdot
\begin{bmatrix}[c]
       R_{\mathrm{cam}} \\[0.3em]
       G_{\mathrm{cam}} \\[0.3em]
       B_{\mathrm{cam}} \end{bmatrix} \right)
\end{equation}
Természetesen az egymás utáni két mátrixszorzás összevonható, így a két $RGB$ színtér között közvetlen lineáris leképzés határozható meg.
Ez a transzformáció jellegzetesen már a kamerán belül megvalósul.
%
Hasonlóképp, megjelenítőoldalon a
\begin{equation} 
\begin{bmatrix}[c]
       R_{\mathrm{cam}} \\[0.3em]
       G_{\mathrm{cam}} \\[0.3em]
       B_{\mathrm{cam}} \end{bmatrix}
       =
       \mathbf{M}_{ X\!Y\!Z \rightarrow R\!G\!B_{\mathrm{TV}} } \cdot 
\left(     \mathbf{M}_{R\!G\!B_{709}  \rightarrow X\!Y\!Z } \cdot
\begin{bmatrix}[c]
       R_{\mathrm{ITU}-709} \\[0.3em]
       G_{\mathrm{ITU}-709} \\[0.3em]
       B_{\mathrm{ITU}-709} \end{bmatrix}
 \right)
\end{equation}
transzformációt kell elvégezni.

Ez az egyszerű transzformációs módszer lehetővé teszi egy adott színtérben mért színpontok másik térbe való ábrázolását.
Ugyanakkor felmerül a probléma, hogy nagyobb gamuttal rendelkező színtérből kisebbe való áttérés esetén az új színtérben nem ábrázolható, gamuton kívüli színek negatív, és egynél nagyobb $RGB$ koordinátákkal jelennek meg, míg kisebb gamutú térből való áttérés esetén a nagyobb gamutú tér egy része kihasználatlan marad.
A probléma megoldására a fenti transzformációk mellett az egyes színterek gamutját valamilyen nemlineáris leképzés segítségével lehet egymásra illeszteni (expandálással, kompresszálással).
Ezek az ún. gamut-mapping technikák.

A következőekben az egyes SD, HD és UHD videóformátumok tárolására és továbbítására alkalmazott eszközfüggő színtereket vizsgáljuk.

\paragraph{A videótechnika színterei:\\}

% http://www.displaymate.com/crtvslcd.html
Az első kodifikált színmérő rendszer az NTSC (National Television System Committee) által 1953-ban szabványosított színes-televíziós műsorszóráshoz alkalmazott, az azt létrehozó bizottság után elnevezett NTSC szabvány volt.
A színteret a korabeli foszfortechnológiával létrehozható CRT kijelzők (TV vevők) alapszíneik megfelelően írták elő, így színtérkorrekció vevő oldalon nem volt szükség.
Egy egyszerű példa CRT kijelző alapszíneinek meghatározása a következőekben lesz látható.
A színmérő rendszer C fehérponttal dolgozott, alapszíneit pedig a \ref{tab:ntsc_colorimetry} táblázat mutatja.
Az így kapott gamut az xy ábrán látható.
\begin{table}[h!]
\caption{Az NTSC szabvány színmérőrendszere}
\renewcommand*{\arraystretch}{1}
\label{tab:ntsc_colorimetry}
\begin{center}
\small\addtolength{\tabcolsep}{15pt}
    \begin{tabular}[h!]{ @{}c | | l | l @{} }%\toprule
		&   x  	&    y \\ \hline
    R   &  0.67 &	0.33 \\
    G   &  0.21 &   0.71  \\
    B   & 0.14   &	0.08\\
    C fehér     &  0.310 &	0.316  \\
    \end{tabular}
\end{center}
\end{table}
Az alapszínekből és a fehérpontból meghatározható az $RGB_{\mathrm{NTSC}} \rightarrow XYZ$ transzformációs mátrix, amely alakja általánosan
\begin{equation}
\begin{bmatrix}[c]
       X \\[0.3em]
       Y \\[0.3em]
       Z \end{bmatrix}
       = 
  \begin{bmatrix}[c c c]
   0.60 & 0.17 & 0.2  \\
   0.30 & 0.59 & 0.11 \\
   0 & 0.07 & 1.11
\end{bmatrix}
\cdot
\begin{bmatrix}[c]
       R \\[0.3em]
       G \\[0.3em]
       B \end{bmatrix}_{\mathrm{NTSC}}
\label{Eq:NTSC_transform}
\end{equation}
Az egyenlet második sora kitüntetett szereppel bír: meghatározza, hogy az NTSC színtérben hogyan számítható adott $RGB$ színpont világossága:
\begin{equation}Y_{\mathrm{NTSC}} = 
   0.30R + 0.59G + 0.11 B. 
\label{Eq:NTSC_luminance}
\end{equation}
A világosságjel számítása egészen a HD formátum megjelenése (azaz közel 50 éven keresztül) a fenti egyenlet szerint történt.

\vspace{3mm}
Az foszfortechnológia fejlődésével az újabb megjelenítők egyre inkább feláldozták a széles gamutot (azaz a minél telítettebb alapszínek használatát) a minél nagyobb fényerő érdekében: 
Az alkalmazott foszforok a nagyobb érzékelt világosság (fénysűrűség) érdekében egyre nagyobb sávszélességben sugároztak, így az alapszínek egyre kevésbé telítettek lettek, a gamut tehát csökkent (más szóval: az alapszínek spektruma a Dirac-impulzus helyett---amely teljesen telített spektrálszín lenne---szélesebb görbe lett, így a görbe alatti terület---és ezzel a szín világossága nőtt---de telítettsége csökkent).
Mivel így a megjelenítő gamutja jelentősen eltért az NTSC szabványtól, ezért ez a képernyőn látható színek torzulását eredményezte.
Ennek megoldásául a TV vevőkbe analóg színtérkonverziós áramköröket ültettek, amelyek az NTSC és a megjelenítő saját színtere közti konverziót valósította meg\footnote{Ahogy látni fogjuk a későbbiekben: a vevőkbe már csak a nem-lineárisan Gamma-előtorzított $RGB$ jelek jutottak, ahol az inverz torzítást maga a kijelző hajtotta végre. Emiatt a színtérkonverziót csak Gamma-torzított $R'G'B'$ jeleken tudták végrehajtani, ami azonban a telített színeknél ismét látható színezet és fénysűrűség-hibát okozott.}
Ettől a ponttól tehát a műsorszórás szabványos színtere és a megjelenítők színtere különválnak.

Az európai színes műsorszórásra az EBU (European Broadcasting Union) a PAL (Phase Alternating Line) rendszert vezette be 1963-ban, újradefiniálva a színmérőrendszert, új alapszínekre és D65 fehéret alkalmazva:
\begin{table}[h!]
\caption{A PAL szabvány színmérőrendszere}
\renewcommand*{\arraystretch}{1}
\label{tab:pal_colorimetry}
\begin{center}
\small\addtolength{\tabcolsep}{15pt}
    \begin{tabular}[h!]{ @{}c | | l | l @{} }%\toprule
		&   x  	&    y \\ \hline
    R   &  0.64 &  0.33 \\
    G   &  0.29 &  0.60  \\
    B   & 0.15 & 0.06\\
    D65 fehér     &  0.3127 & 0.3290 	  \\
    \end{tabular}
\end{center}
\end{table}
%
Ez matematikailag helyesen a transzformációs mátrix és a világosságjel számításának módjának megváltozását jelentené.
Praktikussági szempontokból azonban a PAL rendszer az NTSC-vel azonos módon, \eqref{Eq:NTSC_luminance} alapján állítja elő a világosságjelet, mivel a gyakorlatban a különbség alig volt látható \footnote{Ennek oka, hogy a világosságjel átviteltechnológia szempontjából fontos: a kamera és a kijelző is $RGB$ jeleket használ, a világosságjelet, ahogy a következőekben látjuk csak a képanyag átviteléhez számítjuk ki.}.
Az PAL alapszíneit és a világosságjel számításának módját átvette az első digitális videóformátum, az ITU-601-es SD formátum is 1982-ben.

\begin{figure}[]
	\centering
	\begin{overpic}[width = 0.7\columnwidth ]{figures/gamuts.png}
	\end{overpic}
	\caption{Az NTSC, PAL/SD/HD/sRGB és UHD szabványok gamutja az $xy$-színpatkóban.
	Az NTSC jóval nagyobb gamuttal dolgozott, mint a ma is használt HD és sRGB formátumok. Ennek oka, hogy a korai CRT megjelenítők ugyan telítettebb, de ugyanakkor kisebb fénysűrűségű és nagy időállandójú foszforokkal dolgoztak, amivel bár nagy színtartományt tudtak megjeleníteni, de kis fényerővel, és mozgó objektumoknál a képernyőn akaratlanul is nyomokat hagyva.}
	\label{Fig:gamut}
\end{figure}

A HD formátumot az 1990-ben szabványosították az ITU-709-es ajánlás formájában.
Ez ugyanúgy átvette az PAL rendszer alapszíneit, azonban immáron matematikailag precízen, újraszámította a transzformációs mátrixot és a világosságjel együtthatókat, amely tehát HD esetén
\begin{equation}Y_{\mathrm{ITU}-709} = 
   0.2126\,R + 0.7152\,G + 0.0722\,B. 
\label{Eq:NTSC_luminance}
\end{equation}
alapján számítható.
Fontos megjegyezni, hogy az ITU-709 szabvány színmérőrendszerét átvette az sRGB szabvány is, ami a mai napig a számítógépes alkalmazások (és operációs rendszerek) alapértelmezett színteréül szolgál.

Az alkalmazott alapszíneket végül számottevően csak az UHD formátum változtatta meg az ITU-2020 számú ajánlásában 2012-ben.
Az UHD alkalmazásokra a szabvány egy széles gamutú, spektrál-alapszíneket alkalmazó színteret ajánl a \ref{tab:UHDTV_colorimetry} táblázatban látható paraméterekkel. 
\begin{table}[h!]
\caption{Az ITU-2020 szabvány színmérőrendszere}
\renewcommand*{\arraystretch}{1}
\label{tab:UHDTV_colorimetry}
\begin{center}
\small\addtolength{\tabcolsep}{15pt}
    \begin{tabular}[h!]{ @{}c | | l | l @{} }%\toprule
		&   x  	&    y \\ \hline
    R   &  0.708 &	0.292  \\
    G   &  0.17 &	0.797  \\
    B   & 0.131 &	0.046 \\
    D65 fehér     &  0.3127 & 0.3290 	  \\
    \end{tabular}
\end{center}
\end{table}
A szabvány természetesen újradefiniálta a világosság komponens számításának a módját is, amely tehát UHD esetben
\begin{equation}Y_{\mathrm{ITU}-2020} = 
   0.2627\,R + 0.678 \,G + 0.0593\,B 
\label{Eq:UHD_luminance}
\end{equation}
alapján számítható.
A szabvány természetesen nem igényli, hogy az UHD megjelenítők spektrálszíneket legyenek képesek alapszínekként realizálni, a minél szélesebb gamut inkább a jövőbeli technológiák szempontjából ad ajánlást.
A mai konzumer megjelenítők az UHD képanyagot megjelenítés előtt a saját színterükben konvertálják, amely jellegzetesen jóval kisebb a szabvány színterénél.

\paragraph{Példa CRT kijelző eszközfüggő színterére:\\}

Egyszerű példaként az eddig leírtakra vizsgáljuk, hogyan számítható és illusztrálható egy CRT kijelző által megjelenített színek tartománya, röviden rávilágítva a CRT technológia működési elvére is \footnote{Természetesen az itt leírtak változtatás nélkül alkalmazhatók más technológia alapján működő kijelzőkre is, pl. LCD.}.
Bár a CRT technológia kezd egyre inkább eltűnni, néhány évvel ezelőttig a stúdiómonitorok jelentős része még mindig CRT alapon működött köszönhetően a színhű megjelenítésüknek, és a mai LCD megjelenítőkhöz képest is jóval nagyobb statikus kontrasztjuknak.

\begin{figure}[]

	\centering
	\begin{overpic}[width = 0.5\columnwidth ]{figures/1024px-CRT_color_enhanced.png}
	\end{overpic}
	\caption{CRT megjelenítő felépítése.}
	\label{Fig:crt}
\end{figure}

A katódsugárcsöves (CRT) kijelzők sematikus ábrája az \ref{Fig:crt} ábrán látható.
A CRT-k kijelzők működésének alapja három ún. elektronágyú volt, amelyek egy fűtőtt katódból (1) és egy nagyfeszültségre helyezett anódból állt.
A melegítés hatására a katód környezetébe szabad elektronok léptek ki, így egy elektronfelhőt képezve a katód körül.
A katód közelébe helyezett nagyfeszültségű (néhány száz Volt) gyorsítóanód hatására a szabad elektronok az anód felé kezdtek mozogni, egy szabad elektronáramot (2) indítva a vákuumban (ugyanezen az elven működtek a vákuum-diódák, triódák, pentódák, stb. is).
Elegendően nagy anódfeszültség (és további anódok jelenléte) esetén az elektronok jelentős része nem csapódott be a gyorsítóanódra, hanem továbbhaladt.
Ezt az elektronnyalábot elektrosztatikusan és mágnesesen (3) fókuszálták, majd egy vezérelt mágneses eltérítő (4) sorról sorra végigfuttatta azt egy anódfeszültségű-ernyőn (5), azaz a képernyőn.
Színes kijelző esetén természetesen három elektronágyú üzemelt párhuzamosan.
A képernyő felszínét pixelekre bontva képpontonként három különböző foszforral borították (7-8), amely gerjesztés (becsapódó elektronok) hatására bizonyos ideig adott spektrális sűrűségfüggvényű fényt bocsájtott ki\footnote{Ellentétben a fluoreszkáló anyagok csak a gerjesztés fennállásának idején bocsájtanak ki fényt. 
A foszforeszkálás időállandója előnyös, hiszen megfelelően megválasztott foszforok épp egy képidőig bocsájtanak ki fényt, így a kijelzett kép nem fog villogni.
Ugyanakkor a korai kijelzők ezen időállandója túl nagy volt, ezért a gyors mozgások elmosódtak a kijelzett képen.}, realizálva ezzel az $RGB$ alapszíneket.

\begin{figure}[]
	\centering
	\begin{overpic}[width = 0.54\columnwidth]{figures/sony.png}
	\small
	\put(0,0){(a)}
	\end{overpic}
	\begin{overpic}[width = 0.39\columnwidth]{figures/sony_gamut.png}
	\small
	\put(0,0){(b)}
	\end{overpic}
	\begin{overpic}[width = 0.014\columnwidth]{figures/sony_gamut_2.png}
	\end{overpic}
	\caption{CRT megjelenítő foszforai által kibocsájtott sugárzás spektrális sűrűségfüggvénye (a) a megjelenítő gamutja és az adott spektrumok/alapszínek által keltett színérzet, valamint a színtér fehérpontja (b).
	A jobb oldali oszlop bal fele a Sony monitor alapszíneit és fehérpontját, a jobb fele az sRGB színtér alapszíneit és fehérpontját szemlélteti.}
	\label{Fig:sony}
\end{figure}

Tekintsünk példaként egy Sony F520 CRT kijelzőt: 
A kijelző $RGB$ foszforjai gerjesztés hatására a \ref{Fig:sony} (a) ábrán látható spektrális sűrűségfüggvényű (sugársűrűségű) fényt bocsájtanak ki magukban egységnyi felületről, egységnyi térszögbe, azaz rendelkezésre állnak a $L_{e}^R(\lambda)$, $L_{e}^G(\lambda)$ és $L_{e}^B(\lambda)$ függvények.
Ekkor a $\overline{x}(\lambda)$, $\overline{y}(\lambda)$, $\overline{z}(\lambda)$ szabványos $XYZ$ spektrális színösszetevő függvények alkalmazásával a piros alapszín színkoordinátái rendre
\begin{align}
\begin{split}
\overline{X}_R &= \int_{380~\mathrm{nm}}^{780~\mathrm{nm}} L_{e}^R(\lambda) \cdot \overline{x}(\lambda) \mathrm{d} \lambda = 0.0633 \\
\overline{Y}_R &= \int_{380~\mathrm{nm}}^{780~\mathrm{nm}} L_{e}^R(\lambda) \cdot \overline{y}(\lambda) \mathrm{d} \lambda = 0.0373\\
\overline{Z}_R &= \int_{380~\mathrm{nm}}^{780~\mathrm{nm}} L_{e}^R(\lambda) \cdot \overline{z}(\lambda) \mathrm{d} \lambda  = 0.0035 \\
\end{split}
\end{align}
és persze hasonlóan számíthatóak az zöld és kék alapszínek $XYZ$-koordinátái, az integrálok numerikus kiértékelésével.
A színtér fehérpontja definíció szerint az alapszínvektorok egyenlő súlyú összegeként áll elő, azaz
\begin{equation}
\overline{X}_W = \overline{X}_R + \overline{X}_G + \overline{X}_B, \hspace{6mm} 
\overline{Y}_W = \overline{Y}_R + \overline{Y}_G + \overline{Y}_B, \hspace{6mm} 
\overline{Z}_W = \overline{Z}_R + \overline{Z}_G + \overline{Z}_B,
\end{equation}
amelyből az alapszínvektorok pontos hossza meghatározható, hiszen definíció szerint $Y_W = 1$ érvényes (az alapszínvektorok eszerint normálandók).
Így az alapszín-vektorok, és így a színtér alkalmazásához szükséges transzformáció mátrixok a következők:
\begin{align}
\begin{split}
\begin{bmatrix}[c]
       X \\[0.3em]
       Y \\[0.3em]
       Z \end{bmatrix} &= 
     \underbrace{ \begin{bmatrix}[c|c|c]
       0.5646 &  0.2665 &  0.2068 \\[0.3em]
       0.3174 &  0.5992 &  0.0834 \\[0.3em]
       0.0302 &  0.1443 &  1.0539 \end{bmatrix} }_{\mathbf{M}_{R\!G\!B \rightarrow X\!Y\!Z}}
\begin{bmatrix}[c]
       R \\[0.3em]
       G \\[0.3em]
       B \end{bmatrix}_{\mathrm{F}520}
\\ \vspace{1mm} \\
&\mathbf{M}_{X\!Y\!Z \rightarrow   R\!G\!B} = \mathbf{M}_{R\!G\!B \rightarrow X\!Y\!Z}^{-1}
\end{split}
\end{align}
Az alapszínek és a fehérpont színezete ezután
\begin{equation}
x_R = \frac{X_R}{X_R + Y_R + Z_R}, \hspace{1cm} y_R = \frac{Y_R}{X_R + Y_R + Z_R}
\end{equation}
alapján számolható.
Az így meghatározott színtér gamutja a \ref{Fig:sony} ábrán látható, az alapértelmezett számítógépes sRGB színtérrel együtt.

Természetesen jelen esetben a színtér a megjelenítő színterének alapszíneinek ábrázolásához az $XYZ$ térben adott alapszíneket az $RGB$ térbe kell konvertálni.
Jelen dokumentum sRGB színtérben kerül tárolásra (és megjelenítéskor az sRGB színtér az olvasó kijelzőjének saját színterébe transzformálva), így jelen dokumentumban az $XYZ$ koordinátáival adott alapszínek az sRGB térbe való konverzió után kerülhetnek megjelenítésre, amely pl. a vörös alapszínre
\begin{equation}
\begin{bmatrix}[c]
       R_R \\[0.3em]
       G_R \\[0.3em]
       B_R \end{bmatrix}_{\mathrm{sRGB}}
       =
     \mathbf{M}_{X\!Y\!Z \rightarrow R\!G\!B_{\mathrm{sRGB}}}
      \begin{bmatrix}[c]
       X_R \\[0.3em]
       Y_R \\[0.3em]
       Z_R \end{bmatrix} =      
       \begin{bmatrix}[c]
       1.13 \\[0.3em]
       0.25 \\[0.3em]
       -0.02 \end{bmatrix} 
\end{equation}
alakú.
A Sony megjelenítő alapszíneinek sRGB koordinátáira negatív és 1-nél nagyobb $RGB$ értékek is adódnak.
Ez a \ref{Fig:sony} ábrán is látható gamutok közti eltérést tükrözi.

\section{A TV-technika színkülönbségi jelei}

Az előző szakasz bemutatta egy színes képpont ábrázolásának módját adott $RGB$ eszközfüggő színtérben.
A következő felmerülő kérdés, hogy ezekből az RGB jelekből---amelyek tehát egy pontja egy megjelenítendő képelem RGB koordinátáit írja le---hogyan hozhatóak létre a ténylegesen rögzített és továbbított videójelek.

\paragraph{A világosság és színkülönbségi jelek:\\}
A fő oka, hogy a videójeleket nem közvetlenül az RGB jeleknek választották (bár manapság már gyakori a közvetlen RGB ábrázolás) az NTSC bevezetésének idejében a visszafelé kompatibilitás biztosítása volt:
A színes műsorszórás kezdetén a korabeli háztartásokban szinte kizárólag fekete-fehér TV-vevők voltak találhatók.
Természetes volt az igény a már kiépített fekete-fehér műsorszóró rendszerrel való visszafelé kompatibilitásra színes kép-továbbítás esetén, amelyet a fekete-fehér kép és a színinformáció külön kezelésével volt elérhető.
Természetesen manapság már ez a tradicionális ok nem szempont videójelek megválasztása esetén.
Azonban látni a színinformáció külön kezelése lehetővé teszi a színek csökkentett felbontással való tárolását, amely jelentős adattömörítést (analóg esetben sávszélesség-csökkentést) tesz lehetővé.

A fekete-fehér kép gyakorlatilag egy színes kép világosságinformációjának fogható fel, amely az $RGB$ koordinátákból azok lineáris kombinációjaként számítható.
Az együtthatók az adott eszközfüggő színtértől függenek, az NTSC alapszínei esetén pl. \ref{Eq:NTSC_luminance} alapján adottak.
Mivel a fekete-fehér TV vevők közvetlenül ezt a világosságjelet jelenítették meg, ezért a színes TV esetén is az egyik, változatlanul továbbítandó jelet a \textbf{világosságjelnek (luminance)} választották, amely tehát például NTSC esetén az $RGB$ jelekből
\begin{equation}Y_{\mathrm{NTSC}} = 
   0.30R + 0.59G + 0.11 B. 
   \label{Eq:NTSC_luminance}
\end{equation}
alapján számítható \footnote{Fontos ismét kihangsúlyozni, hogy a világosság-számítás módja színtérfüggő, az alapszínektől és a fehérponttól függ a már bemutatott módon.}.

Egy színes képpont leírásához 3 komponens szükséges, egy lehetséges és hatékony leírás pl. a képpont világossága, színezete és telítettsége.
A világosságjel mellé tehát két független információ kell, amelyek egyértelműen meghatározzák az adott színpont színezetét és telítettségét\footnote{A visszafelé-kompatibilitás biztosításához ezt a két színezetet leíró jelet kellett az NTSC rendszerben a változatlan fekete-fehér jelhez úgy hozzáadni, hogy a meglévő fekete-fehér vevők a világosságjelet demodulálni tudják, és a hozzáadott többletinformáció minimális látható hatással legyen a megjelenített képre.}.
Ugyanakkor fontos szempont volt ezen világosságinformáció-mentes, pusztán színinformációt leíró jelek könnyű számíthatósága az RGB komponensekből az egyszerű analóg áramköri megvalósíthatóság érdekében.

A színinformáció/világosságinformáció-szétválasztás legegyszerűbb (de jól működő) megoldásaként egyszerűen vonjuk ki a világosságot az RGB jelekből.
Mivel az $Y$ együtthatók összege definíció szerint (tetszőleges színtérben) egységnyi, így pl. NTSC esetén \eqref{Eq:NTSC_luminance} mindkét oldalából $Y$-t kivonva igaz a 
\begin{equation} 
   0.30 ( R - Y ) + 0.59 ( G - Y )  + 0.11 ( B - Y )  = 0 
   \label{eq:chrominances}
\end{equation}
egyenlőség.
Az $ ( R - Y ) $, $ ( G - Y ) $ és $ ( B - Y ) $ a TV-technika ún. \textbf{színkülönbségi jelei}, és a következő tulajdonságokkal bírnak:
\begin{itemize}
\item Nem függetlenek egymástól, kettőből számítható a harmadik.
\item Előjeles mennyiségek.
\item Ha két színkülönbségi jel zérus, akkor a harmadik is az.
Ekkor $R = G = B = Y$, így tehát a színtér fehérpontjában vagyunk.
A fehér színre kapott zérus színkülönbségi jelek azt mutatják, hogy a színinformációt valóban a színkülönbségi jelek jelzik, a fénysűrűség (világosság) pedig tőlük független mennyiség.
\item Az adott színkülönbségi jel értéke maximális ha a hozzá tartozó alapszín maximális, és vice versa.
NTSC rendszerben vörös színkülönbségi jelre $R = 1$, $G = B= 0$ esetén
\begin{equation}
Y = 0.30 \cdot 1 + 0.59 \cdot 0 + 0.11 \cdot 0 \hspace{3mm }\rightarrow \hspace{3mm } R - Y  = 0.7,
\end{equation}
és hasonlóan $R=0$, $G = B = 1$ esetén
\begin{equation}
Y = 0.30 \cdot 0+ 0.59 \cdot 1 + 0.11 \cdot 1 \hspace{3mm }\rightarrow \hspace{3mm } R - Y  = -0.7.
\end{equation}
\item A fenti megfontolások alapján a színkülönbségi jelek dinamikatartománya:
\begin{align}
\begin{split}
-0.7 \leq R-Y \leq& 0.7 , \hspace{2cm} -0.89 \leq G-Y \leq 0.89, \\
 &-0.41 \leq B-Y \leq 0.41
\end{split}
\end{align}
\end{itemize}
A három színkülönbségi jelből kettőt kell választani a színpont színinformációjának leírásához.
Mivel jel/zaj-viszony szempontjából ökölszabályszerűen mindig a nagyobb dinamikatartományú jelet célszerű továbbítani, így a választás a vörös és zöld színkülönbségi jelekre esett.

A videótechnikában tehát egy adott színpont ábrázolása a
\begin{align*}
Y&: \text{Luminance }\\
 	\left.\begin{array}{lr}
        R-Y\\
        B-Y
        \end{array}\right\}&: \text{Chrominance}
\end{align*}
ún. \textbf{luminance-chrominance térben} történik, amely gyakorlatilag felfogható egy új színmérőrendszernek is az $RGB$ színtérhez képest.

\paragraph{Az $Y,\,R-Y,\,B-Y$ színtér:\\}
Vizsgáljuk most, hol helyezkednek el az adott $RGB$ eszközfüggő színtérben ábrázolható színek ebben az új, $Y,\, R-Y,\, B-Y$ térben!
Az előzőekben láthattuk, hogy az $XYZ$ térben ez a színhalmaz egy paralelepipedont, az $RGB$ térben egy egységnyi oldalú kockát jelent (lásd \ref{Fig:device_dep} ábra).
Vegyük észre, hogy a $Y,\, R-Y,\, B-Y$ koordinátákat akár az $XYZ$, akár az $RGB$ komponensekből egy lineáris transzformációval előállíthatjuk:
Maradva az NTSC rendszer világosság-együtthatóinál (kiindulva abból, hogy $Y = 0.3R + 0.59G + 0.11B$) a transzformáció alakja
\begin{align}
\begin{bmatrix}[c]
       R- Y \\[0.3em]
       B - Y \\[0.3em]
       Y \end{bmatrix} &= 
\begin{bmatrix}[c c c]
      0.7 &  -0.59&  -0.11  \\[0.3em]
       -0.3 &  -0.59 & 0.89  \\[0.3em]
      0.3 &  0.59&  0.11 \end{bmatrix} 
\begin{bmatrix}[c]
       R \\[0.3em]
       G \\[0.3em]
       B \end{bmatrix}_{\mathrm{NTSC}}.
\end{align}
A lineáris transzformációt az RGB kockán elvégezve megkaphatjuk az ábrázolható színek halmazát.
Az így kapott test az \ref{Fig:YCbCr_space} (a) ábrán látható.
Láthatjuk, hogy az $RGB$ egységkocka egy paralelepipedonba transzformálódott, ahol a paralelepipedon főátlója az $Y$ világosság tengely.
Ennek mentén, az $R-Y = B-Y = 0$ tengelyen helyezkednek el a különböző szürke árnyalatok. 
\begin{figure}[]
	\centering
	\begin{overpic}[width = 0.45\columnwidth ]{figures/YC_space_1.png}
	\small
	\put(0,0){(a)}
	\put(45,90){$Y$}
	\put(48,2){$R\!-\!Y$}
	\put(87,26){$B\!-\!Y$}
	\end{overpic}
	\hspace{6mm}
	\begin{overpic}[width = 0.48\columnwidth ]{figures/YC_space_2.png}
	\small
	\put(0,0){(b)}
	\scriptsize
	\put(39,82){$R$}
	\put(25,24){$G$}
	\put(89,44){$B$}
	\put(12,58){$Y\!e$}
	\put(65,19){$C\!y$}
	\put(78,77){$M\!g$}
	\end{overpic}
	\caption{Az $Y, R-Y, B-Y$ színtér ábrázolható színeinek halmaza oldalnézetből (a) és felülnézetből (b).}
	\label{Fig:YCbCr_space}
\end{figure}

Az eredeti RGB kockához hasonlóan, paralelepipedon főátlón kívüli csúcsaiban (amelyben az $Y=0$ fekete és az $R=G=B=Y=1$ fehér található) az eszközfüggő színtér egy, vagy két $100~\%$-os intenzitású alapszínnel kikeverhető
\begin{equation}
R = \begin{bmatrix}[c] 1\\[0.3em] 0\\[0.3em] 0\end{bmatrix} \hspace{2mm}
G = \begin{bmatrix}[c] 0\\[0.3em] 1\\[0.3em] 0\end{bmatrix}\hspace{2mm}
B = \begin{bmatrix}[c] 0\\[0.3em] 0\\[0.3em] 1\end{bmatrix}\hspace{2mm}
Cy = \begin{bmatrix}[c] 0\\[0.3em] 1\\[0.3em] 1\end{bmatrix}\hspace{2mm}
Mg = \begin{bmatrix}[c] 1\\[0.3em] 1\\[0.3em] 0\end{bmatrix}\hspace{2mm}
Ye = \begin{bmatrix}[c] 1\\[0.3em] 1\\[0.3em] 0\end{bmatrix}
\end{equation}
vörös, zöld, kék alap- és cián, magenta, sárga ún. komplementer színek találhatóak.

Ezen komplementer színek tulajdonsága, hogy az egyes RGB alapszínekkel RGB kockában átlósan helyezkednek el, így a színtérben a lehető legmesszebb elhelyezkedő színpárokat alkotják.
Ennek megfelelően egymás mellé vetítve a komplementer színpárok (vörös-cián, sárga-kék, zöld-magenta) váltják ki a legnagyobb érzékelt kontrasztot.

A paralelepipedonra az $Y$-tengely irányából ránézve (\ref{Fig:YCbCr_space} (b) ábra) láthatjuk a világosságjeltől függetlenül, adott színtérben kikeverhető színek összességét.
Az $R-Y, B-Y, Y$ térben gyakori adott $Y$ világosság mellett a színek ezen $R-Y, B-Y$ síkon való ábrázolása.
Minthogy az $R-Y, B-Y$ jelek meghatározzák adott színpont színezetét és telítettségét, így az ábra azt jelzi, hogy a különböző színezetű és telítettségű színek egy szabályos hatszöget töltenek ki.
A hatszög csúcsai a színtér alap- és komplementerszínei.
Természetesen adott $Y$ érték mellett az ábrázolható színek nem tölti ki teljesen ezt a hatszöget:
adott világosságérték mellett az ábrázolható színek halmaza a $Y, R-Y, B-Y$ paralelepipedon egy adott $Y$ magasságban húzott síkkal vett metszeteként képzelhető el, azaz tetszőleges $0 \leq Y \leq1$ esetén rajzolható egy $R-Y, B-Y$ diagram.
Az így rajzolható diagramokra példákat a \ref{Fig:YCbCr_sect} ábra mutat.
\begin{figure}[]
	\centering
	\begin{overpic}[width = 1\columnwidth ]{figures/YCbCr_2_11.png}
	\small
	\put(0,3){(a)}
	\put(0,37){$Y = 0.11$}
	\end{overpic}
	\vspace{2mm}
	\begin{overpic}[width = 1\columnwidth]{figures/YCbCr_2_30.png}
	\small
	\put(0,37){$Y = 0.3$}
	\put(0,3){(b)}
	\end{overpic}
	\vspace{2mm}
	\begin{overpic}[width = 1\columnwidth]{figures/YCbCr_2_59.png}
	\small
	\put(0,37){$Y = 0.59$}
	\put(0,3){(c)}
	\end{overpic}
	\caption{Különböző $Y$ értékek mellett rajzolható $B-Y, R-Y$ diagramok.}
	\label{Fig:YCbCr_sect}
\end{figure}
Nyilván rögzített $Y$ mellett nem biztos, hogy minden szín $100~\%$-os telítettséggel van jelen a $B-Y,R-Y$ diagramon. 
Például: teljesen telített kékre ($\begin{bmatrix}[c] 0\\[0.3em] 0\\[0.3em] 1\end{bmatrix}$) $Y=0.11$, azaz a $100~\%$ intenzitású kék alapszín ezen magasságban vett diagramon található.
Más magasságban vett  $B-Y, R-Y$ diagramon csak fehérrel higított kék található, azaz nem teljesen telített kék található.

A vizsgált diagramokból leszűrhető, hogy valóban, a világosságjel független a színinformációtól, adott színpont színezetét és telítettségét pusztán az $R-Y$ és $B-Y$ diagramokon vett helye meghatározza.
Vizsgáljuk most, hogyan definiálhatóak ezen érzeti jellemzők, azaz a színezet és telítettség a TV technika $Y, R-Y, B-Y$ színterében!

A könnyebb elképzelhetőség kedvéért ábrázoljuk az $R-Y, B-Y$ koordinátákhoz tartozó színeket, az adott színponthoz tartozó olyan világosságérték mellett, amely esetén minden pontonként teljesül, hogy $X \!+\!Y\!+\!Z = 1$, azaz ezzel gyakorlatilag az adott $RGB$ színtér $xy$-színpatkón vett színét képezzük le az $R-Y, B-Y$ diagramra.
\begin{figure}[]
	\centering
	\begin{minipage}[c]{0.6\textwidth}
	\begin{overpic}[width = 1\columnwidth ]{figures/YCbCr_gamut.png}
	\small
	\put(56,46){$\alpha$}
	\end{overpic} \end{minipage}\hfill
	\begin{minipage}[c]{0.4\textwidth}
	\caption{Adott $Y, R-Y, B-Y$ térben ábrázolható színek gamutja.}
	\label{Fig:ycbcr_gamut}  \end{minipage}
\end{figure}
Az így kapott színhalmaz, amely felfogható az adott alapszínek mellett a luminance-chrominance tér gamutjának is, a \ref{Fig:ycbcr_gamut} ábrán látható.

Megfigyelhető, hogy a diagramon az origóból kiinduló félegyenesen azok a színek vannak, amelyek egymásból kinyerhetők fehér szín hozzáadásával.
Tehát az origóból kiinduló félegyenesen az azonos színezetű, de eltérő telítettségű színek vannak. 
Azaz tetszőleges színpontot vizsgálva, a $B-Y,R-Y$ diagramon a színpontba mutató helyvektor iránya egyértelműen meghatározza az adott pont színezetét.
Ennek megfelelően a TV technikában a színezetet a $B-Y, R-Y$ diagramon a színpont helyvektorának irányszögeként definiáljuk:
\begin{equation}
\text{színezet}_{\mathrm{TV}} = \alpha  = \arctan \frac{R-Y}{B-Y}
\label{eq:hue}
\end{equation}

A telítettség kifejezése már kevésbé egyértelmű, több definíció bevezethető rá.
Mindkét esetben a telítettség természetesen azt fejezi ki, mennyi fehér hozzáadásával keverhető ki egy adott szín a színezetét meghatározó teljesen telített alapszínből.
Az $XYZ$-térben bevezettük a telítettségre a színtartalmat, illetve színsűrűséget.
Mindkét telítettségdefiníció zérus értékű volt a $C$-fehérre, és egységnyi a színpatkót határoló spektrálszínekre.
Felmerül a kérdés, hogyan terjeszthető ki a telítettség fogalma eszközfüggő $RGB$-színterekre.

Ehhez bevezethetjük az \textbf{kávzi-spektrálszínek} fogalmát, amelyek adott $RGB$ színtérben előállítható legtelítettebb színek (a legközelebb vannak az azonos színezetű valódi spektrálszínhez).
Ennek megfelelően a kvázi-spektrálszínek az $xy$-diagramon az adott $RGB$ színtér gamutjának határán helyezkednek el, tehát kikeverhetőek legfeljebb két alapszínből.
Hasonlóképp, \ref{Fig:ycbcr_gamut} diagramon a színteret határoló hatszög csúcsaiban és oldalin találhatóak.

A telítettség ezek után a következő módokon definiálható.
\begin{itemize}
\item  Minthogy egy tetszőleges színnek a fehér színtől, azaz az origótól vett távolsága arányos a szín fehér-tartalmával, így legegyszerűbb módon a telítettség közelíthető a
\begin{equation}
\text{telítettség}_{\mathrm{TV},1} = \sqrt{ (R-Y)^2 +(B-Y)^2}
\label{eq:saturation_1}
\end{equation}
távolsággal.
Később tárgyalt okok miatt az analóg időkben TV technikusok körében ez a definíció volt érvényben.
Az így számolt telítettség valóban $0$ a fehér színre, azonban a kvázi-spektrálszínek telítettsége így nem egységnyi.
%
\item A matematikailag korrekt telítettség-definíció bevezetéséhez kiterjeszthetjük a korábban megismert színsűrűséget eszközfüggő színterekre \footnote{Ismétlésként: az $XYZ$ térben adott pont színsűrűsége $p_c = \frac{Y_d}{Y}$, ahol $Y_d$ az adott színhez tartozó domináns hullámhosszú szín fénysűrűsége, $Y$ a vizsgált szín saját fénysűrűsége.}.
Ennek egyszerűbb értelmezéséhez ábrázoljuk adott színpont paramétereit ún. területdiagramon!
%
\begin{figure}[]
	\centering
	\begin{minipage}[c]{0.6\textwidth}
	\begin{overpic}[width = 1\columnwidth ]{figures/area_chart.png}
	\end{overpic} \end{minipage}\hfill
	\begin{minipage}[c]{0.4\textwidth}
	\caption{Tetszőlegesen választott $R,G,B$ koordináták esetén rajzolható területdiagram.}
	\label{Fig:area_diagram}  \end{minipage}
\end{figure}
%
A területdiagram a következő módon rajzolható fel egy tetszőleges $RGB$ koordinátáival adott szín esetén: 
A vízszintes tengelyt osszuk fel az $Y$ fénysűrűség $RGB$ együtthatóinak megfelelően, majd az egyes $RGB$ komponenseket ábrázoljuk az intenzitásuknak megfelelő magasságú oszlopokkal.
Ekkor egy $Y$ magasságban húzott vonal alatt és fölött a színkülönbségi jeleknek megfelelő magasságú oszlopok alakulnak ki, amely oszlopok előjelesen vett területeinek összege \eqref{eq:chrominances} alapján zérus.
\begin{figure}[b!]
	\centering
	\begin{overpic}[width = 1\columnwidth ]{figures/YCbCr_saturation.png}
	\small
	\put(0,0){(a)}
	\put(50,0){(b)}
	\end{overpic}
	\caption{Az $R-Y,B-Y$ térben ábrázolt színek telítettsége \eqref{eq:saturation_1} (a) és \eqref{eq:saturation_2} (b) alapján számolva}
	\label{Fig:saturations}  
\end{figure}

Válasszuk ki ezután a legkisebb $RGB$ komponenst (a \ref{Fig:area_diagram} ábrán látható példában az $R$) és húzzunk egy vízszintes vonalat ennek magasságában!
Ekkor a vizsgált színt két részre osztottuk: egy fehér színre (amelyre $R=G=B$) és egy kvázi-spektrálszínre, amelynek az egyik $RGB$ komponense zérus, és amelynek fénysűrűsége $Y_d = \min (R,G,B) - Y$.
A domináns hullámhosszú spektrálszín szerepét erre a kvázi-spektrálszínre cserélve kiterjeszthetjük a színsűrűséget az adott eszközfüggő színtérre, amely alapján a telítettség definíciója
\begin{equation}
\text{telítettség}_{\mathrm{TV},2} = \frac{| \min(R,G,B) - Y |}{Y}.
\label{eq:saturation_2}
\end{equation}
Könnyen belátható, hogy az $R = G=B=Y$ fehérpontokra a telítettség definíció szerint 0, míg kvázi-spektrálszínekre ($\min(R,G,B) = 0$) a telítettség azonosan 1.
\end{itemize}
A fent tárgyalt két telítettség-definíció alkalmazásával a \ref{Fig:ycbcr_gamut} ábrán látható színek telítettségét az \ref{Fig:saturations} ábra szemlélteti, megerősítve az eddig elmondottakat.
%
\section{A luma-chroma komponensek és a Gamma-korrekció}

Az előző szakasz bemutatta a TV-(és úgy általában videó-) rendszerek esetében egy színes képpont ábrázolásának módját.
A tényleges videójelek ezen $Y, R-Y, B-Y$ jelekkel rokonmennyiségek, azonban történelmi okokból a feldolgozási lánc egy nem-lineáris transzformációt is tartalmaz, az ún. \textbf{Gamma-korrekciót}.

A Gamma-korrekció bevezetése történeti okokra vezethető vissza.
A CRT megjelenítők elektron-ágyúja erős nem-lineáris karakterisztikával rendelkezik, azaz a képernyő pontjain létrehozott fénysűrűség az anódfeszültség nemlineáris függvénye \footnote{Ez a nemlinearitás az anód-katód feszültség-áram karakterisztikájából származik főleg.
A megjelenítésért felelős foszforok már jó közelítéssel lineárisan viselkednek, azaz a gerjesztéssel egyenesen arányos a létrehozott fénysűrűségük.}.
Ez a karakterisztika jól közelíthető egy 
\begin{equation}
L_{R,G,B} \sim U^{\gamma}
\end{equation} 
hatványfüggvénnyel, ahol a legtöbb korabeli kijelzőre az exponens $\gamma \approx 2.5$, $L_{R,G,B}$ az egyes $RGB$ pixelek fénysűrűsége és $U$ a pixelek vezérlőfeszültsége.
Ez a nemlineáris átvitel természetesen jól látható hatással lenne a megjelenített képre:
Az alacsony $RGB$ szintek kompresszálódnak, míg a világos árnyalatok expandálódnak, ennek hatására a telített színek túltelítődnek, illetve a sötét árnyalatok még sötétebbé válnak.
A nem-kívánatos torzulás az \ref{Fig:gamma} ábrán figyelhető meg.

\begin{figure}[]
	\centering
	\begin{overpic}[width = 1\columnwidth ]{figures/Gamma.png}
	\small
	\put(0,0){(a)}
	\put(52,0){(b)}
	\end{overpic}
	\caption{$RGB$ kép megjelenítése Gamma-korrekcióval (a) és Gamma-korrekció hiányában (b).
	Utóbbi esetben az $R,G,B$ komponensek egy 2.4 exponensű hatványfüggvény általi torzuláson mennek át.}
	\label{Fig:gamma}  
\end{figure}
%
\paragraph{CRT kijelzők kompenzációja:\\}
A torzítás korrekciója kézenfekvő: 
Az $RGB$ komponensek megjelenítés előtti inverz hatványfüggvény szerinti előtorzítása esetén az előtorzítás és a CRT kijelző torzítása együttesen az $RGB$ jelek lineáris megjelenítését teszi lehetővé $\left(U^{\gamma}\right)^{\frac{1}{\gamma}} = U$ alapján.
\begin{figure}[]
	\centering
	\begin{minipage}[c]{0.65\textwidth}
	\begin{overpic}[width = 0.95\columnwidth ]{figures/Gamm2.png}
	\end{overpic} \end{minipage}\hfill
	\begin{minipage}[c]{0.33\textwidth}
	\caption{A Gamma-korrekció alapelve az $RGB$ jelek előtorzításával.}
	\label{Fig:gamma2}  \end{minipage} 
\end{figure}

A korrekció természetesen a megjelenítés előtt bárhol elvégezhető a videófeldolgozási lánc során, azonban a lehető legegyszerűbb felépítésű TV vevők érdekében az előtorzítást az $RGB$ forrás-oldalon célszerű elvégezni \footnote{Természetesen ez a korai TV vevők esetén volt fontos szempont, amikor a Gamma-korrekciót drága/komplex analóg áramkörökkel kellett megvalósítani}.
Ennek megfelelően a Gamma-korrekció már kamera oldalon megvalósul (akár analóg, akár digitális módon) az $RGB$ jelek közvetlen Gamma-torzításával.
A következőkben tehát
\begin{align*}
\begin{split}
R' = R^{\frac{1}{\gamma}}, \hspace{10mm} 
G' = G^{\frac{1}{\gamma}}, \hspace{10mm}
B' = B^{\frac{1}{\gamma}}
\end{split}
\end{align*}
a Gamma-előtorzított $RGB$ összetevőket jelölik, ahol $\frac{1}{\gamma} \approx 0.4-0.6$.

\paragraph{Megjelenítési körülmények kompenzációja:\\}
Az előző gondolatmenet alapján (A CRT kijelzők esetén $\gamma = 2.4$) a Gamma-korrekció exponensére 0.4 adódik.
Mégis, a gyakorlatban ennél gyakran magasabb hatványkitevőket alkalmaznak.
Ennek oka a megjelenítési körülményekre vezethető vissza: \ref{Fig:gamma} ábrán látható, hogy az RGB jelek 1-nél nagyobb hatványkitevőjű torzítása a kontraszt növekedéséhez és a színek telítéséhez vezet.
Ismert tény, hogy a Stevens (Bartelson-Breneman) és a Hunt hatás alapján sötét környezetben a sötét árnyalatok megkülönböztetési képessége romlik, a kép észlelt kontrasztja csökken, a színek színezettsége csökken.
Ha tehát a képi reprodukció helyszínén a környezeti fénysűrűség kicsi (pl. mozi) a megfelelő kontraszt és telítettség eléréséhez a kép előtorzítása szükséges.
Ennek módja olyan Gamma-korrekciós tényező előírása, amely a megjelenítő Gamma-torzítása után egy nem-lineáris, 1-nél kicsivel nagyobb kitevőjű eredő átvitelt valósít meg, így növelve a kontrasztot és a telítettséget.
A megfelelő Gamma-korrekcióval tehát a megjelenítési körülmények hatása kompenzálható.
Ez alapján pl. TV képernyőn való megjelenítéshez az ITU-709-es HD szabvány $\frac{1}{\gamma} = 0.5$ Gamma-korrekciót definiál, ami a 2.4-es CRT torzítással 1.2 eredő torzítást eredményez, míg pl. a mozis célra szánt DCI-P3 szabvány a mozivásznon megjelenített kép 1.5-ös nem-lineáris torzítását írja elő \footnote{Valójában pl. HD esetén a teljes produkciós lánc minden eleme jól definiált, szabványosított.
A képi tartalmat úgy állítják elő, hogy az a kívánt (szubjektíve esztétikus) módon jelenjen meg egy szabványos átlagos megjelenítési környezetben, amelyet a ITU-R BT.2035 szabvány definiál, szabványos ITU-R BT.1886 szabvány szerinti referencia képernyőn megjelenítve.}.
A gyakorlatban természetesen a jelenlegi LCD kijelzők esetében a Gamma-torzítás (vagy éppen az eredő Gamma) szabadon állítható.

\hspace{3mm}
Fontos leszögezni, hogy ugyan a Gamma-korrekciót a CRT képernyők nemlinearitásának kompenzációjára vezették be, manapság a Gamma-korrekció rendszertechnikája a mai napig változatlan (ugyanúgy az $RGB$ jelek kerülnek Gamma-torzításra) annak ellenére, hogy a CRT kijelzők alkalmazását szinte teljesen felváltotta az LCD és LED technológia.
A Gamma-korrekció fennmaradásának oka, hogy a videójel digitalizálása során perceptuális kvantálást valósít meg, ahogyan az a következő fejezetben láthatjuk.

\paragraph{A luma és chroma videójelek:\\}
A Gamma-korrekció ismeretében bevezethetjük a mai videórendszerekben is alkalmazott tárolt és továbbított videójel-komponenseket:
\begin{figure}[]
	\centering
	\begin{overpic}[width = 0.53\columnwidth ]{figures/video_signals.png}
	\end{overpic}
	\hspace{2mm}
	\begin{overpic}[width = 0.44\columnwidth ]{figures/video_signals_2.png}
	\end{overpic}
	\caption{A Gamma-korrekció rendszertechnikája és a videójel-komponensek.}
	\label{Fig:gamma_system}  
\end{figure}
A videókomponensek előállításának rendszertechnikája a \ref{Fig:gamma_system} ábrán látható, az egyszerűség kedvéért most a kamerából ITU szabványba, ITU szabványból megjelenítő saját színterébe való színtérkonverziókat figyelmen kívül hagyva.
\begin{itemize}
\item A Gamma-korrekció a kamera $RGB$-jelein hajtódik végre, SD, illetve HD esetében egy kb. 0.5 kitevőjű hatványfüggvény szerint.
A pontos Gamma-korrekciós görbéket a következőekben fogjuk tárgyalni.
\item Az Gamma-torzított $R',G',B'$ jelekből ezután az adott szabványos színtér előírt világosság-együtthatók alapján előállítjuk az $Y', R'-Y', B'-Y'$ jeleket.
Továbbra is példaként az NTSC rendszer együtthatóinál maradva ezek alakja
\begin{align}
\begin{split}
Y' &= 0.3 \, R' + 0.59 \, G' + 0.11 \, B' \\
R'-Y' &= 0.7 \, R' - 0.59 \, G' - 0.11 \, B' \\
B'-Y' &= -0.3 \, R' - 0.59 \, G' - 0.89 \, B' \\
\end{split}
\end{align}
Ezek tehát az alapvető videójel-komponensek, amelyek végül ténylegesen tárolásra, tömörítésre, továbbításra (pl. műsorszórás) kerülnek.
\item Megjelenítő oldalon a fenti videójelekből a megfelelő inverz-mátrixolással az $R', G', B'$ jelek visszaszámíthatóak.
Megjelenítés során a megjelenítő Gamma-torzításának hatására a kameraoldalon mért $RGB$ komponensekkel lineárisan arányos fénysűrűségű $RGB$ pixelek jelennek meg a kijelzőn.
\end{itemize}
Az így létrehozott $Y', R'-Y', B'-Y'$ jelek kitüntetett szereppel bírnak a videótechnikában, lévén az eddigieket összegezve: ezek adják meg egy színes képpont ábrázolásának módját.
A komponensek neve:
\begin{itemize}
\item $Y'$: \textbf{luma jel} és
\item $R'-Y'$, $B'-Y'$: \textbf{chroma jel}.
\end{itemize}
Fontos észrevenni, hogy a luma jel nem egyszerűen a Gamma-torzított relatív világosság, hanem a Gamma-korrigált $RGB$ jelekből az eredeti $Y$ együtthatókkal számított videójel, azaz
\begin{equation}
Y' = 0.3R^{\frac{1}{\gamma}} + 0.11G^{\frac{1}{\gamma}} + 0.59B^{\frac{1}{\gamma}} \neq Y^{\frac{1}{\gamma}} = \left( 0.3R + 0.59G + 0.11B\right)^{\frac{1}{\gamma}}
\end{equation}
\begin{figure}[]
	\centering
	\begin{overpic}[width = 0.32\columnwidth ]{figures/luma_chroma_0_11.png}
\small
\put(0,0){(a)}
	\end{overpic}
	\begin{overpic}[width = 0.32\columnwidth ]{figures/luma_chroma_0_30.png}
\small
\put(0,0){(b)}
	\end{overpic}
	\begin{overpic}[width = 0.32\columnwidth ]{figures/luma_chroma_0_59.png}
\small
\put(0,0){(c)}
	\end{overpic}
	\caption{A chroma térben ábrázolható színek halmaza fix $Y'$ értékek mellett vizsgálva.}
	\label{Fig:luma_chroma_space}  
\end{figure}
A luma jel fizikai tartalma emiatt nehezen kezelhető: 
Legszorosabban az adott színpont világosságával függ össze, fehér szín speciális esetén pl. ahol $R=G=B=Y_0$
\begin{equation}
Y' = \left( 0.3 + 0.59 +0.11 \right)Y_0^{\frac{1}{\gamma}} = Y_0^{\frac{1}{\gamma}}
\end{equation}
a fenti egyenlőtlenség egyenlőségbe megy át, azaz a luma a Gamma-korrigált világosságjellel egyenlő.
Általánosan azonban a luma jel színinformációt is hordoz magában.
Hasonlóan, a chroma jelek nem szimplán a Gamma-torzított színkülönbségi jelek (de hasonlóan, fehér esetében azonosan nullák), és így világosságinformációt is hordoznak magukban.
	
Adott luma értékek mellett az ábrázolható színek halmaza a \ref{Fig:luma_chroma_space} ábrán látható.
Megfigyelhető, hogy a luminance-chrominance térrel azonosan az ábrázolható színek egy hatszöget feszítenek ki, és a 100\%-osan telített színek helye nem változik (hiszen a 0 és 1 értékeken nem változtat a Gamma-korrekció), ennek megfelelően az egyes pontok színezete a chroma térben változatlan.
Az ábrákon azonban egyértelműen látható, hogy adott $Y'$ értékek mellett is az egyes ábrákon az ábrázolt színek világossága változik, tehát a chroma jelek világosságinformációt is tartalmaznak.
Látható, hogy a Gamma-torzítás hatására---ahogy \ref{Fig:gamma} ábrán is látható---adott $Y'$ mellett a telítetlen (fehérhez közeli) színek sötétebbé válnak, míg a telítettebb színek még telítettebbé válnak. 

%Ez az eddig elmondottak alapján nem kell, hogy problémát okozzon, hiszen pusztán annyit jelent, hogy a világosság és színinformációt nem teljesen szeparáltan kezeljük átvitel tárolás és átvitel során.
%Ugyanakkor látni fogjuk, hogy az emberi látás tulajdonságait kihasználva a színjeleket---azaz a chroma jeleket---csökkentett sávszélességgel, vagy digitális esetben kisebb felbontással továbbítjuk.
%Minthogy a fentiek alapján így kis részben a világosságjel sávszélessége/felbontása is csökken, amelynek már látható hatása lehet a megjelenített képen.-

\section{Videójel-formátumok}

Az előző szakaszban bevezettük a videó-technikában tárolandó, továbbítandó videó-jel komponenseket, a luma és chroma komponenseket.
A következőekben láthatjuk, hogyan vihető át a három egymástól független videójel egy, kettő, illetve három érpáron.
\begin{figure}[]
	\centering
	\begin{overpic}[width = 0.90\columnwidth ]{figures/video_comp.png}
	\end{overpic}
	\caption{A kompozit és komponens videójelek előállításának folyamatábrája}
	\label{Fig:video_components}
\end{figure}
A következőekben bemutatott kompozit és komponens videójelek előállításának folyamatábrája a \ref{Fig:video_components} ábrán látható (kompozit esetben a PAL rendszer példáján).

Az előző fejezetben láthattuk, hogy az emberi szem színezetre vett térbeli felbontóképessége jóval kisebb (kevesebb, mint fele) a világosságéhoz képest.
Ezt kihasználva a különböző videójel formátumokban közös, hogy a színkülönbségi (chroma) jeleket sávkorlátozva, azaz csökkentett felbontással reprezentálják.
Ez analóg formátumok esetén sávszélességet takarít meg, míg digitális esetben már hatékony kompressziós módszerként is felfogható, mint a következőekben látható lesz.

\subsection{A kompozit videójel}
Analóg átviteltechnika szempontjából a legegyszerűbb megoldás a videójel továbbítására a 3 videókomponens egyetlen érpáron való átvitele.
Ebben az esetben a luma és chroma komponensekből egyetlen ún. \textbf{kompozit} jelet kell képzeni, hogy a vevő oldalon az eredeti három komponens különválasztható.
A feladat megoldására három---alapgondolatában azonos---módszer létezik, az NTSC, PAL és SECAM megoldások.
A rendszerek pontos működésétől eltekintve a következő bekezdés az NTSC és PAL kompozitjelek képzésének alapelvét mutatja be.


A kompozit formátum az NTSC rendszer bevezetésével került kidolgozásra a létező fekete-fehér TV-vevőkkel kompatibilis analóg színes műsorszórás megvalósítására.
A feladat a már létező műsorszóró rendszerben alapsávban továbbított luma jelhez (fekete-fehér jelhez) a színinformáció olyan módú hozzáadása volt, hogy a létező monokróm vevőkben a többletinformáció minimális látható hatást okozzon, míg a színes vevő megfelelően külön tudja választani a luma és chroma jeleket.
Tehát más szóval a visszafele-kompatibilitás miatt az új színes rendszerben a luma jelet változatlanul kellett átvinni. 
Minthogy az átvitelhez használt RF spektrum jelentős részét már elfoglalták a frekvenciaosztásban küldött egyes TV csatornák (a képinformáció, és az FM modulált hanginformáció), így a luma és chroma komponensek csak ugyanabban a frekvenciasávban kerülhetnek továbbításra.

Az alapsávi fekete-fehér TV jel felépítése egyszerű:
Egymás után, soronként tartalmazza a CRT elektron-ágyú vezérlőfeszültségének időtörténetét, amely tehát így a műsor vételével teljesen valós időben rajzolja soronként a kijelző képernyőjére az $Y'(t)$ luma jel tartalmát.
Az egyes sorok és képek kijelzése között az elektron-ágyú kikapcsolt állapotban véges idő alatt fut vissza a következő sor, illetve kép elejére. 
%
\begin{figure}[]
	\centering
	\begin{minipage}[c]{0.65\textwidth}
	\begin{overpic}[width = 0.95\columnwidth ]{figures/PAL_line.png}
	\end{overpic} \end{minipage}\hfill
	\begin{minipage}[c]{0.35\textwidth}	\caption{Egyetlen TV sor luma jele és szinkron jelei a PAL rendszer időzítései mellett. Az NTSC esetében a TV sor felépítse jellegere teljesen azonos, a PAL-tól eltérő időzítésekkel.}
	\label{Fig:PAL_line}  \end{minipage}
\end{figure}
%
Ezekben a kioltási időkben a kijelzés vertikális és horizontális szinkronizációjához szükséges sor- és képszinkronjelek kerülnek továbbításra. 
Egy sor felépítése az \ref{Fig:PAL_line} ábrán látható, ahol az aktív soridő a ténylegesen megjelenített világosságjelet tartalmazza, a sorkioltási idő pedig az az időtartam, amíg a CRT kijelző elektronsugara visszafut az adott sor végéről a következő elejére.
Az egyes videósorok felépítését részletesebben a következő fejezetben tárgyaljuk.

A valós idejű átvitel/kijelzés elvéből látható, hogy a színinformáció átvitele időosztásban sem lehetséges, tehát a chroma jeleket a luma jelekkel azonos frekvenciasávban és időben szükséges átvinni.
A megoldás tárgyalása előtt vizsgáljuk külön a chroma jelek továbbításának módját.

\paragraph{A színsegédvivő bevezetése:}
A színformációt hordozó két chroma jel ($Y'(t)-R'(t), Y'(t)-B'(t)$) egyidőben történő átvitele során alapvető feladat a két analóg jel egyetlen jellé való átalakítása.
Erre az kvadratúra amplitúdómoduláció ad lehetőséget, amely egy olyan modulációs eljárás, ahol az információt részben a vivőhullám amplitúdójának változtatásával, részben annak fázisváltoztatásával kódoljuk (ezzel tehát két független jel vihető át egyszerre). 
Mind PAL, mind NTSC rendszer esetében az emberi látás színekre vett alacsony felbontását kihasználva a chroma jeleket erősen (PAL esetében pl. a luma jel ötödére, $1~\mathrm{MHz}$-re) sávkorlátozzák, ezzel az apró, nagyfrekvencián reprezentált részleteket kisimítják. 
Ezután a kvadratúramodulált chroma jeleket pl. PAL esetén
\begin{equation}
c^{\mathrm{PAL}}(t) = \underbrace{U'(t)}_{\left( B'- Y'\right) / 2.03} \cdot \sin \omega_c t + \underbrace{V'(t)}_{\left( R'- Y'\right) / 1.14}  \cdot \cos \omega_c t
\label{Eq:PAL_cr}
\end{equation}
alakban állíthatjuk elő, ahol $\sin \omega_c t$ az ún. \textbf{színsegédvivő}, $\omega_c$ a színsegédvivő frekvencia, $U'(t)$ az ún. fázisban lévő, $V'(t)$ pedig a kvadratúrakomponens.
A kvadratúramodulált színjelek tehát egyszerűen az átskálázott színkülönbségi jelek fázisban és kvadratúrában lévő színsegédvívővel való modulációjával állítható elő.

A színjelek demodulációja koherens (fázishelyes) vevővel egyszerű alapsávba való lekeveréssel és aluláteresztő szűréssel valósítható meg:
\begin{align}
\begin{split}
\sin x \cdot \sin x = \frac{1-\cos 2x}{2}&,\hspace{1cm}
\cos x \cdot \cos x = \frac{1+\cos 2x}{2} \\
\sin x &\cdot \cos x = \frac{1}{2}\sin 2x
\end{split}
\end{align}
trigonometrikus azonosságok alapján $U'(t)$ demodulációja
\begin{multline}
c^{\mathrm{PAL}}_{\mathrm{QAM}}(t)\cdot \sin \omega_c t = U'(t)\cdot \sin \omega_c t\cdot \sin \omega_c t + V'(t) \cdot \cos \omega_c t  \cdot	\sin \omega_c t = \\
\frac{1}{2} U'(t) -
\underbrace{ \xcancel{ \frac{1}{2} U'(t)\cos 2 \omega_c t  + V'(t) \cdot \frac{1}{2}\sin 2 \omega_c t }}_{\text{aluláteresztő szűrés}}
\end{multline}
szerint történik, míg $V'(t)$ demodulálása hasonlóan $\sin \omega_c t$ lekeverés szerint.
A megfelelő demodulációhoz tehát a vevőben a színsegédvivő fázishelye, koherens előállítása elengedhetetlen.
\begin{figure}[]
	\centering
	\hspace{4mm}
	\begin{overpic}[width = 0.90\columnwidth ]{figures/QAM_mod_demod.png}
	\end{overpic}
	\caption{QAM moduláció és demoduláció folyamatábrája}
	\label{Fig:QAM_mod_demod}
\end{figure}

Az NTSC rendszerben a PAL-hoz hasonlóan a színjelek
\begin{equation}
c^{\mathrm{NTSC}}_{\mathrm{QAM}}(t) = I'(t) \cdot \sin \omega_c t + Q'(t) \cdot \cos \omega_c t
\end{equation}
alakban kerültek átvitelre, ahol az in-phase és kvadratúra komponensek rendre
\begin{align}
\begin{split}
I'(t) &= k_1 (R'-Y') + k_2 (B'-Y) ,\\ 
Q'(t) &= k_3 (R'-Y') + k_4 (B'-Y).
\end{split}
\end{align}
A $k_{1-4}$ konstansokat úgy választották meg, hogy az in-phase és kvadratúra modulált jelek nem a kék és piros merőleges bázisvektorok \ref{Fig:ycbcr_gamut} ábrán, hanem ezek kb. $+20^{\circ}$ elforgatottja.
Az így kapott új tengelyek a magenta-zöld és türkiz-narancssárga tengelyek a közvetlen modulálójelek.
Ennek oka, hogy úgy találták, az emberi látás felbontása jóval nagyobb türkiz-narancssárga közti változásokra, mint a magenta-zöld között.
Ezt kihasználva a magenta-zöld $Q'(t)$ színjeleket az $I'(t)$ jelhez képest is jobban sávkorlátozták, sávszélesség-takarékosság céljából.
A PAL rendszer bevezetésének idejére azonban kiderült, hogy ez rendszer felesleges túlbonyolítása, így az új rendszerben megmaradtak az eredeti színkülönbségi jelek modulációjánál.

\vspace{3mm}
Vizsgáljuk végül a modulált színjel fizikai jelentését, az egyszerűség kedvéért $c^{\mathrm{PAL}}(t)$ esetére (PAL rendszerben)!
Az \eqref{Eq:PAL_cr} egyenlet egyszerű trigonometrikus azonosságok alapján átírható a 
\begin{equation}
c^{\mathrm{PAL}}_{\mathrm{QAM}}(t) = \sqrt{U'(t)^2 + V'(t)^2} \, \sin \left( \omega_c t + \arctan \frac{V'(t)}{U'(t)} \right)
\end{equation}
polár alakra.
Minthogy a moduláló $U',V'$ jelek a színkülönbségi jelekkel arányosak, így a fenti kifejezést \eqref{eq:saturation_1} és \eqref{eq:hue}-val összehasonlítva megállapítható, hogy a QAM modulált jel egy olyan szinuszos vivő, amelynek pillanatnyi amplitúdója a továbbított színpont telítettségét, pillanatnyi fázisa a színpont színezetét adja meg.

\begin{figure}[]
	\centering
	\hspace{4mm}
	\begin{overpic}[width = 0.50\columnwidth ]{figures/SMPTE_Color_Bars.png}
\small
\put(-7	,0){(a)}
	\end{overpic} \hfill
	\begin{overpic}[width = 0.395\columnwidth ]{figures/vectorscope.png}
\small
\put(-10,0){(b)}
	\end{overpic}
	\caption{Egy gyakran alkalmazott vizsgálókép (SMPTE color bar) (a) és vektorszkóppal ábrázolva (b).}
	\label{Fig:bar_pattern_vscope}
\end{figure}

A színsegédvivő amplitúdójának és fázisának egyszerű értelmezhetősége miatt az NTSC és PAL jeleket gyakran vizsgálták ún. vektorszkóp segítségével jól meghatározott vizsgálóábrák megjelenítése mellett.
A vektorszkóp kijelzője gyakorlatilag a \ref{Fig:ycbcr_gamut} ábrán is látható $B'-Y', B'-Y'$ térben jeleníti meg a teljes képtartalom (azaz egyszerre az összes képpont) chroma jeleit, $Y'$-tól függetlenül a demodulált chroma-jelek megjelenítésével.
A vektorszkóp gyakorlatilag egy olyan oszcilloszkóp, amelynek $x$ kitérését a demodulált $B'-Y'$, $y$-kitérést a demodulált $R'-Y'$ jel vezérli, így a teljes képtartalom színezetét szinte egyszerre jeleníti meg az előre felrajzol vizsgálati rácson.
Egy tipikus vizsgáló ábra és annak vektorszkópos képe látható a \ref{Fig:bar_pattern_vscope} ábrákon.
A vektorszkóp alkalmazásának előnye, hogy az esetleges amplitúdó és fázishibából származó telítettség és színezethibák jól láthatóvá válnak a kijelzőn az egyes felvetített pontok ''összeszűkülése/tágulása'', illetve a teljes konstelláció elfordulásaként.
Megjegyezhető, hogy a mai digitális videojeleket is gyakran ábrázolják szoftveres vektorszkópon az egyes pixelek színezetének vizsgálatához.

\paragraph{A színsegédvivő frekvencia:}
Vizsgáljuk most, hogyan választható meg a színsegédvivő $\omega_c$ vivőfrekvenciája úgy, hogy a QAM modulált $c^{\mathrm{PAL}}(t)$ jelet a luma jelhez hozzáadva a vevő oldalon lehetséges legyen a vett $c^{\mathrm{PAL}}(t) + Y'(t)$ jelből az eredeti chroma és luma jelek szétválasztása!

A jelek vevőoldali szétválasztására a luma és chroma jelek spektruma ad lehetőséget:
Láthattuk, hogy a videójel az egyes TV sorokban megjelenítendő világosság és színinformáció sorfolytonos időtörténeteként fogható fel.
Természetes képeken a képtartalom sorról sorra csak lassan változik (természetesen a képtartalomban jelenlévő vízszintes éleket leszámítva), így mind a luma, mind a chroma jelek ún. kvázi-periodikusak, azaz közel periodikusak.
Jel- és rendszerelméleti ismereteink alapján tudjuk, hogy egy periodikus jel spektruma vonalas, a jelfrekvencia egész számú többszörösein tartalmaz csak komponenseket.
Ennek megfelelően mind a luma, mind a chroma jelek spektruma közel vonalas: az energiájuk a sorfrekvencia egész számú többszörösein csomósodik.
Természetesen a luma jel az alapsávban helyezkedik el ($0~\mathrm{Hz}$ környezetében), kb. $6~\mathrm{MHz}$ sávszélességben.
A QAM modulált chroma jel spektruma a sávkorlátozás miatt keskenyebb ($1~\mathrm{MHz}$), és középpontját $\omega_c$ vivőfrekvencia határozza meg.
\begin{figure}[]
	\centering
	\hspace{4mm}
	\begin{overpic}[width = 0.80\columnwidth ]{figures/LC_interlace.png}
	\end{overpic} \hfill
	\caption{A luma és chroma jelek spektrális közbeszövésének alapelve a teljes spektrumokat ábrázolva (a) és a spektrális csomókat felnagyítva (b)}
	\label{Fig:YC_interlace}
\end{figure}

A luma-chroma jel összegzése ennek ismeretében egyszerű: 
Az $\omega_c$ vivőfrekvencia megfelelő megválasztásával elérhető, hogy a chroma jel spektrumvonalai (spektrumcsomói) éppen a luma jel spektrumvonalai közé essen, azaz a spektrumukat átlapolódás nélkül közbeszőhetjük.
Az eljárás alapötletét \ref{Fig:YC_interlace} ábra illusztrálja $f_{\mathrm{H}}$-val a sorfrekvenciát jelölve.
A szétválaszthatóság feltétele ekkor 
\begin{equation}
f_c = f_{\mathrm{H}} \cdot \left( \mathrm{n} + \frac{1}{2}\right), \hspace{1.5cm} \mathrm{n} \in \mathcal{N} 
\end{equation}
azaz a színsegédvivő frekvenciája a sorfrekvencia felének egész szűmú többszörösének kell, hogy legyen \footnote{Megjegyezhető, hogy PAL esetében az előre adott sorfrekvenciához egyszerű volt a színsegédvivő-frekvencia megválasztása, míg NTSC esetén bizonyos okok miatt a sorfrekvencia és ebből következően a képfrekvencia megváltoztatására volt szükség. 
Innen származnak a ma is használatos $59.94$ és $29.97~\mathrm{Hz}$ képfrekvenciák, amelyeket a következő fejezet tárgyal részletesen.}.

\paragraph{A CVBS kompozit videójel és luma-chroma szétválasztás:}
Ennek ismeretében végül a teljes kompozitjel a 
\begin{equation}
\text{CVBS}(t) = \mathrm{Sy}(t) + Y'(t) + c_{\mathrm{QAM}}(t)
\end{equation}
alakban áll elő, ahol $Y'$ a luma jel, $c_{\mathrm{QAM}}$ a QAM modulált chroma jelek és $\mathrm{S\!y}(t)$ a kioltási időben jelen lévő sorszinkron és képszinkron jelek.
A CVBS elnevezés gyakori szinoníma a kompozit videójelre, jelentése C: color, V: video (luma), B: blanking (azaz kioltás) és S: sync (azaz szinkronjelek).

Az így létrehozott videójel a fekete-fehér képhez képest csak a modulált színsegédvivőt tartalmazza többletinformációnak.
Egyszerű fekete-fehér vevőn a CVBS jelet megjelenítve a színinformáció nagyfrekvenciás zajként, pontozódásként (ún. \href{http://www.techmind.org/colrec/}{chroma dots}) jelenik csak meg a kijelzőn, így a visszafelé kompatibilitás biztosítva volt.
Színes vevőkben a CVBS jelből a chromajel elméletileg fésűszűréssel szeparálható, amely egy soridejű késleltetést igényel.
Minthogy ez a PAL megjelenése előtt nem állt rendelkezésre, ezért a korai NTSC vevők egyszerű alul/felüláteresztő szűrőkkel, vagy egyszerű chroma jelre állított lyukszűrőkkel szeparálták a luma-chroma jeleket.
Ennek eredményeképp még a színes vevőkben is a chroma jelen kisfrekvenciás tartalomként jelen lehetett a világosságinformáció látható hatással a megjelenített képre.
A megfelelő analóg PAL fésűszűrő-tervezés még a 90-es években is aktív \href{https://www.renesas.com/in/en/www/doc/application-note/an9644.pdf}{K+F} alatt álló terület volt.

\begin{figure}[]
	\centering
	\begin{overpic}[width = 0.45\columnwidth ]{figures/ntsc_color_line.png}
	\end{overpic} \hfill
	\begin{overpic}[width = 0.48\columnwidth ]{figures/Waveform_monitor.jpg}
	\end{overpic} \hfill
	\caption{Az SMPTE color bar vizsgáló ábrának egy, illetve két sorának hullámformája sematikusan (a), és egy hulláforma monitoron (b) vizsgálva}
	\label{Fig:NTSC_line}
\end{figure}

Az elmondottak alapján az NTSC rendszerben a \ref{Fig:bar_pattern_vscope} ábrán látható vizsgálóábrának egy sorának kompozit ábrázolását a \ref{Fig:NTSC_line} mutatja be jellegre helyesen, és egy konkrét hullámforma monitoron mérve.
Az ábrán megfigyelhető az egyes oszlopokhoz tartozó hullámalak: látható, hogy a csökkenő világosságú oszlopokra (amelyek világosságát szaggatott vonal jelzi) hogyan ültették rá a QAM modulált chroma jeleket.
Az első és utolsó fehér, illetve fekete oszlop esetén a chroma jelek amplitúdója zérus (fehérpont), egyéb esetekben a szinuszos színsegédvivő amplitúdója az oszlopok színének telítettségével, fázisa a színezetükkel arányos.
Megjegyezzük, hogy a tényleges hullámforma már átskálázott chroma jeleket ábrázol, amely átskálázás épp azért történik, hogy a teljes CVBS jel beleférjen a fizikai interface dinamikatartományába (ez természetesen a nagy telítettségű színek esetén okozna problémát).
Ez magyarázza tehát az eddig figyelmen kívül hagyott 2.03 és 1.14 skálafaktorokat pl. \eqref{Eq:PAL_cr} esetében.


Az NTSC jel felépítése alapján egyértelmű, hogy a megfelelő színek helyreállításához a vevőben a színsegédvivő fázisának nagyon pontos ismerete szükséges.
Ahhoz, hogy ez biztosítva legyen a sorkioltási időben az ún. hátsó vállra (ld. \ref{Fig:PAL_line} ábra) beültetésre került néhány periódusnyi (9) képtartalom nélküli referenciavivő, az ún. color burst, vagy burst jel.
Ez a burst jel megfigyelhető a \ref{Fig:NTSC_line} ábrán is.

Ennek ellenére az NTSC rendszer továbbra is fázisérzékeny volt, hiszen fázishibát a vevőben is bármelyik alkatrész okozhatott.
A QAM moduláció jelege miatt már a legkisebb fázishiba is látható színezetváltozást okozott a megjelenített képen.
A PAL rendszer tervezésének egyik fő célja épp ezért a rendszer fázishibára vett érzékenységének csökkentése volt

\paragraph{A PAL rendszer:}
Míg az egyszerű NTSC rendszer már 1953-ban bevezetésre került Amerikában, addig Európában egészen az 1960-as éveikg vártak a színes műsorszórás bevezetésére.
Ennek oka, hogy az eltérő hálózati frekvencia miatt az NTSC-t nem lehetett egy az egyben átemelni Európába (ld. később).
Mire az európai rendszert kifejlesztették, az NTSC rendszer jó néhány gyengeségére fény derült, így az újonnan kifejlesztett PAL (Phase Alternate Lines) ezek kijavítását célozta főként meg.
Ennek eredményeképp a PAL rendszer más QAM modulációval dolgozik (a chroma jelek közvetlenül a modulálójelek), eltérő a színsegédvivő frekvencia, és legfontosabb újításként: egy egyszerű megoldással szinte érzéketlen a fázishibára.
\begin{figure}[]
	\centering
	\begin{overpic}[width = 0.45\columnwidth ]{figures/PAL1.png}
	\end{overpic} \hfill
	\begin{overpic}[width = 0.45\columnwidth ]{figures/PAL2.png}
	\end{overpic} \hfill
	\caption{Az SMPTE color bar vizsgáló ábrának egy, illetve két sorának hullámformája sematikusan (a), és egy hulláforma monitoron (b) vizsgálva}
	\label{Fig:PAL1}
\end{figure}

Láthattuk, hogy a vevő oldalán bármilyen fázishiba a színezet jól látható torzulását okozza.
Mivel a fázishiba gyakran elkerülhetetlen, ezért hatásának kiküszöbölésére a PAL rendszer a következő egyszerű megoldást alkalmazza:
\begin{itemize}
\item Az adó oldalon (a PAL jel létrehozása során) képezzük QAM moduláció során a V' chromajel előjelét minden második TV-sorban negáljuk meg, azaz sorról sorra fordított előjellel vigyük át (ez ekvivalens a sorról sorra változó $\pm \cos \omega_c t$ vivővel való modulációval)!
Az eljárás szemléltetésére tegyük fel, hogy két egymás utána sorban minden horizontális pozícióban a színinformáció azonos.
Ekkor egy adott pontra az n. és (n+1). sorban átvitt $U',V'$ jeleket a \ref{Fig:PAL1} (a) ábra szemlélteti pl egy lila képpont átvitele esetén.
\item Tegyük fel, hogy a vevő oldalon a vett jelhez $\Delta \alpha$ fázishiba adódik az átvitel és demoduláció során.
Természetesen a fázishiba hatására az így vett színvektor mind az n., mind az (n+1). sorban azonos irányba fordul az $U'-V'$ konstellációs diagramon (azaz a $R'-Y', B'-Y'$ síkon), ahogy az a \ref{Fig:PAL1} (b) ábrán látható.
\item A vevő oldalán forgassuk vissza minden második sorban a vett $V'$ komponens előjelét és képezzük az (n+1). sor és az n. sor átlagát.
Ezzel természetesen a színjelek vertikális felbontását csökkentjük (az átlagképzés az apró részleteket elsimítja), azonban ennek eredménye az emberi szem színezetre vett felbontása eredményeképp az információveszteség nem látható (a horizontális felbontás már egyébként is jelentősen lecsökkent az egyszerű sávkorlátozás hatására).
Könnyen belátható, hogy a két vektor átlagát képezve éppen az eredeti, hibamentes színvektort kapjuk eredményül.
Két sor esetén azonos sortartalom esetén tehát ezzel az egyszerű trükkel a fázishiba hatása teljesen kiküszöbölhető, míg levezethető, hogy változó sortartalom esetén a fázishiba az átlagvektor hosszának csökkenését okozza, tehát színezetváltozás helyett csak telítettségváltozást okoz.
\end{itemize}
A bemutatott módosított modulációs módszerrel még aránylag nagy fázishibák hatása is minimális hatással van a megjelenített képre.
Az ok, hogy mégis több, mint egy évtizedet kellett várni a PAL rendszer bevezetésére az volt, hogy a módszer alkalmazásához (az átlagolás elvégzéséhez) a videójel soridejű késleltetésére volt szükség.
Ez az 50-es években analóg módon nem megoldható probléma volt amely a PAL implementálását hátráltatta.

A PAL bevezetését végül az olcsón tömeggyártható ún akusztikus művonalak megjelenése tette lehetővé.
Ez az akusztikus művonal, vagy \href{https://www.google.com/search?q=PAL+delay+line&client=firefox-b-d&sxsrf=ALeKk03EUTzVwc7dkYJFnEK-nlEI_p3hng:1586379019108&source=lnms&tbm=isch&sa=X&ved=2ahUKEwi90Kav2tnoAhXJ-ioKHWz6AJcQ_AUoAXoECA0QAw&biw=1407&bih=675}{PAL delay line} egy egyszerű üvegtömb, amelyre egy piezo aktuátor és piezo vevő csatlakozik.
Az adó a TV chroma jelével arányos mechanikai rezgéseket (ultrahang) \href{https://www.youtube.com/watch?v=-qerYLM-eEg}{hoz létre}, amely többszörös visszaverődések után épp egy soridőnyi késleltetést szenvedve ér a vevő elektródához.
Az ultrahang alapú késleltetővonalak egészen a 90-es évek végéig a PAL dekóderek részét képezték.


\begin{figure}[]
	\centering
	\begin{overpic}[width = 0.82\columnwidth ]{figures/PAL_coder.png}
	\end{overpic} \hfill
	\caption{A PAL kódoló felépítése}
	\label{Fig:PAL_coder}
\end{figure}
Az egyszerű PAL kódoló felépítése az eddig elmondottak alapján a \ref{Fig:PAL_coder} ábrán látható.
Röviden összefoglalva, mind a PAL, mind NTSC esetén a kompozitjel létrehozása során a feladat a Gamma-torzított $R',G',B'$ jelekből az $Y',U',V$ (PAL) és $Y',I',Q'$ (NTSC) jelek létrehozása, majd az $U',V'$ és $I'Q'$ jelek megfelelő QAM modulációja. 
Az így létrehozott jeleket összeadva és a kioltási időben továbbított szinkronjelekkel ellátva előáll a CVBS kompozit jel.

\vspace{3mm}
A kompozit videójel fizikai interface megvalósítása szabványról szabványra változó.
Konzumer felhasználás (pl. kézikamerák, videólejátszók, DVD lejátszók) szempontjából a legelterjedtebb csatlakozó a sárga jelölésű RCA végződés, amely az esetleges kísérő hangtól szigetelve, külön érpáron továbbítja a kompozit videójelet.
\begin{figure}[]
	\centering
	\begin{minipage}[c]{0.6\textwidth}
	\begin{overpic}[width = 0.45\columnwidth ]{figures/Composite-video-cable.jpg}
	\end{overpic} 
		\begin{overpic}[width = 0.45\columnwidth ]{figures/s_video.jpg}
	\end{overpic} \end{minipage}\hfill
	\begin{minipage}[c]{0.4\textwidth}
	\caption{Konzumer alkalmazásokhoz használt sárga jelölésű RCA csatlakozó (a) és a luma-QAM chroma jeleket külön érpáron átvivő S-videó csatlakozó (b)}
	\label{Fig:composite_video}  \end{minipage}
\end{figure}

\paragraph{Az S-video interface}
A kompozit és komponens jelek közti kompromisszumként az S-video formátum a luma és chroma jeleket külön érpáron viszi át.
Ezt leszámítva az interface jele teljesen a kompozit videóval azonosak, továbbíthat akár NTSC, akár PAL (akár SECAM) videókomponenseket:
A luma tehát változatlanul alapsávban, míg a chroma a színsegédvivővel modulálva kerül átvitelre.
A chroma jelek modulációja elkerülhetetlen, hiszen a két független színkülönbségi jel egy érpárra való ültetéséhez azokat legalább a sávszélességükkel megegyező frekvenciájú vivőjellel való moduláció szükséges az átlapolódás elkerüléséhez.
Az S-video szabvány csatlakozója a \ref{Fig:composite_video} (b) ábrán látható.

\subsection{A komponens videójel}
