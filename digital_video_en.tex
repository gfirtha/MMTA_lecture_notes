Az előző fejezet bemutatta az egyes képelemek ábrázolásának módját, és bemutatta, hogyan épülnek fel a kompozit és komponens videójelek.
A digitális videójel ezen komponens videójelek közvetlen digitalizálásával nyerhetők.
A digitalizálandó komponensek ritkább esetben közvetlenül a Gamma-korrigált $R'G'B'$ jelek, míg leggyakrabban a luma-chroma $YP'_BP'_R$ komponensek.
A videójel digitális luma-chroma ábrázolását az ún. $\mathbf{Y'C_{B}C'_{R}}$ jeleknek nevezzük\footnote{Az elnevezéseket gyakran hibásan használják, pl. \ycbcr helyett $Y'U'V'$ jelölést alkalmazva, ami helyesen nyilvánvalóan az analóg PAL kompozitjelet jelöli.}.
A következő fejezet a \ycbcr jelek előállításának kérdéseivel foglalkozik.

\section{Az \ycbcr, mint színtér}

A vonatkozó szakirodalomban gyakran a digitális videójel \ycbcr reprezentációjára mintegy eszközfüggő színtér tekintenek.
Ismétlésként: az \ycbcr rendszer $Y'$ luma és $C'$ croma-jeleit az $RGB$ jelekből a 
\begin{align}
\begin{split}
Y' &= k_r \, R' + k_g \, G' + k_b \, B' ,\\
P'_R &= k_1 \, \left( R' - Y' \right) = \frac{1}{2} \frac{1}{1 - k_r} \, \left( R' - Y' \right)\\
P'_B &=  k_2 \, \left( B' - Y' \right) = \frac{1}{2} \frac{1}{1 - k_b} \, \left( B' - Y' \right)
\end{split}
\end{align}
számítható \ypbpr komponensek közvetlen digitalizálásával kaphatjuk.
Az egyenletekben $R', G', B'  \in \lbrace 0, 1 \rbrace$ a Gamma-korrigált színkoordinátái az ábrázolt színpontnak adott eszközfüggő színtérben, $k_{r,g,b,1,2}$ pedig színtértől függő konstansok.
A Gamma-korrekcióról az előző fejezetben láthattuk, hogy analóg esetben egy kb. $0.4-0.5$ hatványfüggvényű nem-lineáris előtorzítást jelent.
Digitális formátumok esetén a korrekció szigorúan definiált, ahogy később látni fogjuk.
A $k_{r,g,b}$ konstansok az adott színtérhez tartozó relatív fénysűrűség-együtthatók, míg $k_1, k_2$ együtthatók a színkülönbségi jeleket a $P'_R \in \lbrace-0.5, 0.5\rbrace$, $P'_B \in \lbrace-0.5, 0.5\rbrace$ tartományba skálázzák (az előző fejezetben láthattuk, hogy $R'-Y'$ dinamikatartománya $\pm 1 - k_1$, $B'-Y'$ dinamikatartománya $\pm 1 - k_3$.
Ebből következik az egyenletek jobb oldala).

Egyszerű példaként tekintsük a HD szabvány színterét.
A szabványos alapszínek használatával a luma jel az $R',G',B'$ jelekből az
\begin{align}
Y' &= 0.2126 \, R' + 0.7152 \, G' + 0.0722 \, B'
\end{align}
együtthatókkal számítható.
Ekkor az $R'G'B'$-színtérből a \ypbpr térbe valót transzformációt
\begin{align}
\begin{split}
\begin{bmatrix}[c]
       Y' \\[0.3em]
       P'_\mathrm{B} \\[0.3em]
       P'_{\mathrm{R}} \end{bmatrix}
       =& 
  \begin{bmatrix}[c c c]
   k_r & k_g & k_b  \\
   -\frac{1}{2}\frac{k_r}{1-k_b} & -\frac{1}{2}\frac{k_g}{1-k_b} & \frac{1}{2} \\
   \frac{1}{2}& -\frac{1}{2}\frac{k_g}{1-k_r} & -\frac{1}{2}\frac{k_b}{1-k_r} \\
\end{bmatrix}
\cdot
\begin{bmatrix}[c]
       R' \\[0.3em]
       G' \\[0.3em]
       B' \end{bmatrix} = \\
&=  \begin{bmatrix}[c c c]
   0.2126  &  0.7152 & 0.0722 \\
   -0.1146 & -0.3854 &  0.5000 \\
    0.5000 & -0.4542 & -0.0458
\end{bmatrix}
\cdot
\begin{bmatrix}[c]
       R' \\[0.3em]
       G' \\[0.3em]
       B' \end{bmatrix} 
\end{split}
\end{align}
írja le.
Az adott így megadott \ypbpr színtérben fix $Y'$ értékek mellett az ábrázolható színek halmazát xy ábra mutatja be.

Az analóg videójel tehát a megjelenítendő kép képpontjainak ezen \ypbpr koordinátáit tartalmazza sorról sorra a kioltási időkkel és az ebben jelenlévő szinkronjelekkel együtt.
A videójel digitalizációja során ezen komponens-jelek digitális ábrázolása a feladat.
A digitalizálás három fő lépésből áll: az időben folytonos videójel átlapolódásgátló szűrése, az időben folytonos videójel mintavétele és az értékkészletben folytonos jelek megfelelő kvantálása, azaz digitális ábrázolása.

A mintavételi frekvencia pontos megválasztása adott képformátum paraméterei (pl. képfrekvencia, sorok száma, stb.) alapján történik.
Ezzel a kérdéssel konkrét képformátumok esetét tárgyalva a következő fejezet foglalkozik részletesen.
A következőekben a színes \ypbpr képpontok digitális reprezentációját, azaz az \ycbcr komponenseket tárgyaljuk.

\section{A perceptuális kvantálás a mintánkénti bitszám}
Vizsgáljuk elsőként, milyen kvantálási karakterisztikát kell alkalmazni a minél hatékonyabb diszkrét ábrázolás eléréséhez, ezzel rávilágítva a Gamma-korrekció valódi jelenlegi szerepére!
Tételezzük fel, hogy a kvantálandó jel a világosság, azaz a lineáris $Y$ relatív fénysűrűség (tehát nem a Gamma torzított $Y'$ luma jel)!
A kvantálási karakterisztika meghatározásához kiindulásul az emberi látás világosságérzékelésének tulajdonságai szolgálnak.

Ökölszabályként elmondható, hogy képi reprodukció során a képen megjelenített csúcsfehér \footnote{maximális diffúz fehér fény} fénysűrűségéhez képest az $1~\%$-nyi fénysűrűségszintek alatt az emberi szem nem képes az árnyalatok között különbséget tenni, pusztán fekete színt érzékel.
Más szóval ez azt jelenti, hogy a a reprodukálandó fénysűrűségszintek dinamikatartománya 100:1.

Emellett láthattuk, hogy átlagos megvilágítási körülmények között az emberi szem fénysűrűségre vett világosságérzékelése közel logaritmikus (a fénysűrűség-érzékelt világosság görbe közel logaritmikus függvény) a kontrasztérzékenység kb. $1~\%$.
Ez azt jeleneti, hogy két különböző fénysűrűségű felület akkor megkülönböztethető, ha a fénysűrűségeik aránya legalább 1.01.
Később ezt az összefüggést pontosították, a CIE L* karakterisztikája alapján---amely nagyobb tartományban írja pontosabban le az emberi látás világosságérzéklését---az érzékelt világosság a fénysűrűség kb. 0.4 hatványkitevőjű hatványfüggvénye.

Ez két elvárást támaszt a kvantálás során a teljes dinamikatartomány kvantálásához látható kvantálási zaj nélkül
\begin{itemize}
\item A legnagyobb és legkisebb ábrázolt fénysűrűségszint aránya legalább 100:1
\item A szomszédos (N. és (N-1).) kódszavak fénysűrűségeinek hányadosa legfeljebb 1.01, azaz relatív különbségük $1~\%$
\end{itemize}

A megfelelő kvantálásra ellenpéldaként vizsgáljuk, mi a probléma pl. lineáris kvantálás alkalmazásával, azaz ha az $Y$ világosságjel kvantálása során a relatív fénysűrűség $Y \in \lbrace Y_0, 100 Y_0 \rbrace$ dinamikatartományának elemihez lineárisan rendelnénk hozzá a digitális kódokat.
A kvantálást tehát a 
\begin{equation}
q =  \nint{ \left( 2^N - 1 \right) \cdot  \frac{Y - Y_0 }{Y_1 - Y_0 } }
\end{equation}
leképzés valósítja meg, ahol $\nint{}$ az legközelebbi egészhez kerekítés operátora, és a kódolandó dinamika tartományból eredően $Y_1 = 100Y_0$.
Hasonlóan az $q.$ kódszóhoz tartozó világosságérték
\begin{equation}
Y^q = q \cdot \frac{Y_1 - Y_0}{2^N - 1} + Y_0.
\end{equation}

\begin{figure}[]
	\centering
	\begin{overpic}[width = 1\columnwidth ]{figures/linear_vs_perc_quant.png}
	\end{overpic}
	\caption{Az szomszédos kódszavakhoz tartozó fénysűrűségek relatív különbsége lineáris (a) és perceptuális (b) kvantálás esetén.}
	\label{Fig:linear_vs_perc_quant}
\end{figure}

Vizsgáljuk ekkor a szomszédos kódszavak relatív különbségét:
\begin{equation}
\frac{Y^{q+1}-Y^q}{Y^q} = \frac{1 }{q + \frac{2^N - 1}{99}}.
\label{Eq:rel_dif}
\end{equation}
Ez alapján pl. $N = 8$  bites kvantálás esetén a $q = 100$-as kód esetére
\begin{equation*}
\frac{Y^{101}-Y^{100}}{Y^{100}} \approx 0.01 = 1~\%,
\end{equation*}
azaz a szomszédos kódokhoz tartozó világosságértékek éppen a megkülönböztethetőség határán vannak.
Kisebb és nagyobb kódszavak (pl. 20 és 21, 200 és 201) esetén ugyanez a relatív különbség
\begin{equation*}
\frac{Y^{21}-Y^{20}}{Y^{20}} \approx 0.05 = 5~\%, \hspace{1cm} \frac{Y^{201}-Y^{200}}{Y^{200}} \approx 0.005 = 0.5~\%,
\label{eq:code_100}
\end{equation*}
Egyértelmű, hogy a 100-as kód alatti kódok esetében a egyes kódszavakhoz tartozó világosságszintek jól megkülönböztethetőek.
Ennek hatásaként a világosságjel lineáris kvantálása esetén a sötét árnyalatok esetén a különböző kvantálási szintek határa láthatóvá válna, ún. sávosodás (banding) jelenne meg a képen.
A bitmélység növelésével természetesen ez a sávosodás elkerülhető:
a legnagyobb relatív eltérés a 0 és 1-es kód között található. 
%
\begin{figure}[]
	\centering
	\begin{overpic}[width = 0.8\columnwidth ]{figures/linear_vs_perc_quant_2.png}
	\end{overpic}
	\caption{Kvantált szürkeskála 7 bites lineáris (a) és perceptuális (b) kvantálás alkalmazása esetén.
	Lineáris kvantálás esetén a $1 / q + \frac{2^7 - 1}{99} = 0.01$ alapján a kvantálási zaj láthatósági határa $q \approx 99$.}
	\label{Fig:linear_vs_perc_quant_2}
\end{figure}
%
A minimális bitszám amelyre \eqref{Eq:rel_dif} relatív különbség már $q=0$-re is $1~\%$-nál nagyobb
\begin{equation}
\frac{99}{2^N - 1} \leq 0.01 \hspace{1cm} N \geq 13.27 
\end{equation}
alapján $N=14$ esetén teljesül, azaz a 14 bites lineáris kvantálás megfelelő lenne \footnote{Ennek megfelelően a kamerák gyakran a nyers képformátum 14 bites lineáris ábrázolást alkalmaznak, amelyből a digitális Gamma-torzítás után újrakvantálással hozzák létre a végleges kimenetet.}.
Ugyanakkor \eqref{eq:code_100} összefüggésből látszik, hogy a 100-as kód fölötti tartományban a kvantálási lépcsők kihasználtsága rossz, a világosságjel feleslegesen finoman kvantált.

Kézenfekvő az ötlet, hogy az $Y$ relatív fénysűrűség helyett az érzékelt, szubjektív világosságinformációt kvantáljuk, így a kvantálási szintek közti szubjektív világosságérzet-különbség állandó.
A CIE $L^*$ görbéje alapján a szubjektív világosság az $Y$ relatív fénysűrűség 0.4-es kitevőjű hatványaként közelíthető.
A kvantálás előtt az $Y$ koordináta ennek megfelelő nem-lineáris előtorzításával tehát \textbf{perceptuális kvantálás} érhető el.
A kvantálási leképzést és az egyes kvantálási szintekhez tartozó fénysűrűségértékeket ebben az esetben a
\begin{equation}
q =  \nint{ \left( 2^N - 1 \right) \cdot  \frac{Y^{0.4} - Y_0^{0.4} }{Y_1^{0.4} - Y_0^{0.4}} }
\hspace{1cm}
Y^q = \left( q \cdot \frac{Y_1^{0.4} - Y_0^{0.4}}{\left( 2^N - 1 \right) } + Y_0^{0.4} \right)^{\frac{1}{0.4}} 
\end{equation}
alakban adhatjuk meg.
Ebből az összefüggésből már könnyen meghatározhatjuk az szomszédos kódszavakhoz tartozó fénysűrűségértékek relatív távolságát, ahogy az a \ref{Fig:linear_vs_perc_quant} (b) ábrán látható.
Megfigyelhetjük, hogy perceptuális kvantálás alkalmazásával a szomszédos fénysűrűségek relatív különbsége a dinamikatartomány nagy részén állandó, és már 10 bites kvantálás esetén is alig nagyobb, mint $1~\%$ \footnote{Harmadik opcióként logaritmikus kvantálást alkalmazva elérhető lenne, hogy a kódszavak elhelyezkedése feleljen meg az 1.01 relatív lépésköznek.
Ekkor $1.01^q \geq 100$ alapján $q = 463$ kódszóra van szükség, aminek reprezentációjára $N=9$ bit elegendő.
Történelmi okokból (Gamma-korrekció) azonban az itt bemutatott perceptuális kvantálás került szabványosításra.}.

Ebből kifolyólag videó esetén stúdiószabványokban perceptuális, nem-lineáris kvantálás mellett legalább 10 bites kvantálást alkalmaznak, míg konzumer célokra (pl. állókép esetén JPEG, műsorszórás, MPEG) elegendőnek ítélték a  8 bites ábrázolást.
Ennek megfelelően pl. az ITU-601 standard defintion és az ITU-709 HD szaványokban 8 és 10 bites reprezentációt írnak elő, a perceptuális kvantálás megvalósításának módjának és a pontos karakterisztikájának szigorú előírásával.
Ezt tárgyaljuk a következő szakaszban.

\section{A gamma-korrekció célja és megvalósítása}

Az előzőekben láthattuk, hogy digitalizálás előtt az $Y$ fénysűrűség kb. 0.4-es hatványfüggvénnyel való előtorzításával közelítően perceptuális kvantálás érhető el (azaz a szubjektív világosságérzet kvantálása).
Ennek hatására a kvantálási zaj a teljes dinamikatartományban egyenletesen oszlik el, és nem okoz jól látható sávosodást a kis fénysűrűségű területeken, mint lineáris kvantálás esetén, ahogy az a \ref{Fig:linear_vs_perc_quant_2} ábrán megfigyelhető.

Vizsgáljuk most, hogyan valósítható meg a perceptuális kvantálás a gyakorlatban a videó-rendszertechnikában!
Az elmondottak alapján egyszerű $RGB$ forrást feltételezve a perceptuális kvantálás az $RGB$ jelekből képzett $Y$ relatív fénysűrűség nem-lineáris, kb. 0.4-es hatványfüggvénnyel való torzítás utáni (tehát az $L^*$ szubjektív világosságjrl) kvantálásával oldható meg.
Ekkor vevőoldalon (kijelző) digitális-analóg átalakítás után a megfelelő inverz-torzítás után az $Y$ összetevő visszanyerhető és a megfelelő inverz-transzformáció után az $RGB$ jelek kijelezhetők.

A kvantálás kérdései során nem vettük eddig figyelembe eddig a gamma-korrekciót: 
Gamma-korrigálás nélkül a kijelző nem-lineáris átvitele a megjelenített kép jól látható gamma-torzításához vezetne, amelyet tehát még a megjelenítés előtt egy kb. 0.4 hatványkitevőjű gamma-korrekciós görbével korrigálni kell.
Ez egy újabb, második nem-lineáris torzítás alkalmazását igényelné a dekóderoldalon, amely az analóg időkben természetesen drága, és feleslegesen bonyolult megoldásnak számított.

Vegyük észre azonban azt a szerencsés, de teljesen véletlen tényt, hogy a gamma-korrekciós hatványfüggvény kitevője pontosan megegyezik az emberi látás világosságérzékelést leíró $L^*$ hatványfüggvényével, azaz a perceptuális kvantálást megvalósító görbéjével.
Az egyszerűsítés kedvéért szakadjunk el a szigorú perceptuális kvantálás elvétől és adó oldalon cseréljük meg a nem-lineáris torzítást és a $P$-vel jelölt $RGB \rightarrow Y, R-Y, B-Y$ lineáris transzformációt!
Ennek hatására két dolog történik:
%
\begin{figure}[]
	\centering
	\begin{overpic}[width = 1\columnwidth ]{figures/gamma_flow_1.png}
	\small	
	\put(0,0){(a)}
	 \vspace{5mm}
 	\end{overpic}
	\begin{overpic}[width = 1\columnwidth ]{figures/gamma_flow_2.png}
	\small	
	\put(0,0){(b)}
	\end{overpic} \vspace{5mm}
	\begin{overpic}[width = 0.98\columnwidth ]{figures/gamma_flow_3.png}
	\small	
	\put(0,0){(c)}
	\end{overpic}
	\caption{.}
	\label{Fig:gamma_flow}
\end{figure}
%
\begin{itemize}
\item Az adó oldalon nem a $L^* = Y^{0.4}$ jelet kvantáljuk, hanem az $R^{0.4},G^{0.4},B^{0.4}$ jelekből képzett $Y'$ \textbf{luma} jelet.
Mint láthattuk, fehér színekre ($R=G=B$) a luma jelre igaz, hogy $Y^{0.4} = Y'$, egyéb színekre a luma jel tartalmaz színinformációt is.
Az így megvalósított kvantálás tehát fehér színekre valóban perceptuális kvantálást valósít meg, egyéb színekre ezt csak jól közelíti.
\item A vevő oldalon a CRT korrekciós görbéje és az $L^* \rightarrow Y$ görbe eredőben épp lineáris átvitelt valósít meg (a görbék ,,kioltják egymást'').
\end{itemize}
Az így kapott rendszertechnika tehát egyetlen nem-lineáris átvitel alkalmazását igényli forrásoldalon és a teljes rendszer az eddig is tárgyalt Gamma-korrekciót valósítja meg.

Ezzel tehát most már láthatjuk a Gamma-korrekció tényleges, jelenlegi szerepét:
Habár manapság már a CRT kijelzőket szinte teljesen leváltották az LCD és LED alapon működő kijelzők, mégis a Gamma-korrekciót változatlanul alkalmazzák forrásoldalon.
Jelentősége ma már \textbf{nem} a CRT képcsövek karakterisztikájának kompenzálása, hanem az érzékeléshez illeszkedő perceptuális kvantálás megvalósítása.

\vspace{3mm}
Manapság természetesen a nem-lineáris átvitel digitális megvalósítása nem számít költséges feladatnak.
A jelenlegi megjelenítők vezérlőfeszültség-fénysűrűség karakterisztikája erősen nemlineáris, kijelzőről kijelzőre változik és jellemzően a CRT-k átvitelétől eltérő.
Emiatt kijelző oldalon egyrészt a gamma-torzítást semlegesíteni kell, illetve a kijelző átvitelének megfelelően az $RGB$ jelek előtorzítása szükséges.
Ezt a feladatot jellegzetesen közvetlenül kijelzés előtt megfelelő Look-up-Table alkalmazásával oldják meg.

\vspace{3mm}
A perceptuális kvantálás megvalósításához a különböző digitális formátumok a gamma-torzítást szabványos módon előírják.
Ezen nem lineáris transzferkarakterisztikák a szabványban ún. \textbf{opto-elektronikus transzfer karakterisztika} néven szerepelnek.
Pontos megválasztásuknak két fő szempontja van:
\begin{itemize}
\item A perceptuális kvantálás megvalósítása
\item A megjelenítési körülmények kompenzációja
\end{itemize}

\paragraph{Megjelenítési körülmények kompenzációja:\\}
Eddigi vizsgálatunk során az emberi látás jellemzői alapján a Gamma-korrekció exponensét 0.4-re választottuk.
Mégis, a gyakorlatban ennél gyakran magasabb hatványkitevőket alkalmaznak.
Ennek oka a megjelenítési körülményekre vezethető vissza: \ref{Fig:gamma} ábrán látható, hogy az RGB jelek 1-nél nagyobb hatványkitevőjű torzítása a kontraszt növekedéséhez és a színek telítéséhez vezet.
Ismert tény, hogy a Stevens (Bartleson-Breneman) és a Hunt hatás alapján sötét környezetben a sötét árnyalatok megkülönböztetési képessége romlik, a kép észlelt kontrasztja csökken, a színek színezettsége csökken.
%
\begin{figure}[]
	\centering
	\begin{overpic}[width = 1\columnwidth ]{figures/stevens.png}
	\end{overpic}
	\caption{A Bartleson-Breneman hatás illusztrációja.}
	\label{Fig:stevens_effect}
\end{figure}
A Bartleson-Breneman hatás pl. a \ref{Fig:stevens_effect} ábrán figyelhető meg.
Látható, hogy az emberi látás világosságérzékelése már képen belül is jelentősen változik a háttérvilágítás függvényében:
Egyrészt világos háttér előtt az érzékelt kontraszt (legvilágosabb és legsötétebb árnyalat aránya) nagyobb, a sötét háttérhez képest.
Másrészt világos háttér előtt a világos árnyalatok által keltett világosságkülönbség nő, a sötét árnyalatok által keltett kontraszt csökken.
Sötét háttér előtt a helyzet megfordul.
Ez azt jelenti, hogy a világos háttér előtt a látás világosság-érzékelése 0.4-nél kicsit nagyobb, sötét háttér előtt 0.4-nél kisebb hatványfüggvénnyel közelíthető.

Ha tehát a képi reprodukció helyszínén a környezeti fénysűrűség kicsi (pl. mozi) a megfelelő kontraszt és telítettség eléréséhez a kép előtorzítása szükséges.
Ennek módja olyan Gamma-korrekciós tényező előírása, amely a megjelenítő Gamma-torzítása után egy nem-lineáris, 1-nél kicsivel nagyobb kitevőjű eredő átvitelt valósít meg, így növelve a kontrasztot és a telítettséget.
A megfelelő Gamma-korrekcióval tehát a megjelenítési körülmények hatása kompenzálható.

Ez alapján pl. TV képernyőn való megjelenítéshez az ITU-709-es HD szabvány által definiált opto-elektronikus átviteli függvény a
\begin{equation}
E = 
\begin{cases}
4.500 L, \hspace{20mm} \mathrm{ha}\, L < 0.018 \\
1.099 L^{0.45} - 0.099, \hspace{3mm} \mathrm{ha}\, L \geq 0.018,
\end{cases}
\end{equation}
alakú, ahol $L \in \{ R, G, B \}$.
A görbe egy hatványfüggvényből és egy lineáris kezdeti szakaszból áll.
Ez a lineáris szakasz megakadályozza, hogy a görbe meredeksége (azaz a sötét árnyalatok erősítése) végtelen nagy legyen, amely erősítés a kamera érzékelőjének zaját látható nagyságúra erősítené.
Ahogy az ref{Fig:itu709} ábrán látható, a teljes görbe jól közelíthető egy $L^{0.5}$ függvénnyel.
\begin{figure}[]
	\centering
	\begin{overpic}[width = 0.7\columnwidth ]{figures/itu709.png}
	\end{overpic}
	\caption{Az ITU-709 HD szabvány (és SD szabvány) és az NTSC ($1/2.2$-es) gamma-karakterisztikája.}
	\label{Fig:itu709}
\end{figure}
A HD szabvány a kijelző oldalon minden esetben egy $\gamma_D \approx 2.5$-kitevőjű átvitelt feltételez.
Ez a gamma-korrekcióval 1.25 eredő hatványkitevőjű torzítást eredményez, amely egy átlagos nappali megvilágítása mellett a készítők által elérni kívánt kontrasztot eredményezi.

Ezzel szemben pl. a mozis célra szánt DCI-P3 szabvány a mozivásznon megjelenített kép 1.5-ös nem-lineáris torzítását írja elő \footnote{Valójában pl. HD esetén a teljes produkciós lánc minden eleme jól definiált, szabványosított.
A képi tartalmat úgy állítják elő, hogy az a kívánt (szubjektíve esztétikus) módon jelenjen meg egy szabványos átlagos megjelenítési környezetben, amelyet a ITU-R BT.2035 szabvány definiál, szabványos ITU-R BT.1886 szabvány szerinti referencia képernyőn megjelenítve.}.
A gyakorlatban természetesen a jelenlegi LCD kijelzők esetében a Gamma-torzítás (vagy éppen az eredő Gamma) szabadon állítható a néző számára optimális kontraszt beállítására.

\section{A digitális ábrázolás dinamikatartománya}

Az előzőekben láthattuk, hogy a perceptuális kvantálás lehetővé teszi az SD és HD tartalomra szánt színtér 10 biten való ábrázolását, valamint konzumer és műsorszórási célokra már 8 bites ábrázolás is kielégítő eredményt ad.
Kézenfekvő lenne a rendelkezésre álló teljes dinamikatartomány kihasználása a digitális tartalom ábrázolására, azaz pl. a világosságjel 8 bites ábrázolása esetén a $\lbrace 0, 255 \rbrace$ tartomány kihasználása úgy, hogy 0 a feketéhez, 255 a fehérhez tartozó kódszó.
Ezt az ún. full range hozzáállást alkalmazza pl. a JPEG kódoló, illetve számos közvetlenül $RGB$ koordinátákkal dolgozó képszerkesztő szoftver.
%https://books.google.hu/books?id=hOu5DQAAQBAJ&pg=PA427&lpg=PA427&dq=RGB+headroom+footroom&source=bl&ots=NsT6C3TiLr&sig=ACfU3U1HLa7oM0fxBZLHjs6PFfuyi2kflg&hl=en&sa=X&ved=2ahUKEwjSlf-Uw-PoAhWkw4sKHebNCl0Q6AEwC3oECA0QLw#v=onepage&q=RGB%20headroom%20footroom&f=false 

\begin{figure}[]
	\centering
	\begin{overpic}[width = 0.7\columnwidth ]{figures/ycbcr_dyn_range.png}
	\end{overpic}
	\caption{A digitális \ycbcr kódszavak dinamikatartománya.}
	\label{Fig:ycbcr_dyn_range}
\end{figure}
Videótechnika szempontjából \ycbcr ábrázolás mellett gyakoribb az \textbf{narrow range} dinamika tartomány alkalmazása.
Ebben az esetben a teljes dinamikatartomány csak egy részét töltik ki az érvényes \ycbcr (vagy éppen $RGB$) értékek, az érvényes kódszavak alatt és fölött ún. \textbf{footroom} és \textbf{headroom} található.
Ezek a kódtartományok csak feldolgozás során (szűrések, mintavételi konverzió stb.) kihasználtak, a tárolás és továbbítás során nem tartalmaznak érvényes videóadatot.
A headroom és a footroom célja a kép szűrése, feldolgozása során esetlegesen fellépő túl- és alullövések (Gibbs-jelenség) kezelése, tárolása adatvesztés (clipping) nélkül.

8 bites reprezentáció esetén luma ($Y'$), illetve esetlegesen $R',G',B'$ jelekre a footroom 15 kódszónyi, míg a headroom 19 kód széles, így a luma jel 0 és 219 között vehet fel értékeket, tehát $16 \leq Y' \leq 235$ (A 0 és 255 kód szinkronizációs célokra foglalt).
A headroom és footroom aszimmetriájának valódi nyomós oka nincsen.

A $C'_\mathrm{B}$ és $C'_\mathrm{R}$ jelek esetén a nullszint a dinamikatartomány közepe, azaz a 128-as kód, míg headroom és footroom azonosan 15 kódnyi, azaz $16 \leq C'_\mathrm{B}, C'_\mathrm{R} \leq 240$.

Magasabb bitszámon való ábrázolás esetén minden előbb leírt jelszint arányosan skálázódik.
Az \ycbcr jelszintek tehát az \ypbpr analóg jelszintekből a 
\begin{equation}
\begin{bmatrix}[c]
       Y' \\[0.3em]
       C'_{\mathrm{B}} \\[0.3em]
       C'_{\mathrm{R}} \end{bmatrix}
       =
D\cdot
\begin{bmatrix}[c]
       16 \\[0.3em]
       128 \\[0.3em]
       128 \end{bmatrix}
+
D\cdot
\begin{bmatrix}[c]
       219 Y' \\[0.3em]
       224 P'_\mathrm{B} \\[0.3em]
       224 P'_{\mathrm{R}} \end{bmatrix}
\end{equation}
összefüggéssel számítható, ahol $D = 2^{N-8}$, $N$-el a bitszámot jelölve.

\section{A színkülönbségi jelek alulmintavételezése}

\begin{figure}[]
	\centering
	\begin{overpic}[width = 1\columnwidth ]{figures/umbrella.png}
 	\end{overpic}
	\caption{Az \ycbcr komponensek tartalma egyszerű tesztkép esetén.
	Megjegyezhető, hogy a $C_B$ és $C_R$ komponensek a színpont színezetét írják le $\lbrace -0.5, 0.5\rbrace$ dinamikatartományban, a tartalmuk ábrázolása így nem egyértelmű.
	Jelen ábrán az egyszerűség kedvéért $Y=0.5$ mellett mutatja a színtartalmat ()}
	\label{Fig:umbrella}
\end{figure}
%
Az világosság és színinformáció külön kezelésének---azaz az $R'G'B'$ színpontok luma+chroma reprezentációjának---két nagy előnye volt bevezetésük idejében.
Egyrészt történelmileg fontos előny a korai fekete-fehér TV vevőkkel való visszafelé kompatibilitás.
Másrészt az ábrázolásmód jobban illeszkedik az emberi színérzékelés modelljéhez \footnote{Habár az emberi szemben a fényérzékelés három különböző, jellemzően vörös, zöld és kék árnyalatokra érzékeny fotoreceptorral történik, az opponens színelmélet alapján a látóidegeken már egy világosságinformáció-jellegű és két színezetet leíró ingerület terjed}:
az emberi látás színezet-információra vett kisebb felbontása lehetővé teszi az analóg chroma jelek sávszélesség-csökkentését.
Digitális ábrázolás, azaz \ycbcr reprezentáció esetén ez a csökkentett felbontású színjel-ábrázolást a \textbf{chroma-alulmintavételezés} (\textbf{chroma subsampling}) valósítja meg.

\subsection{A mintavételi struktúra jelzése}

Az \ycbcr komponensek tartalma egy egyszerű mintakép esetén a \ref{Fig:umbrella} ábrán látható.
Egyértelmű, hogy a színezet-tartalom ritkán tartalmaz apró részleteket (nagyfrekvenciás tartalmat), így a chroma jeleket elegendő kisebb felbontással tárolni.
A chroma jelek felbontása a horizontális és vertikális irányban is csökkenthető, jellegzetesen a luma jel felbontása felére, vagy negyedére választják, azaz alulmintavételezik.
Így különböző \textbf{mintavételi struktúrák,} vagy \textbf{sémák} (\textbf{subsampling scheme}) alakíthatóak ki.
A chroma jel felbontásának a lumáéhoz képesti változását a horizontális és vertikális irányban a
\begin{equation}
J : a : b : \alpha
\end{equation}
mintavételi struktúra jelzéssel adható meg, ahol az egyes betűk a következőket jelölik:
\begin{itemize}
\item $J$: a horizontális referencia szám.
Eredetileg (NTSC és SD esetén) a luma jel mintavételi frekvenciáját jelölte $f^{Y'}_s = J \cdot 3\,\frac{3}{8}~\mathrm{MHz}$ formában (azaz hányszorosa a luma jel mintavételi frekvenciája az NTSC rendszer színsegédvivő-frekvenciájának).
A HD formátumok esetében már $J=22$ nagyságú értékeket kellett volna jelölni.
Ehelyett végül referenciaértékként fixen 4-re választják, amelyhez képest megadhatjuk a chroma jelek alulmintavételezésének mértékét.
\item $a$: a $J$ pixelre eső chroma ($C_{\mathrm{B}}$ és $C_{\mathrm{R}}$) minták száma egy sorban, tehát a színkülönbségi jelek vízszintes irányú alulmintavételezésének mértéke a lumához képest.
Így pl. $J:a= 4:2$ a chromaminták számának felezését jelöli a vízszintes irányban.
\item $b$: a chroma minták függőleges irányú alulmintavételezésének jelzése.
Ha $b = a$, akkor nincs vertikális alulmintavételezés.
Ha $b = 0$, akkor a 2:1 arányú alulmintavételezés történik (azaz a chroma jel vertikális felbontása a luma fele).
\item $\alpha$: a színkulcsolási (pl. green box) csatorna jelenlétét jelzi. 
Ha van színkulcsolás, akkor $\alpha = J$, máskülönben nem jelöljük.
\end{itemize}
\begin{figure}[]
	\centering
	\begin{overpic}[width = 1\columnwidth ]{figures/chroma_subsampling.png}
 	\end{overpic}
	\caption{A leggyakrabban alkalmazott chroma mintavételi struktúrák szemléltetése.}
	\label{Fig:chroma_subsampling}
\end{figure}
Természetesen a chroma jelek alulmintavételezése egyszerű veszteséges tömörítésként fogható fel, a tárolás és továbbítás során létrejövő adatmennyiség csökkentésére alkalmas.
Megjelenítés előtt az alumintavételezett chroma jelek felbontását megfelelő eljárással vissza kell állítani az eredetire, amely után a megjelenítendő $R'G'B'$ jelek kiszámíthatók.

\subsection{A gyakran alkalmazott mintavételi struktúrák}

A leggyakrabban alkalmazott chroma mintavételezési struktúra a \ref{Fig:chroma_subsampling} ábrán láthatóak:
\begin{itemize}
\item \textbf{4:4:4}: Ezen mintavételi struktúra esetén sem horizontális, sem vízszintes alulmintavételezés nem történik, a chroma jelek felbontása a luma jelével megegyező.
Ez esetben választható közvetlen $R'G'B'$ ábrázolás is az \ycbcr helyett.
Konzumer berendezésekben nem alkalmazzák, filmstúdiókban (pl. CGI feldolgozás, filmszkennelés, utómunkálatok során gyakrabban).
Elsőként az ITU-2020 UHD ajánlásban jelent meg szabványosított formában.
Egy pixel ábrázolása 8 bites bitmélység esetén $3 \cdot 8 ~\mathrm{bit} = 24~\mathrm{bit}$ adatigényű.
%
\item \textbf{4:2:2}: A chroma jelek vízszintes felbontása a luma jelek fele, míg függőlegesen nincs alulmintavételezés.
A chroma minták vízszintesen minden második luma mintával megegyező pozícióban helyezkednek el (\textbf{cosited}) (ld. később).
Ez az alapvető SD és HD stúdióformátum, konzumer célra inkább csak high-end berendezések alkalmazzák.
Mivel jut pixelre egy $C_{\mathrm{R}}$ és egy $C_{\mathrm{R}}$ minta, egy pixel ábrázolása 8 bit/komponens esetén $(1+ 2\cdot \frac{1}{2})\cdot 8~\mathrm{bit} = 16~\mathrm{bit}$-et vesz igénybe, így a tömörítési tényező 4:4:4-hez képest $\frac{2}{3}$.
%
\item \textbf{4:1:1}: Manapság ritkábban alkalmazott mintavételi struktúra, DV kézikamerák, olcsóbb konzumer elektronikák használták.
A chroma minták vízszintes felbontása a luma felbontásának negyede, függőlegesen nincs alulmintavételezés.
Mivel 4 pixelre jut egy-egy chroma minta, így egy pixel ábrázolása 8 bit/komponens esetén $(1+ 2\cdot \frac{1}{4})\cdot 8~\mathrm{bit} = 12~\mathrm{bit}$-et vesz igénybe. 
A tömörítési faktor 4:4:4-hez képest $\frac{1}{2}$.
%
\item \textbf{4:2:0}: A manapság is legelterjedtebb, a digitális műsorszórás és lokális, digitális videótárolás mintavételi struktúrája.
A chroma jelek mind horizontálisan, mind vertikálisan felezett felbontásúak a lumához képest.
Hasonlóan a 4:1:1-hez, mivel 4 pixelre jut egy-egy chroma minta, így egy pixel ábrázolása 8 bit/komponens esetén $(1+ 2\cdot \frac{1}{4})\cdot 8~\mathrm{bit} = 12~\mathrm{bit}$-et vesz igénybe, és a tömörítési faktor 4:4:4-hez képest $\frac{1}{2}$.

A 4:2:0 mintavételi struktúra megvalósítása két módon lehetséges, ahol a módszerek között a chroma minták horizontális helye tesz különbséget:
\begin{itemize}
\item A JPEG és MPEG-1 mintavételi struktúrája esetében a chroma minták horizontálisan minden második luma-minta \textbf{közé} esnek.
\item MPEG-2 és újabb kódolok esetén a chroma minták horizontálisan \textbf{egybeesnek} minden második luma-mintával.
\end{itemize}
\end{itemize}
Felmerülhet a kérdés, hogy hogyan értelmezhető a chroma minták helye a luma-mintákhoz képest.
A ennek megválaszolásához fontos ismerni, milyen jelfeldolgozási lépések szükségesek a különböző mintavételi struktúrák előállításához.

\subsection{A mintavételi struktúra-konverzió kérdései}

Láthattuk, hogy a chroma jelek csökkentett felbontású tárolása két fontos jelfeldolgozási lépést igényel:
\begin{itemize}
\item adó oldalon az $R'G'B'$ jelekből képzett chroma jelek csökkentett mintavételi frekvenciával való mintavételezése, vagy eleve digiális ábrázolás esetén a mintavételi struktúrának megfelelően chroma minták elhagyása, azaz \textbf{decimálása}
\item vevő oldalon a megjelenítéshez a hiányzó minták pótlása, azaz \textbf{interpolációja}, majd az $R'G'B'$ komponensek számítása és megjelenítése.
\end{itemize}

\paragraph{A decimálás megvalósítása}
Általánosan, decimálás során a feladat a bemeneti jel mintavételi frekvenciájának csökkentése.
Jól ismert tény, hogy egy jel mintavételezése hatására a folytonos jel spektruma a mintavételi egész számú többszörösein ismétlődni fog.
Emiatt ha a jel sávszélessége nagyobb, mint a mintavételi frekvencia fele a spektrumok átlapolódnak, és a jel nem állítható vissza mintáiból.
\begin{figure}[]
	\centering
	\begin{overpic}[width = 0.8\columnwidth ]{figures/decimation.png}
 	\end{overpic}
	\caption{A decimálás (alulmintavételezés) folyamatábrája: 
	(1) a bemeneti jel, és $F(1)$ a bemenő jel spektruma.
	(2) az átlapolódásgátló szűrő kimenete és $F(2)$ ennek spektruma.
	(3) a kimeneti jel.}
	\label{Fig:decimation}
\end{figure}
Videótechnikában az átlapolódás jól látható zavaró hatással van a visszaállított képre, így a szabványok szigorúan definiálják a mintavételezés előtt az analóg jel sávkorlátozásához szükséges aluláteresztő szűrőket.
Ugyanez a jelenség fennáll eleve mintavett jel mintavételi frekvenciájának csökkentése esetén is:
A mintavételi frekvencia csökkentése után a mintavett jel spektruma az új mintavételi frekvencia egész számú többszörösein fog ismétlődni, ahogy a \ref{Fig:decimation} ábrán látható.
Emiatt az átlapolódás/visszahajlás elkerülése érdekében a jelet újramintavételezés előtt az új mintavételi frekvencia felére kell sávkorlátozni ún. \textbf{átlapolódásgátló szűrés} (\textbf{antialising filtering}) alkalmazásával.
A decimálás folyamatábráját a \ref{Fig:decimation} ábra mutatja.

\vspace{3mm}
Vegyük észre, hogy pl. 4:4:4-ről 4:2:2-re való konverzió során a horizontális minták eldobása során a chroma jelek mintavételi frekvenciáját a felére csökkentjük.
Így tehát sávkorlátozás nélkül---amely a nagyfrekvenciás részletek elkenését jelenti---a minták elhagyása után a chroma jel jó eséllyel átlapolódik.
Ennek megfelelően decimálás során a mintacsökkentés előtt minden esetben a chroma-jel aluláteresztő szűrése szükséges, 4:4:4 $\rightarrow$ 4:2:2 konverzió esetén pl. az eredeti horizontális irányú sávszélesség felére történő sávkorlátozással.
4:4:4 $\rightarrow$ 4:2:0 mintastruktúra elérése esetén vertikális alulátersztő szűrés is szükséges.

\begin{figure}[]
	\centering
	\begin{overpic}[width = 0.9\columnwidth ]{figures/aliasing2.png}
 	\end{overpic}
	\caption{Példa a színkülönbségi jelek átlapolódására nem megfelelő átlapolódásgátló szűrés esetén.
 	A tesztkép (a) egy mind vertikális, mind horizontális irányban növekvő frekvenciájú vörös és zöld alapszín között ingadozó térbeli szinusz.
 	A (b) ábra a tesztkép átlapolódásgátló szűrés nélküli 4:4:4 $\rightarrow$ 4:1:1 konverziójának eredményét mutatja be, inteproláció során egyszerű mintaismétlést alkalmazva ($\mathbf{h}_H= \left[ 1\,1 \,1 \,1 \right]^{\mathrm{T}}$).
	A (c) ábra ideális aluláteresztő szűrést alkalmaz mind átlapolódásgátló, mind interpolációs szűrőként.}
	\label{Fig:chroma_subsampling}
\end{figure}
\vspace{3mm}
Az átlapolódásgátló szűrés elhagyásának hatását az \ref{Fig:chroma_subsampling} (a) ábra mutatja $4:4:4 \rightarrow 4:1:1$ konverzió esetén.
Látható, hogy az átlapolódási jelenségek ún. Moiré ábrák kirajzolódásában manifesztálódnak térben periodikus képelemek esetén \footnote{A jelen mintakép a horizontális irányban periodikus, emiatt az átlapolódás hatása fokozottan látható.
A mintavételi tétel be nem tartása miatt (a térbeli mintázat frekvenciája nagyobb, mint amit a fennmaradó chroma mintákkal ábrázolni lehetne) az eredeti minta helyett új, belapolódó komponensek jelennek meg.}.
A megfelelő átlapolódásgátló szűrés tehát kritikus jelentőségű.

Részletek nélkül: a képtartalom aluláteresztő szűrése, azaz a részletek ,,elkenése'' legegyszerűbben a szomszédos minták (pixelek) súlyozott átlagolásával történik \footnote{Más szóval a kép FIR szűrésével}: 
Tételezzük fel, hogy mind vízszintes, mind függőleges irányban szűrjük a képet, azaz átlagoljuk a szomszédos mintákat.
Az átlagolás során a vízszintes, illetve függőleges minták súlyát rendre $\mathbf{h}_H = h_H(n)$ és $\mathbf{h}_V = h_V(n)$ vektorok tartalmazzák.
Jelöljük az aktuális kép adott komponensének (pl. $C_{\mathrm{B}}$, $C_{\mathrm{R}}$) $m.$ sorának $n.$ oszlopának intenzitását $x(m,n)$-el.	
Ekkor a szűrt (azaz súlyozott átlagolt) képelemek intenzitása
\begin{equation}
y(m,n) = \sum_{k = -\infty}^{\infty} \sum_{l = -\infty}^{\infty} x(k,l)\, h_V(m-k) \, h_H(n-l)
\end{equation}
alakban adható, amely egy függőleges és vízszintes irányú konvolúciót ír le.
A $h_H(n)$ és $h_V(n)$ vektorok a horizontális és vertikális szűrőegyütthatók, vagy szűrőkernelek (impulzusválaszok).

Legegyszerűbb esetben mind vertikális, mind horizontális irányban egyszerűen a két szomszédos minta átlagát képezzük.
Ezt az egyszerű átlagolást a 
\begin{equation}
\mathbf{h}_H =
\mathbf{h}_V =
\begin{bmatrix}[c]
       1/2 \\[0.3em]
       1/2\end{bmatrix}
\end{equation}
szűrőegyütthatókkal való szűrés valósítja meg.
Ez a megoldást használja a JPEG és MPEG-1 kódoló a szabványos 4:2:0 formátumra való konverzió során a chroma jelek szűrésére, minden második horizontális és vertikális minta eldobása előtt.
Mivel minden chroma minta 4 szomszédos minta egyszerű átlagaként áll elő, ezért az eredményként kapott átlagminta a négy eredeti minta közé esik.

A szűrés számításigényének növelésével, hosszabb szűrőket alkalmazva a szűrés hatékonysága növelhető, a zárósávban (az decimálás utáni mintavételi frekvencia fele fölött) nagyobb elnyomás valósítható meg.
Az MPEG-2 kódoló a vertikális irányban az MPEG-1-el megegyező egyszerű átlagolást alkalmaz, míg a horizontális irányban 3 minta átlagolásával hozza létre a szűrt jelet, így az MPEG-1-nél pontosabb szűrést elérve.
A szűrőegyütthatók így MPEG-2 esetében
\begin{equation}
\mathbf{h}_V =
\begin{bmatrix}[c]
       1/2 \\[0.3em]
       1/2\end{bmatrix}
,
\hspace{1cm}
\mathbf{h}_H =
\begin{bmatrix}[c]
       1/4 \\[0.3em]
       1/2 \\[0.3em]
       1/4\end{bmatrix}
\end{equation}
Mivel a vízszintes irányban 3 minta átlagolódik, amelyek közül a középső szerepel a kimenetben a legnagyobb súllyal, így MPEG-2 esetében a szűrt és decimált chroma minták horizontálisan minden második luma minta (azaz az eredeti chroma minta) helyével esnek egybe.

\paragraph{Az interpoláció megvalósítása}
A decimálás ellentétes műveleteként, a vevő oldali interpoláció során a feladat a bemeneti jel mintavételi frekvenciájának növelése, azaz az adó oldalon eldobott chroma minták becslése.
\begin{figure}[]
	\centering
	\begin{overpic}[width = 0.8\columnwidth]{figures/interpolation.png}
 	\end{overpic}
	\caption{Az egyszerű lineáris szűréssel megvalósított interpoláció folyamatábrája:
	(1) a bemeneti jel, és $F(1)$ a bemenő jel spektruma.
	(2) a nullákkal kibővített bemenő jel és $F(2)$ ennek spektruma.
	(3) az interpolációs szűrő kimenete.}
	\label{Fig:interpolation}
\end{figure}
Jelfeldolgozás szempontjából az interpoláció folyamata a \ref{Fig:interpolation} ábrán látható:
A mintavételi frekvencia növeléséhez az ismert minták közé az új mintavételi pontokban 0 kezdeti értékű mintákat helyezünk el.
A csupa-nulla értékű jel az eredeti jelhez adása természetesen a spektrumokat nem módosítja, azonban a valódi mintavételi frekvencia értéke azonban megnőtt, az ábrán látható példában kétszeresére.
Ez alapján egyértelmű, hogy az új mintavételi pontokban a jel értéke a szomszédos minták alapján történő becslése egyenértékű a fennmaradó ,,image'' spektrumok kiszűrésével.
Az interpoláció tehát a bemenet egyszerű aluláteresztő szűrésével megvalósítható, ez az ún. \textbf{rekonstrukciós szűrő}.

Hasonlóan a decimáláshoz, az aluláteresztő szűrés legegyszerűbben a szomszédos minták súlyozott átlagolásaként valósítható meg mind a horizontális, és vertikális irányban. 
Legegyszerűbb esetben a horizontális és vertikális rekonstrukciós szűrőegyütthatók a
\begin{equation}
\mathbf{h}_H =
\mathbf{h}_V =
\begin{bmatrix}[c]
       1 \\[0.3em]
       1\end{bmatrix}
\end{equation}
alakban adhatók meg.
Könnyen belátható, hogy a szűrő mindkét irányban egyszerű mintaismétlést valósít meg (azaz a visszaállított képelem ,,pixeles'' lesz).
A szükséges számításigény növelésével a hiányzó minták pontosabban is becsülhetők, pl. a 
\begin{equation}
\mathbf{h}_H =
\mathbf{h}_V =
\begin{bmatrix}[c]
       1/2 \\[0.3em]
       1 \\[0.3em]
       1/2 \end{bmatrix}
\end{equation}
szűrőegyütthatókkal való szűrés (2:1 arányú interpoláció esetén) egyszerű lineáris interpolációt valósít meg.
A lineáris szűrés (azaz az egyszerű súlyozott átlagolás) mellett egyéb, bonyolultabb módszerek is léteznek az interpoláció megvalósítására, pl. magasabb rendű hatványfüggvényekkel való közelítések (bicubic interpolation).
Az \ref{Fig:interpolation} ábrán látható példa természetesen kirívó, extrém esetet szemléltet.
\begin{figure}[]
	\centering
	\begin{overpic}[width = 1\columnwidth]{figures/subsampling_Example.png}
 	\end{overpic}
	\caption{Egyszerű példa a színkülönbségi jelek alulmintavételezésére}
	\label{Fig:chroma_subsampling_Ex}
\end{figure}

Látható, hogy a bemutatott decimálásra és interpolációra alkalmazott aluláteresztő szűrőegyütthatók formailag megegyeznek.
Fontos különbség, hogy a decimáláshoz alkalmazott szűrőegyütthatók összege 1, hiszen az ettől eltérő súlyozás a jelenergia megváltozását eredményezné (a kép intenzitása változna), míg $N:1$ arányú interpoláció esetén a rekonstrukciós szűrő együtthatóinak összege $N$, hiszen rekonstrukció során egy minta értékéből $N$ mintát becslünk.

A \ref{Fig:interpolation} (c) ábrán látható példában mind az átlapolódásgátló, mind a rekonstrukciós szűrő ideális szűrést valósít meg.
Látható, hogy a szűrés eredményeképp a kép azon részén, ahol a színezet változása a decimálás után már nem ábrázolható a $C_B$ és $C_R$ jelek ,,elkenése'' miatt az apró részletek helyett egy átlagos színezet jelenik meg (ami jelen esetben sárga).
A bemutatott példában az átlapolódás teljesen elkerülhető.

\vspace{3mm}
A gyakorlatban előforduló természetes képek esetében elmondható, hogy a chroma komponens ritkán tartalmaz olyan nagyfrekvenciás komponenseket amelyek kiszűrése látható hatással lenne a teljes, megjelenített (visszainterpolált) képre.
A \ref{Fig:chroma_subsampling_Ex} ábra egy ilyen gyakrabban előforduló tesztképre mutatja be a chroma-alulmintavételezés hatását.
Látható, hogy még a példa kedvéért létrehozott 4:1:0 formátum esetében is---amely esetben a chroma jelek felbontása horizontális és vertikális irányban is a luma negyede---, a visszaállított képen a színjelek felbontáscsökkentése nem zavaró.
A 4:2:0 formátum ehhez képest közel az eredetivel megegyező minőséget biztosít felezett horizontális és vertikális chroma felbontás mellett.