%\documentclass{wileySev}
%%\documentclass{wileySix}
\documentclass{book}
\usepackage[T1]{fontenc}
\usepackage{lmodern}
\usepackage{fancyhdr}
\usepackage{amsmath}
\usepackage[utf8]{inputenc}
\usepackage{overpic}
\usepackage{tabularx} % in the preamble
\usepackage{hyperref}
\usepackage{caption,graphicx,enumitem}
\usepackage{array}
\usepackage{mathtools}
\usepackage{movie15}
\usepackage{makecell}
\usepackage[makeroom]{cancel}
\makeatletter
\renewcommand*\env@matrix[1][*\c@MaxMatrixCols c]{%
  \hskip -\arraycolsep
  \let\@ifnextchar\new@ifnextchar
  \array{#1}}
\makeatother
\usepackage{nccmath}
\makeatletter
\setlength{\@fptop}{0pt}
\makeatother


\usepackage{amssymb}

\DeclareRobustCommand{\bbone}{\text{\usefont{U}{bbold}{m}{n}1}}

\DeclareMathOperator{\EX}{\mathbb{E}}% expected value
\DeclarePairedDelimiter{\nint}\lfloor\rceil

\DeclarePairedDelimiter{\abs}\lvert\rvert
\newcommand{\ycbcr}{$Y'C_{\mathrm{B}}C_{\mathrm{R}}$}
\newcommand{\ypbpr}{$Y'P_{\mathrm{B}}P_{\mathrm{R}}$}
\newcommand{\xyz}{$XY\!Z$}
\renewcommand\theadalign{bc}
\renewcommand\theadfont{\bfseries}
\renewcommand\theadgape{\Gape[4pt]}
\renewcommand\cellgape{\Gape[4pt]}

\hypersetup{
    colorlinks=true,
    linkcolor=blue,
    filecolor=magenta,      
    urlcolor=cyan,
}

\usepackage[american]{babel}
%%%%%%%
%% for times math: However, this package disables bold math (!)
%% \mathbf{x} will still work, but you will not have bold math
%% in section heads or chapter titles. If you don't use math
%% in those environments, mathptmx might be a good choice.

% \usepackage{mathptmx}

% For PostScript text
\usepackage{w-bookps}
\usepackage{sidenotes}

\setcounter{secnumdepth}{3}
\setcounter{tocdepth}{2}


\title{Foundations of Multimedia Technologies}
\author{Dr. Firtha Gergely}
\date{\today}
  

\begin{document}
\sloppy 


\titlepage
\maketitle
\cleardoublepage 
  
\tableofcontents

\pagenumbering{arabic}			% arabic page numbering
\setcounter{page}{1}			% set page counter 


\chapter{Representation of colors in video technologies}
\label{sec:video_color_representation}
Az előző fejezet bemutatta az emberi látás képi reprodukció szempontjából legfontosabb tulajdonságait és részletesen tárgyalta a fény- és színmérés alapjait, bevezetve a világosság fogalmát és a CIE XYZ színteret.
Ez a fejezet ezekre az ismeretekre építve bemutatja a színes képpontok videótechnikában alkalmazott analóg és digitális reprezentációs módját.

\vspace{3mm}
Videótechnika szempontjából az XYZ színteret ritkán alkalmazzák képpontok színkoordinátáinak tárolására, kivétel ez alól a \href{https://en.wikipedia.org/wiki/Digital_Cinema_Package}{digitális mozi} és mozifilm-archiválási alkalmazások\footnote{Ennek oka, hogy egyrészt reprodukcióra közvetlenül nem használható, hiszen az XYZ alapszínek nem valós színek (az X,Y,Z bázisvektorok helyén nem található látható szín), másrészt a teljes látható színek tartománya igen nagy bitmélységet igényel, ráadásul feleslegesen:
Az XYZ tér pozitív térnyolcadát a látható színek csak részben töltik ki (sok olyan kód lenne, amihez nem tartozik látható szín), ráadásul a ezen belül is a megjelenítők a látható színeknek csak egy részét képesek reprodukálni.}.
Ugyanakkor az XYZ tér lehetővé teszi a különböző megjelenítők és kamerák által reprodukálható színek halmazának egyszerű vizsgálatát, valamint az ezen eszközök színterei közti átjárást.
A következő szakasz ezeket a konkrét videóeszközökre jellemző, ún. \textbf{eszközfüggő színtereket} mutatja be.

\section{Device-dependent color spaces}

Az előző fejezetben láthattuk, hogy az emberi látás trikromatikus jellegének, valamint linearitásának (illetve az egyszerű lineáris modelljének) köszönhetően a látható színek egy lineáris 3D vektortérben ábrázolhatóak, amelyben a vektorok összegzési szabálya érvényes: 
Két tetszőleges szín keverékéből származó eredő színinger meghatározható a két színbe mutató helyvektorok összegeként (függetlenül az eredeti színingereket létrehozó fény spektrumától).
Az $xy$-színpatkón ennek megfelelően két szín összege a két színpontot összekötő szakasz mentén fog elhelyezkedni.

Ebből következik, hogy az emberi látás metamerizmusát kihasználva a látható színek nagy része előállítható mesterségesen, megfelelően megválasztott \textbf{alapszínek} (\textbf{primary}) összegeként.
Ez általánosan véve a színes képreprodukció alapja.
Természetesen nem lehet célunk az összes látható szín visszaállítása: 
Minthogy a színpatkón a látható színek határa---amely mentén a spektrálszínek találhatók---folytonos, nem nulla görbületű (azaz végtelen számú infinitezimálisan rövid egyenes szakaszból állítható össze), így elvben végtelen számú spektrálszínt kéne alapszínként alkalmazni az összes látható szín kikeveréséhez.
Felmerül tehát a kérdés, hány alapszín szükséges a színpatkó megfelelő lefedéséhez.

\begin{figure}[]
	\centering
	\begin{overpic}[width = 1\columnwidth]{figures_en/Video_colorspaces/color_space_gamut.png}
	\small
	\put(0,0){(a)}
	\put(50,0){(b)}
	\put(22,22){Gamut}	
	\end{overpic}
	\caption{Az azonos alapszínekkel dolgozó SD, HD és a sRGB színtér gamutja $xy$ (a) és $u^*v^*$ (b) diagramon ábrázolva.}
	\label{Fig:gamut}
\end{figure}

A színdiagramban könnyen felvehető 4 színpont úgy, hogy a négy szín keverékeit lefedő négyszög (azaz a reprodukálható színek területe) közel azonos területű legyen a színpatkó területével.
Ugyanakkor az $L^*u^*v^*$ színtér színpatkójából láthattuk, hogy az emberi felbontás zöld árnyalatokra vonatkozó felbontása rossz, és az perceptuálisan egyenletes színdiagram inkább háromszög alakú.
Ez azt jelenti, hogy három megfelelően megválasztott alapszínnel---amelynek különböző arányú keverékeinek színezete egy háromszögön belül helyezkedik el---az egyenletes színezetű ($u^*v^*$) színpatkó jelentős része lefedhető.
Ebből kifolyólag az additív színkeverésen alapuló képreprodukciós eszközök szinte kizárólag három megfelelően megválasztott piros, zöld és kék alapszínnel dolgozik.

Az ezekből a színekből pozitív együtthatókkal (RGB intenzitásokkal) kikeverhető színek összességét egy adott \textbf{eszközfüggő színtérnek} nevezzük, míg ezzel ellentétben a kolorimetrikus, abszolút színterek (mint pl. a CIE XYZ, $L^*u^*v^*$, $L^*a^*b^*$) közösen ún. \textbf{eszközfüggetlen színterek}.
Az adott eszközfüggő színtérben reprodukálható különböző színezetű színek az $xy$-színpatkóban egy háromszög mentén és belsejében helyezkednek el.
Ezt a háromszögét a színtér \textbf{gamutjának} nevezzük.
Egy egyszerű példa adott RGB színtér gamutjára a \ref{Fig:gamut} ábrán látható\footnote{Természetesen nem csak RGB színterek léteznek, nyomdatechnikában pl a CMYK eszközfüggő színterek elterjedtek, amelyek esetében a négy alapszín a nyomdában alkalmazott tinták színét jelzi.
A következőekben a vizsgálatunkat kizárólag RGB színterekre végezzük el.}.
A színtér gamutjának határán (a háromszög csúcsaiban és oldalain) az adott RGB alapszínekkel elérhető legtelítettebb, a spektrál színekhez legközelebb elhelyezkedő színek találhatóak.
Ezek az ún. \textbf{kvázi-spektrál színek}, amelyek közös tulajdonsága, hogy legfeljebb két alapszínből kikeverhetők.

\vspace{3mm}
Ha egy RGB színtér megfelelően definiált, tetszőleges $C$ színre meghatározhatóak azok az RGB intenzitások, amelyekkel az RGB alapszíneket súlyozva a $C$ szín kikeverhető (amennyiben az RGB értékek pozitívak).
Ezek az adott $C$ szín $\mathbf{c}_{RGB} = \begin{bmatrix}
       R_c \\[0.3em] G_c \\[0.3em] B_c \end{bmatrix}$ \textbf{RGB koordinátái} és a színpont adott RGB térbeli pozícióját írják le.
A színkoordináták definíció szerint 0 és 1 között vehetnek fel értékeket, így a
\begin{equation}
\mathbf{r}_{R\!G\!B} = \begin{bmatrix}
       1 \\[0.3em]
       0 \\[0.3em]
       0 \end{bmatrix}, \hspace{4mm}
\mathbf{g}_{R\!G\!B} = \begin{bmatrix}
       0 \\[0.3em]
       1 \\[0.3em]
       0 \end{bmatrix}, \hspace{4mm}
\mathbf{b}_{R\!G\!B} = \begin{bmatrix}
       0 \\[0.3em]
       0 \\[0.3em]
       1 \end{bmatrix}.
\end{equation}
vektorok rendre a $100~\%$-os intenzitású vörös, zöld és kék alapszínvektort jelölik.

A következő szakasz bemutatja, hogyan definiálnak egy adott eszközfüggő RGB színteret a gyakorlatban, azaz hogy hogyan kell megadni a színtér alapvető jellemzőit ahhoz, hogy ezután tetszőleges szín RGB koordinátái számíthatók legyenek.

\subsection{Definition of device-dependent color spaces}

Vizsgáljunk egy három alapszínt alkalmazó RGB színteret!
Az R, G és B alapszínek természetesen egy-egy vektort határoznak meg az $XYZ$ koordináta-rendszerben, és az egységsíkon vett vetületük/metszéspontjuk adja meg a színpatkón vett $xy$ koordinátáikat.
Ezt illusztrálja a \ref{Fig:device_dep} ábra.
Az alapszín-vektorok $XYZ$ koordinátáit jelölje rendre 
\begin{equation}
\mathbf{r}_{X\!Y\!Z} = \begin{bmatrix}
       X_r \\[0.3em]
       Y_r \\[0.3em]
       Z_r \end{bmatrix}, \hspace{4mm}
\mathbf{g}_{X\!Y\!Z} = \begin{bmatrix}
       X_g \\[0.3em]
       Y_g \\[0.3em]
       Z_g \end{bmatrix}, \hspace{4mm}
\mathbf{b}_{X\!Y\!Z} = \begin{bmatrix}
       X_b \\[0.3em]
       Y_b \\[0.3em]
       Z_b \end{bmatrix}.
\end{equation}
%
\begin{figure}[]
	\centering
	\begin{minipage}[c]{0.65\textwidth}
	\begin{overpic}[width = 1\columnwidth ]{figures_en/device_dep.png}
	\small
	\put(89,19){$X$}
	\put(12,96){$Y$}
	\put(0,4){$Z$}
	\put(36,64){$(X_g,Y_g,Z_g)$}
	\put(10,8){$(X_b,Y_b,Z_b)$}
	\put(39,33){$(X_r,Y_r,Z_r)$}
	\end{overpic}\end{minipage}\hfill
	\begin{minipage}[c]{0.35\textwidth}
	\caption{RGB színtér alapszíneinek helye, és metszéspontja az egységsíkkal az XYZ színtérben.}
	\label{Fig:device_dep}  \end{minipage}
\end{figure}
Amennyiben a három alapszín $XYZ$ koordinátái ismertek, úgy a színtér teljesen definiálva van:
tetszőleges $\mathbf{c}_{X\!Y\!Z}$ színvektor koordinátái meghatározhatóak az adott eszközfüggő $RGB$ térben, amely $\mathbf{c}_{RGB}$ vektor tehát azt írja le, milyen súlyozással keverhető ki az adott $\mathbf{c}$ szín az RGB alapszínekből:
\begin{equation} 
\underbrace{\begin{bmatrix}[c]
       R_c \\[0.3em]
       G_c \\[0.3em]
       B_c \end{bmatrix}}_{\mathbf{c}_{RGB}}
       =
     \mathbf{M}_{X\!Y\!Z \rightarrow R\!G\!B}
      \underbrace{\begin{bmatrix}[c]
       X_c \\[0.3em]
       Y_c \\[0.3em]
       Z_c \end{bmatrix}}_{\mathbf{c}_{X\!Y\!Z}},
\end{equation}
ahol $ \mathbf{M}_{X\!Y\!Z \rightarrow R\!G\!B}$ egy bázistranszformációs mátrix. 
Vice versa, az $RGB$ színtérben adott szín $XYZ$ koordinátái meghatározhatók a 
\begin{equation}
      \underbrace{\begin{bmatrix}[c]
       X_c \\[0.3em]
       Y_c \\[0.3em]
       Z_c \end{bmatrix}}_{\mathbf{c}_{X\!Y\!Z}} = 
     \mathbf{M}_{R\!G\!B \rightarrow X\!Y\!Z}
\underbrace{\begin{bmatrix}[c]
       R_c \\[0.3em]
       G_c \\[0.3em]
       B_c \end{bmatrix}}_{\mathbf{c}_{RGB}}
\end{equation}
egyenletből.
Természetesen fennáll a $\mathbf{M}_{R\!G\!B \rightarrow X\!Y\!Z} = \mathbf{M}_{X\!Y\!Z \rightarrow R\!G\!B}^{-1}$ összefüggés.

Utóbbi transzformációs mátrix egyszerűen meghatározható elemi lineáris algebra ismeretek alapján:
Az $\mathbf{M}_{R\!G\!B \rightarrow X\!Y\!Z}$  mátrix oszlopai egyszerűen az $RGB$ színtér bázisainak $XYZ$-ben vett reprezentációja, azaz általánosan igaz a
\begin{equation}
\begin{bmatrix}[c]
       X_c \\[0.3em]
       Y_c \\[0.3em]
       Z_c \end{bmatrix}
       = 
       \underbrace{
  \begin{bmatrix}[c|c|c]
   X_r & X_g & X_b  \\
   Y_r & Y_g & Y_b \\
   Z_r & Z_g & Z_b  \\
\end{bmatrix}}_{\mathbf{M}_{R\!G\!B \rightarrow X\!Y\!Z}}
\cdot
\begin{bmatrix}[c]
       R_c \\[0.3em]
       G_c \\[0.3em]
       B_c \end{bmatrix}
\label{Eq:CS_transform}
\end{equation}
összefüggés\footnote{Az összefüggés érvényessége könnyen belátható pl. $\mathbf{c}_{RGB} = \begin{bmatrix}[c]
       1 \\[0.3em]
       0 \\[0.3em]
       0 \end{bmatrix}$ helyettesítéssel, amely vektor az $R$ alapszín RGB-ben vett reprezentációja, és \eqref{Eq:CS_transform} egyenletben a transzformációs mátrix első oszlopát választja ki.}.
% POynoton 250.oldal
A transzformációs mátrixok több szempontból fontosak: 
Egyrészt lehetővé teszik a különböző RGB terek közti színtérkonverziókat (ld. következő bekezdés).
Másrészt egy $\mathbf{c}$ színpont $Y_c$ koordinátája a színinger fénysűrűségével arányos, amely az érzékelt világosságot határozza meg.
\emph{A $\mathbf{M}_{R\!G\!B \rightarrow X\!Y\!Z}$ transzformációs második sora tehát meghatározza, hogyan számítható ki egy RGB térben megadott színpont (relatív) fénysűrűsége, azaz világossága.}

\vspace{3mm}
Felmerül a kérdés, milyen teret testet feszítenek ki az $R$,$G$, $B$ alapszínekkel kikeverhető színek összessége, azaz az RGB eszközfüggő színtér az $XYZ$ térben.
Könnyen belátható, hogy a három alapszínvektor pozitív együtthatókkal képzett összes lineáris kombinációja egy paralelepipedont feszít ki, azaz adott eszközfüggő RGB színtér az $XYZ$ térben egy paralelepipedonként ábrázolható.
\begin{figure}[]
	\centering	
	\small
	(a)
	\begin{overpic}[width = 0.45\columnwidth ]{figures_en/Video_colorspaces/device_dep_2.png}
	\small
	\put(-2,5){$Z$}
	\put(89,17){$X$}
	\put(11,97){$Y$}
	\end{overpic}
	(b)
	\begin{overpic}[width = 0.45\columnwidth ]{figures_en/Video_colorspaces/The-RGB-colour-cube.png}
	\end{overpic}
	\caption{Egy adott RGB színtér ábrázolása az $XYZ$ térben (a) és az RGB kockában (b). Az (a) ábrán szereplő vektorok színe a végpontjukban található színt jelzi.}
	\label{Fig:device_dep_2}
\end{figure}

Tekintve, hogy az RGB együtthatók definíció szerint 0 és 1 között vehetnek fel értékeket, ennek megfelelően egy adott RGB térben az ebben a színtérben reprodukálható színek egy kockában helyezkednek el\footnote{Emiatt az RGB színtereket gyakran RGB kockaként említik.}, ahol a kocka origóból induló három éle mentén az alkalmazott RGB alapszínek helyezkednek el.
A transzformációs mátrixok tehát gyakorlatilag olyan lineáris transzformációt valósítanak meg, amelyek a paralelepipedont kockába, és a kockát paralelepipedonba viszik.

\paragraph{A relatív fénysűrűség bevezetése:\\}
Egy RGB színtér tehát teljes egészében adott, amennyiben az alapszín-vektorok $XYZ$ koordinátái ismertek.
A gyakorlatban azonban egy RGB színtér definiálása során az $XYZ$ koordináták helyett az RGB alapszínek és a fehérpontjának színezetét, azaz $xy$ színkoordinátáit adják meg.
Definíció szerint egy adott színtér \textbf{fehérpontja} az adott térben elérhető legvilágosabb pont, amelyet az alapszínek egyenlő arányú keverékével érhetünk el.
Az adott eszközfüggő színtérben a 100\%-os ez alapján (hasonlóan az $XYZ$-beli fehérhez), definíció szerint 
\begin{equation}
\mathbf{w}_{RGB} = \begin{bmatrix}[c]
       1 \\[0.3em]
       1 \\[0.3em]
       1 \end{bmatrix}, \hspace{5mm} \text{és} \hspace{5mm} 
Y_w = 1,
\end{equation}
ahol $Y_w$ a színpont \textbf{relatív fénysűrűsége}, amely tehát 0 és 1 között vehet fel értékeket.
 A \ref{Fig:device_dep} ábrán látható példában a fehér szín vektora a paralelepipedon szürkével jelölt főátlója, ezen vonal mentén helyezkednek el a különböző világosságértékű (árnyalatú) fehér színek.
A fehér szín színezete, azaz $x_w$ és $y_w$ koordinátái ezen vektor az egységsíkkal vett döféspontja határozza meg.

A színteret tehát úgy definiáljuk, hogy a három alapszínvektor $xy$ koordinátája (azaz az iránya) mellett megadjuk az alapszínek egyenlő energiájú keverékének a színezetét, (azaz a három bázisvektor összegének irányát), és rögzítjük, hogy az összegvektor $Y$ koordinátája egységnyi.
Ebből a 9 adatból meghatározhatók az RGB bázisvektorok tényleges hossza, és így a szükséges transzformációs mátrixok felírhatók.

\vspace{3mm}
Az RGB színterek ilyen módú definíciója mögött a motíváció a következő:
Láthattuk, hogy az $XYZ$ koordináták a színérzetet létrehozó spektrummal szorosan összefüggnek, az $Y$ koordináta pl. a fényinger fénysűrűségét adja meg ([$\mathrm{cd}/\mathrm{m}^2$]-ben, vagy nit-ben).
A gyakorlati alkalmazások során azonban nem szempont egy RGB színtér alapszíneinek---pl. egy RGB kijelző LCD alapszíneinek---fizikai jellemzőinek pontos ismerete (azaz pl. hány nit fénysűrűséget hoz létre az R, G, vagy B pixel-elem).
Ennek oka, hogy képi reprodukció során a tényleges, fotometriai abszolút fénysűrűséget szinte soha nem célunk visszaállítani (nem is tudnánk, ha a képernyő maximális létrehozható fénysűrűsége kisebb, mint az eredeti mért fénysűrűség).
Ehelyett az adott megjelenítő eszköz által létrehozható legvilágosabb színhez képest reprodukáljuk az adott képpontok relatív fénysűrűségét.
Az, hogy ez a legvilágosabb pont ténylegesen hány nit fénysűrűséget hoz létre eszközről eszközre változhat, és a megjelenítők fontos paramétere (ez az általában [$\mathrm{cd}/\mathrm{m}^2$]-ben megadott maximális fényerő paraméter).
Az eszközfüggő színterek fenti definíciója tehát azt biztosítja, hogy az $Y$ koordináta az RGB alapszínek fizikai jellemzőitől függetlenül a relatív fénysűrűséget írja le.

\paragraph{A fehér színről általában:\\}
Látható tehát, hogy a fehér szín önmagában szubjektív fogalom: adott környezetben a leginkább akromatikus fényingert nevezzük fehérnek, amelynek spektrális sűrűségfüggvénye minél inkább egységnyi (azaz minél több spektrális komponenst tartalmaz), és ezzel analóg módon RGB színtérben ábrázolva minél közelebb van a csupa-egy vektorhoz.
A fehér fogalom egységesítéséhez bevezettek ún. szabványos megvilágításokat (standard illuminants), amelyet szabványosított RGB  színterek esetén előírnak, mint fehérpont.
Ezeknek a szabványos megvilágításoknak a spektrális sűrűségfüggvénye (és persze az általa keltett színinger $xy$-koordinátái) adott, jól-definiált.
Ilyen szabványos megvilágítások a következők:
\begin{figure}[]
	\centering
	\begin{minipage}[c]{0.6\textwidth}
	\begin{overpic}[width = 0.9\columnwidth ]{figures_en/Video_colorspaces/PlanckianLocus.png}
	\end{overpic} \end{minipage}\hfill
	\begin{minipage}[c]{0.4\textwidth}
	\caption{Különböző hőmérsékletű feketetest sugárzók által keltett színek összessége, azaz a Planck görbe.}
	\label{Fig:planck}  \end{minipage}
\end{figure}
\begin{itemize}
\item E fehér: egyenlő energiájú fehér, a CIE XYZ színtér fehérpontja. Kolorimetria szempontjából jelentős, videótechnikában kevésbé fontos a szerepe, mivel a gyakorlatban nem fordul elő olyan fényforrás, amely minden hullámhosszon azonos energiával sugároz.
\item A fehér: a CIE által szabványosított, egyszerű háztartási wolfram-szálas izzó fényét (azzal azonos színérzetet keltő) fényforrás spektruma és színe, $T_{\mathrm{C}} = 2856~\mathrm{K}$ korrelált színhőmérséklettel \footnote{A korrelált színhőmérséklet (correlated color temperature, CCT, $T_{\mathrm{C}}$) azon feketetest sugárzó hőmérsékletét jelzi, amely az emberi szemben a minősítendő fényforrással azonos színérzetet kelt.
A feketetest (hőmérsékleti) sugárzó által keltett színingerek az $xy$ színdiagramon az ún. Planck-görbét járják be, amely a \ref{Fig:planck} ábrán látható.}.
\item B és C fehér: Az A fehérből egyszerű szűréssel nyerhető, napfényt szimuláló megvilágítások.
A B fehér a déli napsütést modellezi $4874~\mathrm{K}$ színhőmérséklettel, míg a C fehér a teljes napra vett átlagos fény színét (spektrumát) modellezi $6774~\mathrm{K}$ színhőmérséklettel.
\item D fehér: szintén a napfény közelítésére alkalmazott megvilágítások sora.
Videótechnika szempontjából a legfontosabb a D65 fehér, amely jelenleg is az UHD formátumok színterének szabványos fehérpontja.
\end{itemize}

\subsection{Color space conversions}
Az eddigiekben látható volt, hogyan definiálható egy eszközfüggő színtér az alapszíneivel.
Ahogy az elnevezés is mutatja, ezek a színterek jellegzetesen adott eszközre érvényesek, pl. egy kamera a beépített RGB szenzorok, egy kijelző az alkalmazott RGB kristályok által meghatározott RGB színtérben dolgoznak.
Emellett léteznek szabványos RGB színterek amelyek a képi tartalom tárolására, továbbítására szolgálnak egységesített, szabványos módon.
A következő szakasz ezeket a szabványos videószíntereket tárgyalja részletesebben.
%TODO Lab, luv spaces: conversion
Felmerül tehát a természetes igény az egyes színterek közti átjárásra, amelyet \textbf{színtér konverziónak} nevezünk.

A színtérkonverziót az $XYZ$ színtér teszi lehetővé, amely egy eszközfüggetlen, abszolút színtér:
egyes színterek közti konverzió a forrás által létrehozott jelek $XYZ$ színtérbe való transzformációjával, majd ezen reprezentáció a nyelő színterébe való transzformációval történik.
Az $XYZ$ színtér így tehát színterek közti átjárást biztosít, ún. Profile Connection Space-ként működik (hasonlóan pl. a gyakran azonos célra alkalmazott $Lab$ színtérhez).

\begin{figure}[]
	\centering
	\begin{overpic}[width = 1\columnwidth]{figures_en/Video_colorspaces/cs_conversion.png}
	\small
	\put(1,37){$RGB_{\mathrm{cam}}$}
	\put(35,37.5){$XYZ$}
	\put(67,39){$RGB_{\mathrm{ITU}-709}$}
	\put(13,18){$RGB_{\mathrm{ITU}-709}$}
	\scriptsize
	\put(15,29.25){$\mathbf{M}_{\!R\!G\!B_{\mathrm{c\!a\!m}} \!\!\rightarrow \!\!X\!Y\!Z}$}
	\scriptsize
	\put(49,29.25){$\mathbf{M}_{\!X\!Y\!Z \!\rightarrow \!R\!G\!B_{7\!0\!9}} $}
	\small
	\put(87,29){\parbox{.86in}{MPEG kódolás, műsorszórás, tárolás}}
	\put(52,18){$XYZ$}
	\put(87,17){$RGB_{\mathrm{TV}}$}
	\scriptsize
	\put(32.5,9.5){$\mathbf{M}_{\!R\!G\!B_{\mathrm{7\!0\!9}} \!\!\rightarrow \!\!X\!Y\!Z}$}
	\scriptsize
	\put(66.5,9.6){$\mathbf{M}_{\!X\!Y\!Z \!\rightarrow \!R\!G\!B_{7\!0\!9}} $}	
	\end{overpic} 	
	\caption{Színtér-konverzió folyamatábrája.}
	\label{Fig:cs_conversion}
\end{figure}
Egy tipikus színtér konverziós folyamatot az \ref{Fig:cs_conversion} ábra mutat.
Tegyük fel, hogy adott egy HD kamera által rögzített képanyag, ahol a kamera színterét $RGB_{\mathrm{cam}}$ jelöli.
A HD formátum szabványos színteret alkalmaz, amelyet az ITU-709 ajánlásban rögzítettek (lásd később).
A kamera RGB jeleit tehát az esetleges kódolás és tárolás előtt ebbe a HD színtérbe kell konvertálni.
Ez a konverzió a kamerajelek $XYZ$ térbe, majd innen az ITU-709 színtérbe való transzformációval oldható meg, amely a megfelelő transzformációs-mátrixszal való szorzással valósítható meg:
\begin{equation} 
\begin{bmatrix}[c]
       R_{\mathrm{ITU}-709} \\[0.3em]
       G_{\mathrm{ITU}-709} \\[0.3em]
       B_{\mathrm{ITU}-709} \end{bmatrix}
       =
       \mathbf{M}_{ X\!Y\!Z \rightarrow R\!G\!B_{709} } \cdot 
\left(     \mathbf{M}_{R\!G\!B_{\mathrm{cam}} \rightarrow X\!Y\!Z } \cdot
\begin{bmatrix}[c]
       R_{\mathrm{cam}} \\[0.3em]
       G_{\mathrm{cam}} \\[0.3em]
       B_{\mathrm{cam}} \end{bmatrix} \right)
\end{equation}
Természetesen az egymás utáni két mátrixszorzás összevonható, így a két $RGB$ színtér között közvetlen lineáris leképzés határozható meg.
Ez a transzformáció jellegzetesen már a kamerán belül megvalósul.
%
Hasonlóképp, megjelenítőoldalon a
\begin{equation} 
\begin{bmatrix}[c]
       R_{\mathrm{cam}} \\[0.3em]
       G_{\mathrm{cam}} \\[0.3em]
       B_{\mathrm{cam}} \end{bmatrix}
       =
       \mathbf{M}_{ X\!Y\!Z \rightarrow R\!G\!B_{\mathrm{TV}} } \cdot 
\left(     \mathbf{M}_{R\!G\!B_{709}  \rightarrow X\!Y\!Z } \cdot
\begin{bmatrix}[c]
       R_{\mathrm{ITU}-709} \\[0.3em]
       G_{\mathrm{ITU}-709} \\[0.3em]
       B_{\mathrm{ITU}-709} \end{bmatrix}
 \right)
\end{equation}
transzformációt kell elvégezni.

Ez az egyszerű transzformációs módszer lehetővé teszi egy adott színtérben mért színpontok másik színtérben való ábrázolását.
Ugyanakkor felmerül a probléma, hogy a nagyobb gamuttal rendelkező színtérből kisebbe való áttérés esetén az új színtérben nem ábrázolható, gamuton kívüli színek negatív és egynél nagyobb RGB koordinátákkal jelennek meg, míg a kisebb gamutú térből való áttérés esetén a nagyobb gamutú tér egy része kihasználatlan marad.
A probléma megoldására a fenti transzformációk mellett az egyes színterek gamutját valamilyen nemlineáris leképzés segítségével lehet egymásra illeszteni (expandálással, kompresszálással).
Ezek az ún. gamut-mapping technikák.

A következőekben az egyes SD, HD és UHD videóformátumok tárolására és továbbítására alkalmazott eszközfüggő színtereket tárgyaljuk.

\subsection{Color spaces of video technology}

% http://www.displaymate.com/crtvslcd.html
\paragraph{Az NTSC színmérőrendszere:\\}
Az első kodifikált színmérő rendszer az NTSC (National Television System Committee) által 1953-ban szabványosított színes-televíziós műsorszóráshoz alkalmazott NTSC szabvány volt.
A színteret a korabeli foszfortechnológiával létrehozható CRT kijelzők (TV vevők) alapszíneik megfelelően írták elő, így színtérkorrekció vevő oldalon nem volt szükség.
A színmérő rendszer C fehérponttal dolgozott, alapszíneit pedig a \ref{tab:ntsc_colorimetry} táblázat mutatja.
Az így kapott gamut az \ref{Fig:gamut} ábrán látható.
\begin{table}[h!]
\caption{Az NTSC szabvány színmérőrendszere}
\renewcommand*{\arraystretch}{1}
\label{tab:ntsc_colorimetry}
\begin{center}
\small\addtolength{\tabcolsep}{15pt}
    \begin{tabular}[h!]{ @{}c | | l | l @{} }%\toprule
		&   x  	&    y \\ \hline
    R   &  0.67 &	0.33 \\
    G   &  0.21 &   0.71  \\
    B   & 0.14   &	0.08\\
    C fehér     &  0.310 &	0.316  \\
    \end{tabular}
\end{center}
\end{table}
Az alapszínekből és a fehérpontból meghatározható az $RGB_{\mathrm{NTSC}} \rightarrow XYZ$ transzformációs mátrix, amely alakja általánosan
\begin{equation}
\begin{bmatrix}[c]
       X \\[0.3em]
       Y \\[0.3em]
       Z \end{bmatrix}
       = 
  \begin{bmatrix}[c c c]
   0.60 & 0.17 & 0.2  \\
   0.30 & 0.59 & 0.11 \\
   0 & 0.07 & 1.11
\end{bmatrix}
\cdot
\begin{bmatrix}[c]
       R \\[0.3em]
       G \\[0.3em]
       B \end{bmatrix}_{\mathrm{NTSC}}
\label{Eq:NTSC_transform}
\end{equation}
Az egyenlet második sora kitüntetett szereppel bír: meghatározza, hogy az NTSC színtérben hogyan számítható adott $RGB$ színpont relatív fénysűrűsége (világossága):
\begin{equation}Y_{\mathrm{NTSC}} = 
   0.30R + 0.59G + 0.11 B. 
\label{Eq:NTSC_luminance}
\end{equation}
A világosságjel számítása egészen a HD formátum megjelenése (azaz közel 50 éven keresztül) a fenti egyenlet szerint történt.

\vspace{3mm}
Az foszfortechnológia fejlődésével az újabb megjelenítők egyre inkább feláldozták a széles gamutot (azaz a minél telítettebb alapszínek használatát) a minél nagyobb fényerő érdekében: 
Az alkalmazott foszforok a nagyobb érzékelt világosság (fénysűrűség) érdekében egyre nagyobb sávszélességben sugároztak, így az alapszínek egyre kevésbé telítettek lettek, a gamut tehát csökkent (más szóval: az alapszínek spektruma a Dirac-impulzus helyett---amely teljesen telített spektrálszín lenne---szélesebb görbe lett, így a görbe alatti terület---és ezzel a szín világossága nőtt---de telítettsége csökkent).

Mivel így a megjelenítő gamutja jelentősen eltért az NTSC szabványtól, ezért ez a képernyőn látható színek torzulását eredményezte.
Ennek megoldásául a TV vevőkbe analóg színtérkonverziós áramköröket ültettek, amelyek az NTSC és a megjelenítő saját színtere közti konverziót valósította meg\footnote{Ahogy látni fogjuk a későbbiekben: a vevőkbe már csak a nem-lineárisan Gamma-előtorzított $RGB$ jelek jutottak, ahol az inverz torzítást maga a kijelző hajtotta végre. Emiatt a színtérkonverziót csak Gamma-torzított $R'G'B'$ jeleken tudták végrehajtani, ami azonban a telített színeknél ismét látható színezet és fénysűrűség-hibát okozott.}.
Ettől a ponttól tehát a műsorszórás szabványos színtere és a megjelenítők színtere különváltak.

\begin{figure}[]
	\centering
	\begin{overpic}[width = 0.7\columnwidth ]{figures_en/Video_colorspaces/gamuts.png}
	\end{overpic}
	\caption{Az NTSC, PAL/SD/HD/sRGB és UHD szabványok gamutja az $xy$-színpatkóban.
	Az NTSC jóval nagyobb gamuttal dolgozott, mint a ma is használt HD és sRGB formátumok. Ennek oka, hogy a korai CRT megjelenítők ugyan telítettebb, de ugyanakkor kisebb fénysűrűségű és nagy időállandójú foszforokkal dolgoztak, amivel bár nagy színtartományt tudtak megjeleníteni, de kis fényerővel, és mozgó objektumoknál a képernyőn akaratlanul is nyomokat hagyva.}
	\label{Fig:gamut}
\end{figure}
\paragraph{A PAL és az SD színmérőrendszere:\\}
Az európai színes műsorszórás bevezetéséhez az EBU (European Broadcasting Union) 1963-ban szabványosította a PAL (Phase Alternating Line) rendszert, újradefiniálva a színmérőrendszert, új alapszíneket és D65 fehéret alkalmazva:
\begin{table}[h!]
\caption{A PAL szabvány színmérőrendszere}
\renewcommand*{\arraystretch}{1}
\label{tab:pal_colorimetry}
\begin{center}
\small\addtolength{\tabcolsep}{15pt}
    \begin{tabular}[h!]{ @{}c | | l | l @{} }%\toprule
		&   x  	&    y \\ \hline
    R   &  0.64 &  0.33 \\
    G   &  0.29 &  0.60  \\
    B   & 0.15 & 0.06\\
    D65 fehér     &  0.3127 & 0.3290 	  \\
    \end{tabular}
\end{center}
\end{table}
%
Ez matematikailag helyesen a transzformációs mátrix és a világosságjel számításának módjának megváltozását jelentené.
Praktikussági szempontokból azonban a PAL rendszer az NTSC-vel azonos módon, \eqref{Eq:NTSC_luminance} alapján állítja elő a világosságjelet, mivel a gyakorlatban a különbség alig volt látható \footnote{Ennek oka, hogy a világosságjel átviteltechnológia szempontjából fontos: a kamera és a kijelző is $RGB$ jeleket használ, a világosságjelet, ahogy a következőekben látjuk csak a képanyag átviteléhez számítjuk ki.}.
Az PAL alapszíneit és a világosságjel számításának módját átvette az első digitális videóformátum, az ITU (International Telecommunication Union) által szabványosított ITU-601-es SD formátum is 1982-ben.

\paragraph{A HD és UHD formátumok színmérőrendszere:\\}
A HD formátumot az 1990-ben szabványosították az ITU-709-es ajánlás formájában.
Az ajánlás átvette az PAL rendszer alapszíneit, azonban immáron matematikailag precízen, újraszámította a transzformációs mátrixot és a világosságjel együtthatókat, amely tehát HD esetén
\begin{equation}Y_{\mathrm{ITU}-709} = 
   0.2126\,R + 0.7152\,G + 0.0722\,B. 
\label{Eq:NTSC_luminance}
\end{equation}
alapján számítható.
Fontos megjegyezni, hogy az ITU-709 szabvány színmérőrendszerét átvette az sRGB szabvány is, ami a mai napig a számítógépes alkalmazások (és operációs rendszerek) alapértelmezett színteréül szolgál.

Az alkalmazott alapszíneket végül számottevően csak az UHD formátum változtatta meg az ITU-2020 számú ajánlásában 2012-ben.
Az UHD alkalmazásokra a szabvány egy széles gamutú, spektrál-alapszíneket alkalmazó színteret ajánl a \ref{tab:UHDTV_colorimetry} táblázatban látható paraméterekkel. 
\begin{table}[h!]
\caption{Az ITU-2020 szabvány színmérőrendszere}
\renewcommand*{\arraystretch}{1}
\label{tab:UHDTV_colorimetry}
\begin{center}
\small\addtolength{\tabcolsep}{15pt}
    \begin{tabular}[h!]{ @{}c | | l | l @{} }%\toprule
		&   x  	&    y \\ \hline
    R   &  0.708 &	0.292  \\
    G   &  0.17 &	0.797  \\
    B   & 0.131 &	0.046 \\
    D65 fehér     &  0.3127 & 0.3290 	  \\
    \end{tabular}
\end{center}
\end{table}
A szabvány természetesen újradefiniálta a világosság komponens számításának a módját is, amely tehát UHD esetben
\begin{equation}Y_{\mathrm{ITU}-2020} = 
   0.2627\,R + 0.678 \,G + 0.0593\,B 
\label{Eq:UHD_luminance}
\end{equation}
alapján számítható.
A szabvány természetesen nem igényli, hogy az UHD megjelenítők spektrálszíneket legyenek képesek alapszínekként realizálni, a minél szélesebb gamut inkább a jövőbeli technológiák szempontjából ad ajánlást.
A mai konzumer megjelenítők az UHD képanyagot megjelenítés előtt a saját színterükben konvertálják, amely jellegzetesen jóval kisebb a szabvány színterénél.

\subsection{Example for device-dependent color space}
\label{sec:CRT}

Egyszerű példaként az eddig leírtakra vizsgáljuk, hogyan számítható és illusztrálható egy CRT kijelző által megjelenített színek tartománya, röviden rávilágítva a CRT technológia működési elvére is \footnote{Természetesen az itt leírtak változtatás nélkül alkalmazhatók más technológia alapján működő kijelzőkre is, pl. LCD.}.
Bár a CRT technológia kezd egyre inkább eltűnni, néhány évvel ezelőttig a stúdiómonitorok jelentős része még mindig CRT alapon működött köszönhetően a színhű megjelenítésüknek, és a mai LCD megjelenítőkhöz képest is jóval nagyobb statikus kontrasztjuknak.

\begin{figure}[]

	\centering
	\begin{overpic}[width = 0.5\columnwidth ]{figures_en/Video_colorspaces/1024px-CRT_color_enhanced.png}
	\end{overpic}
	\caption{CRT megjelenítő felépítése.}
	\label{Fig:crt}
\end{figure}

A katódsugárcsöves (CRT) kijelzők sematikus ábrája az \ref{Fig:crt} ábrán látható.
A CRT-k kijelzők működésének alapja három ún. elektronágyú volt, amelyek egy fűtőtt katódból (1) és egy nagyfeszültségre helyezett anódból állt.
A melegítés hatására a katód környezetébe szabad elektronok léptek ki, így egy elektronfelhőt képezve a katód körül.
A katód közelébe helyezett nagyfeszültségű (néhány száz Volt) gyorsítóanód hatására a szabad elektronok az anód felé kezdtek mozogni, egy szabad elektronáramot (2) indítva a vákuumban (ugyanezen az elven működtek a vákuum-diódák, triódák, pentódák, stb. is).
Elegendően nagy anódfeszültség (és további anódok jelenléte) esetén az elektronok jelentős része nem csapódott be a gyorsítóanódra, hanem továbbhaladt.
Ezt az elektronnyalábot elektrosztatikusan és mágnesesen (3) fókuszálták, majd egy vezérelt mágneses eltérítő (4) sorról sorra végigfuttatta azt egy anódfeszültségű-ernyőn (5), azaz a képernyőn.
Színes kijelző esetén természetesen három elektronágyú üzemelt párhuzamosan.
A képernyő felszínét pixelekre bontva képpontonként három különböző foszforral borították (7-8), amely gerjesztés (becsapódó elektronok) hatására bizonyos ideig adott spektrális sűrűségfüggvényű fényt bocsájtott ki\footnote{Ellentétben a fluoreszkáló anyagok csak a gerjesztés fennállásának idején bocsájtanak ki fényt. 
A foszforeszkálás időállandója előnyös, hiszen megfelelően megválasztott foszforok épp egy képidőig bocsájtanak ki fényt, így a kijelzett kép nem fog villogni.
Ugyanakkor a korai kijelzők ezen időállandója túl nagy volt, ezért a gyors mozgások elmosódtak a kijelzett képen.}, realizálva ezzel az RGB alapszíneket.

\begin{figure}[]
	\centering
	\begin{overpic}[width = 0.54\columnwidth]{figures_en/Video_colorspaces/sony.png}
	\small
	\put(0,0){(a)}
	\end{overpic}
	\begin{overpic}[width = 0.39\columnwidth]{figures_en/Video_colorspaces/sony_gamut.png}
	\small
	\put(0,0){(b)}
	\end{overpic}
	\begin{overpic}[width = 0.014\columnwidth]{figures_en/Video_colorspaces/sony_gamut_2.png}
	\end{overpic}
	\caption{CRT megjelenítő foszforai által kibocsájtott sugárzás spektrális sűrűségfüggvénye (a) a megjelenítő gamutja és az adott spektrumok/alapszínek által keltett színérzet, valamint a színtér fehérpontja (b).
	A jobb oldali oszlop bal fele a Sony monitor alapszíneit és fehérpontját, a jobb fele az sRGB színtér alapszíneit és fehérpontját szemlélteti.}
	\label{Fig:sony}
\end{figure}

Tekintsünk példaként egy Sony F520 CRT kijelzőt: 
A kijelző RGB foszforjai gerjesztés hatására a \ref{Fig:sony} (a) ábrán látható spektrális sűrűségfüggvényű (sugársűrűségű) fényt bocsájtanak ki egységnyi felületről, egységnyi térszögbe, azaz rendelkezésre állnak a mért $L_{e}^R(\lambda)$, $L_{e}^G(\lambda)$ és $L_{e}^B(\lambda)$ függvények.
Fejezzük ki ezek segítségével a kijelző működéséhez szükséges RGB vezérlőjeleket, illetve vizsgáljuk a megjeleníthető színek tartományát!

A $\overline{x}(\lambda)$, $\overline{y}(\lambda)$, $\overline{z}(\lambda)$ szabványos $XYZ$ spektrális színösszetevő függvények alkalmazásával a piros (és persze a zöld és kék) alapszín abszolút $XYZ$ színkoordinátái rendre a
\begin{align}
\begin{split}
\overline{X}_R &= K_m \int_{380~\mathrm{nm}}^{780~\mathrm{nm}} L_{e}^R(\lambda) \cdot \overline{x}(\lambda) \mathrm{d} \lambda = 45.3, \hspace{5mm} \overline{X}_G = 21.4,\hspace{5mm}  \overline{X}_B = 16.6 \\
\overline{Y}_R &= K_m \int_{380~\mathrm{nm}}^{780~\mathrm{nm}} L_{e}^R(\lambda) \cdot \overline{y}(\lambda) \mathrm{d} \lambda = 25.5
, \hspace{5mm} \overline{Y}_G = 48,\hspace{5mm}  \overline{Y}_B = 6.7 \\
\overline{Z}_R &= K_m \int_{380~\mathrm{nm}}^{780~\mathrm{nm}} L_{e}^R(\lambda) \cdot \overline{z}(\lambda) \mathrm{d} \lambda  = 2.4, \hspace{5mm} \overline{Z}_G = 11.6,\hspace{5mm}  \overline{Z}_B =84.6\\
\end{split}
\end{align}
integrálok numerikus kiértékelésével számítható, ahol $K_m = 683~\mathrm{lm/W}$ fényhasznosítási tényező.
A színtérben előállítható fehér szín definíció szerint az alapszínvektorok egyenlő súlyú összegeként áll elő, azaz
\begin{equation}
\overline{X}_W = \overline{X}_R + \overline{X}_G + \overline{X}_B, \hspace{6mm} 
\overline{Y}_W = \overline{Y}_R + \overline{Y}_G + \overline{Y}_B, \hspace{6mm} 
\overline{Z}_W = \overline{Z}_R + \overline{Z}_G + \overline{Z}_B,
\end{equation}
azaz pl. a fehér szín abszolút fénysűrűsége $80.2~\mathrm{cd/m^2}$.
Ez egészen pontosan megegyezik az sRGB szabvány által előírt referenciamonitor fénysűrűségével ($80~\mathrm{cd/m^2}$).

Természetesen az alapszíneknek nem az abszolút XYZ koordinátái a fontosak, hanem a relatív koordináták, amelyekre teljesül, hogy $Y_W=1$, és így $Y$ a relatív fénysűrűség.
A fenti alapszínvektorok tehát $\overline{Y}_W$ értékével normálandók.
Az így kapott relatív alapszínvektorokból már összeállíthatók a színtér alkalmazásához szükséges transzformáció mátrixok:
\begin{align}
\begin{split}
\begin{bmatrix}[c]
       X \\[0.3em]
       Y \\[0.3em]
       Z \end{bmatrix} &= 
     \underbrace{ \begin{bmatrix}[c|c|c]
       0.5646 &  0.2665 &  0.2068 \\[0.3em]
       0.3174 &  0.5992 &  0.0834 \\[0.3em]
       0.0302 &  0.1443 &  1.0539 \end{bmatrix} }_{\mathbf{M}_{R\!G\!B \rightarrow X\!Y\!Z}}
\begin{bmatrix}[c]
       R \\[0.3em]
       G \\[0.3em]
       B \end{bmatrix}_{\mathrm{F}520}
\\ \vspace{1mm} \\
&\mathbf{M}_{X\!Y\!Z \rightarrow   R\!G\!B} = \mathbf{M}_{R\!G\!B \rightarrow X\!Y\!Z}^{-1}
\end{split}
\end{align}
Az alapszínek és a fehérpont színezete ezután
\begin{equation}
x_R = \frac{X_R}{X_R + Y_R + Z_R}, \hspace{1cm} y_R = \frac{Y_R}{X_R + Y_R + Z_R}
\end{equation}
alapján számolható.
Az így meghatározott színtér gamutja a \ref{Fig:sony} ábrán látható, az alapértelmezett számítógépes sRGB színtérrel együtt.

Jelen dokumentum sRGB színtérben kerül tárolásra (és megjelenítéskor az sRGB színtér az olvasó kijelzőjének saját színterébe transzformálva), így jelen dokumentumban az $XYZ$ koordinátáival adott alapszínek az sRGB térbe való konverzió után kerülhetnek megjelenítésre (ahogy \ref{Fig:sony} ábrán látható), amely pl. a vörös alapszínre
\begin{equation}
\begin{bmatrix}[c]
       R_R \\[0.3em]
       G_R \\[0.3em]
       B_R \end{bmatrix}_{\mathrm{sRGB}}
       =
     \mathbf{M}_{X\!Y\!Z \rightarrow R\!G\!B_{\mathrm{sRGB}}}
      \begin{bmatrix}[c]
       X_R \\[0.3em]
       Y_R \\[0.3em]
       Z_R \end{bmatrix} =      
       \begin{bmatrix}[c]
       1.13 \\[0.3em]
       0.25 \\[0.3em]
       -0.02 \end{bmatrix} 
\end{equation}
alakú.
A Sony megjelenítő alapszíneinek sRGB koordinátáira negatív és 1-nél nagyobb $RGB$ értékek is adódnak.
Ez a \ref{Fig:sony} ábrán is látható gamutok közti eltérést tükrözi.

\section{The $Y,\,R-Y,\,B-Y$ representation}

Az előző szakasz bemutatta egy színes képpont ábrázolásának módját adott RGB eszközfüggő színtérben.
Láthattuk, hogy egy színinger leírására a fő érzeti jellemzők a színpont világossága, színezete és telítettsége volt.
Felmerül tehát a kérdés, hogy létezik-e hatékonyabb reprezentációja az egyes színpontoknak, ami jobban leírja a fent említett szubjektív jellemzőket, így kevesebb redundáns információt tartalmaz az RGB reprezentációnál.

\subsection{The color difference signals}
A fő oka, hogy a korai TV- és videójeleket nem közvetlenül az RGB jeleknek választották (bár manapság már gyakori a közvetlen RGB ábrázolás) az NTSC bevezetésének idejében a visszafelé kompatibilitás biztosítása volt:
A színes műsorszórás kezdetén a korabeli háztartásokban szinte kizárólag fekete-fehér TV-vevők voltak találhatók.
Természetes volt az igény a már kiépített fekete-fehér műsorszóró rendszerrel való visszafelé kompatibilitásra színes kép-továbbítás esetén, amelyet a fekete-fehér kép és a színinformáció külön kezelésével volt elérhető.
Természetesen manapság már ez a tradicionális ok nem szempont videójelek megválasztása esetén.
Azonban látni a színinformáció külön kezelése lehetővé teszi a színek csökkentett felbontással való tárolását, amely jelentős adattömörítést (analóg esetben sávszélesség-csökkentést) tesz lehetővé.

A fekete-fehér kép egy színes kép világosságinformációjának fogható fel, amely a színpont relatív fénysűrűségével arányos, és így az $RGB$ koordináták lineáris kombinációjaként számítható.
Az együtthatók az adott eszközfüggő színtértől függnek, az NTSC alapszínei esetén pl. \eqref{Eq:NTSC_luminance} alapján adottak.
Ebből kifolyólag színes TV esetén is az változatlanul továbbítandó jelnek a \textbf{világosságjelet (luminance)} választották, amely tehát a relatív fénysűrűséggel megegyezik, és így pl. NTSC esetén az RGB jelekből
\begin{equation}Y_{\mathrm{NTSC}} = 
   0.30R + 0.59G + 0.11 B. 
   \label{Eq:NTSC_luminance}
\end{equation}
alapján számítható \footnote{Fontos ismét kihangsúlyozni, hogy a világosság-számítás módja színtérfüggő, az alapszínektől és a fehérponttól függ a már bemutatott módon.}.

Egy színes képpont leírásához 3 komponens szükséges, egy lehetséges és hatékony leírás pl. a képpont világossága, színezete és telítettsége.
A világosságjel mellé tehát két független információ kell, amelyek egyértelműen meghatározzák az adott színpont színezetét és telítettségét\footnote{A visszafelé-kompatibilitás biztosításához ezt a két színezetet leíró jelet kellett az NTSC rendszerben a változatlan fekete-fehér jelhez úgy hozzáadni, hogy a meglévő fekete-fehér vevők a világosságjelet demodulálni tudják, és a hozzáadott többletinformáció minimális látható hatással legyen a megjelenített képre.}.
Ugyanakkor fontos szempont volt ezen világosságinformáció-mentes, pusztán színinformációt leíró jelek könnyű számíthatósága az RGB komponensekből az egyszerű analóg áramköri megvalósíthatóság érdekében.

A színinformáció/világosságinformáció-szétválasztás legegyszerűbb (de jól működő) megoldásaként egyszerűen vonjuk ki a világosságot az RGB jelekből!
Mivel az $Y$ együtthatóinak összege definíció szerint (tetszőleges színtérben) egységnyi, így pl. NTSC esetén \eqref{Eq:NTSC_luminance} mindkét oldalából $Y$-t kivonva igaz a 
\begin{equation} 
   0.30 ( R - Y ) + 0.59 ( G - Y )  + 0.11 ( B - Y )  = 0 
   \label{eq:chrominances}
\end{equation}
egyenlőség.
Az $ ( R - Y ) $, $ ( G - Y ) $ és $ ( B - Y ) $ a TV-technika ún. \textbf{színkülönbségi jelei}, és a következő tulajdonságokkal bírnak:
\begin{itemize}
\item Nem függetlenek egymástól, kettőből számítható a harmadik.
\item Előjeles mennyiségek.
\item Ha két színkülönbségi jel zérus, akkor a harmadik is az.
Ekkor $R = G = B = Y$, így tehát a színtér fehérpontjában vagyunk.
A fehér színre kapott zérus színkülönbségi jelek azt mutatják, hogy a színinformációt valóban a színkülönbségi jelek jelzik, a fénysűrűség (világosság) pedig tőlük független mennyiség.
\item Az adott színkülönbségi jel értéke maximális ha a hozzá tartozó alapszín maximális intenzitású, és vice versa.
NTSC rendszerben vörös színkülönbségi jelre $R = 1$, $G = B= 0$ esetén
\begin{equation}
Y = 0.30 \cdot 1 + 0.59 \cdot 0 + 0.11 \cdot 0 \hspace{3mm }\rightarrow \hspace{3mm } R - Y  = 0.7,
\end{equation}
és hasonlóan $R=0$, $G = B = 1$ esetén
\begin{equation}
Y = 0.30 \cdot 0+ 0.59 \cdot 1 + 0.11 \cdot 1 \hspace{3mm }\rightarrow \hspace{3mm } R - Y  = -0.7.
\end{equation}
\item A fenti megfontolások alapján a színkülönbségi jelek dinamikatartománya:
\begin{align}
\begin{split}
-0.7 \leq R-Y \leq& 0.7 , \hspace{2cm} -0.89 \leq G-Y \leq 0.89, \\
 &-0.41 \leq B-Y \leq 0.41
\end{split}
\end{align}
\end{itemize}
A három színkülönbségi jelből kettő elegendő a színpont színinformációjának leírásához.
Mivel jel/zaj-viszony szempontjából ökölszabályszerűen mindig a nagyobb dinamikatartományú jelet célszerű továbbítani, így a választás a vörös és zöld színkülönbségi jelekre esett.

A videótechnikában tehát egy adott színpont ábrázolása a
\begin{align*}
Y&: \text{Luminance }\\
 	\left.\begin{array}{lr}
        R-Y\\
        B-Y
        \end{array}\right\}&: \text{Chrominance}
\end{align*}
ún. \textbf{luminance-chrominance térben} történik, amely felfogható egy új színmérőrendszernek/színtérnek is az $RGB$ színtérhez képest.

\subsection{The luminance-chrominance color space}
Vizsgáljuk most, hol helyezkednek el az adott RGB eszközfüggő színtérben ábrázolható színek ebben az új, $Y,\, R-Y,\, B-Y$ térben!
Az előzőekben láthattuk, hogy az $XYZ$ térben ez a színhalmaz egy paralelepipedont, az RGB térben egy egységnyi oldalú kockát jelent (lásd \ref{Fig:device_dep} ábra).
Vegyük észre, hogy a $Y,\, R-Y,\, B-Y$ koordinátákat akár az $XYZ$, akár az RGB komponensekből egy lineáris transzformációval előállíthatjuk:
Jelöljük adott RGB alapszínek esetén a relatív fénysűrűség RGB együtthatóit $k_r, k_g, k_b$-vel.
Ekkor általánosan a színkülönbségi jelek a 
\begin{align}
\begin{bmatrix}[c]
       Y \\[0.3em]
       B - Y \\[0.3em]
       R - Y\end{bmatrix} &= 
\begin{bmatrix}[c c c]
      k_r &  k_g&  k_b  \\[0.3em]
      -k_r &  -k_g&  1-k_b  \\[0.3em]
      1-k_r &  -k_g&  -k_b \end{bmatrix} 
\begin{bmatrix}[c]
       R \\[0.3em]
       G \\[0.3em]
       B \end{bmatrix}
\end{align}
transzformációval számíthatók.
Példaképp maradva az NTSC rendszer világosság-együtthatóinál (kiindulva abból, hogy $Y = 0.3R + 0.59G + 0.11B$) a transzformáció alakja
\begin{align}
\begin{bmatrix}[c]
       Y \\[0.3em]
       B - Y \\[0.3em]
       R - Y \end{bmatrix} &= 
\begin{bmatrix}[c c c]
      0.3 &  0.59&  0.11  \\[0.3em]
       -0.3 &  -0.59 & 0.89  \\[0.3em]
      0.7 &  -0.59&  -0.11  \end{bmatrix} 
\begin{bmatrix}[c]
       R \\[0.3em]
       G \\[0.3em]
       B \end{bmatrix}_{\mathrm{NTSC}}.
\end{align}
A lineáris transzformációt az RGB kockára végrehajtva megkaphatjuk az ábrázolható színek halmazát.
Az így kapott test az \ref{Fig:YCbCr_space} (a) ábrán látható.
Láthatjuk, hogy az RGB egységkocka egy paralelepipedonba transzformálódott, ahol a paralelepipedon főátlója az $Y$ világosság tengely.
Ennek mentén, az $R-Y = B-Y = 0$ tengelyen helyezkednek el a különböző szürke árnyalatok. 
\begin{figure}[htp]
	\centering
	\begin{overpic}[width = 0.45\columnwidth ]{figures_en/Video_colorspaces/LC_space_1.png}
	\small
	\put(0,0){(a)}
	\put(45,90){$Y$}
	\put(48,2){$R\!-\!Y$}
	\put(87,26){$B\!-\!Y$}
	\end{overpic}
	\hspace{6mm}
	\begin{overpic}[width = 0.48\columnwidth ]{figures_en/Video_colorspaces/LC_space_2.png}
	\small
	\put(0,0){(b)}
	\scriptsize
	\put(39,82){$R$}
	\put(25,24){$G$}
	\put(89,44){$B$}
	\put(12,58){$Y\!e$}
	\put(65,19){$C\!y$}
	\put(78,77){$M\!g$}
	\end{overpic}
	\caption{Az $Y, R-Y, B-Y$ színtér ábrázolható színeinek halmaza oldalnézetből (a) és felülnézetből (b).}
	\label{Fig:YCbCr_space}
\end{figure}

Az eredeti RGB kockához hasonlóan, paralelepipedon főátlón kívüli csúcsaiban (amelyben az $Y=0$ fekete és az $R=G=B=Y=1$ fehér található) az eszközfüggő színtér egy, vagy két $100~\%$-os intenzitású alapszínnel kikeverhető
\begin{equation}
R = \begin{bmatrix}[c] 1\\[0.3em] 0\\[0.3em] 0\end{bmatrix} \hspace{2mm}
G = \begin{bmatrix}[c] 0\\[0.3em] 1\\[0.3em] 0\end{bmatrix}\hspace{2mm}
B = \begin{bmatrix}[c] 0\\[0.3em] 0\\[0.3em] 1\end{bmatrix}\hspace{2mm}
Cy = \begin{bmatrix}[c] 0\\[0.3em] 1\\[0.3em] 1\end{bmatrix}\hspace{2mm}
Mg = \begin{bmatrix}[c] 1\\[0.3em] 1\\[0.3em] 0\end{bmatrix}\hspace{2mm}
Ye = \begin{bmatrix}[c] 1\\[0.3em] 1\\[0.3em] 0\end{bmatrix}
\end{equation}
vörös, zöld, kék alap- és cián, magenta, sárga ún. komplementer színek találhatók \footnote{Ezen komplementer színek tulajdonsága, hogy az egyes RGB alapszínekkel RGB kockában átlósan helyezkednek el, így a színtérben a lehető legmesszebb elhelyezkedő színpárokat alkotják.
Ennek megfelelően egymás mellé vetítve a komplementer színpárok (vörös-cián, sárga-kék, zöld-magenta) váltják ki a legnagyobb érzékelt kontrasztot.}.

A paralelepipedonra az $Y$-tengely irányából ránézve (\ref{Fig:YCbCr_space} (b) ábra) láthatjuk a világosságjeltől függetlenül, adott színtérben kikeverhető színek összességét.
Az $R-Y, B-Y, Y$ térben gyakori adott $Y$ világosság mellett a színek ezen $R-Y, B-Y$ síkon való ábrázolása.
Minthogy az $R-Y, B-Y$ jelek meghatározzák adott színpont színezetét és telítettségét, így az ábra azt jelzi, hogy a különböző színezetű és telítettségű színek egy szabályos hatszöget töltenek ki.
A hatszög csúcsai a színtér alap- és komplementerszínei.
Természetesen adott $Y$ érték mellett az ábrázolható színek nem tölti ki teljesen ezt a hatszöget:
adott világosságérték mellett az ábrázolható színek halmaza a $Y, R-Y, B-Y$ paralelepipedon egy adott $Y$ magasságban húzott síkkal vett metszeteként képzelhető el, azaz tetszőleges $0 \leq Y \leq1$ esetén rajzolható egy $R-Y, B-Y$ diagram.
Az így rajzolható diagramokra példákat a \ref{Fig:YCbCr_sect} ábra mutat.
\begin{figure}[htp]
	\centering
	\begin{overpic}[width = 1\columnwidth ]{figures_en/Video_colorspaces/YCbCr_2_11.png}
	\small
	\put(0,3){(a)}
	\put(0,37){$Y = 0.11$}
	\end{overpic}
	\vspace{2mm}
	\begin{overpic}[width = 1\columnwidth]{figures_en/Video_colorspaces/YCbCr_2_30.png}
	\small
	\put(0,37){$Y = 0.3$}
	\put(0,3){(b)}
	\end{overpic}
	\vspace{2mm}
	\begin{overpic}[width = 1\columnwidth]{figures_en/Video_colorspaces/YCbCr_2_59.png}
	\small
	\put(0,37){$Y = 0.59$}
	\put(0,3){(c)}
	\end{overpic}
	\caption{Különböző $Y$ értékek mellett rajzolható $B-Y, R-Y$ diagramok.}
	\label{Fig:YCbCr_sect}
\end{figure}
Nyilván rögzített $Y$ mellett nem biztos, hogy minden szín $100~\%$-os telítettséggel van jelen a $B-Y,R-Y$ diagramon. 
Például: teljesen telített kékre $Y=0.11$, azaz a $100~\%$ intenzitású kék alapszín ezen magasságban vett diagramon található.
Más magasságban vett  $B-Y, R-Y$ diagramon csak fehérrel higított kék található, azaz nem teljesen telített kék található.

A vizsgált diagramokból leszűrhető, hogy a világosságjel valóban független a színinformációtól, adott színpont színezetét és telítettségét pusztán az $R-Y$ és $B-Y$ diagramokon vett helye meghatározza.
Vizsgáljuk most, hogyan definiálhatóak ezen érzeti jellemzők, azaz a színezet és telítettség a TV technika $Y, R-Y, B-Y$ színterében!

\subsection{Hue and saturation in device-dependent color spaces}
A könnyebb elképzelhetőség kedvéért ábrázoljuk az $R-Y, B-Y$ koordinátákhoz tartozó színeket, az adott színponthoz tartozó olyan világosságérték mellett, amely esetén pontonként teljesül, hogy $X \!+\!Y\!+\!Z = 1$: 
ezzel gyakorlatilag az adott RGB színtér $xy$-színpatkón felvett színét képezzük le az $R-Y, B-Y$ diagramra.
\begin{figure}[htp]
	\centering
	\begin{minipage}[c]{0.6\textwidth}
	\begin{overpic}[width = 1\columnwidth ]{figures_en/Video_colorspaces/YCbCr_gamut.png}
	\small
	\put(56,46){$\alpha$}
	\end{overpic} \end{minipage}\hfill
	\begin{minipage}[c]{0.4\textwidth}
	\caption{Adott $Y, R-Y, B-Y$ térben ábrázolható színek gamutja.}
	\label{Fig:ycbcr_gamut}  \end{minipage}
\end{figure}
Az így kapott színhalmaz, amely felfogható az adott alapszínek mellett a luminance-chrominance tér gamutjának is, a \ref{Fig:ycbcr_gamut} ábrán látható.

\paragraph{Színezet:}
Megfigyelhető, hogy a diagramon az origóból kiinduló félegyenesen azok a színek vannak, amelyek egymásból kinyerhetők fehér szín hozzáadásával.
Tehát az origóból kiinduló félegyenesen az azonos színezetű, de eltérő telítettségű színek vannak. 
Azaz tetszőleges színpontot vizsgálva, a $B-Y,R-Y$ diagramon a színpontba mutató helyvektor iránya egyértelműen meghatározza az adott pont színezetét.
Ennek megfelelően a TV technikában a színezetet a $B-Y, R-Y$ diagramon a színpont helyvektorának irányszögeként definiáljuk:
\begin{equation}
\text{színezet}_{\mathrm{TV}} = \alpha  = \arctan \frac{R-Y}{B-Y}
\label{eq:hue}
\end{equation}
a \ref{Fig:ycbcr_gamut} ábrán látható jelölés alkalmazásával.

\paragraph{Telítettség:}
A telítettség kifejezése már kevésbé egyértelmű, több definíció bevezethető rá.
Általánosan, a telítettség azt fejezi ki, mennyi fehér hozzáadásával keverhető ki egy adott szín a színezetét meghatározó teljesen telített alapszínből.
Az $XYZ$-térben bevezettük a telítettségre a színtartalmat, illetve színsűrűséget.
Felmerül a kérdés, hogyan terjeszthető ki a telítettség fogalma eszközfüggő RGB színterekre.
Láthattuk, hogy az adott RGB színtérben előállítható legtelítettebb színek a gamut határán elhelyezkedő kvázi-spektrál színek, amelyek a legközelebb vannak az azonos színezetű valódi spektrálszínhez.
A bevezetendő telítettség-mennyiség célszerűen a kvázi-spektrál színekre tehát maximális, egységnyi értékű.

A telítettség ezek után a következő módokon definiálható.
\begin{itemize}
\item  Minthogy egy tetszőleges színnek a fehér színtől, azaz az origótól vett távolsága arányos a szín fehér-tartalmával, így legegyszerűbb módon a telítettség közelíthető a
\begin{equation}
\text{telítettség}_{\mathrm{TV},1} = \sqrt{ (R-Y)^2 +(B-Y)^2}
\label{eq:saturation_1}
\end{equation}
távolsággal.
Később tárgyalt okok miatt az analóg időkben TV technikusok körében ez a definíció volt érvényben.
Az így számolt telítettség valóban $0$ a fehér színre, azonban a kvázi-spektrálszínek telítettsége így nem egységnyi.
%
\item A matematikailag korrekt telítettség-definíció bevezetéséhez kiterjeszthetjük a korábban megismert színsűrűséget eszközfüggő színterekre\footnote{Ismétlésként: az $XYZ$ térben adott pont színsűrűsége $p_c = \frac{Y_d}{Y}$, ahol $Y_d$ az adott színhez tartozó domináns hullámhosszú szín fénysűrűsége, $Y$ a vizsgált szín saját fénysűrűsége.}.
Ennek egyszerűbb értelmezéséhez ábrázoljuk adott színpont paramétereit ún. területdiagramon!
%
\begin{figure}[htp]
	\centering
	\begin{minipage}[c]{0.6\textwidth}
	\begin{overpic}[width = 1\columnwidth ]{figures_en/Video_colorspaces/area_chart.png}
	\end{overpic} \end{minipage}\hfill
	\begin{minipage}[c]{0.4\textwidth}
	\caption{Tetszőlegesen választott $R,G,B$ koordináták esetén rajzolható területdiagram.}
	\label{Fig:area_diagram}  \end{minipage}
\end{figure}
%
A területdiagram a következő módon rajzolható fel egy tetszőleges RGB koordinátáival adott szín esetén: 
A vízszintes tengelyt osszuk fel az $Y$ fénysűrűség RGB együtthatóinak megfelelően, majd az egyes RGB komponenseket ábrázoljuk az intenzitásuknak megfelelő magasságú oszlopokkal.
Ekkor egy $Y$ magasságban húzott vonal alatt és fölött a színkülönbségi jeleknek megfelelő magasságú oszlopok alakulnak ki, amely oszlopok előjelesen vett területeinek összege \eqref{eq:chrominances} alapján zérus.
\begin{figure}[b!]
	\centering
	\begin{overpic}[width = 1\columnwidth ]{figures_en/Video_colorspaces/YCbCr_saturation.png}
	\small
	\put(0,0){(a)}
	\put(50,0){(b)}
	\end{overpic}
	\caption{Az $R-Y,B-Y$ térben ábrázolt színek telítettsége \eqref{eq:saturation_1} (a) és \eqref{eq:saturation_2} (b) alapján számolva}
	\label{Fig:saturations}  
\end{figure}

Válasszuk ki ezután a legkisebb $RGB$ komponenst (a \ref{Fig:area_diagram} ábrán látható példában az $R$) és húzzunk egy vízszintes vonalat ennek magasságában!
Ekkor a vizsgált színt két részre osztottuk: egy fehér színre (amelyre $R=G=B$) és egy kvázi-spektrálszínre, amelynek az egyik $RGB$ komponense zérus, és amelynek fénysűrűsége $Y_d = \min (R,G,B) - Y$.
A domináns hullámhosszú spektrálszín szerepét erre a kvázi-spektrálszínre cserélve kiterjeszthetjük a színsűrűséget az adott eszközfüggő színtérre, amely alapján a telítettség definíciója
\begin{equation}
\text{telítettség}_{\mathrm{TV},2} = \frac{| \min(R,G,B) - Y |}{Y}.
\label{eq:saturation_2}
\end{equation}
Könnyen belátható, hogy az $R = G=B=Y$ fehérpontokra a telítettség definíció szerint 0, míg kvázi-spektrálszínekre ($\min(R,G,B) = 0$) a telítettség azonosan 1.
\end{itemize}
A fent tárgyalt két telítettség-definíció alkalmazásával a \ref{Fig:ycbcr_gamut} ábrán látható színek telítettségét az \ref{Fig:saturations} ábra szemlélteti, megerősítve az eddig elmondottakat.
%
\section{The $Y', R'-Y', B'-Y'$ components}

Az előző szakasz bemutatta, hogyan választható legegyszerűbben szét a világosság és színezet/telítettség információ.
A tényleges videójelek ezen $Y, R-Y, B-Y$ jelekkel rokonmennyiségek, azonban történelmi okokból a feldolgozási lánc egy nem-lineáris transzformációt is tartalmaz, az ún. \textbf{gamma-korrekciót}.

\subsection{The role of Gamma-correction}
A gamma-korrekció bevezetése történeti okokra vezethető vissza.
A CRT megjelenítők elektron-ágyúja erős nem-lineáris karakterisztikával rendelkezik, azaz a képernyő pontjain létrehozott fénysűrűség az anódfeszültség nemlineáris függvénye\footnote{Ez a nemlinearitás az anód-katód feszültség-áram karakterisztikájából származik főleg.
A megjelenítésért felelős foszforok már jó közelítéssel lineárisan viselkednek, azaz a gerjesztéssel egyenesen arányos a létrehozott fénysűrűségük.}.
Ez a karakterisztika jól közelíthető egy 
\begin{equation}
L_{R,G,B} \sim U^{\gamma}
\end{equation} 
hatványfüggvénnyel, ahol a legtöbb korabeli kijelzőre az exponens $\gamma \approx 2.5$, $L_{R,G,B}$ az egyes RGB pixelek fénysűrűsége és $U$ a pixelek vezérlőfeszültsége.
Ez a nemlineáris átvitel természetesen jól látható hatással lenne a megjelenített képre:
Az alacsony RGB szintek kompresszálódnak, míg a világos árnyalatok expandálódnak, ennek hatására a telített színek túltelítődnek, illetve a sötét árnyalatok még sötétebbé válnak.
A nem-kívánatos torzulás az \ref{Fig:gamma} ábrán figyelhető meg.

\begin{figure}[]
	\centering
	\begin{overpic}[width = 1\columnwidth ]{figures_en/Video_colorspaces/Gamma.png}
	\small
	\put(0,0){(a)}
	\put(52,0){(b)}
	\end{overpic}
	\caption{RGB kép megjelenítése Gamma-korrekcióval (a) és Gamma-korrekció hiányában (b).
	Utóbbi esetben az $R,G,B$ komponensek egy 2.4 exponensű hatványfüggvénnyel előtorzítottak.}
	\label{Fig:gamma}  
\end{figure}
\vspace{3mm}
A torzítás korrekciója kézenfekvő: 
Az RGB komponensek megjelenítés előtti inverz hatványfüggvénnyel való előtorzítása esetén az előtorzítás és a CRT kijelző torzítása együttesen az $RGB$ jelek lineáris megjelenítését teszi lehetővé $\left(U^{\gamma}\right)^{\frac{1}{\gamma}} = U$ alapján.
Ez a nemlineáris előtorzítás az ún. \textbf{gamma-korrekció}.
\begin{figure}[b!]
	\centering
	\begin{minipage}[c]{0.65\textwidth}
	\begin{overpic}[width = 0.95\columnwidth ]{figures_en/Video_colorspaces/gamma2.png}
	\end{overpic} \end{minipage}\hfill
	\begin{minipage}[c]{0.33\textwidth}
	\caption{A Gamma-korrekció alapelve az RGB jelek előtorzításával.}
	\label{Fig:gamma2}  \end{minipage} 
\end{figure}

A korrekció természetesen a megjelenítés előtt bárhol elvégezhető a videófeldolgozási lánc során, azonban a lehető legegyszerűbb felépítésű TV vevők érdekében az előtorzítást az RGB forrás-oldalon célszerű elvégezni\footnote{Természetesen ez a korai TV vevők esetén volt fontos szempont, amikor a gamma-korrekciót drága/komplex analóg áramkörökkel kellett megvalósítani}.
Ennek megfelelően a gamma-korrekció már kamera oldalon megvalósul (akár analóg, akár digitális módon) az RGB jelek közvetlen gamma-korrigálásával.
A következőkben tehát
\begin{align*}
\begin{split}
R' = R^{\frac{1}{\gamma}}, \hspace{10mm} 
G' = G^{\frac{1}{\gamma}}, \hspace{10mm}
B' = B^{\frac{1}{\gamma}}
\end{split}
\end{align*}
a Gamma-előtorzított RGB összetevőket jelölik, ahol $\frac{1}{\gamma} \approx 0.4-0.6$ szabványtól függően (ld. később).

\hspace{3mm}
Fontos leszögezni, hogy ugyan a Gamma-korrekciót a CRT képernyők nemlinearitásának kompenzációjára vezették be, a gamma-korrekció rendszertechnikája manapság is változatlan annak ellenére, hogy a CRT kijelzők alkalmazását szinte teljesen felváltotta az LCD és LED technológia.
A gamma-korrekció fennmaradásának oka, hogy a videójel digitalizálása során perceptuális kvantálást valósít meg, ahogyan az a következő fejezetben láthatjuk.

\subsection{The luma and chroma components}
A Gamma-korrekció ismeretében bevezethetjük a mai videórendszerekben is alkalmazott tárolt és továbbított videójel-komponenseket:
\begin{figure}[]
	\centering
	\begin{overpic}[width = 0.53\columnwidth ]{figures_en/Video_colorspaces/video_signals.png}
	\end{overpic}
	\hspace{2mm}
	\begin{overpic}[width = 0.44\columnwidth ]{figures_en/Video_colorspaces/video_signals_2.png}
	\end{overpic}
	\caption{A Gamma-korrekció rendszertechnikája és a videójel-komponensek.}
	\label{Fig:gamma_system}  
\end{figure}
A videókomponensek előállításának rendszertechnikája a \ref{Fig:gamma_system} ábrán látható, az egyszerűség kedvéért most a kamerából ITU szabványba, ITU szabványból megjelenítő saját színterébe való színtérkonverziókat figyelmen kívül hagyva.
\begin{itemize}
\item A gamma-korrekció a kamera RGB-jelein hajtódik végre, SD, illetve HD esetében egy kb. 0.5 kitevőjű hatványfüggvény szerint.
A pontos gamma-korrekciós görbéket a következőekben fogjuk tárgyalni.
\item Az gamma-torzított $R',G',B'$ jelekből ezután az adott színtér előírt világosság-együtthatói alapján előállíthatók az $Y', R'-Y', B'-Y'$ jelek.
Továbbra is példaként az NTSC rendszer együtthatóinál maradva ezek alakja
\begin{align}
\begin{split}
Y' &= 0.3 \, R' + 0.59 \, G' + 0.11 \, B' \\
R'-Y' &= 0.7 \, R' - 0.59 \, G' - 0.11 \, B' \\
B'-Y' &= -0.3 \, R' - 0.59 \, G' - 0.89 \, B' \\
\end{split}
\end{align}
Ezek tehát az alapvető videójel-komponensek, amelyek végül ténylegesen tárolásra, tömörítésre, továbbításra (pl. műsorszórás) kerülnek.
\item Megjelenítő oldalon a fenti videójelekből a megfelelő inverz-mátrixolással az $R', G', B'$ jelek visszaszámíthatóak.
Megjelenítés során a megjelenítő gamma-torzításának hatására a kameraoldalon mért RGB komponensekkel lineárisan arányos fénysűrűségű RGB pixelek jelennek meg a kijelzőn.
\end{itemize}
Az így létrehozott $Y', R'-Y', B'-Y'$ jelek kitüntetett szereppel bírnak a videótechnikában.
Az eddigieket összegezve: ezek adják meg egy színes képpont ábrázolásának módját.
A komponensek neve:
\begin{itemize}
\item $Y'$: \textbf{luma jel}
\item $R'-Y'$, $B'-Y'$: \textbf{chroma jel}.
\end{itemize}

\paragraph{A luma és chroma jelek fizikai tartalma:\\}
Fontos észrevenni, hogy a luma jel nem egyszerűen a gamma-korrigált relatív fénysűrűség, hanem a gamma-korrigált RGB jelekből az eredeti $Y$ együtthatókkal számított videójel, azaz
\begin{equation}
Y' = 0.3R^{\frac{1}{\gamma}} + 0.11G^{\frac{1}{\gamma}} + 0.59B^{\frac{1}{\gamma}} \neq Y^{\frac{1}{\gamma}} = \left( 0.3R + 0.59G + 0.11B\right)^{\frac{1}{\gamma}}
\end{equation}
\begin{figure}[]
	\centering
	\begin{overpic}[width = 0.32\columnwidth ]{figures_en/Video_colorspaces/luma_chroma_0_11.png}
\small
\put(0,0){(a)}
	\end{overpic}
	\begin{overpic}[width = 0.32\columnwidth ]{figures_en/Video_colorspaces/luma_chroma_0_30.png}
\small
\put(0,0){(b)}
	\end{overpic}
	\begin{overpic}[width = 0.32\columnwidth ]{figures_en/Video_colorspaces/luma_chroma_0_59.png}
\small
\put(0,0){(c)}
	\end{overpic}
	\caption{A chroma térben ábrázolható színek halmaza fix $Y'$ értékek mellett vizsgálva.}
	\label{Fig:luma_chroma_space}  
\end{figure}
A luma jel fizikai tartalma emiatt nehezen kezelhető: 
Legszorosabban az adott színpont világosságával függ össze, fehér szín speciális esetén pl. ahol $R=G=B=Y_W$
\begin{equation}
Y'_W = \left( 0.3 + 0.59 +0.11 \right)Y_W^{\frac{1}{\gamma}} = Y_W^{\frac{1}{\gamma}}
\end{equation}
az egyenlőtlenség egyenlőségbe megy át, azaz a luma megegyezik a gamma-korrigált világosságjellel.
Általánosan azonban a luma jel színinformációt is hordoz magában.
Hasonlóan, a chroma jelek nem szimplán a gamma-korrigált színkülönbségi jelek (de hasonlóan, fehér esetében azonosan nullák), és így világosságinformációt is hordoznak magukban.
	
Adott luma értékek mellett az ábrázolható színek halmaza a \ref{Fig:luma_chroma_space} ábrán látható.
Megfigyelhető, hogy a luminance-chrominance térrel azonosan az ábrázolható színek egy hatszöget feszítenek ki, és a 100\%-osan telített színek helye nem változik (hiszen a 0 és 1 értékeken nem változtat a gamma-korrekció), ennek megfelelően az egyes pontok színezete a chroma térben változatlan.
Az ábrákon azonban egyértelműen látható, hogy adott $Y'$ értékek mellett is az ábrázolt színek világossága változik, tehát a chroma jelek világosságinformációt is tartalmaznak.
Látható, hogy a gamma-torzítás hatására---ahogy \ref{Fig:gamma} ábrán is megfigyelhető---adott $Y'$ mellett a telítetlen (fehérhez közeli) színek sötétebbé válnak, míg a telítettebb színek még telítettebbé válnak. 

%Ez az eddig elmondottak alapján nem kell, hogy problémát okozzon, hiszen pusztán annyit jelent, hogy a világosság és színinformációt nem teljesen szeparáltan kezeljük átvitel tárolás és átvitel során.
%Ugyanakkor látni fogjuk, hogy az emberi látás tulajdonságait kihasználva a színjeleket---azaz a chroma jeleket---csökkentett sávszélességgel, vagy digitális esetben kisebb felbontással továbbítjuk.
%Minthogy a fentiek alapján így kis részben a világosságjel sávszélessége/felbontása is csökken, amelynek már látható hatása lehet a megjelenített képen.-

\section{The \ypbpr color space}

A luma és chroma központi szerepet játszanak videótechnikában, a leggyakrabban ezek a jelek a színes képpont ábrázolásának alapja mind komponens, mind kompozit (több komponens kombinációjaként létrehozott videó) formátumok esetén.
Utóbbi formátum létrehozásával a következő fejezet foglalkozik részletesen.
Analóg, komponens videótechnikában egy színes képpont luma-chroma térben való leírását az \textbf{\ypbpr színtérben} való ábrázolásnak nevezzük (az ezekből képzett \ypbpr videójeleket a következő fejezet részletezi).

Az \ypbpr színtér $Y'$ jele maga a luma komponens, míg a $P'_{\mathrm{B}}, P'_{\mathrm{R}}$ jelek szimplán az átskálázott chroma komponensek, a skálafaktort úgy megválasztva, hogy dinamikatartományuk $\pm 0.5$ legyen.

Jelölje az adott RGB színtérben a relatív fénysűrűség együtthatóit $k_r, k_g$ és $k_b$.
Minthogy az $R'-Y$ és $B'-Y'$ komponensek dinamikatartományra rendre $1 - k_r$ és $1 - k_b$, ezért általános az \ypbpr jelek az luma-chroma jelekből a
\begin{align}
\begin{split}
Y' &= k_r \, R' + k_g \, G' + k_b \, B' ,\\
P_R &= k_1 \, \left( R' - Y' \right) = \frac{1}{2} \frac{1}{1 - k_r} \, \left( R' - Y' \right)\\
P_B &=  k_2 \, \left( B' - Y' \right) = \frac{1}{2} \frac{1}{1 - k_b} \, \left( B' - Y' \right)
\end{split}
\end{align}
összefüggés alapján számítható.
Az egyenletekben $R', G', B'  \in \lbrace 0, 1 \rbrace$ a Gamma-korrigált színkoordinátái az ábrázolt színpontnak adott eszközfüggő

Hasonlóan meghatározhatjuk általános $R',G',B'$ komponensekre az \ypbpr jelek kiszámításához szükséges transzformációs mátrixot
\begin{align}
\begin{bmatrix}[c]
       Y' \\[0.3em]
       P_{\mathrm{B}} \\[0.3em]
       P_{\mathrm{R}} \end{bmatrix}
       =& 
  \begin{bmatrix}[c c c]
   k_r & k_g & k_b  \\
   -\frac{1}{2}\frac{k_r}{1-k_b} & -\frac{1}{2}\frac{k_g}{1-k_b} & \frac{1}{2} \\
   \frac{1}{2}& -\frac{1}{2}\frac{k_g}{1-k_r} & -\frac{1}{2}\frac{k_b}{1-k_r} \\
\end{bmatrix}
\cdot
\begin{bmatrix}[c]
       R' \\[0.3em]
       G' \\[0.3em]
       B' \end{bmatrix},
\end{align}
míg az inverz-transzformációt 
\begin{align}
\begin{bmatrix}[c]
       R' \\[0.3em]
       G' \\[0.3em]
       B' \end{bmatrix}
       =& 
  \begin{bmatrix}[c c c]
   1 & 0 & 2 - 2 \cdot k_r  \\
   1 & -\frac{k_b}{k_g} \cdot (2-2k_b) &  -\frac{k_r}{k_g} \cdot (2-2k_r)  \\
   1 & 2 - 2 \cdot k_b & 0 \\
\end{bmatrix}
\cdot       \begin{bmatrix}[c]
       Y' \\[0.3em]
       P_{\mathrm{B}} \\[0.3em]
       P_{\mathrm{R}} \end{bmatrix}
\end{align}
írja le.

\begin{figure}[]
	\centering
	\begin{minipage}[c]{0.6\textwidth}
	\begin{overpic}[width = 1\columnwidth ]{figures_en/Video_colorspaces/YPbPr.png}
	\end{overpic} \end{minipage}\hfill
	\begin{minipage}[c]{0.4\textwidth}
	\caption{Az \ypbpr térben ábrázolható színek gamutja.}
	\label{Fig:ypbpr_gamut}  \end{minipage}
\end{figure}
Egyszerű példaként a HD szabvány színterében 
\begin{equation}
k_r = 0.2126, \hspace{7mm}
k_g = 0.7152, \hspace{7mm}
k_b = 0.0722
\end{equation}
Az adott alapszínek mellett az ábrázolható színek tartománya a \ref{Fig:ypbpr_gamut} ábrán látható.

\section{Digital representation of color information}

So far, the current chapter has introduced the color representation of video technologies by assuming continuous RGB, luma and chroma values.
The digital representation of color pixels can be obtained by the direct digitization of the $R'G'B'$, or more often the \ypbpr components.
The digital representation of the \ypbpr signals have its own terminology: it is termed as the \ycbcr color space \footnote{
The \ycbcr signals are sometimes incorrectly referred to as $Y'U'VI$ signals (e.g. in VLC player), which term was originally used for the components of the PAL composite video format.}.

The \ycbcr digital color space can be obtained by the quantization of the \ypbpr components, with representing the originally continuous values at discrete levels.
It is therefore obvious that \ycbcr is a device dependent representation, depending on the RGB primaries and its gamut coincides with the gamut of the color gamut of the \ypbpr color space, depicted in Figure \ref{Fig:ypbpr_gamut} (with of course only discrete number of the reproducible colors due to digitization).
In the following the current chapter deals with the questions arising at the quantization of the \ypbpr color space.

\subsection{Perceptual quantization and bit depth}
First the optimal quantizer transfer characteristics is investigated, in order to achieve bit-efficient digital representation.
As a result, the real role of gamma correction is highlighted.

For the sake of simplicity first it is assumed that the signal-to-quantize is the $Y$ component, i.e. the linear relative luminance signal (without gamma correction).
The starting point for defining an appropriate quantizer transfer characteristics is given by the perceptual properties of the human visual system:
As a rule of thumb it can be stated that in case of image reproduction, the HVS can not discern luminance levels below $1~\%$ of the maximal luminance on the given scene.
Loosely speaking relative luminances below $\frac{1}{100}$ just appear black for the human observer, therefore, the dynamic range of luminance levels to be reproduced is 100:1.

Within this dynamic range the lightness perception of the HVS is approximately logarithmic function of luminance with the contrast sensitivity being $1~\%$.
This means that two luminance levels can be distinguished only if their relative difference is larger than 1.01.
Later the relative luminance-percepted lightness characteristics ($L(Y)$) was given more accurately by the CIE $L^*$ function, describing the lightness as the power function of luminance with the exponent being approximately 0.4.

These properties of human vision establishes the following requirements for quantizing the luminance signal without visible quantization noise:
\begin{itemize}
\item The ratio of the largest and the smallest quantized luminance levels should be at least 100:1
\item The ratio of the adjacent quantized luminance levels should be at most 1.01, i.e. their relative difference should be less than or equal to $1~\%$
\end{itemize}

In the following, as a counterexample for the appropriate quantization strategy the problem with linear quantization is investigated.
In this case the digital signal levels are assigned to the relative luminance levels within the dynamic range of $Y \in \lbrace Y_0, 100 Y_0 \rbrace$ linearly. 
The quantization can be, therefore, performed by simple rounding to the nearest integer.
In case of representing the digital samples at $N$ bits the mapping is given by 
\begin{equation}
q =  \nint{ \left( 2^N - 1 \right) \cdot  \frac{Y - Y_0 }{Y_1 - Y_0 } },
\end{equation}
where $\nint{}$ is the rounding operation, and $Y_1 = 100Y_0$ is the maximal quantized luminance value.
Similarly, the inverse mapping, i.e. the luminance levels of the $q$-th digital code is given by
\begin{equation}
Y^q = q \cdot \frac{Y_1 - Y_0}{2^N - 1} + Y_0.
\end{equation}

\begin{figure}[]
	\centering
	\begin{overpic}[width = 1\columnwidth ]{figures_en/linear_vs_perc_quant.png}
	\end{overpic}
	\caption{
Relative difference of adjacent digital codes in case of linear (a) and perceptual (b) quantization.}
	\label{Fig:linear_vs_perc_quant}
\end{figure}

The relative difference of the adjacent digital codes is then given as
\begin{equation}
\frac{Y^{q+1}-Y^q}{Y^q} = \frac{1 }{q + \frac{2^N - 1}{99}}.
\label{Eq:rel_dif}
\end{equation}
As an example of quantization with $N = 8$ bits, the relative difference between the 101-st and 100-th code (with $q = 100$) the relative difference is
\begin{equation*}
\frac{Y^{101}-Y^{100}}{Y^{100}} \approx 0.01 = 1~\%,
\end{equation*}
meaning that the luminance levels of the adjacent codes are just noticeable.
For smaller, or larger codes (e.g. 20 and 21, or 200 and 201) the relative difference is given by
\begin{equation*}
\frac{Y^{21}-Y^{20}}{Y^{20}} \approx 0.05 = 5~\%, \hspace{1cm} \frac{Y^{201}-Y^{200}}{Y^{200}} \approx 0.005 = 0.5~\%.
\label{eq:code_100}
\end{equation*}
Obviously, the luminance levels of codes under 100 are easily distinguishable, meaning that quantization noise at these code levels is clearly visible.
As a consequence, in case of the linear quantization of the luminance signal for dark shades the boundary of the different quantization levels would be easily noticeable, leading to so-called banding artifact.

Straightforwardly, by increasing the bit depth ($N$) banding could be avoided:
It is clear that the largest relative difference is between codes 0 and 1.
%
\begin{figure}[]
	\centering
	\begin{overpic}[width = 0.8\columnwidth ]{figures_en/linear_vs_perc_quant_2.png}
	\end{overpic}
	\caption{Quantized luminance ramp by applying linear (a) and perceptual (b) quantization with $N=7$ bits.
	In case of linear quantization the just noticeable quantization is found at $q \approx 99$ (based on $1 / q + \frac{2^7 - 1}{99} = 0.01$).}
	\label{Fig:linear_vs_perc_quant_2}
\end{figure}
%
By setting $q = 0$ the smallest bit depth for which \eqref{Eq:rel_dif} is larger than $1~\%$, according to
\begin{equation}
\frac{99}{2^N - 1} \leq 0.01 \hspace{1cm} N \geq 13.27 
\end{equation}
is given by $N = 14~\mathrm{bits}$.
This means that linear quantization ensures unnoticeable qunatization noise by applying the bit depth of 14\footnote{
As a consequence digital cameras often digitize the pixel levels by using 14 bits linear quantization, which is requantized to the final bit depth after digital gamma correction.}.
On the other hand \eqref{eq:code_100} reflects that codes above 100 have decreasing perceptual utility: luminance at these regimes is quantized with ineffectively fine resolution.

\vspace{3mm}
A straightforward strategy in order to avoid the problem with linear quantization would be to quantize the percepted lightness ($L$) instead of the luminance information, resulting in uniform perceptual difference between adjacent codes.
By employing the lightness definition of the CIE ($L^*$) this \textbf{perceptual quantization} can be achieved by the distortion of the relative luminance with the power function of 0.4 before quantization.

The quantization mapping and the inverse mapping in this perceptual case are given by
\begin{equation}
q =  \nint{ \left( 2^N - 1 \right) \cdot  \frac{Y^{0.4} - Y_0^{0.4} }{Y_1^{0.4} - Y_0^{0.4}} }
\hspace{1cm}
Y^q = \left( q \cdot \frac{Y_1^{0.4} - Y_0^{0.4}}{\left( 2^N - 1 \right) } + Y_0^{0.4} \right)^{\frac{1}{0.4}} .
\end{equation}
Based on these formulae the relative difference of adjacent codes can be expressed for perceptual quantization.
The result is depicted in Figure \ref{Fig:linear_vs_perc_quant} (b).
It is verified that the relative difference is approximately constant over the entire dynamic range, and even in case of $N = 10$ quantization, it is only slightly higher than $1~\%$.

As a consequence, in the field of video technologies in studio standards the luminance is quantized perceptually, representing the luminance in 10 bits, while in most consumer electronics (e.g. JPEG image compression, MPEG video compression and video broadcasting) representation with $N=8$ is satisfactory\footnote{
As a third option logarithmic quantization could be performed by setting the ratio of the luminance levels of the adjacent codes to 1.01.
In this case according to $1.01^q \geq 100$ the number of required codes is $q = 463$, which can be represented in $N = 9$ bits.
Due to historical reasons (due to gamma correction) the presented, power function-based perceptual quantization was introduced in the video standards.}.
Therefore, SD and HD studio standards (Recommendations ITU-601 and ITU-709) include the digital representation applying $N = 8$ or $N = 10$ bits, with a rigorous definition of the implementation of quantization and the non-linear pre-distortion curve.
This curve is investigated in thew following section in details.

\subsection{Gamma correction: goal and implementation}
The previous section introduced the basic principle of perceptual quantization: pre-distortioning the luminance values by a power function with the exponent being approximately 0.4 the quantization noise can be uniformly distributed over the entire dynamic range.
As a result the visible banding of dark shades can be avoided, as it is illustrated in Figure \ref{Fig:linear_vs_perc_quant_2}.
In the following the actual implementation of perceptual quantization within the image/video processing chain is discussed.

The optimal signal processing scheme would be the following, as shown on Figure \ref{Fig:gamma_flow} (a):
At the source of the RGB signals perceptual quantization is achieved by the direct quantization of the perceived lightness ($L^* \sim Y^{0.4}$), obtained from the luminance $Y$, that can be calculated as the linear combination of the RGB coordinates.
At the receiver (e.g. display, TV receiver) following D/A conversion the original luminance value is regained by the inverse distorting the perceived lightness.
Finally, from the luminance inforamtion the RGB values to be displayed are obtained by the corresponding inverse transformation.

Note that so far linear RGB values and luminance levels were assumed, without taking the gamma correction into consideration:
As a consequence, although perceptual quantization could be achieved, the non-linear transfer characteristics of the CRT display would still result in distorted dynamics of the displayed image.
The required neutralization of the CRT distortion would introduce a further non-linear transfer function---as depicted in Figure \ref{Fig:gamma_flow} (b)---overcomplicating the signal processing chain (as well as making it more expensive).

However, as a lucky coincidence the luminance-lightness characteristics of the human vision exactly coincides with the required CRT gamma compensation function, both described by $\sim x^{0.4}$, allowing the simplification of the signal processing.
As an engineering approximation, both at the source and the receiver side the order of the non-linear mapping and the linear transformation $P: RGB \rightarrow Y, R-Y, B-Y$ is interchanged.
This interchange of operations will have two consequences:

%
\begin{figure}[]
	\centering
	\begin{overpic}[width = 1\columnwidth ]{figures_en/gamma_flow_1.png}
	\small	
	\put(0,0){(a)}
	 \vspace{5mm}
 	\end{overpic}
	\begin{overpic}[width = 1\columnwidth ]{figures_en/gamma_flow_2.png}
	\small	
	\put(0,0){(b)}
	\end{overpic} \vspace{5mm}
	\begin{overpic}[width = 0.98\columnwidth ]{figures_en/gamma_flow_3.png}
	\small	
	\put(0,0){(c)}
	\end{overpic}
	\caption{Signal processing scheme of gamma correction:
	Figure (a) realizes true perceptual quantization, however, with the CRT gamma distortion still leading to visible artifacts in the reproduced image.
	Figure (b) shows a possible, but over-complicated solution for this problem.
	As an engineering approximation the order of linear and non-linear operations are interchanged on Figure (c), giving a mathematically imprecise solution, but allowing the considerable simplification of the block diagram.}
	\label{Fig:gamma_flow}
\end{figure}
%
\begin{itemize}
\item By leaving the principle of strict perceptual quantization, on the source side the quantized signal is not the perceived lightness $L^* = Y^{0.4}$, but the \textbf{luma signal} $Y'$, obtained from the gamma corrected color-coordinates, $R^{0.4},G^{0.4},B^{0.4}$.
As it was already declared, for white shades ($R=G=B$) the luma signal is the gamma corrected luminance signal, i.e. $Y_W^{0.4} = Y_W' = L_W^*$ holds, while for other colors the luma signal carries color information as well.
Therefore, the presented signal processing chain realizes perceptual quantization of white shades, and only approximates it for any other color.
%
\item On the receiver side the CRT correction function exactly neautralizes the inverse quantization function $L^* \rightarrow Y$, resulting in linear resultant transfer characteristics.
\end{itemize}
The signal processing chain obtained consists only a single non-linear transfer block, with the entire system resulting in the gaama correction technology, discussed already in the previous sections.

Hence, the actual present role of gamma correction is highlighted:
Although the role of CRT imaging systems has been almost entirely superseded by LCD and LED based displays, gamma correction is applied in the video processing chain in a completely unchanged manner.
However, its real role nowadays is \textbf{not} the CRT transfer compensation, but it achieves and approximation of perceptual quantization, adapting the quantization characteristics to the properties of human vision.


\vspace{3mm}
Obviously, nowadays the implementation of non-linear transfer functions is computationally inexpensive.
Similarly to CRTs, the currently used displays also exhibit highly non-linear driving voltage-display luminance characteristics, which has to be compensated prior to reproducing the input image.
Therefore, in current display first the gamma distortion (due to perceptual quantization) has to be neutralized, afterwards the actual display characteristics has to be compensated.
These non-linear operations are usually implemented before D/A conversion, based on lookup tables (LUTs).

\vspace{3mm}
The actual form of gamma correction that has to be implemented before A/D conversion is rigorously codified in SD and HD recommendations.
Within these standards the non-linear distortion is referred to as the \textbf{opto-electronic transfer function (OETF)}.
The actual choice of the OETF is depends on two aspects:
\begin{itemize}
\item to achieve approximately perceptual quantization
\item to compensate the effects of the viewing environment
\end{itemize}
On the other hand the standards also codify the non-linear transfer function at the receiver side, which neutralizes the effect of OETF.
This receiver side non-linear compensation is termed as the \textbf{electro-optical transfer function (EOTF)}.

\paragraph{Compensation of the viewing environment:\\}
So far it was inherently assumed that the exponent of the gamma correction curve is 0.4.
However, in actual reproduction systems the exponent of the standardized OETF is usually a higher value, with the actual choice depending on the supposed average illumination level in the reproduction environment.

It was illustrated in Figure \ref{Fig:gamma} that the non-linear distortion of the RGB components with the exponent being larger than 1 will result in the compression of deep shades (increase in contrast) and in the saturation of colors.
This fact allows the compensation of loss of contrast and saturation due to dim viewing environments:
According to the Stevens (Bartleson-Breneman) and Hunt effects in a dark viewing environment the ability of discerning dark shades, the perceived contrast of the image and the perceived saturation of colors decreases.
%
\begin{figure}[]
	\centering
	\begin{overpic}[width = 1\columnwidth ]{figures_en/stevens.png}
	\end{overpic}
	\caption{Illustration of the Bartleson-Breneman effect.}
	\label{Fig:stevens_effect}
\end{figure}

The Bartleson-Breneman effect is illustrated in Figure \ref{Fig:stevens_effect}.
It is verified that within a displayed image, the image contrast  increases with the luminance of surround lighting:
\begin{itemize}
\item with bright surround luminance the image contrast (the perceived ratio of the brightest and darkest shades) is larger, than in case of a dark background.
\item also the entire luminance-lightness characteristic curve changes depending on the surround luminance: in case of bright surround the perceived contrast between dark shades increases, while bright shades can be distinguished less efficiently, and vice versa.
\end{itemize}
This means that instead of the well-known $L^* \sim Y^{0.4}$ characteristics, the exponent slightly increases for bright surround luminance levels, and decreases in dark backgrounds.

As a consequence if the environment in image reproduction is dark (e.g. a cinema), in order to avoid the loss of contrast and saturation the overall non-linear transfer function of the signal processing chain should have an exponent of about 1.2-1.5 instead of the linear transfer.
This can be achieved with choosing the source gamma correction, i.e. the OETF to be higher than the original value of 0.4, which was originally chosen to compensate the effect of CRT distortion.

Based on these considerations as an example, the ITU-709 HDTV recommendation defines the following OETF
\begin{equation}
E = 
\begin{cases}
4.500 L, \hspace{20mm} \mathrm{ha}\, L < 0.018 \\
1.099 L^{0.45} - 0.099, \hspace{3mm} \mathrm{ha}\, L \geq 0.018,
\end{cases}
\end{equation}
where $L \in \{ R, G, B \}$.
The entire curve consists of a power function and a linear segment.
This linear segment is required in order to avoid the infinite slope of the power function around the origin, that would result in infinite gain of noise around the black level.
The entire curve can be well-approximated with a power function of 0.5 (i.e. a square root function), as depicted in \ref{Fig:itu709}.

\begin{figure}[]
	\centering
	\begin{overpic}[width = 0.7\columnwidth ]{figures_en/itu709.png}
	\end{overpic}
	\caption{The opto-electronic transfer function of the ITU-709 standard and the NTSC standard.}
	\label{Fig:itu709}
\end{figure}
The HD standard assumes the receiver gamma distortion to be $\gamma_D \approx 2.5$ (i.e. the EOTF is assumed to be a power function with the exponent being 0.4).
Along with the prescribed gamma correction this results in a resultant transfer function of $0.5 \cdot 2.5 = 1.25$, which ensures the desired contrast and saturation in an average living room at daylight.

As an other example: the current most widespread digital cinema standard, called the DCI-P3, with the correctly chosen OETF and EOTF achieves a resultant non-linear transfer with the exponent being 1.5.
\footnote{
Actually, in case of e.g. the HD standard, each element of the production and reproduction chain is rigorously defined.
The content is produced in such a manner that it would be reproduced with the desired aesthetic properties in a standardized viewing environment (defined by ITU-R BT.2035) displayed on a standardized display apparatus (standardized by ITU-R BT.1886).}.
In practice, in case of current computer and TV displays the actual value of the EOTF, i.e. the display gamma can be freely adjusted in order to achieve the desired contrast.

\subsection{Dynamic range of \ycbcr representation}

In the foregoing it was presented how the gamma correction allows the representation of SD and HD image content as low as 8 (for consumer) or 10 (for studio and professional processing) bits.
It seems to be straightforward to utilize the entire possible dynamic range in order to represent the video data, e.g. to let the $Y'$ luma data to cover the $\lbrace 0, 255 \rbrace$ code range in case of 8 bit representation.
This so called \textbf{full range} approach is however only applied in the JPEG encoder, and several image editor softwares, operating directly in the RGB color space.

%https://books.google.hu/books?id=hOu5DQAAQBAJ&pg=PA427&lpg=PA427&dq=RGB+headroom+footroom&source=bl&ots=NsT6C3TiLr&sig=ACfU3U1HLa7oM0fxBZLHjs6PFfuyi2kflg&hl=en&sa=X&ved=2ahUKEwjSlf-Uw-PoAhWkw4sKHebNCl0Q6AEwC3oECA0QLw#v=onepage&q=RGB%20headroom%20footroom&f=false 

\begin{figure}[]
	\centering
	\begin{overpic}[width = 0.7\columnwidth ]{figures_en/ycbcr_dyn_range.png}
	\end{overpic}
	\caption{The dynamic range of the \ycbcr codes.}
	\label{Fig:ycbcr_dyn_range}
\end{figure}
In the field of video technologies in case of \ycbcr representation the image/video data is usually stored and transmitted with a \textbf{narrow range} approach.
In this case the valid video data is allowed to fill only the part of the available digital dynamic range:
In high-quality video, it is necessary to preserve transient signal undershoots below black, and overshoots above white, that are liable to result from processing by digital and analog filters without clipping the video data. 
Studio video standards provide \textbf{footroom} below reference black, and \textbf{headroom} above reference white. 
This code ranges are only containing video data during video processing, and their content is discarded during video storing and transmission.

For the case of 8 bit representation the luma ($Y'$) and (in case of RGB representation) and the $R',G',B'$ components have a headroom of 15 and a footroom of 19 codes, thus the dynamic range of the components is $16 \leq Y' \leq 235$ (codes 0 and 255 are reserved for synchronization in digital interfaces).
The asymmetry of the headroom and footroom has no important reason.

For the $C'_\mathrm{B}$ and $C'_\mathrm{R}$ chroma components the zero level is the middle of the dynamic range, (i.e. code 128 digitally), while the headroom and footroom are symmetrically 15 codes, i.e. $16 \leq C'_\mathrm{B}, C'_\mathrm{R} \leq 240$ holds.

For higher bit depths the width of headroom and footroom increases proportionally.
As a summary the \ycbcr digital levels can be obtained from the \ypbpr analog signal levels according to
\begin{equation}
\begin{bmatrix}[c]
       Y' \\[0.3em]
       C'_{\mathrm{B}} \\[0.3em]
       C'_{\mathrm{R}} \end{bmatrix}
       =
D\cdot
\begin{bmatrix}[c]
       16 \\[0.3em]
       128 \\[0.3em]
       128 \end{bmatrix}
+
D\cdot
\begin{bmatrix}[c]
       219 Y' \\[0.3em]
       224 P'_\mathrm{B} \\[0.3em]
       224 P'_{\mathrm{R}} \end{bmatrix},
\end{equation}
where $D = 2^{N-8}$ holds, with $N$ denoting the bit depth.

\subsection{Chroma subsampling}
%
The luma-chroma representation of image and video content has two great advantages over the direct RGB representation:
On one the transmission of the separated luminance information and the color information allowed the color TV transmission system to be fully backward compatible with the then-existing black-and-white TV receivers, as it is discussed in the next chapter.
On the other hand it is more adapted to the properties of human vision\footnote{
The conversion of light to stimulus is ensured by the frequency-selective photoreceptors of the retina, with the three types of cone cells being sensitive to mainly red, green and blue lights.
However, the transmission of the stimulus towards the brain on three types of optic nerves, one carrying black and white and two carrying color information.}: 
The perceptual spatial resolution of the HVS (i.e. the visual acuity) is much lower for spatial change of color information than for that of the luminance.
Therefore, the separate transmission of color information allows the bandwidth reduction of the chroma signals, i.e. their transmission with lower spatial resolution.
In case of digital representation the reduction of chroma resolution is performed by \textbf{chroma subsampling}.

\begin{figure}[]
	\centering
	\begin{overpic}[width = 1\columnwidth ]{figures_en/umbrella.png}
 	\end{overpic}
	\caption{The content of the \ycbcr components for a simple test image.}
	\label{Fig:umbrella}
\end{figure}

\subsection*{Notation of subsampling schemes}
The content of the \ycbcr components in case of a simple test image is illustrated in Figure \ref{Fig:umbrella}.
Clearly, the color information carries minor high frequency details---and even where details are present they are hardly noticeable for the human vision---thus the transmission of color information with lower resolution is satisfactory.
The resolution of chroma components can be decreased both in the horizontal and vertical dimensions, usually to the half, or quarter of the luma resolution, therefore, different types of chroma subsampling schemes can be implemented.
The rate of decimation along the horizontal/vertical dimension are usually denoted by using the following notation:
\begin{equation}
J : a : b : \alpha,
\end{equation}
where the digits denote the following properties:
\begin{itemize}
\item $J$: the first digit stands for the horizontal sampling reference of the luma component.
Originally (in the NTSC and SD systems) the first digit indicated the sampling frequency of the luma signal a the multiple of the color subcarrier frequency, via $f^{Y'}_s = J \cdot 3\,\frac{3}{8}~\mathrm{MHz}$.
In case of HD formats based on this interpretation often $J=22$ and higher values should be used, so for the sake of simplicity the first digit is usually fixed to the value 4, serving as a reference number for the following digits.
\item $a$: is the $C_{\mathrm{B}}$ and $C_{\mathrm{R}}$ horizontal decimation factor, given relatively to the first digit.
more informally it gives  the number of chroma samples in the first row of $J$ pixels.
As an example: $J=a = 4:2$ means that the horizontal resolution of the color information is half of the luma resolution.
\item $b$: is the number of changes of chroma samples between first and second row of $J$ pixels.
If $b=a$, then no vertical subsampling of the chroma samples is performed.
If $b=0$, then the vertical resolution of chroma is half of that of the luma samples.
\item $\alpha$: indicates the presence of alpha channel (e.g. chroma keying).
May be omitted, if alpha component is not present, and is equal to $J$ when present.
\end{itemize}
\begin{figure}[]
	\centering
	\begin{overpic}[width = 1\columnwidth ]{figures_en/chroma_subsampling.png}
 	\end{overpic}
	\caption{Illustration of frequently used chroma subsampling schemes.}
	\label{Fig:chroma_subsampling}
\end{figure}

The chroma subsampling process can be interpreted as a simple lossy compression technique, allowing to reduce the amount of video or image data during storing or transmission.
Before reproducing the original RGB signals on the display side, obviously, the discarded chroma samples has to be reconstructed by some interpolation technique.

\subsubsection*{Frequently used subsampling schemes}

The most common chroma subsampling schemes are summarized in Figure \ref{Fig:chroma_subsampling}:
\begin{itemize}
\item \textbf{4:4:4}: In this case no subsampling is performed neither in the horizontal, nor in the vertical dimension.
The spatial resolution of the chroma data is the same as the luma's.
If no subsampling is applied, the direct $R'G'B'$ representation of color pixels is often used instead of the \ycbcr space.
The scheme is, however, scarcely applied in consumer electronics, but mainly used in the field of film archivation, CGI and movie production.
The first video studio-standard to support 4:4:4 sampling was the UHD format, published in the ITU-2020 recommendation.
The representation of one pixel requires $3 \cdot 8 ~\mathrm{bit} = 24~\mathrm{bits}$ in case of the bit depth of 8.
%
\item \textbf{4:2:2}: The horizontal resolution of the chroma samples is the half of the luma resolution, while in the vertical dimension no subsampling is performed.
The chroma samples are \textbf{cosited} with every second luma samples horizontally, meaning that the horizontal center of the luma samples coincide with the center of every second luma samples.
4:2:2 is the default chroma sampling scheme of SD, HD and UHD studio standards, with also many high-end digital video formats and interfaces using this scheme.
Since two pixels consists of two luma, one $C_{\mathrm{B}}$ and one $C_{\mathrm{R}}$ sample, therefore the representation of one pixel requires$(\frac{2 + 1 + 1}{2})\cdot 8~\mathrm{bit} = 16~\mathrm{bit}$, meaning that the compression factor from 4:4:4 to 4:2:2 is $\frac{2}{3}$.
%
\item \textbf{4:1:1}: The horizontal resolution of chroma is quarter of that of the luma samples, and in the vertical dimension no subsampling is performed.
Originally it was the default sampling scheme of low-end consumer digital electronics, e.g. handheld DVCAMs, but nowadays it is rarely used.
Since every 4 pixels contains 4 luma and 1-1 chroma samples, therefore the representation of one pixel requires $( \frac{4 + 1 + 1}{4})\cdot 8~\mathrm{bit} = 12~\mathrm{bits}$ and the compression factor from 4:4:4 to 4:1:1 is $\frac{1}{2}$.
%
\item \textbf{4:2:0}: Both the horizontal and the vertical resolution of the chroma samples are half of the luma resolution.
This is the most commonly used subsampling scheme, used for digital video storing, local playback and broadcasting.
Similarly to 4:1:1, 4 pixels contain 4 luma and 1-1 chroma samples, therefore the representation of one pixel requires $( \frac{4 + 1 + 1}{4})\cdot 8~\mathrm{bit} = 12~\mathrm{bits}$ and the compression factor from 4:4:4 to 4:2:0 is $\frac{1}{2}$.

There are two main variants of 4:2:0 schemes, having different horizontal siting (different position of the subsampled chroma samples)
\begin{itemize}
\item In JPEG, H.261, and MPEG-1, $C_{\mathrm{B}}$ and $C_{\mathrm{R}}$ are sited \textbf{interstitially}, halfway between alternate luma samples.
\item In MPEG-2, $C_{\mathrm{B}}$ and $C_{\mathrm{R}}$ are \textbf{cosited} horizontally and are sited between pixels in the vertical direction (interstitially).
\end{itemize}
\end{itemize}
The question may arise, how the siting of the chroma samples can be interpreted.
In order to answer this the basic steps of chroma subsampling process has to be investigated in more details.


\subsubsection*{Signal processing questions of chroma subsampling}

The practical realization of the chroma subsampling concept requires two basic digital signal processing steps:
\begin{itemize}
\item on the source side (e.g. camera) the chroma samples---obtained from the $R'G'B'$ components---has to be sampled with a reduced sampling frequency. 
Digitally speaking, if the digital chroma samples are directly available then the chroma samples has to be \textbf{decimated} (resampled with reduced sampling frequency) according to the current subsampling scheme.
The subsampled chroma samples can be stored, transmitted between digital equipment, or broadcasted. 
\item On the receiver side (e.g. at a display) the $R'G'B'$ components has to be calculated from the luma-chroma representation.
In order to do so the discarded chroma samples has to be approximated based on the received chroma data, i.e. the missing chroma samples have to be \textbf{interpolated} (resampled with increased sampling frequency) .
\end{itemize}

\paragraph{Decimation of the chroma samples:}
As a well-known fact, the frequency content of a discrete signal contains the spectrum of the underlying continuous signal, repeating on the multiples of the sampling frequency.
Therefore if the bandwidth of the underlying continous signal is larger than the half of the sampling frequency (the so-called \textbf{Nyquist frequency}), after discretization the repeating spectra will overlap, leading to \textbf{aliasing} artifacts, and the original continuous signal can not be reproduced from its discrete samples.
In order to avoid the spectral overlaping, the signal has to be bandlimited to the half of the sampling frequency with an \textbf{antialiasing filter}, being a low pass filter with the cut-off frequency being the Nyquist frequency.

In the field of image reproduction appropriate antialiasing filtering is crucial: 
aliasing manifests in clearly visible low-frequency patterns on the sampled image.
As these aliasing patterns, termed as \textbf{Moiré patterns} are the result of the spatial sampling of continuous images the type of aliasing is termed as \textbf{spatial aliasing}.
\begin{figure}[]
	\centering
	\begin{overpic}[width = 0.8\columnwidth ]{figures_en/decimation.png}
 	\end{overpic}
	\caption{Signal processing chain of decimation (subsampling): 
	(1) is the input signal and $F(1)$ the input spectrum.
	(2) is the output of the antialiasing filter and $F(2)$ is its spectrum.
	(3) is the output signal.}
	\label{Fig:decimation}
\end{figure}

Generally speaking, decimation is the reduction of the sampling frequency of an arbitrary discrete input signal.
The new sampling frequency is lower than the initial one, and after decimation the spectrum of the input signal will repeat on the multiples of the new sampling frequency.
Therefore, prior to setting the new sampling frequency by discarding e.g. every second input samples, the original input signal has to be antialiasing filtered below the half of the new sampling frequency.
The process is illustrated in Figure \ref{Fig:decimation}.

\vspace{3mm}
Subsampling from e.g. 4:4:4 to 4:2:2 means the reduction of the chroma samples' horizontal sampling frequency by discarding every second samples in a line of samples, i.e. decimating it horizontally with the ratio of 2:1.
Therefore, without antialiasing filtering prior to decimation, the spectrum of the resampled chroma lines would overlap and Moiré patterns would appear in the displayed image.
In the present example the horizontal frequency content of the chroma signals has to be bandlimited to the half of the original bandwidth  by applying an appropriate \textbf{horizontal (spatial) low pass filter}, or horizontal antialiasing filter.
In case of converting from 4:4:4 $\rightarrow$ 4:2:0 also a further, \textbf{vertical low pass filtering} is required.

\begin{figure}[h!]
	\centering
	\begin{overpic}[width = 0.75\columnwidth ]{figures_en/aliasing2.png}
 	\end{overpic}
	\caption{
	Example for the aliasing of the chroma samples.
	The original 4:4:4 test image (a) contains a sine signal with its frequency increasing along both the horizontal and vertical dimensions, oscillating between the red and green primaries.
	Figure (b) shows the result of chroma subsampling conversion 4:4:4 $\rightarrow$ 4:1:1, with the interpolation performed by simply repeating the nearest sample ($\mathbf{h}_H= \left[ 1\,1 \,1 \,1 \right]^{\mathrm{T}}$).
	Figure (c) shows the result of ideal low pass filtering for both antialiasing and interpolation filtering.}
	\label{Fig:chroma_subsampling}
\end{figure}
\vspace{3mm}
The effect of omitting the antialiasing filter prior to decimation is illustrated in Figure \ref{Fig:chroma_subsampling} via the example of 4:1 decimation along the horizontal direction, i.e. in the case of the subsampling scheme conversion from 4:4:4 to 4:1:1.
Figure \ref{Fig:chroma_subsampling} (a) depicts the 4:4:4 representation of the image.
In Figure \ref{Fig:chroma_subsampling} the chroma samples are subsampled and reconstructed without any antialiasing filter applied.
The resulting aliasing Moiré patterns are cleary visible, seriously degrading the image quality.
Loosely speaking, the color information of the image contains high-frequency components (small details) that can not be represented in the reduced sampling grid, therefore the Nyquist sampling condition is violated, resulting in the visible patterns\footnote{Aliasing images/Moiré patterns are even more enhanced in case of spatially periodical images, since the aliasing components are also periodical, like the one in the present example.}.
Hence, the application of appropriate antialiasing filtering is crucial.
Obviously, since high frequency components in the spectrum represent small details in the image, therefore, spatial low pass filtering can be interpreted as ,,smoothing'' the image by blurring the small details.

\vspace{3mm}
Without going deep into signal processing details: the blurring of the small details in a signal can be most easily performed by weighted averaging of the adjacent samples (pixels), i.e. with FIR filtering.
Assume that the image is filtered in both the horizontal and vertical dimensions, meaning that both the horizontally and vertically adjacent samples are averaged.
The weights of the horizontal samples in the calculated sample is given by vector $\mathbf{h}_H = h_H(n)$.
Similarly, the vertical weight factors are given by $\mathbf{h}_V = h_V(n)$.
Let $x(m,n)$ denote the intensity of the input sample in the $m$-th row and $n$-th column, e.g. in this case being either $C_{\mathrm{B}}$ or  $C_{\mathrm{R}}$ samples.
The intensity of the filtered (or averaged) output sample can be expressed as
\begin{equation}
y(m,n) = \sum_{k = -\infty}^{\infty} \sum_{l = -\infty}^{\infty} x(k,l)\, h_V(m-k) \, h_H(n-l),
\end{equation}
describing consequent horizontal and vertical 1D convolutions.
Vectors $h_H(n)$ and $h_V(n)$ are the horizontal and vertical filter coefficients, or filter kernels (impulse responses).

As the simplest filtering approach the output is generated as the average of two adjacent samples both horizontally and vertically.
This simple averaging process is described by convolution with the filter coefficients
\begin{equation}
\mathbf{h}_H =
\mathbf{h}_V =
\begin{bmatrix}[c]
       1/2 \\[0.3em]
       1/2\end{bmatrix}.
\end{equation}
As a result, the output sample is obtained as the simple sum of 4 adjacent samples, therefore, the average sample is located between the input samples both horizontally and vertically, i.e. it is sited interstitially.
This is the antialiasing filtering approach, applied by the JPEG and MPEG-1 encoders during conversion to 4:2:0 subsampling scheme.

With increasing the computational complexity of filtering---i.e. by applying filters with longer impulse response/involving more input samples to the averaging process---the accuracy of the filtering can be improved, with higher achieved attenuation factor above the filter's cutoff frequency.
As an example: MPEG-2 encoder applies the same vertical filter coefficients as the MEPG-1, but in the horizontal direction every output sample is obtained from 3 adjacent input samples, resulting in improved antialiasing performance.
The filter coefficients for MPEG-2 are given by
\begin{equation}
\mathbf{h}_V =
\begin{bmatrix}[c]
       1/2 \\[0.3em]
       1/2\end{bmatrix}
,
\hspace{1cm}
\mathbf{h}_H =
\begin{bmatrix}[c]
       1/4 \\[0.3em]
       1/2 \\[0.3em]
       1/4\end{bmatrix}
\end{equation}
Since along the horizontal dimension 3 adjacent samples are symmetrically averaged, therefore, the output sample will be located coinciding with the input sample at the center position, i.e. they are cosited.
\begin{figure}[]
	\centering
	\begin{overpic}[width = 0.8\columnwidth]{figures_en/interpolation.png}
 	\end{overpic}
	\caption{Interpolation process, realized by simple linear filtering:
	(1) is the input signal, and $F(1)$ illustrating its spectrum.
	(2) zero stuffed input signal and its spectrum $F(2)$.
	(3) output of the interpolation filter.}
	\label{Fig:interpolation}
\end{figure}

\paragraph{Interpolation of the chroma samples:}
The inverse operation of decimation is the increasing of the sampling frequency, by approximating the samples of the input signal in intermediate sampling positions.
Interpolation of the missing chroma samples has to be performed in the receiver of the video signal in order to restore the chroma resolution prior to calculating the RGB signal to be displayed.

The signal processing chain of interpolation is illustrated in Figure \ref{Fig:interpolation}:
In order to increase the sampling frequency the original signal is stuffed with zeroes in the new sampling positions.
This zero stuffing leaves the input spectrum unchanged, since a zero-only spectrum is added to the input spectrum, but with an increased sampling frequency.
The approximation of the intermediate samples can be interpreted as filtering out the image spectrum, i.e. the repeating baseband spectrum on the original sampling frequency.
Hence, interpolation max be performed by simple low pass filtering of the zero-stuffed signal with an appropriate \textbf{reconstruction filter}.

\begin{figure}[t!]
	\centering
	\begin{overpic}[width = 0.8\columnwidth]{figures_en/subsampling_Example.png}
 	\end{overpic}
	\caption{Simple example for the chroma subsampling of natural images.
	The upper row presents the chroma subsampled image after reconstruction, the lower row presents the chroma information only.}
	\label{Fig:chroma_subsampling_Ex}
\end{figure}

Similarly to decimation, low pass filtering for interpolation can be performed by calculating the weighted average of the adjacent samples.
For chroma interpolation in case of 4:2:0 scheme interpolation must be performed both in the horizontal and vertical dimensions.
The simplest horizontal and vertical reconstruction filters are given by the coefficients
\begin{equation}
\mathbf{h}_H =
\mathbf{h}_V =
\begin{bmatrix}[c]
       1 \\[0.3em]
       1\end{bmatrix}.
\end{equation}
Obviously, the filter coefficients above realize the repetition of the nearest input samples in the interpolation position.
By increasing the computational cost, the missing samples can be approximated more accurately.
As an example: Filtering with the horizontal and vertical coefficients
\begin{equation}
\mathbf{h}_H =
\mathbf{h}_V =
\begin{bmatrix}[c]
       1/2 \\[0.3em]
       1 \\[0.3em]
       1/2 \end{bmatrix}
\end{equation}
realize linear interpolation (in case of the interpolation ratio 2:1). 

Also, interpolation can be carried out by more sophisticated methods, besides linear filtering, e.g. based on fitting higher order polynomials to the input data (bicubic interpolation).

\vspace{3mm}
Clearly, the introduced antialiasing and reconstruction filter coefficients coincide up to a constant.
However, the sum of the antialiasing filtering coefficients has to be equal to 1, otherwise the averaging process would change the intensity of the input signal (e.g. the saturation of the colors).
On the other hand, the reconstruction filter in case of $N:1$ interpolation should have the total energy $N$, in order to keep the signal energy unchanged.

Figure \ref{Fig:interpolation} (c) illustrates the effect of chroma subsampling and reconstruction by applying ideal low pass filters for both antialiasing and reconstruction (meaning that its frequency transfer characteristics is a rectangular window).
It is clearly shown that on those parts of the image where the rapid change of color information can not be represented in the subsampled grid, the blurring the color information results in an average yellow hue instead of the oscillation between red and green.
In practice, in case of natural images the chroma component rarely contains alternating high frequency components, which would result in such clearly visible artifacts after filtering.

Figure \ref{Fig:chroma_subsampling_Ex} depicts the chroma subsampling and reconstruction process of a more natural test image.
As it is demonstrated, even in the case of a theoretical 4:1:0 subsampling scheme---in which case the chroma resolution is the quarter of the luma information both in the horizontal and vertical dimensions---the final quality of the reproduced image is only slightly degraded.
The subsampling scheme 4:2:0, on the other hand, ensures the image quality, approximately indistinguishable from the original.

\vspace{2cm}
\noindent\rule{12cm}{0.4pt}

\subsection*{End-of-Chapter Questions}

\begin{itemize}
\item What are the most commonly used chroma subsampling schemes? 
What is the compression factor of the 4:2:0 scheme, compared to the 4:2:2 scheme?
\item What is the difference between the subsampling scheme of MPEG-1 and MPEG-2?
\end{itemize} 

\chapter{Video formats}
\label{sec:video_formats}
The previous chapter introduced the representation of color pixels in analog and digital systems.
The current chapter presents how analog and digital video signals can be formed by using these pixel representations and discusses how the parameters of the resulting video formats were chosen.

\section{Structure and properties of the video signal}

The discussion starts with the questions that arose at the establishment of the early analog TV broadcast systems, beginning with the format parameters elaborated for the NTSC and PAL standards and later adopted by the SD digital format.
These analog systems are, of course, already superseded by digital transmission and broadcasting standards.
However, their discussion is still of great importance, on one hand because the principles of their actual parameter choices hold unchanged for introducing modern systems as well.
On the other hand, a large number of parameters, used in HD and UHD formats are the legacy of these early systems.

\subsection{Structure of the analog video signal}

Before discussing the actual video parameters the structure of the video signals is introduced.
The term video signal refers to the signal, carrying video information either from a broadcasting video source, or between local electronic devices (e.g. from set-top-box to the TV on a HDMI interface, or from a computer/laptop to an external display through VGA interface).
Even today, the structure of the digital video signal is completely identical with the analog representation, that's structure is the result of the operation principle of early CRT displays.

In the previous chapter \ref{sec:CRT} gave a detailed explanation on the principle of cathode-ray tube based imaging devices:
The realization of the RGB primaries was achieved by phosphores covering the screen, emitting visible light when excited.
The excitation was ensured by an electron gun, producing a narrow electron beam, with the current density being approximately proportional to the 2.2 power of the input voltage (hence the original need for gamma correction).

The electron beams (one in case of black and white, three beams in case of colored display), driven with the input voltage of the video signal, are focused and steered by properly driven magnetic coils, and scan the display screen line by line, along \textbf{scan lines} in a predefined \textbf{raster scanning pattern}\footnote{The choice of scanning the screen linewise is not absolutely obvious, while creating the early black and white TV standards other solution, e.g. columnwise or back-and-fortha linewise scanning types were also investigated.}.
As the electron beam reaches the end of a given scan line, it retraces horizontally to the beginning of the next line.
Similarly, by reaching the lower end of the screen the beam retraces vertically to the beginning of the next frame.

An important principle of CRT displays is that the video data is received and displayed real-time, instantaneously (when early TV receivers were introduced circuits existed for storing video data).
Hence, the analog video signal is basically the driving voltage of the CRT electron guns, containing the video data line by line: 
In case of a black and white display, the video signal consists of the continuous luma information $Y'$, while in color receivers the receiver either receives directly the $R'G'B'$ signals (not used in video broadcasting), or demodulates them from the luma-chroma representation.
The principle of raster scanning is depicted in Figure \ref{Fig:TV_signal} (a).
%
\begin{figure}[]
	\centering
	\begin{minipage}[c]{0.3\textwidth}
		\begin{overpic}[width = 1\columnwidth ]{figures/FormatJargon-1.png}	
		\small
		\put(0,0){(a)}		
		\end{overpic}
				\begin{overpic}[width = 1\columnwidth ]{figures/TV.png}	
		\small
		\put(0,0){(d)}		
		\end{overpic}
	\end{minipage} \hfill
	\begin{minipage}[c]{0.68\textwidth}
		\centering
		\begin{overpic}[width = 0.86\columnwidth ]{figures/FormatJargon-3.png}		\small
		\put(0,0){(b)}		\end{overpic}
		\begin{overpic}[width = 1\columnwidth ]{figures/FormatJargon-7.png}		\small
		\put(0,0){(c)}		\end{overpic}
	\end{minipage}
%
	\caption{General structure of analog video signal}
	\label{Fig:TV_signal}
\end{figure}
%

Obviously, it takes a finite time for the electron beam to retrace from the end of a line to the beginning of the next one, and similarly to retrace from the end of the screen to the beginning.
During these \textbf{retracing intervals} the electron beam is turned off, i.e. \textbf{blanked}, otherwise undesired traces would appear on the screen.
In practice this means that during these \textbf{blanking intervals} the video signal is zero, or negative valued, i.e. it takes black level, or synch level.
In the blanking interval synchronizing pulses are added to the video signal that ensure that the oscillators in the receiver remain locked with the transmitted signal, so that the image can be reconstructed on the receiver screen.

Based on the foregoing the linewise video signal contains three distinct time intervals:
\begin{itemize}
\item The \textbf{active video content}, containing the actual video information of one line.
In analog formats the length of the active video interval is expressed in $\mu \mathrm{s}$, and in $\mathrm{px}$-s in the digital case.
\item The \textbf{horizontal blanking interval}, the time it takes for the electron beam to retrace from the end of the actual scan line to the beginning of the next scan line.
The horizontal blanking interval contains the horizontal sync pulse (\textbf{HSYNC}) with negative signal level (i.e. ,,blacker than black''), serving as the trigger pulse for the horizontal retracing circuit.
The sync pulse is separated from the active video data with black level intervals before and after the sync pulse, termed as the front porch and black porch.
The length of horizontal blanking is expressed in $\mu \mathrm{s}$ in the analog, and in $\mathrm{px}$-s in the digital case.
\item The \textbf{vertical blanking interval (VBI)}, the time it takes for the electron beam to retrace from the end of the actual frame to the beginning of the next frame.
The vertical blanking interval contains the vertical sync pulses (\textbf{VSYNC}) at negative signal level, allowing the vertical synchronization of the display frames on the screen.
The length of the VBI is expressed in lines (number of the inactive lines).
\end{itemize}
The video signal, therefore, contains in one entire line period time ($T_{\mathrm{H}}$, for horizontal) both \textbf{active video interval} and \textbf{inactive interval}, carrying no video information.
The structure of video lines is depicted in Figure \ref{Fig:TV_signal} (b).
Similarly, the entire period time of one frame ($T_{\mathrm{V}}$, as vertical) contains \textbf{active lines} and \textbf{inactive lines}, containing no actual video data.
The structure of active and inactive lines is depicted in Figure \ref{Fig:TV_signal} (c).

In the following it is revealed what principles lead to the actual timing parameters (e.g. the lengths of the above intervals, line frequency, frame frequency) and to the standard definition resolution parameters. 

\subsection{Parameters of analog video formats}

\subsubsection*{Aspect ratio and display size}

First it is explained what display size should be the optimal format parameters chosen for.

It was already discussed in \ref{Sev:HVS} that the human color vision is ensured by the macula, containing mostly cones, located around the center of the retina.
In the center of the macula the fovea is responsible for sharp central vision (also called foveal vision).
Due to the size of the fovea, the human central vision with high visual acuity covers about $10-15^{\circ}$ from the entire field of view of $\approx 200^{\circ}$ horizontally.
During the creation of standard definition television standard the basic goal was to fill only the central vision with content, therefore, the SD television should cover about $10^{\circ}$ from the horizontal field of view (i.e. the peripheral vision does not contribute to imaging).
Obviously, the actual display size then depends on the viewing distance, as it will be discussed later in this chapter.

\vspace{3mm}
Another important spatial attribute of video formats is the ratio of the horizontal and vertical screen size, i.e. the \textbf{aspect ratio} ($a_r$).
The basis of all the color TV formats was the NTSC standard and its black and white predecessor, introduced in the 1940s.
As an obvious endeavor the introduced broadcasting format was engineered in order to be compatible with the then-existing video sources, e.g. movie films.
Until the 1950s during the entire silent film era movies were almost exclusively captured with the aspect ratio of 4:3, i.e. the ratio of the horizontal and vertical screen size was $1.3\dot{3}$
\footnote{
The introduction of the aspect ratio of 4:3 is connected to the work of Thomas Alva Edison, who \href{https://en.wikipedia.org/wiki/35_mm_movie_film}{defined} the standard image exposure length on $35~\mathrm{mm}$ film for movies is four perforations ($19~\mathrm{mm}$) per frame along both edges. 
From the available film width between perforations ($25.375~\mathrm{mm}$) the active are has an aspect ratio of 4:3.
This \href{https://en.wikipedia.org/wiki/Negative_pulldown}{4-perf negative pulldown} became the official standard in 1909, allowing the emerging of standard movie cameras, movie projectors, and hence emerging of cinema technologies.}.
Although by the 1950s first widescreen movie formats have already emerged, the NTSC standard adopted the \textbf{4:3} aspect ratio, which remained the standard TV and video aspect ratio until the introduction of the HD format.

%TODO anamorphic lenses?
% Forrás: https://www.shutterstock.com/blog/4-3-aspect-ratio
% https://www.cinematographers.nl/FORMATS1.html

\subsubsection*{Refresh rate and frame rate}

The next parameter to investigate is the temporal sampling frequency of the video data, i.e. the number of frame fed to the display device per second.
First three closely related terms are introduced:
\begin{itemize}
\item The \textbf{refresh rate} ($\mathbf{f_{\mathrm{r}}}$) refers to the number of times in a second that a display ,,flashes'', i.e. redraws its content, expressed usually in $\mathrm{Hz}$.
\item The \textbf{frame rate} ($\mathbf{f_{\mathrm{V}}}$) express the number of time in a second that the content of the display changes, i.e. the number of frames, contained by the video signal per second, usually expressed in $\mathrm{fps}$ (frame per second) or in $\mathrm{Hz}$.
\item The \textbf{field rate} ($\mathbf{f_{\frac{\mathrm{V}}{2}}}$) can be defined for interlaced video data, denoting the number of fields (half frames) per second in the video data, generally expressed in $\mathrm{fps}$.
Universally $f_{\frac{\mathrm{V}}{2}} = 2\cdot f_{\mathrm{V}}$ holds.
\end{itemize}
The key difference between refresh rate and frame rate is that refresh rate is the property of the display device (e.g. LCD display) and includes the repeated drawing of identical frames, while frame rate measures how often a video source can feed an entire frame of new data to a display.

In order to arrive at practical frame rate and refresh rate choice for video data two perceptual aspects have to be taken into consideration:
\begin{itemize}
\item On one hand when reproducing objects under motion it is crucial to display sufficient number of motion phases in order to ensure that the observer perceives a smooth, continuous motion.
This requirement constitutes a lower limit for the applicable frame rate.
\item On the other hand the refresh rate has to be chosen high enough in order to avoid \textbf{flickering} and to reduce eye strain.
\end{itemize}
Due to the \textbf{beta movement} phenomenon the frame rate can be significantly lower, than the refresh rate:
Beta movement is an optical illusion whereby viewing a rapidly changing series of static images creates the illusion of a smoothly flowing scene, occurring if the frame rate is greater than 10 to 12 $\mathrm{fps}$. 
Therefore, due to the beta movement the frame rate should satisfy
\begin{equation}
f_{\mathrm{frame}} > \sim20~\mathrm{Hz},
\end{equation}
\footnote{
It is worth noting that the above frame rate only allows the perception of motion instead of distinct static images, higher frame rates )(usually $60~\mathrm{fps}$) still ensure much smoother motion reproduction.
In order to increase the effective frame rate of the video stream, modern displays and software allow temporal interpolation by estimating the intermediate frames, based on some motion prediction algorithm, similarly to MPEG encoding.
However, the average viewer already adapted to the frame rate of $24~\mathrm{fps}$, being the standard frame rate of movie films, therefore the increased frame rate often generates an unpleasant, unnatural effect.
This is termed as the \textbf{soap opera effect}, originating from the fact that usually low cost TV shows are recorded directly to digital video cameras (being much cheaper than capturing to film), allowing higher frame rates, usually set to $60 \mathrm{fps}$.}
however, applying the same refresh rate would lead to serious perceived flickering artifacts.

In over to avoid flickering the refresh rate has to be higher than the \textbf{flicker fusion threshold}
\footnote{
The flickering fusion threshold is the frequency at which an intermittent light stimulus appears to be completely steady to the average human observer.
For surfaces with temporally alternating luminances above the flickering threshold the average luminance is perceived.
The threshold depends on numerous factors:
Depends on the average illumination intensity, the adaptation state, the color of vibration (above $15-20~\mathrm{Hz}$ fluctuation of hue information can not be perceived) of and the position on the retina at which the stimulation occurs}.
For the purposes of presenting moving images, the human flicker fusion threshold is usually taken between 60 and $90~\mathrm{Hz}$:
in the central vision, dominated by the cones the response time is high, and the flickering fusion threshold is around $50~\mathrm{Hz}$.
The peripheral vision is dominated by the rods, with a much lower response time and the flickering fusion threshold is higher.

In case of analog TV formats the goal was to fill the central vision of the viewer with, therefore, with a refresh rate of
\begin{equation}
f_{\mathrm{r}} > 50-60~\mathrm{Hz} 
\end{equation}
flickering can be avoided\footnote{This is true only for the central vision.
The flickering of CRTs can be easily observed with a display watched from the peripheral vision}.
The choice of the actual refresh rate was, however, a consequence of the CRT technology's imperfection: a trick in order to avoid the effect of the supply voltage ripple.

\begin{figure}[]
	\centering
	\begin{overpic}[width = 1\columnwidth ]{figures/ripple.png}
	\end{overpic}
	\caption{Source of the supply voltage ripple in a single-way rectifier}
	\label{Fig:ripple}
\end{figure}

\paragraph*{Effect of the mains frequency:}
Voltage ripple is the residual periodic variation of the DC voltage in a power supply due to the imperfect rectification of the alternating means AC voltage, as illustrated in \ref{Fig:ripple}.
The ripple frequency coincides with the means frequency in case of single-way rectification, or with its double in the two-way case.
As in a CRT display the anode is directly connected to the supply voltage, therefore, any perturbation in the DC voltage is directly added to the video signal and is displayed on the screen.

Assume a screen consisting of $N_{\mathrm{V}}$ horizontal scan lines, with the refresh rate denoted by $f_\mathrm{r}$.
The number of scan lines displayed in a second, i.e. the line frequency is given by
\begin{equation}
f_\mathrm{H} = N_\mathrm{V} \cdot f_\mathrm{r}.
\end{equation}
In order to investigate the effect of supply ripple, as a generalization, assume a periodic black and white video signal, oscillating between 0 and 1, described by
\begin{equation}
Y(t) = \frac{1}{2} \left( 1 + \sin 2 \pi f t \right),
\end{equation}
with $f$ being the signal frequency. 
For the sake of simplicity the blanking intervals are assumed to be of zero length.
In this case the sine wave is displayed on the screen line by line.
Depending on the signal frequency $f$ the content of the screen can be the following:
\begin{itemize}
\item If $f = f_\mathrm{H}$ each scan line consists of exactly one period of the sine wave, resulting in a horizontal sine on the screen, as depicted in Figure \ref{Fig:ripple_display} (a).
\item If $f > f_\mathrm{H}$ each scan line consists of less then one period of the sine wave, with the initial phase in the beginning of the scan lines increasing line by line and frame by frame.
Therefore, the horizontal sine on the screen is slightly steered, and moves towards left.
Similarly, with $f < f_\mathrm{H}$ the image moves slowly towards right.
\item If $f = f_\mathrm{r}$ each frame consists of a single period of the sine wave.
Assuming $N_\mathrm{V} \gg 1$ the value of the sine function changes insignificantly over one scan line, therefore, the displayed image is a vertical sine.
\item If $f > f_\mathrm{r}$ the initial phase of the vertical sine increases frame by frame, thus, the sine wave moves slowly upwards.
Similarly, for $f < f_\mathrm{r}$ the vertical sine moves downwards.
%
\end{itemize}
\begin{figure}[]
	\centering
	\begin{overpic}[width = 0.45\columnwidth ]{figures/horizontal_sine.png}
	\small
	\put(0,0){(a)}
	\end{overpic}
	\hspace{5mm}
	\begin{overpic}[width = 0.45\columnwidth ]{figures/vertical_sine.png}
	\small
	\put(0,0){(b)}
	\end{overpic}
	\caption{Periodic video signal with $f = f_\mathrm{H}$ (a) and $f = f_V$ (b) as displayed line by line}
	\label{Fig:ripple_display}
\end{figure}
%
Based on the foregoing, with the appropriate synchronization of the signal frequency and the refresh rate a periodic video signal can be displayed as a still, non-moving image.
The effect of voltage ripple acts as such a periodic noise signal, with the frequency determined by the mains voltage of the given region and displayed inevitably on the screen.
Early tests with black and white TV receiver suggested the visible effect of this periodic noise of less disturbing if the noise image is still.
Therefore, both in the American, and later in the European systems refresh rate was chosen to coincide the mains frequency
\footnote{
In the American NTSC system with the introduction of color information the choice of refresh rate became more complicated, as the frequency of the color subcarrier could not be correctly chosen.
Without more details: as a consequence both the refresh rate (and the field rate) and the line frequency had to be decreased by $0.1~\%$.
Therefore, in the American system the refresh rate is $f_r = 60\cdot \frac{1000}{1001} = 59.94~\mathrm{Hz}$.
Due to the presence of vertical and horizontal sync pulses, this change did not had and effect on the then-existing TV receivers.}.
\begin{equation}
f_{\mathrm{r,USA}} = 60~\mathrm{Hz}, \hspace{1cm} f_{\mathrm{r,Eu}} = 50~\mathrm{Hz}
\end{equation}
ensuring that supply ripple only manifests in a non-moving vertical noise image on the screen.

Having found the refresh rate for early TV systems, which has been also in use by modern CRTs and LCD displays, still, the actual frame rate is an open question.
An obvious choice for the frame rate would be the refresh rate itself.
Due to bandwidth efficiency, however, instead a special raster scan order was introduced. 

\subsubsection*{Raster scan orders}

The raster scan order is the rectangular pattern of image capture and reconstruction in television.
In the receiver side it defines the systematic process how the electron beam scans the entire screen over a given time interval.
The following section introduces two frequently used raster scan order, leading to the definition of frame rate and field rate for.

Obviously, for currently used LCD and OLED displays the raster scan order can not be interpreted as the scanning pattern of the display, since the entire display content is updated in the same time instant.
Still, scan order can be understood also as the storing and transmission order of video data, hence scan order is interpretable in modern video systems as well.

\paragraph{Progressive scan:}
As the most straightforward scanning order, \textbf{progressive scan} is a format of displaying, storing, or transmitting moving images in which all the lines of each frame are drawn in sequence.
In the receiver side it means that the electron beam scans and updates the content of the entire screen over one frame time ($T_{\mathrm{V}} = 1/f_{\mathrm{V}}$), line by line.
\begin{figure}[]
	\centering
	\begin{minipage}[c]{0.6\textwidth}
	\begin{overpic}[width = 1\columnwidth ]{Figures/progressive_scan.png}
	\end{overpic}   \end{minipage}\hfill
		\begin{minipage}[c]{0.3\textwidth}
	\caption{Illustration of progressive scanning (for the sake of simplicity with the exemplary number of 11 lines), with the active line content (1), the vertical retrace/blanking interval (2) and the horizontal retrace/blanking interval (3).}
	\label{Fig:progressive}  \end{minipage}
\end{figure}
Progressive scanning is illustrated in Figure \ref{Fig:progressive}.

In the aspect of transmission in case of video data with progressive scanning the content of the entire frames has to be transmitted over the frame time via the given interface (e.g. HDMI) sequentially.
Progressive scan is usually denoted by letter ,,p'' in the format designation.

Although progressive scan seems to be the most simple and obvious scan order, still, until the emergence of the UHD format progressive scan was only occasionally utilized due to the reasons discussed in the following.

\vspace{3mm}
\paragraph{Interlaced scan:}
The previous section have already discussed that in order to avoid flickering the display refresh rate should be risen above $50~\mathrm{Hz}$, while for ensuring continuous motion the video frame rate of around $20~\mathrm{Hz}$ may be sufficient.
This fact suggests that frame rate should be handled independently by choosing a high refresh rate and a efficiently low frame rate, allowing the reduction of video data and the required bandwith for transmission.

This concept could be easily implemented in cinematic techniques:
As a heritage of the early silent movie era (where the frame rate varied between $16-24~\mathrm{Hz}$) movie films are commonly captured with the frame rate of 24 frames per second.
In order to avoid flickering the film projectors are equipped with special \href{https://www.youtube.com/watch?v=jrSzRAch930}{two or three blade shutters}, rotating in front of the projector beam and flashing each frame two or three times before the film would travel to the next frame.
With the simple trick of presenting the same frame multiple times the effective refresh rate can be increased to $48\mathrm{fps}$ or $72~\mathrm{fps}$, and flickering may be avoided.

Also, nowadays frame rate and refresh rate can be handled independently in digital systems (e.g. in a video card and a VGA monitor), where a \textbf{display buffer} is present, containing the data of an entire frame, generating and feeding the video signals towards the display.
Modern LCD displays are usually built with the LED backlight panel flashing at around $200~\mathrm{Hz}$\footnote{
Obviously, the LED panel could be built with continuous driving voltage of well, instead of flashing, but the dynamic adjustment of its lightness (based on the video content) can be much cost-efficiently solved with PWM modulation.}.
Still, as a rule of thumb even in the presence of a video buffer the content of the video buffer may change during the vertical blanking of the display in order to avoid \href{https://en.wikipedia.org/wiki/Screen_tearing#/media/File:Tearing_(simulated).jpg}{screen tearing}.
As a consequence, generally $f_{\mathrm{r}} = \mathrm{n} \cdot f_{\mathrm{V}}$ holds, where $\mathrm{n} \in \lbrace 1,2,3,... \rbrace$.

In analog TV receiver no frame buffer could be present due to the lack of appropriate technology, the transmitted signal had to be displayed by the receiver on the fly.
Obviously, transmitting the same frames multiple times requires a significant increase of bandwidth. 
Therefore, a more simple engineering workaround was needed for the bandwidth efficient video transmission, which was eventually provided by the concept of \textbf{interlace scanning}.

\begin{figure}[]
	\centering
	\begin{overpic}[width = 0.85 \columnwidth ]{Figures/interlaced_scan.png}
	\end{overpic}
	\caption{Illustration of interlaced scanning (for the sake of simplicity with the exemplary number of 21 lines)
	Assuming that scanning starts with the beginning of the first scan line, in order to cover the entire screen the first field has to end in a half-line, and the second field has to start with a half-line.
	This requirement can be satisfied only by applying odd number of scan lines (each field consists of $N_{\frac{V}{2}} + \frac{1}{2}$ lines, thus the total line number is given by $2 N_{\frac{V}{2}} + 1$, being an odd number by definition). }
	\label{Fig:interlaced}
\end{figure}

The basic concept of interlaced scanning is illustrated in Figure \ref{Fig:interlaced}:
Instead of scanning each line of the display within one frame period time, the screen is divided into two \textbf{fields}
\begin{itemize}
\item the \textbf{odd field}, containing every odd scan line 
\item the \textbf{even field}, containing every even scan numbered scan line
\end{itemize}
The interlaced signal contains the two fields of a video frame captured consecutively (i.e. the even and odd fields are captured in consequent time instants).
Similarly, in the receiver the electron beam scans first all the odd lines, displaying the content of the odd field, then draws the content of the even field into all the even lines.
Interlaced scan is usually denoted by letter ,,i'' in the format designation.

By applying interlaced scanning the content of the display is redrawn with the frequency of the fields, thus, for interlaced video the refresh rate is given by the \textbf{field rate} ($f_{\frac{\mathrm{V}}{2}}$).
Therefore, the field rate is determined by the mains frequency of the given region, being $50~\mathrm{Hz}$ for the European and $60~\mathrm{Hz}$ (more precisely $59.94~\mathrm{Hz}$) for the American electric network.
Furthermore, the period time of an entire frame, consisting both even and odd fields is obviously twice the field period time, thus the frame rate is half of the field rate.
As a result in the frame rate is $25~\mathrm{Hz}$ in the European and $30~\mathrm{Hz}$ ($29.97~\mathrm{Hz}$) in the American systems, being high enough to ensure the perception of continuous motion, while the refresh rate is sufficient in order to avoid flickering.
As a summary for the foregoing in the European and American system 
\begin{align}
\begin{split}
f_{\mathrm{r,USA}} &= f_{\frac{\mathrm{V}}{2},\mathrm{USA}} = 2\cdot f_{\mathrm{V},\mathrm{USA}} = 60~\mathrm{Hz} , \hspace{1cm}  f_{\mathrm{V},\mathrm{USA}}= 30~\mathrm{Hz} \\
f_{\mathrm{r,Eu}} &= f_{\frac{\mathrm{V}}{2},\mathrm{Eu}} =  2\cdot f_{\mathrm{V},\mathrm{Eu}} = 50~\mathrm{Hz}, \hspace{1cm}  f_{\mathrm{V},\mathrm{Eu}}= 25~\mathrm{Hz}
\end{split}
\end{align}
hold.

Opposed to the solution of cinema, i.e. displaying the same frame multiple times the even and odd fields in interlaced video are taken in consequent time instants.
Therefore, in interlaced video the temporal resolution is effectively doubled at the cost of halving the vertical spatial resolution of the video data.
This means the following properties of interlaced video
\begin{itemize}
\item Compared to progressive video with the same refresh rate, interlaced scanning achieves the compression factor of 2:1, resulting in halved analog bandwidth
\item In case of still, and slowly moving scenes the vertical resolution coincides with the progressive resolution (since the even and odd fields complete the same frame)
\item In case of rapid motion the vertical resolution if half of the progressive resolution
\end{itemize}
Generally speaking for video content with slowly changing content, e.g. movies, interlaced scanning ensures a sufficiently high vertical resolution with improved motion reproduction besides feasible signal bandwidth.
In case of fast motion, e.g. camera movements in sport programs the artifacts due to the reduced vertical resolution become visible.
\vspace{3mm}

As bandwidth efficiency was of primary importance during the introduction of early analog television broadcasting, all the analog and digital standard definition video formats adopted exclusively interlaced scanning.
Later the high definition video standard introduced progressive video formats for the first time, while the latest ultra high definition format supports progressive scanning mode only.

\vspace{3mm}
\paragraph{Questions of interlaced scanning:}
Besides its obvious advantages, the application of interlaced video rises a number of questions and challenges.

As an example, it introduces the phenomenon of \textbf{interline twitter}:
The previous chapter explained that the violation of the sampling theorem---i.e. sampling frequency components above half the sampling frequency---leads to spatial aliasing phenomena.
In the field of image processing it manifests in visible Moiré pattern on spatially periodic textures (e.g. a brick wall, or squared shirts).
In case of interlaced video the vertical resolution is halved, compared to progressing scanning, therefore, vertical aliasing artifacts are more likely to emerge.
Since the even and odd fields are displayed alternatively, therefore also the potential Moiré patterns alternate from field to field.
This may produce a shimmering effect, termed as twittering, even when a still image is \href{https://en.wikipedia.org/wiki/File:Indian_Head_interlace.gif}{reproduced}.
Interline twittering is the main reason, why television professionals avoid wearing clothing with fine striped patterns, while professional video cameras apply a low-pass filter to the vertical resolution of the signal to prevent interline twitter. 

\begin{figure}  
\small
  \begin{minipage}[c]{0.64\textwidth}
	\begin{overpic}[width = 1\columnwidth ]{Figures/Interlaced_video_frame_(car_wheel).jpg}
	\end{overpic}   \end{minipage}\hfill
	\begin{minipage}[c]{0.3\textwidth}
    \caption{
    Displaying interlaced video on a progressive display without deinterlacing.}
\label{fig:deinterlacing}  \end{minipage}
\end{figure}

Further questions arise in case of the conversion between interlaced and progressive formats.
The progressive to interlaced conversion can be straightforwardly solved by dividing the entire frame to odd and even lines.
Interlaced to progressive conversion emerges more frequently in practice: e.g. in case of the playback of a DVD disc on a usually progressive computer monitor.
As the simplest solution, two adjacent field can be combined to form a single frame, however, this naive approach leads to the ,,combing'' of moving objects on the screen, as illustrated in Figure \ref{fig:deinterlacing}.
Generally speaking techniques for the interpolation of the content of intermediate scan lines are termed as \textbf{deinterlacing} methods.
Deinterlacing is an often emerging problem even today, as broadcasting most often transmits interlaced HD format (generally using format 1080i), while modern LCD displays do not support native interlaced scanning anymore.


\section{Analog video formats}

\begin{figure}[]
	\centering
	\begin{overpic}[width = 0.8\columnwidth ]{figures_en/NTSC_PAL.png}
	\end{overpic}
	\caption{Distribution of analog television formats by nation}
	\label{Fig:NTSC_map}
\end{figure}

Based on the foregoing, the main properties of the video format used for analog broadcasting can be introduced.
Historically, three analog video formats were introduced with the advent of color television broadcasting:
\begin{itemize}
\item The first color TV broadcast system was the \textbf{NTSC} format, introduced by the Federal Communications Commission (FCC) in the United States in 1954 and have been adopted by western America and Japan.
For vertical resolution NTSC selected $N_V = 525$ scan lines\footnote{
The choice of the vertical line number was a compromise between the existing black and white systems, e.g. as used by RCA's NBC TV network, and the intention of manufactures, desiring to increase the line number to 600-800.
The actual value originates from a limitation of the CRT technology of the day:
In early TV receivers a master oscillator ran at twice the line-frequency, and this frequency ($2\cdot 15750~\mathrm{Hz}$) was divided down by the number of lines used (in this case $N_V = 525$) to give the field frequency ($f_{\frac{\mathrm{V}}{2}} = 60~\mathrm{Hz}$).
This frequency was compared with the mains frequency ($60~\mathrm{Hz}$ in America) and the master oscillator's frequency was corrected by the discrepancy in order to avoid a moving noise image due to supply ripple.
At the time, frequency division was performed use of a chain of multivibrators, the overall division ratio being the mathematical product of the division ratios of the chain.
Since the line number has to be odd an odd number (see previous section), therefore it can be factorized only to odd numbers as well, which had do be relatively small in order to avoid thermal drift of the oscillator.
The closest sequence up to about 500 lines that meets all these requirements is $525 = 3\cdot 5\cdot 5 \cdot 7$, giving the number of scan lines.}
with interlaced scanning and the	 field rate of $f_{\mathrm{\frac{V}{2}}} = 60~\mathrm{Hz}$ in agreement with the mains frequency.
\item As an improved version of NTSC, the \textbf{PAL} system was introduced by the European Broadcasting Union (EBU) in 1967 and besides Europe it was adopted in Australia, South America, Africa and a part of Asia.
The PAL format applied interlaced scanning with the number of scan lines of $N_V = 625$ along with the field rate of $50~\mathrm{Hz}$.
\item Also as an improvement on NTSC, emerging from France the \textbf{SECAM} system was introduced in France and in the Soviet Union.
The SECAM system will not be discussed in details in the present book.
\end{itemize}
The main parameters of the NTSC and PAL formats are summarized in Table \ref{tab:sd_formats}.
As a summary it can be stated that NTSC provides a higher temporal resolution (i.e. higher refresh rate/field rate) with lower spatial resolution, while PAL ensures a higher spatial resolution due to the increased number of scan lines at the cost of lower field rate.

\begin{table}[h!]
\caption{Parameters of the NTSC and PAL analog formats}
\renewcommand*{\arraystretch}{2.25}
\label{tab:sd_formats}
\begin{center}
    \begin{tabular}[h!]{ @{}c | | l | l @{} }%\toprule
				         &   NTSC  							       & PAL \\ \hline
    Total line number ($N_{\mathrm{V}}$):	 &  525   								   &  625 \\
    Number of active lines ($N_{\mathrm{V,A}}$):   &  480   								   &  576 \\
    Frame rate ($f_{\mathrm{V}}$):    &  $30~\mathrm{Hz}$ ($29.97~\mathrm{Hz}$) & $25~\mathrm{Hz}$ \\
    Field rate ($f_{\frac{\mathrm{V}}{2}}$): &  $60~\mathrm{Hz}$ ($59.94~\mathrm{Hz}$) & $30~\mathrm{Hz}$ \\
    Line frequency ($f_{\mathrm{H}}$):    &  $525 \cdot 30 = 15750~\mathrm{Hz}$ ($15734~\mathrm{Hz}$) & $15625~\mathrm{Hz}$ \\
    Line period time ($T_{\mathrm{H}}$):           &  $63.49~\mathrm{\mu s}$ ($63.55~\mathrm{\mu s}$) & $64~\mathrm{\mu s}$ \\
    Active line time ($T_{\mathrm{H,a}}$):           &  $\approx 52 ~\mathrm{\mu s}$ ($63.55~\mathrm{\mu s}$) & $52~\mathrm{\mu s}$ \\
    \end{tabular}
\end{center}
\end{table}

\begin{figure}[t!]
\captionsetup{singlelinecheck=off}
\small
  \begin{minipage}[c]{0.64\textwidth}
	\begin{overpic}[width = 1\columnwidth ]{Figures/Timing_PAL_FrameSignal.png}
	\end{overpic}
    \end{minipage} \hfill
	  \begin{minipage}[c]{0.3\textwidth}
    \caption[]{ The line structure of an entire frame in case of interlaced scan, illustrated for a single video component:
    \begin{itemize}
    \item active line interval: grey
    \item horizontal blanking interval: magenta, cyan and yellow
	\item vertical blanking interval: green, orange and white
    \end{itemize}
    }\label{fig:PAL_frame}
    \end{minipage}
\end{figure}
The structure of an entire frame (i.e. two consecutive fields) in the PAL system is illustrated in Figure \ref{fig:PAL_frame}.
Obviously, the figure depicts only a single video component, the content of the components is discussed in the following.
The figure depicts the horizontal and vertical blanking intervals, with the VSYNC (denoted by green color) and the HSYNC (yellow interval) sync pulses.
These pulses ensure the synchronization of the display to the received signal by triggering the vertical and horizontal retrace of the electron beam, therefore, it is ensured that the individual ,,pixel'' values are displayed on the correct position.
With the loss of these sync pulses the displayed image is misaligned vertically (in case of lost VSYNC), resulting in the \textbf{jitter} or horizontally (in case of lost HSYNC) causing the \textbf{rolling} of the video signal.

Although the characteristics of the video signal---with containing vertical and horizontal blanking intervals---originates from the operation principle of CRT displays, even in modern digital systems (e.g. in case of HDMI or SDI interfaces) the structure of the video signal is identical with the presented analog one.
Obviously, modern LCD displays do not need the presence of blanking intervals at all, since they update the content of all the pixels quasi-instantaneously, still, the blanking intervals are present in digital video signals as well.
One reason for this is that digital video signal is the legacy of the CRT era (and even today, professional CRT studio monitors are often employed), with the newly introduced standards all based on the previous ones.
On the other hand the blanking intervals allow the transmission of auxiliary data, including teletext, captions, subtitles, and in case of digital standards, \textbf{audio streams} of the media content as well.
The actual location of the digital data is codified in ITU-R BT.1364 and SMPTE 291M.
As and example, the audio streams accompanying video data are positioned in the vertical blanking intervals in the HDMI standards, i.e. between the content of active lines
\footnote{
As a simple example for audio transmission via HDMI interface:
In case of a HD format of 1080p (with the total number of lines 1125), with the frame rate of $60~\mathrm{Hz}$, the line frequency is given by $f_V = 67.5~\mathrm{kHz}$.
Assuming an audio stream with 8 channels, sampled at $f_s = 192~\mathrm{kHz}$ the data to be transmitted over one entire frame is $\frac{8 \cdot 192 000 }{60} = 25600~\mathrm{samples}$.
This means the transmission of 23 samples transmitted with one video line, which results in the maximum bandwidth of the HDMI 1.0 \href{https://www.sciencedirect.com/science/article/pii/B9780128016305000049}{standard}.}.

\vspace{3mm}
% https://www.sciencedirect.com/topics/computer-science/blanking-interval
% https://www.sciencedirect.com/topics/computer-science/horizontal-blanking
So far only the general structure of the video signals was discussed, without the investigation of the actual data in the active intervals, e.g. in case of analog video, without investigating what the actual video signals are.
Obviously, the representation of one color pixel requires the transmission of three individual video components, which in the field of video transmission is most often the luma-chroma representation.

Based on the number of actual individual transmitted components video signals can be categorized to two main formats 
\begin{itemize}
\item \textbf{Component video} transmits the video signal on three individual signal paths, i.e. on three individual cable pairs.
\item \textbf{Composite video} transmits the three video signals multiplexed into one single signal, transmitted over a single path, or physically speaking on one pair of cables.
\end{itemize}
\begin{figure}[]
	\centering
	\begin{overpic}[width = 0.90\columnwidth ]{figures/video_comp2.png}
	\end{overpic}
	\caption{Block diagram of the production of composite and component video signals}
	\label{Fig:video_components}
\end{figure}
The signal processing scheme of composite and component video is illustrated in Figure \ref{Fig:video_components}.
As it is depicted, in both cases the basis of video representation is given by the luma-chroma separation of the RGB signals\footnote{Although in case of component video the direct transmission of RGB signals is also allowed in case of UHD video.}.
In the following the actual steps of this production chain is investigated.

\vspace{3mm}
\paragraph{Bandwidth of the video signals:}
Obviously, both the luma and chroma signals are transmitted with finite bandwidth.

As has been already discussed, in case of CRT displays the time domain video signal is drawn on the screen real-time, therefore, the bandwidth of the video signal (defined in $\mathrm{Hz}$) limits the actual horizontal detailedness of the displayed on image.
Thus, the decrease of temporal bandwidth results in the reduction of the spatial, horizontal resolution.
Besides bandwidth efficiency at transmission, the actual choice of video bandwidth has a direct, practical reasons.

The screen of the CRT display is divided both horizontally and vertically to individual RGB pixels.
Therefore, even in case of an continuous CRT driving voltage, the screen itself samples the video signals, that may lead to spatial aliasing artifacts, manifesting in visible Moiré patterns on the screen.
In order to avoid aliasing the video signal has to be temporally bandlimited \textbf{at least} to the half of the spatial sampling frequency, in accordance with the sampling theorem.

The required temporal bandwidth is investigated for the case of the NTSC format:
From table \ref{tab:sd_formats} the number of active lines in the format is $N_{V,A} = 480$, while the active line time is $t_{H,A} \approx 52~\mu \mathrm{s}$.
As the aspect ratio is 4:3 the number of horizontal pixels is approximately $N_{H,A} = \frac{4}{3} N_{V,A} = 640$ (assuming square pixels).
According to the sampling theorem the sine signal with the largest spatial frequency that can be represented with this resolution contains $N_{H,A}$ periods in one line (this is the consecutive ,,black-white-black-white...'' content).
The period time and the frequency of this signal can be calculated as
\begin{equation}
T_{\mathrm{max}} = \frac{t_H}{N_{H,A}/2}, \hspace{1cm} f_{\mathrm{max}} = \frac{N_{H,A}}{2 t_H} = 6.15~\mathrm{MHz}
\end{equation}
which is the theoretical upper frequency limit, based on the Nyquist criterion.

However, experimental results showed that even this theoretical maximal frequency can not be displayed without artifacts, due to the \textbf{Kell effect}.
The Kell effect is caused by the finite size of the CRT's electron beam: instead of the theoretical infinitely narrow beam, which is assumed when applying the sampling theorem, the electron beam has a \textbf{point spread function (PSF)} (i.e. the intensity profile, when projected to a single point on the screen) described approximately by a Gaussian distribution.
\begin{figure}[]
	\centering
	\begin{minipage}[c]{0.65\textwidth}
	\begin{overpic}[width = 0.95\columnwidth ]{figures/kell.png}
	\end{overpic} \end{minipage}\hfill
	\begin{minipage}[c]{0.35\textwidth}	\caption{Illustration of the Kell effect.}
	\label{Fig:Kell}  \end{minipage}
\end{figure}

A simple example is illustrated in Figure \ref{Fig:Kell} for the effect of the beam PSF:
The continuous image to be displayed oscillates vertically at the Nyquist frequency, with alternating black and white lines.
In case the lines of the image exactly coincides with the sampled lines on the screen (i.e. they are aligned) the image can be reconstructed perfectly.
However, if the lines of the image are located between two actual scan lines, due to the non-zero extension of the electron beam PSF the CRT will display the average of the two lines, i.e. a homogeneous grey image is displayed.

As a conclusion, the theoretical maximal spatial frequency can not be displayed on the screen without artifact.
The ratio of the experimental, subjective highest displayable spatial frequency and the theoretical limit is given by the \textbf{Kell factor}, being approximately 0.7 for the NTSC video parameters.

In order to avoid spatial aliasing and by taking the Kell effect into consideration the largest horizontal frequency that can be displayed on the screen without visible artifacts is given by
\begin{equation}
f^{\mathrm{NTSC}}_{\mathrm{max}} = \frac{N_{H,A} \cdot K}{2 t_H} = 4.3~\mathrm{MHz},
\hspace{3mm}
f^{\mathrm{PAL}}_{\mathrm{max}} = \frac{N_{H,A} \cdot K}{2 t_H} \approx 5.2 ~\mathrm{MHz}
\end{equation}
in the NTSC and PAL systems respectively, where the Kell factor is $K = 0.7$.
Therfore, the luma signal (similarly to the early black and white TV signal) is bandlimited to $4.2~\mathrm{MHz}$ in the NTSC and to $5~\mathrm{MHz}$ in the PAL system.

\vspace{3mm}
As discussed earlier, the visual acuity (i.e. the spatial resolution) of the human visual system is less than the half for color information than for luminance.
This property of human vision can be exploited in order to achieve significant video compression.
For the \ycbcr representation this served as the starting point for the subsampling of chroma signals.
In the analog case this phenomena allows the bandwidth reduction of chroma signals, therefore, in component systems the chroma signals are bandlimited to the half of the luma signal (resulting in a halved horizontal color resolution), while in composite video the color information is transmitted with even more reduced bandwidth.

\subsection{Composite video signal}

Analóg átviteltechnika szempontjából a legegyszerűbb megoldás a videójel továbbítására a 3 videókomponens egyetlen érpáron való átvitele.
Ebben az esetben a luma és chroma komponensekből egyetlen ún. \textbf{kompozit} jelet kell képzeni, hogy a vevő oldalon az eredeti három komponens különválasztható.
A feladat megoldására három---alapgondolatában azonos---módszer létezik, az NTSC, PAL és SECAM megoldások.
A rendszerek pontos működésétől eltekintve a következő bekezdés az NTSC és PAL kompozitjelek képzésének alapelvét mutatja be.

A kompozit formátum az NTSC rendszer bevezetésével került kidolgozásra a létező fekete-fehér TV-vevőkkel kompatibilis analóg színes műsorszórás megvalósítására.
A feladat a már létező műsorszóró rendszerben alapsávban továbbított luma jelhez (azaz a fekete-fehér jelhez) a színinformáció olyan módú hozzáadása volt, hogy a létező monokróm vevőkben a többletinformáció minimális látható hatást okozzon, míg a színes vevő megfelelően külön tudja választani a luma és chroma jeleket.
Tehát más szóval a visszafele-kompatibilitás miatt az új színes rendszerben a luma jelet változatlanul kellett átvinni. 
Minthogy az átvitelhez használt RF spektrum jelentős részét már elfoglalták a frekvenciaosztásban küldött egyes TV csatornák (a képinformáció, és az FM modulált hanginformáció), így a luma és chroma komponensek csak ugyanabban a frekvenciasávban kerülhetnek továbbításra.

Az alapsávi fekete-fehér TV jel felépítése egyszerű a már bemutatott \ref{fig:PAL_frame} ábrán látható felépítéssel megegyező:
Egymás után, soronként tartalmazza a CRT elektron-ágyú vezérlőfeszültségének időtörténetét, amely tehát így a műsor vételével teljesen valós időben rajzolja soronként a kijelző képernyőjére az $Y'(t)$ luma jel tartalmát.
Az egyes sorok és képek kijelzése között az elektron-ágyú kikapcsolt állapotban véges idő alatt fut vissza a következő sor, illetve kép elejére. 
Egy fekete-fehér TV sor felépítése az \ref{Fig:PAL_line} ábrán látható.

%
\begin{figure}[]
	\centering
	\begin{minipage}[c]{0.65\textwidth}
	\begin{overpic}[width = 0.95\columnwidth ]{figures/PAL_line.png}
	\end{overpic} \end{minipage}\hfill
	\begin{minipage}[c]{0.35\textwidth}	\caption{Egyetlen TV sor luma jele és szinkron jelei a PAL rendszer időzítései mellett. Az NTSC esetében a TV sor felépítse jellegere teljesen azonos, a PAL-tól eltérő időzítésekkel.}
	\label{Fig:PAL_line}  \end{minipage}
\end{figure}
%

A valós idejű átvitel/kijelzés elvéből látható, hogy a színinformáció átvitele időosztásban sem lehetséges, tehát a chroma jeleket a luma jelekkel azonos frekvenciasávban és időben szükséges átvinni.
A megoldás tárgyalása előtt vizsgáljuk külön a chroma jelek továbbításának módját.

\paragraph{A színsegédvivő bevezetése:}
A színformációt hordozó két chroma jel ($Y'(t)-R'(t), Y'(t)-B'(t)$) egyidőben történő átvitele során alapvető feladat a két analóg jel egyetlen jellé való átalakítása.
Erre az kvadratúra amplitúdómoduláció ad lehetőséget, amely egy olyan modulációs eljárás, ahol az információt részben a vivőhullám amplitúdójának változtatásával, részben annak fázisváltoztatásával kódoljuk (ezzel tehát két független jel vihető át egyszerre). 
Mind PAL, mind NTSC rendszer esetében az emberi látás színekre vett alacsony felbontását kihasználva a chroma jeleket erősen (PAL esetében pl. a luma jel ötödére, $1~\mathrm{MHz}$-re) sávkorlátozzák, ezzel az apró, nagyfrekvencián reprezentált részleteket kisimítják. 
Ezután a kvadratúramodulált chroma jeleket pl. PAL esetén
\begin{equation}
c^{\mathrm{PAL}}(t) = \underbrace{U'(t)}_{\left( B'- Y'\right) / 2.03} \cdot \sin \omega_c t + \underbrace{V'(t)}_{\left( R'- Y'\right) / 1.14}  \cdot \cos \omega_c t
\label{Eq:PAL_cr}
\end{equation}
alakban állíthatjuk elő, ahol $\sin \omega_c t$ az ún. \textbf{színsegédvivő}, $\omega_c$ a színsegédvivő frekvencia, $U'(t)$ az ún. fázisban lévő, $V'(t)$ pedig a kvadratúrakomponens.
A kvadratúramodulált színjelek tehát egyszerűen az átskálázott színkülönbségi jelek fázisban és kvadratúrában lévő színsegédvívővel való modulációjával állítható elő.

A színjelek demodulációja koherens (fázishelyes) vevővel egyszerű alapsávba való lekeveréssel és aluláteresztő szűréssel valósítható meg:
\begin{align}
\begin{split}
\sin x \cdot \sin x = \frac{1-\cos 2x}{2}&,\hspace{1cm}
\cos x \cdot \cos x = \frac{1+\cos 2x}{2} \\
\sin x &\cdot \cos x = \frac{1}{2}\sin 2x
\end{split}
\end{align}
trigonometrikus azonosságok alapján $U'(t)$ demodulációja
\begin{multline}
c^{\mathrm{PAL}}_{\mathrm{QAM}}(t)\cdot \sin \omega_c t = U'(t)\cdot \sin \omega_c t\cdot \sin \omega_c t + V'(t) \cdot \cos \omega_c t  \cdot	\sin \omega_c t = \\
\frac{1}{2} U'(t) -
\underbrace{ \xcancel{ \frac{1}{2} U'(t)\cos 2 \omega_c t  + V'(t) \cdot \frac{1}{2}\sin 2 \omega_c t }}_{\text{aluláteresztő szűrés}}
\end{multline}
szerint történik, míg $V'(t)$ demodulálása hasonlóan $\sin \omega_c t$ lekeverés szerint.
A megfelelő demodulációhoz tehát a vevőben a színsegédvivő fázishelye, koherens előállítása elengedhetetlen.
\begin{figure}[]
	\centering
	\hspace{4mm}
	\begin{overpic}[width = 0.70\columnwidth ]{figures/QAM_mod_demod.png}
	\end{overpic}
	\caption{QAM moduláció és demoduláció folyamatábrája}
	\label{Fig:QAM_mod_demod}
\end{figure}

Az NTSC rendszerben a PAL-hoz hasonlóan a színjelek
\begin{equation}
c^{\mathrm{NTSC}}_{\mathrm{QAM}}(t) = I'(t) \cdot \sin \omega_c t + Q'(t) \cdot \cos \omega_c t
\end{equation}
alakban kerültek átvitelre, ahol az in-phase és kvadratúra komponensek rendre
\begin{align}
\begin{split}
I'(t) &= k_1 (R'-Y') + k_2 (B'-Y) ,\\ 
Q'(t) &= k_3 (R'-Y') + k_4 (B'-Y).
\end{split}
\end{align}
A $k_{1-4}$ konstansokat úgy választották meg, hogy az in-phase és kvadratúra modulált jelek nem közvetlenül a kék és piros merőleges bázisvektorok (ld. \ref{Fig:ycbcr_gamut} ábra), hanem ezek kb. $+20^{\circ}$ elforgatottja.
Az így kapott új tengelyek a magenta-zöld és türkiz-narancssárga tengelyek a közvetlen modulálójelek.
Ennek oka, hogy úgy találták, az emberi látás felbontása jóval nagyobb türkiz-narancssárga közti változásokra, mint a magenta-zöld között.
Ezt kihasználva a magenta-zöld $Q'(t)$ színjeleket az $I'(t)$ jelhez képest is jobban sávkorlátozták, sávszélesség-takarékosság céljából.
A PAL rendszer bevezetésének idejére azonban kiderült, hogy ez rendszer felesleges túlbonyolítása, így az új rendszerben megmaradtak az eredeti színkülönbségi jelek modulációjánál.

\vspace{3mm}
Vizsgáljuk végül a modulált színjel fizikai jelentését, az egyszerűség kedvéért $c^{\mathrm{PAL}}(t)$ esetére (PAL rendszerben)!
Az \eqref{Eq:PAL_cr} egyenlet egyszerű trigonometrikus azonosságok alapján átírható a 
\begin{equation}
c^{\mathrm{PAL}}_{\mathrm{QAM}}(t) = \sqrt{U'(t)^2 + V'(t)^2} \, \sin \left( \omega_c t + \arctan \frac{V'(t)}{U'(t)} \right)
\end{equation}
polár alakra.
Minthogy a moduláló $U',V'$ jelek a színkülönbségi jelekkel arányosak, így a fenti kifejezést \eqref{eq:saturation_1} és \eqref{eq:hue}-val összehasonlítva megállapítható, hogy a QAM modulált jel egy olyan szinuszos vivő, amelynek pillanatnyi amplitúdója a továbbított színpont telítettségét, pillanatnyi fázisa a színpont színezetét adja meg.

\begin{figure}[]
	\centering
	\hspace{4mm}
	\begin{overpic}[width = 0.50\columnwidth ]{figures/SMPTE_Color_Bars.png}
\small
\put(-7	,0){(a)}
	\end{overpic} \hfill
	\begin{overpic}[width = 0.395\columnwidth ]{figures/vectorscope.png}
\small
\put(-10,0){(b)}
	\end{overpic}
	\caption{Egy gyakran alkalmazott vizsgálókép (SMPTE color bar) (a) és vektorszkóppal ábrázolva (b).}
	\label{Fig:bar_pattern_vscope}
\end{figure}

A színsegédvivő amplitúdójának és fázisának egyszerű értelmezhetősége miatt az NTSC és PAL jeleket gyakran vizsgálták ún. vektorszkóp segítségével jól meghatározott vizsgálóábrák megjelenítése mellett.
A vektorszkóp kijelzője gyakorlatilag a \ref{Fig:ycbcr_gamut} ábrán is látható $B'-Y', B'-Y'$ térben jeleníti meg a teljes képtartalom (azaz egyszerre az összes képpont) chroma jeleit, $Y'$-tól függetlenül a demodulált chroma-jelek megjelenítésével.
A vektorszkóp gyakorlatilag egy olyan oszcilloszkóp, amelynek $x$ kitérését a demodulált $B'-Y'$, $y$-kitérést a demodulált $R'-Y'$ jel vezérli, így a teljes képtartalom színezetét szinte egyszerre jeleníti meg az előre felrajzol vizsgálati rácson.
Egy tipikus vizsgáló ábra és annak vektorszkópos képe látható a \ref{Fig:bar_pattern_vscope} ábrákon.
A vektorszkóp alkalmazásának előnye, hogy az esetleges amplitúdó és fázishibából származó telítettség és színezethibák jól láthatóvá válnak a kijelzőn az egyes felvetített pontok ''összeszűkülése/tágulása'', illetve a teljes konstelláció elfordulásaként.
Megjegyezhető, hogy a mai digitális videojeleket is gyakran ábrázolják szoftveres vektorszkópon az egyes pixelek színezetének vizsgálatához.

\paragraph{A színsegédvivő frekvencia:}
Vizsgáljuk most, hogyan választható meg a színsegédvivő $\omega_c$ vivőfrekvenciája úgy, hogy a QAM modulált $c^{\mathrm{PAL}}(t)$ jelet a luma jelhez hozzáadva a vevő oldalon lehetséges legyen a vett $c^{\mathrm{PAL}}(t) + Y'(t)$ jelből az eredeti chroma és luma jelek szétválasztása!

A jelek vevőoldali szétválasztására a luma és chroma jelek spektruma ad lehetőséget:
Láthattuk, hogy a videójel az egyes TV sorokban megjelenítendő világosság és színinformáció sorfolytonos időtörténeteként fogható fel.
Természetes képeken a képtartalom sorról sorra csak lassan változik (természetesen a képtartalomban jelenlévő vízszintes éleket leszámítva), így mind a luma, mind a chroma jelek ún. kvázi-periodikusak, azaz közel periodikusak.
Jel- és rendszerelméleti ismereteink alapján tudjuk, hogy egy periodikus jel spektruma vonalas, a jelfrekvencia egész számú többszörösein tartalmaz csak komponenseket.
Ennek megfelelően mind a luma, mind a chroma jelek spektruma közel vonalas: az energiájuk a sorfrekvencia egész számú többszörösein csomósodik.
Természetesen a luma jel az alapsávban helyezkedik el ($0~\mathrm{Hz}$ környezetében), kb. $5.6~\mathrm{MHz}$ sávszélességben\footnote{Ez a sávszélesség eredményezi az azonos horizontális és vertikális képfelbontást.}.
A QAM modulált chroma jel spektruma a sávkorlátozás miatt keskenyebb ($1~\mathrm{MHz}$), és középpontját $\omega_c$ vivőfrekvencia határozza meg.
\begin{figure}[]
	\centering
	\hspace{4mm}
	\begin{overpic}[width = 0.80\columnwidth ]{figures/LC_interlace.png}
	\end{overpic} \hfill
	\caption{A luma és chroma jelek spektrális közbeszövésének alapelve a teljes spektrumokat ábrázolva (a) és a spektrális csomókat felnagyítva (b)}
	\label{Fig:YC_interlace}
\end{figure}

A luma-chroma jel összegzése ennek ismeretében egyszerű: 
Az $\omega_c$ vivőfrekvencia megfelelő megválasztásával elérhető, hogy a chroma jel spektrumvonalai (spektrumcsomói) éppen a luma jel spektrumvonalai közé essen, azaz a spektrumukat átlapolódás nélkül közbeszőhetjük.
Az eljárás alapötletét \ref{Fig:YC_interlace} ábra illusztrálja $f_{\mathrm{H}}$-val a sorfrekvenciát jelölve.
A szétválaszthatóság feltétele ekkor 
\begin{equation}
f_c = f_{\mathrm{H}} \cdot \left( \mathrm{n} + \frac{1}{2}\right), \hspace{1.5cm} \mathrm{n} \in \mathcal{N} 
\end{equation}
azaz a színsegédvivő frekvenciája a sorfrekvencia felének egész szűmú többszörösének kell, hogy legyen \footnote{Megjegyezhető, hogy PAL esetében az előre adott sorfrekvenciához egyszerű volt a színsegédvivő-frekvencia megválasztása, míg NTSC esetén bizonyos okok miatt a sorfrekvencia és ebből következően a képfrekvencia megváltoztatására volt szükség. 
Innen származnak a ma is használatos $59.94$ és $29.97~\mathrm{Hz}$ képfrekvenciák, amelyeket a következő fejezet tárgyal részletesen.}.

\paragraph{A CVBS kompozit videójel és luma-chroma szétválasztás:}
Ennek ismeretében végül a teljes kompozitjel a 
\begin{equation}
\text{CVBS}(t) = \mathrm{Sy}(t) + Y'(t) + c_{\mathrm{QAM}}(t)
\end{equation}
alakban áll elő, ahol $Y'$ a luma jel, $c_{\mathrm{QAM}}$ a QAM modulált chroma jelek és $\mathrm{S\!y}(t)$ a kioltási időben jelen lévő sorszinkron és képszinkron jelek.
A CVBS elnevezés gyakori szinoníma a kompozit videójelre, jelentése C: color, V: video (luma), B: blanking (azaz kioltás) és S: sync (azaz szinkronjelek).

Az így létrehozott videójel a fekete-fehér képhez képest csak a modulált színsegédvivőt tartalmazza többletinformációnak.
Egyszerű fekete-fehér vevőn a CVBS jelet megjelenítve a színinformáció nagyfrekvenciás zajként, pontozódásként (ún. \href{http://www.techmind.org/colrec/}{chroma dots}) jelenik csak meg a kijelzőn, így a visszafelé kompatibilitás biztosítva volt.
Színes vevőkben a CVBS jelből a luma és chromajel elméletileg fésűszűréssel szeparálható a sorfrekvencia felének egész számú többszöröseit elnyomva.
Ez ideálisan egy soridejű késleltetést igényel \footnote{A bizonyításhoz vizsgáljuk $h(t) = \delta(t) + \delta(t-t_{\mathrm{H}})$ szűrő átviteli karakterisztikáját, amely szűrő a jelből kivonja $t_H$-val késleltetett önmagát!}.
A fésűszűrős luma-chroma szeparáció lehetősége már az NTSC fejlesztésének idején ismert volt, azonban a szükséges soridejű késleltető nem állt rendelkezésre, ezért a korai NTSC vevők egyszerű alul/felüláteresztő szűrőkkel, vagy egyszerű chroma jelre állított lyukszűrőkkel szeparálták a luma-chroma jeleket.
Ennek eredményeképp még a színes vevőkben is a chroma jelen kisfrekvenciás tartalomként jelen lehetett a világosságinformáció látható \href{https://en.wikipedia.org/wiki/Dot_crawl}{hatással a megjelenített képre}.
A megfelelő analóg PAL fésűszűrő-tervezés még a 90-es években is aktív \href{https://www.renesas.com/in/en/www/doc/application-note/an9644.pdf}{K+F} alatt álló terület volt.

\begin{figure}[]
	\centering
	\begin{overpic}[width = 0.45\columnwidth ]{figures/ntsc_color_line.png}
	\end{overpic} \hfill
	\begin{overpic}[width = 0.48\columnwidth ]{figures/Waveform_monitor.jpg}
	\end{overpic} \hfill
	\caption{Az SMPTE color bar vizsgáló ábrának egy, illetve két sorának hullámformája sematikusan (a), és egy hulláforma monitoron (b) vizsgálva}
	\label{Fig:NTSC_line}
\end{figure}

Az elmondottak alapján az NTSC rendszerben a \ref{Fig:bar_pattern_vscope} ábrán látható vizsgálóábrának egy sorának kompozit ábrázolását a \ref{Fig:NTSC_line} mutatja be jellegre helyesen, és egy konkrét hullámforma monitoron mérve.
Az ábrán megfigyelhető az egyes oszlopokhoz tartozó hullámalak: látható, hogy a csökkenő világosságú oszlopokra (amelyek világosságát szaggatott vonal jelzi) hogyan ültették rá a QAM modulált chroma jeleket.
Az első és utolsó fehér, illetve fekete oszlop esetén a chroma jelek amplitúdója zérus (fehérpont), egyéb esetekben a szinuszos színsegédvivő amplitúdója az oszlopok színének telítettségével, fázisa a színezetükkel arányos.
Megjegyezzük, hogy a tényleges hullámforma már átskálázott chroma jeleket ábrázol, amely átskálázás épp azért történik, hogy a teljes CVBS jel beleférjen a fizikai interface dinamikatartományába (ez természetesen a nagy telítettségű színek esetén okozna problémát).
Ez magyarázza tehát az eddig figyelmen kívül hagyott 2.03 és 1.14 skálafaktorokat pl. \eqref{Eq:PAL_cr} esetében.


Az NTSC jel felépítése alapján egyértelmű, hogy a megfelelő színek helyreállításához a vevőben a színsegédvivő fázisának nagyon pontos ismerete szükséges.
Ahhoz, hogy ez biztosítva legyen a sorkioltási időben az ún. hátsó vállra (ld. \ref{Fig:PAL_line} ábra) beültetésre került néhány periódusnyi (9) képtartalom nélküli referenciavivő, az ún. color burst, vagy burst jel.
Ez a burst jel megfigyelhető a \ref{Fig:NTSC_line} ábrán is.

Ennek ellenére az NTSC rendszer továbbra is fázisérzékeny volt, hiszen fázishibát a vevőben is bármelyik alkatrész okozhatott.
A QAM moduláció jelege miatt már a legkisebb fázishiba is látható színezetváltozást okozott a megjelenített képen.
A PAL rendszer tervezésének egyik fő célja épp ezért a rendszer fázishibára vett érzékenységének csökkentése volt

\paragraph{A PAL rendszer:}
Míg az egyszerű NTSC rendszer már 1953-ban bevezetésre került Amerikában, addig Európában egészen az 1960-as éveikg vártak a színes műsorszórás bevezetésére.
Ennek oka, hogy az eltérő hálózati frekvencia miatt az NTSC-t nem lehetett egy az egyben átemelni Európába (ld. később).
Mire az európai rendszert kifejlesztették, az NTSC rendszer jó néhány gyengeségére fény derült, így az újonnan kifejlesztett PAL (Phase Alternate Lines) ezek kijavítását célozta főként meg.
Ennek eredményeképp a PAL rendszer más QAM modulációval dolgozik (a chroma jelek közvetlenül a modulálójelek), eltérő a színsegédvivő frekvencia, és legfontosabb újításként: egy egyszerű megoldással szinte érzéketlen a fázishibára.
\begin{figure}[]
	\centering
	\begin{overpic}[width = 0.45\columnwidth ]{figures/PAL1.png}
	\end{overpic} \hfill
	\begin{overpic}[width = 0.45\columnwidth ]{figures/PAL2.png}
	\end{overpic} \hfill
	\caption{Az SMPTE color bar vizsgáló ábrának egy, illetve két sorának hullámformája sematikusan (a), és egy hulláforma monitoron (b) vizsgálva}
	\label{Fig:PAL1}
\end{figure}

Láthattuk, hogy a vevő oldalán bármilyen fázishiba a színezet jól látható torzulását okozza.
Mivel a fázishiba gyakran elkerülhetetlen, ezért hatásának kiküszöbölésére a PAL rendszer a következő egyszerű megoldást alkalmazza:
\begin{itemize}
\item Az adó oldalon (a PAL jel létrehozása során) képezzük QAM moduláció során a V' chromajel előjelét minden második TV-sorban negáljuk meg, azaz sorról sorra fordított előjellel vigyük át (ez ekvivalens a sorról sorra változó $\pm \cos \omega_c t$ vivővel való modulációval)!
Az eljárás szemléltetésére tegyük fel, hogy két egymás utána sorban minden horizontális pozícióban a színinformáció azonos.
Ekkor egy adott pontra az n. és (n+1). sorban átvitt $U',V'$ jeleket a \ref{Fig:PAL1} (a) ábra szemlélteti pl egy lila képpont átvitele esetén.
\item Tegyük fel, hogy a vevő oldalon a vett jelhez $\Delta \alpha$ fázishiba adódik az átvitel és demoduláció során.
Természetesen a fázishiba hatására az így vett színvektor mind az n., mind az (n+1). sorban azonos irányba fordul az $U'-V'$ konstellációs diagramon (azaz a $R'-Y', B'-Y'$ síkon), ahogy az a \ref{Fig:PAL1} (b) ábrán látható.
\item A vevő oldalán forgassuk vissza minden második sorban a vett $V'$ komponens előjelét és képezzük az (n+1). sor és az n. sor átlagát.
Ezzel természetesen a színjelek vertikális felbontását csökkentjük (az átlagképzés az apró részleteket elsimítja), azonban ennek eredménye az emberi szem színezetre vett felbontása eredményeképp az információveszteség nem látható (a horizontális felbontás már egyébként is jelentősen lecsökkent az egyszerű sávkorlátozás hatására).
Könnyen belátható, hogy a két vektor átlagát képezve éppen az eredeti, hibamentes színvektort kapjuk eredményül.
Két sor esetén azonos sortartalom esetén tehát ezzel az egyszerű trükkel a fázishiba hatása teljesen kiküszöbölhető, míg levezethető, hogy változó sortartalom esetén a fázishiba az átlagvektor hosszának csökkenését okozza, tehát színezetváltozás helyett csak telítettségváltozást okoz.
\end{itemize}
A bemutatott módosított modulációs módszerrel még aránylag nagy fázishibák hatása is minimális hatással van a megjelenített képre.
Az ok, hogy mégis több, mint egy évtizedet kellett várni a PAL rendszer bevezetésére az volt, hogy a módszer alkalmazásához (az átlagolás elvégzéséhez) a videójel soridejű késleltetésére volt szükség.
Ez az 50-es években analóg módon nem megoldható probléma volt amely a PAL implementálását hátráltatta.

A PAL bevezetését végül az olcsón tömeggyártható ún akusztikus művonalak megjelenése tette lehetővé.
Ez az akusztikus művonal, vagy \href{https://www.google.com/search?q=PAL+delay+line&client=firefox-b-d&sxsrf=ALeKk03EUTzVwc7dkYJFnEK-nlEI_p3hng:1586379019108&source=lnms&tbm=isch&sa=X&ved=2ahUKEwi90Kav2tnoAhXJ-ioKHWz6AJcQ_AUoAXoECA0QAw&biw=1407&bih=675}{PAL delay line} egy egyszerű üvegtömb, amelyre egy piezo aktuátor és piezo vevő csatlakozik.
Az adó a TV chroma jelével arányos mechanikai rezgéseket (ultrahang) \href{https://www.youtube.com/watch?v=-qerYLM-eEg}{hoz létre}, amely többszörös visszaverődések után épp egy soridőnyi késleltetést szenvedve ér a vevő elektródához.
Az ultrahang alapú késleltetővonalak egészen a 90-es évek végéig a PAL dekóderek részét képezték.


\begin{figure}[]
	\centering
	\begin{overpic}[width = 0.82\columnwidth ]{figures/PAL_coder.png}
	\end{overpic} \hfill
	\caption{A PAL kódoló felépítése}
	\label{Fig:PAL_coder}
\end{figure}
Az egyszerű PAL kódoló felépítése az eddig elmondottak alapján a \ref{Fig:PAL_coder} ábrán látható.
Röviden összefoglalva, mind a PAL, mind NTSC esetén a kompozitjel létrehozása során a feladat a Gamma-torzított $R',G',B'$ jelekből az $Y',U',V$ (PAL) és $Y',I',Q'$ (NTSC) jelek létrehozása, majd az $U',V'$ és $I'Q'$ jelek megfelelő QAM modulációja. 
Az így létrehozott jeleket összeadva és a kioltási időben továbbított szinkronjelekkel ellátva előáll a CVBS kompozit jel.

\vspace{3mm}
A kompozit videójel fizikai interface megvalósítása szabványról szabványra változó.
Konzumer felhasználás (pl. kézikamerák, videólejátszók, DVD lejátszók) szempontjából a legelterjedtebb csatlakozó a sárga jelölésű RCA végződés, amely az esetleges kísérő hangtól szigetelve, külön érpáron továbbítja a kompozit videójelet.
\begin{figure}[]
	\centering
	\begin{minipage}[c]{0.6\textwidth}
	\begin{overpic}[width = 0.45\columnwidth ]{figures/Composite-video-cable.jpg}
	\end{overpic} 
		\begin{overpic}[width = 0.45\columnwidth ]{figures/s_video.jpg}
	\end{overpic} \end{minipage}\hfill
	\begin{minipage}[c]{0.4\textwidth}
	\caption{Konzumer alkalmazásokhoz használt sárga jelölésű RCA csatlakozó (a) és a luma-QAM chroma jeleket külön érpáron átvivő S-videó csatlakozó (b)}
	\label{Fig:composite_video}  \end{minipage}
\end{figure}

A kompozit és komponens jelek közti kompromisszumként az S-video formátum a luma és chroma jeleket külön érpáron viszi át.
Ezt leszámítva az interface jele teljesen a kompozit videóval azonosak, továbbíthat akár NTSC, akár PAL (akár SECAM) videókomponenseket:
A luma tehát változatlanul alapsávban, míg a chroma a színsegédvivővel modulálva kerül átvitelre.
A chroma jelek modulációja elkerülhetetlen, hiszen a két független színkülönbségi jel egy érpárra való ültetéséhez azokat legalább a sávszélességükkel megegyező frekvenciájú vivőjellel való moduláció szükséges az átlapolódás elkerüléséhez.
Az S-video szabvány csatlakozója a \ref{Fig:composite_video} (b) ábrán látható.


\subsection{Component video signal}

The idea of transmitting the video signal by separated components is straightforward, still, it was allowed by technology only decades after composite video was introduced:
While interfaces for device-to-device video transmission could be achieved even with analog data, the broadcasting of component video could be only resolved with the introduction of digital video standards.
In case of component video the transmitted signals are directly the luma and the chroma signals (or less often the $R'G'B'$ signals), or more precisely the \ypbpr components.

The \ypbpr representation consists of the three following signals:
\begin{itemize}
\item $Y'$: the luma component, calculated by the luma coefficients defined by the RGB primaries.
The required sync pulses are added to the luma component.
Therefore, a device with composite video output can be connected to the $Y'$ interface of a display with component input.
As a result the black and white image is displayed with the modulated chroma signal resulting in high frequency noise (,,chroma dotting'') on the screen.
\item $P'_{\mathrm{B}},P'_{\mathrm{R}}$: the $B'-Y', R'-Y'$ chroma components, rescaled to the actual physical interface, usually requiring the dynamic range of $\pm 0.5~\mathrm{V}$.
Notation $P$ is the legacy of the composite systems, where color information is carried by the \textbf{P}hase of the subcarrier.
\end{itemize}
\begin{figure}[]
	\centering
	\begin{overpic}[width = 0.45\columnwidth ]{figures/1280px-Component-cables.jpg}
	\end{overpic} \hfill
	\begin{overpic}[width = 0.45\columnwidth ]{figures/YPBPR_signals.png}
	\end{overpic} \hfill
	\caption{RCA connectors, applied frequently for the physical interface of component video (a) and the component video signals of the bar test pattern (b)}
	\label{Fig:comp_video}
\end{figure}
Similarly to composite video, the actual parameters of the component formats depend on the region of application: 
components formats can be considered as the transmission of NTSC and PAL video formats without the QAM modulation of the chroma signals.
Therefore, both 625 and 525 scan line interlaced component formats exist with $60~\mathrm{Hz}$ and $50~\mathrm{Hz}$ field rate.
A simple example is depicted for the separately transmitted \ypbpr signals in case of the SMPTE color bar test pattern is depicted in Figure \ref{Fig:comp_video} (b).

In consumer electronic the component video interface is equipped with RCA connectors, with the red and blue cables carrying the red and blue chroma signals and the green connector transmitting the luma information.

\hspace{3mm}
Finally, from the numerous analog video interfaces two most widespread are mentioned here:
\begin{itemize}
\item The SCART connector (or EuroSCART, shorthand for Syndicat des Constructeurs d'Appareils Radiorécepteurs et Téléviseurs) was designed for the transmission of bidirectional composite, S-video and RGB components in the same time, with also carrying stereo audio and digital signing signals in the same time.
Before the advent of the HDMI interface SCART connectors also allowed the transmission of high definition 1080p format via \ypbpr component signals.
The typical 21-pin SCART connector is depicted in Figure \ref{Fig:scart_vga} (a).
\item The VGA (Video Graphics Array) connector is still a commonly used analog component interface between video cards and displays (external monitors, projectors, etc.).
The VGA interface transmits analog RGB components (in the color space of the video card) along with dedicate vertical and horizontal sync (VSYNC and HSYNC) cables.
Nowadays the VGA interface is superseded with DVI, HDMI and DisplayPort digital interfaces.
\end{itemize}

\begin{figure}[]
	\centering
	\begin{minipage}[c]{0.63\textwidth}
	\begin{overpic}[width = 0.47\columnwidth ]{figures/scart.jpg}
	\end{overpic} \hfill
		\begin{overpic}[width = 0.4\columnwidth ]{figures/vga.jpg}
	\end{overpic} \end{minipage}\hfill
	\begin{minipage}[c]{0.35\textwidth}
	\caption{The analog composite/component SCART (a) and RGB component VGA (a) connectors}
	\label{Fig:scart_vga}  \end{minipage}
\end{figure}

\section{Digital video formats}
	
\subsection{The SD format}

The first digital video format was developed by the ITU (International Telecommunication Union) and published in Rec. 601 (or ITU-601) in 1982\footnote{The CCIR (the predecessor of ITU) received the 1982–83 Technology and Engineering Emmy Award for its development}.

The SD format is basically the digital representation of the component analog formats, discussed in the foregoing:
The structure of digital video is obtained by the sampling of the analog video signal, as depicted in Figure \ref{fig:PAL_frame}, containing all the blanking intervals besides the active video content.
Therefore, digital video is the direct digitized version of the \ypbpr signals.
As already discussed in the previous chapter, the components of the video format are the digital representation of the color pixels, referred to as the \ycbcr components.
The signal processing scheme of digital video representation is illustrated in Figure \ref{Fig:SD_production}.
\begin{figure}[]
	\centering
	\begin{overpic}[width = 0.8 \columnwidth ]{Figures/YCbCr_production.png}
	\end{overpic}
	\caption{Production of SD video format by the digitization of \ypbpr components}
	\label{Fig:SD_production}
\end{figure}

The digitization of the analog video signal consists of two main steps:
\begin{itemize}
\item sampling of the continuous video signal with a-priori antialiasing filtering
\item quantization of the continuous signal levels to discrete codes
\end{itemize}
The questions of quantization, i.e. digital representation of the individual pixels has been already discussed in the previous chapter.
Still, the parameters of sampling---or more specifically the sampling frequency---needs to be specified, defining the number of pixels per line, which, along with the number of lines gives the spatial resolution of the digital format.

\paragraph*{Sampling frequency of the video signal:\\}
The sampling frequency of the analog video components was chosen based on the following considerations:
\begin{itemize}
\item For several decades analog formats varied between from region to region due to the co-existence of NTSC, PAL and SECAM systems.
As a straightforward endeavour, a basic goal was to find a common sampling frequency, compatible with all the existing analog formats.
Furthermore, orthogonal sampling was an obvious need, meaning that in each system a single scan line should contain an integer number of samples (pixels).
Mathematically these requirements can be formulated, as the line period time has to be dividable by the sampling period, and equivalently
the sampling frequency has to be the multiple of the line frequency both in the PAL and the NTSC system, satisfying
\begin{equation}
f_s = n \cdot f_H^{\mathrm{PAL}} = m \cdot f_H^{\mathrm{NTSC}},
\end{equation}
%
where $n, m$ are integers.
The line frequencies in the two systems are given by
\begin{align}
f_H^{\mathrm{PAL}} &= 25 \cdot 625 = 15625~\mathrm{Hz} \\
f_H^{\mathrm{NTSC}} &= 30 \cdot \frac{1000}{1001} \cdot 525 = 15734.2~\mathrm{Hz} ,
\end{align}
with the smallest common multiple being
\begin{equation}
144 \cdot f_H^{\mathrm{PAL}} = 143 \cdot f_H^{\mathrm{NTSC}} = 2.25~\mathrm{MHz}.
\end{equation}
Therefore, the sampling frequency must be chosen as the multiple of $2.25~\mathrm{MHz}$.
\item On the other hand, according to the sampling theory, the sampling frequency has to be at least twice the bandwidth of the sampled signal in order to avoid aliasing artifacts.
In the foregoing it was discussed that---by taking also the Kell effect into consideration---the bandwidth of the luma signal is around $5.6~\mathrm{MHz}$, with the chroma bandwidth being the half of it.
\end{itemize}
The smallest frequency satisfying both requirements is given by 
\begin{equation}
f^{\mathrm{Y,SD}}_s = 13.5~\mathrm{MHz},
\end{equation}
being chosen as the sampling frequency of the luma signal of standard definition video.
As the chroma signals are bandlimited to the half of luma signal due to the lower visual acuity of the HVS, therefore, the chroma signals are sampled with halved sampling frequency ($f^{\mathrm{Ch,SD}}_s = 6.75~\mathrm{MHz}$).

The number of pixels in a scan line---given by the line period time divided by the sampling interval---is $N_{H,t} = \frac{T_H}{1/f_s} = 858$ pixels in the NTSC and $864$ in the PAL system, including the horizontal blanking intervals as well.
%
\begin{figure}[]
	\centering
	\begin{overpic}[width = 0.65 \columnwidth ]{Figures/SD_formats.png}
	\end{overpic}
	\caption{The SD video format in the American and European systems}
	\label{Fig:SD_format}
\end{figure}
For the sake of further unification of the standard definition format the number of active pixels in a line was chosen to be 720 pixels in both the American and European systems.
Note that the active line period is $52~\mu \mathrm{s}$ in both systems, from which the number of active pixels is $N_{H,a} = T_{\mathrm{H,a}} \cdot f_s = 702$.
Hence, the 720 pixels contains a small interval from the blanking time as well, due to the uncertainty in the position of the actual analog video signal, distortions and smears near the edges, originating from analog processing.
The actual digital video data is taken as 704 pixels from the center of the nominal 720 pixels.
Obviously, the number of the vertical total and active pixels are simply given by the number of total and active scan lines.

The spatial resolution/pixel number of the SD formats is illustrated in \ref{Fig:SD_format}.
The two formats are named after the number of the active lines, resulting the the two SD formats being \textbf{480i} and \textbf{576i}, with ,,i'' referring to the fact that for both formats only interlaced scan is defined.

As it has been already discussed, the aspect ratio of the reproduced SD images should have the aspect ratio of 4:3, which is termed as the \textbf{display aspect ratio (DAR)}.
However, the ratio of the number of the horizontal and vertical pixel numbers---termed as the \textbf{storage aspect ratio (SAR)} does not coincide this ratio in neither formats.
The target display aspect ratio, therefore, can be achieved only by applying non-square pixels on the display side, described by the \textbf{pixel aspect ratio (PAR)}, defined as the ratio of the horizontal and vertical pixel size.
Therefore, format 480i has a pixel aspect ratio of 10:11, and 576i has the PAR of 12:11 with the DAR of 4:3 for both cases.
In contrary, computer displays are employing squared pixels with the PAR of 1:1, and the standard VGA resolution being 640x480 pixels.

\begin{table}[h!]
\caption{Spatial and temporal parameters of SD video formats}
\renewcommand*{\arraystretch}{2}
\label{tab:SD_formats}
\begin{center}
    \begin{tabular}[h!]{ @{}c | | l | l | l @{} }%\toprule
				         &   480i	   &    576i \\ \hline
    active pixel number: &  704 x 480  &   704 x 576   \\
    total pixel number:  &  858 x 525  &  864 x 625 \\
    display aspect ratio:&  4:3 &  4:3 \\
    pixel aspect ratio   &  10:11  & 12:11  \\
    field rate:          &  $59.94~\mathrm{Hz}$ &   $50~\mathrm{Hz}$ \\
    frame rate:          &   $29.97~\mathrm{Hz}$ &  $25~\mathrm{Hz}$ \\
    \end{tabular}
\end{center}
\end{table}

As a short summary of the foregoing the properties of the SD formats are the following:
\begin{itemize}
\item The chromaticity of the primaries and the gamut of the device RGB color space is illustrated in Figure \ref{Fig:SD_gamut} (a).
The white point of the colors space is D65 white.
The luma is calculated from the nonlinear RGB components as 
\footnote{
It is noted here that the above coefficients---unmathematically---do not coincide with the second row of the $XYZ \rightarrow RGB$ transformation matrix (which can be calculated from the primaries and the white point).
This means that strictly speaking the $Y$ component---calculated from the RGB values with the above coefficients---does not equal to the relative luminance.
Instead, for the sake of simplicity the luma coefficients of the NTSC system were used.
These mathematical inconsistencies are usual in videotechnologies.}
\begin{equation}
Y' = 0.299 R' + 0.587 G' + 0.112 B'.
\end{equation}
\item In order to achieve perceptual quantization the source RGB signals are pre-distorted with the opto-electronic transfer function, given by
%
\begin{figure}[]
	\centering
	\begin{overpic}[width = 0.4 \columnwidth ]{Figures/sd_gamut.png}
	\small
	\put(0,0){(a)}
	\end{overpic}
	\hspace{2mm}
	\begin{overpic}[width = 0.55 \columnwidth ]{Figures/sd_OETF.png}
	\small
	\put(0,0){(b)}
	\end{overpic}
	\caption{Gamut (a) and OETF (b) of the ITU-601 SD format}
	\label{Fig:SD_gamut}
\end{figure}
\begin{equation}
E = 
\begin{cases}
4.500 L, \hspace{20mm} \mathrm{ha}\, L < 0.018 \\
1.099 L^{0.45} - 0.099, \hspace{3mm} \mathrm{ha}\, L \geq 0.018,
\end{cases}
\end{equation}
where $L \in \{ R, G, B \}$.
The entire curve can be approximated by $L^{0.5}$.
\item The original SD display aspect ratio was 4:3.
After the introduction of the HD format also the SD format was extended with the aspect ratio of 16:9.
\item SD format exclusively supports interlaced scanning.
\item ITU-601 originally only allowed the chroma subsampling scheme 4:2:2.
Later it was extended with the scheme 4:2:0 for consumer applications.
\item The SD format applies the sampling frequency of $f_s = 13.5~\mathrm{MHz}$. 
The format strictly prescribes the antialiasing low-pass filters, bandlimiting the luma signal to $5.75~\mathrm{MHz}$ as the upper frequency limit and $6.75~\mathrm{MHz}$ as the lower limit of the stopband.
The chroma is sampled with halved sampling frequency, and prefiltered with corresponding antialiasing filters.
\item The sampled luma and chroma sampled are quantized with 8 (for consumer electronics and broadcasting applications) or 10 bits (in studio applications) of bit depths.
\end{itemize}

\subsection{The HD format}

The starting point for the introduction of the analog video formats, leading directly to SD digital format was to cover approximately about $10^{\circ}$ of the central vision from the viewer's field of view.
Obviously, this requirement does not directly define the display size, or the spatial resolution.
Instead at a given resolution and display size the optimal viewing distance can be derived.
In the following, first this optimal viewing distance is investigated, highlighting the basic motivation behind the introduction of the HD and UHD formats.

\subsubsection*{The optimal viewing distance}

\begin{figure}[]
	\centering
	\begin{overpic}[width = 0.67 \columnwidth ]{Figures/hd_pixel_angle_mod.png}
	\small
	\end{overpic}
	\caption{Geometry for deriving the optimal viewing distance}
	\label{Fig:optimal_vd}
\end{figure}

Generally speaking the goal of pixel based image reproduction is to ensure that the light rays, arriving from adjacent pixels include an angle below the spatial resolution of the human vision.
Once it is achieved it is inherently ensured that the pixel structure of the image is not visible, and allows RGB based color reproduction, since instead of perceiving the individual RGB primaries, only the mixture of the primaries is perceived.
In the foregoing it has been discussed that the visual acuity of human vision is $\frac{1}{60}^{\circ}$ (at least for luminance, for colors the resolution is significantly lower), from which for a given pixel size the minimal viewing distance can be derived.
In practice instead of the pixel size, the dimensions and the horizontal and vertical resolution of displays are given, hence it is useful to express the minimal viewing distance as the function of these quantities.

In the following, the geometry depicted in Figure \ref{Fig:optimal_vd} is investigated in case of a display with given height $H$ and number of active lines $N_V$, with the display located at a distance $D$ from the viewer.
The vertical pixel size is then $\frac{H}{N_V}$.

The perceived angle between the adjacent pixels ate the observer position is then given as 
\begin{equation}
\tan \frac{\Phi}{2} = \frac{H}{2 N_V D}.
\end{equation}
For the sake of simplicity the small argument approximation of the tangent function can be applied, i.e. $\tan x \approx x$, ha $x \ll 1$, leading to
\begin{equation}
\Phi = \frac{H}{N_V D} \hspace{3mm} \rightarrow \hspace{3mm} D = \frac{H}{N_V \Phi}.
\end{equation}
By substituting the visual activity of the HVS ($\frac{1}{60}\cdot \frac{\pi}{180}~\mathrm{rad} = 2.9 \cdot 10^{-4}$) for a display with given vertical resolution the optimal (minimal) viewing distance is given by
\begin{equation}
D = H \frac{1}{N_V \,  2.9 \cdot 10^{-4}}.
\end{equation}
This is the so-called \textbf{Lechner distance}, formulating the optimal viewing distance, taken into consideration during the deigning of a display with given size and resolution.

\begin{table}[h!]
\caption{Optimal viewing distance of common SD and HD formats and the covered observer field of view}
\renewcommand*{\arraystretch}{2.25}
\label{tab:viewing_dist}
\begin{center}
    \begin{tabular}[h!]{ @{}c | | l | l | l @{} }%\toprule
				         &   SD, 480i & SD, 576i &	 HDTV \\ \hline
    Active lines:	 &     480 	  		   &   576   				&	 1080\\
    Viewing distance:   &  7 x height &  6 x height & 3 x height\\
    Viewing distance:       &  4.25 x diameter &  3.6 x diameter & 1.5 x diameter\\
    Horizontal field of view &  $\approx 11^{\circ}$ &    $\approx 13^{\circ}$ & $\approx 32^{\circ}$ \\
    \end{tabular}
\end{center}
\end{table}

In case that also the display aspect ratio (denoted by $a_r$) is given (for SD 4:3 and for HD 16:9) the Lechner distance can be expressed as the function of the horizontal dimension (for the sake of simplicity assuming square pixels):
\begin{equation}
D = \frac{W}{a_r} \frac{1}{N_V \,2.9 \cdot 10^{-4}},
\end{equation}
where $W$ is the horizontal dimension (width) of the display.
Furthermore, if the display is observed from the optimal viewing distance, the horizontal field of view angle can be expressed, included by the display:
\begin{equation}
\tan \frac{\Phi_H}{2} = \frac{W}{2 D} \hspace{1cm} \rightarrow \hspace{1cm} D = \frac{W}{2 \tan \frac{\Phi_H}{2}}, 
\end{equation}
and the field of view is written as
\begin{equation}
\Phi_H = 2\arctan \left( \frac{a_r \, N_V \, 2.9\cdot 10^{-4}}{2} \right).
\end{equation}

The evaluation of these results for the discussed SD formats and the upcoming HD formats is summarized in Table \ref{tab:viewing_dist}.
As a result it is obtained that displays with SD resolution should be observed from a distance 6-7 times the height of the screen.
In this case, it is verified that the display covers about 10-13 degrees from the observer's field of view horizontally.

The table already foreshadows the main motivation behind the introduction of HD video format: increasing the perceived reality of the reproduced scene, by filling an increased field of view with video content, compared to SDTV. Therefore, the basic goal was not to squeeze six times the number of pixels into the same visual angle opposed to the popular misbelief.
Instead the angular subtense of a single pixel should be maintained, and the entire image shoud occupy a much larger area of the viewer's visual field.

\subsubsection*{Short HD history}

Although today the term high definition format is associated with digital video, even in the era of early analog TV system increased resolution, high definition initiations existed.
Years before the introduction of color television, in 1949 France started analog monochromatic transmission with a high resolution standard at 819 lines (with 737 active lines), a system that should have been high definition even by today's standards, discontinued only in 1983. 
In 1958, the Soviet Union developed the Transfomator, the first high-resolution television system capable of producing an image composed of 1125 lines of resolution aimed at providing teleconferencing for military command. It was a research project and the system was never deployed by either the military or consumer broadcasting.

In 1979, the Japanese public broadcaster NHK (Japan Broadcasting Corporation) developed consumer high definition television with 1125 scan lines and a 5:3 display aspect ratio.
The system, known as Hi-Vision or MUSE required about twice the bandwidth of the existing NTSC system but provided about four times the resolution, applied for analog regular HD broadcasting beginning in 1994.
The limited standardization of analog HDTV in the 1990 did not lead to global HDTV adoption as technical and economic constraints did not permit HDTV to use bandwidths greater than normal television.

The introduction of the HD format was finally motivated by the emerging of digital compression methods (mainly of MPEG-1 and MPEG-2).
The HD format was codified in the \href{https://www.itu.int/dms_pubrec/itu-r/rec/bt/R-REC-BT.709-6-201506-I!!PDF-E.pdf}{ITU-709} (Rec. 709) standard, published in 1990.

\subsubsection*{HD parameters}

The ITU-709 recommendation prescribed the following properties for HD video format:
\begin{itemize}
\item \textbf{Aspect ratio:} The first standardized parameter for the new video format was its aspect ration, being chosen to 16:9.
The choice is not obvious, since at the time of the standardization process no video content was captured with this aspect ratio: 
TV movies were shot with the aspect ratio of SD format (4:3), while movie films used widescreen formats (most commonly the 1.85:1 and 2.2:1 widescreen, or 2.35:1 anamorphic format).
It has been found that rectangles with the side ratios of the above popular aspect ratios and with equal areas exactly fit within an outer rectangle with the aspect ratio of 16:9.
Furthermore, the intersection of all these rectangles is an inner rectangle with the aspect ratio of 16:9.
Therefore, the aspect ratio of 16:9 ensured the compatibility with most of the then-existing formats.
The geometry is shown in Figure \ref{Fig:kerns_powers}.
%
\begin{figure}[]
	\centering
	\begin{overpic}[width = 0.9 \columnwidth ]{Figures/KernsPowers.png}
	\small
	\end{overpic}
	\caption{The 16:9 aspect ratio as the outer and inner rectangle of rectangles with common aspect ratios}
	\label{Fig:kerns_powers}
\end{figure}
%
\item \textbf{Raster scan, field and frame rate:}
In the HD system besides interlaced scanning the possibility of progressive scan was introduced.
As the legacy of the SD system, numerous field rate and field rates are specified by the format, being $24~\mathrm{Hz}$, $25~\mathrm{Hz}$, $30~\mathrm{Hz}$, $50~\mathrm{Hz}$ and $60~\mathrm{Hz}$ field/frame rates, along with fractional rates, having the above values multiplied by $\frac{1000}{1001}$.


\item \textbf{Sampling frequency:}
The HD sampling frequency can be derived from the SD sampling rate:
One of the main goals of the HD format was to double both the horizontal ($\times 2$) and vertical ($\times 2$) spatial resolution, along with the aspect ratio changed to 16:9 ($4 : 3 \times 4/3$), which the total pixel number being at least $2\times 2 \times \frac{4}{3} = 5.33$ times that of the SD format.
In order to ensure orthogonal sampling (only full pixels in a line) for most field and frame rates the optimal choice for the sampling frequency was $5.5 \cdot 13.5 = 74.25~\mathrm{MHz}$ and the fractional sampling frequency $\frac{1000}{1001} \cdot 74.25~\mathrm{MHz}$ for fractional field/frame rates.
In case of progressive scanning for frame rates $50-60~\mathrm{Hz}$ the data rate is double compared to the interlaced case, therefore, a doubled sampling frequency of $fs = 148.5~\mathrm{MHz}$ is required.
%TODO : kibogarászni KovácsI jegyzetéből a mintavételi frek. megválasztását (VidStTech 01.pdf, 46.o)

\item \textbf{Spatial resolution:} 
After a lengthy debate \footnote{The first version of the standard specified 1035 active lines, which was later withdrawn.} the number of active line has been codified to 1080 and the total number of lines to 1125, including the vertical blanking with both progressive and interlaced scan modes (this raster format was also codified in SMPTE 274M).
The horizontal number of active pixels has been chosen to a fixed number of 1920 regardless of the frame/field rate, resulting in the active resolution of $1920 \times 1080$.
Therefore, the HD formats with different field and frame rates only differ in the number of total horizontal number of pixels (i.e. in the length of the horizontal blanking time).
%
\begin{table}[h!]
\caption{Sampling frequency, total number of lines and columns and active resolution of commonly used HD formats.
The total number of pixels in a line can be calculated according to $N_\mathrm{H} = f_s \times \frac{f_{\mathrm{V}}}{N_\mathrm{V}}$, with $f_{\mathrm{V}}$ being the frame rate.}
\renewcommand*{\arraystretch}{2.25}
\label{tab:viewing_dist}
\begin{center}
    \begin{tabular}[h!]{ @{}c | l | l | l | l | l @{} }%\toprule
		Format       &   $f_s \, [\mathrm{MHz}]$ 				& $N_\mathrm{H}$ & $N_\mathrm{V}$ & Active resolution \\ \hline
		$720p50$     &   $74.25~\mathrm{MHz}$    				& 1980     &  750  & $1280 \times 720$ \\
		$720p59.94$  &$74.25\cdot\frac{1000}{1001}~\mathrm{MHz}$& 1650     &  750  & $1280 \times 720$ \\
		$1080i25$ 	 &   $74.25~\mathrm{MHz}$    				& 2640     & 1125  & $1920 \times 1080$ \\
		$1080i30$  	 &   $74.25~\mathrm{MHz}$    				& 2200     & 1125  & $1920 \times 1080$ \\
		$1080p50$ 	 &   $148.5~\mathrm{MHz}$    		    	& 2640     & 1125  & $1920 \times 1080$ \\
		$1080p59.94$ &$148.5\cdot\frac{1000}{1001}~\mathrm{MHz}$& 2200     & 1125  & $1920 \times 1080$ 
        \end{tabular}
\end{center}
\end{table}
\vspace{3mm}
The different HD formats are denoted using the following convention:
\begin{itemize}
\item Active resolution in pixels, denoting often only the vertical resolution
\item Scanning mode: $p$ for progressive and $i$ for interlaced
\item frame rate.
For interlaced video often instead of frame frate---incorrectly---field rate is marked.
\end{itemize}
As an example: $1080i25$ denotes interlaced video with 1080 active lines, and the frame rate of $25~\mathrm{Hz}$ (i.e. field rate is $50~\mathrm{Hz}$).
\begin{figure}[]
	\centering
	\begin{overpic}[width = 1 \columnwidth ]{Figures/HD_format.png}
	\small
	\end{overpic}
	\caption{Az 1080 soros és 720 soros HD formátum szemléltetése}
	\label{Fig:HD_formats}
\end{figure}

Due to the large data rate of progressive HD formats the ITU-709 has been extended with a lower spatial resolution format, containing 720 active lines and 1280 active pixels in a line, with exclusively progressive scanning mode.
\footnote{
As already discussed, the basic goal of 1080 lined format was to double the number of scan lines.
The line number of the 720p format---as an intermediate format between SD and HD---applies $\frac{3}{2} \cdot 480 = 720$ scan lines, with the horizontal number of pixels obtained from the aspect ratio of 16:9}.
It is denoted by: \textbf{720p}.
Two examples for the two basic HD formats can be seen in \ref{Fig:HD_formats}.
\item The primaries of the HD format coincide with that of the ITU-601 SD format, resulting in the same gamut.
The luma (and the relative luminance) is calculated as
\begin{equation}
Y' = 0.2126 R' + 0.7152 G' + 0.0722B'.
\end{equation}
\begin{figure}[]
	\centering
	\begin{overpic}[width = 1\columnwidth ]{Figures/hd_line.png}
	\end{overpic} 
	\caption{Structure of one line of the 1080 lined HD format.}
	\label{Fig:hd_line}
\end{figure}
Opposed to the SD formats, in this case the luma coefficients are mathematically correct, obtained from the relative luminance coefficients of the chosen primaries and its white point, i.e. the coefficients correspond to the second row of the $RGB \hspace{3mm} \rightarrow \hspace{3mm} XYZ$ transformation matrix.
\item The opto-electronic transfer function coincides with the SD's OETF:
\begin{equation}
E = 
\begin{cases}
4.500 L, \hspace{20mm} \mathrm{ha}\, L < 0.018 \\
1.099 L^{0.45} - 0.099, \hspace{3mm} \mathrm{ha}\, L \geq 0.018,
\end{cases}
\end{equation}
where $L \in \{ R, G, B \}$.
\item The bit depth is 8 bits for consumer and 10 bits for studio applications.
\item The studio standard codifies the chroma subsampling scheme of 4:2:2.
\end{itemize}
The structure of the resulting standardized HD video signal is the same as the SD video signal, discussed in the foregoing, illustrated in Figure \ref{Fig:hd_line}.
The sole difference is the application of tri-state sync pulses, with the physical dynamic range usually being $0, \pm300~\mathrm{mV}$.

\subsubsection*{Uncompressed data rate of HD formats}

In the following the uncompressed data rate of different video formats is discussed.
With denoting the active and total pixel dimensions the notations showed in Figure \ref{Fig:size} the total data rate can be calculated as
\begin{equation}
B\!R_T = \underbrace{N_{\mathrm{H}} \cdot N_{\mathrm{V}} \cdot f_{\mathrm{V}} \cdot n_{\mathrm{bit}}}_{\text{bitrate per sample}} \cdot n_{\mathrm{CS}},
\end{equation}
with denoting $f_{V}$ the frame rate, $n_{\mathrm{bit}}$ the bit depth and $n_{\mathrm{CS}}$ the number of components, depending on the chroma subsampling scheme applied.
The value of the latter is $n_{\mathrm{CS}} = 3$ for 4:4:4, $n_{\mathrm{CS}} = 2$ for 4:2:2 and $n_{\mathrm{CS}} = 1.5$ for 4:2:0 schemes.

Similarly the active data rate is calculated as
\begin{equation}
B\!R_A = N_{\mathrm{H,A}} \cdot N_{\mathrm{V,A}} \cdot f_f \cdot n_{\mathrm{bit}} \cdot n_{\mathrm{CS}}
\end{equation}.

The active and total bit rates of several frequently used video formats are summarized in Table \ref{tab:bitrate}\footnote{
The vertical blanking interval of UHD format is \href{http://programmersought.com/article/7908103552/}{fixed} to 90 lines, with the horizontal blanking depending on the frame rate of the video.
For a video with the frame rate of $60~\mathrm{Hz}$ the horizontal blanking interval is 560 samples.
}.
\begin{figure}[]
\captionsetup{singlelinecheck=off}
	\centering
	\begin{minipage}[c]{0.4\textwidth}
	\begin{overpic}[width = 1\columnwidth ]{Figures/size.png}
	\end{overpic}   \end{minipage}\hfill
		\begin{minipage}[c]{0.55\textwidth}
	\caption[]{Notation-convention for calculating the data rate of video
	\begin{itemize}
	\item $N_{\mathrm{H}}$: Total pixel number/line
	\item $N_{\mathrm{H,A}}$: Active pixel number/line
	\item $N_\mathrm{V}$: Total number of lines/frame
	\item $N_\mathrm{V,A}$: Number of active lines/frame
	\end{itemize}}
	\label{Fig:size}  \end{minipage}
\end{figure}
%
\begin{table}[h!]
\caption{The total and active bitrate of frequently used video formats}
\renewcommand*{\arraystretch}{2.25}
\label{tab:bitratet}
\begin{center}
    \begin{tabular}[h!]{ @{}c | l | l | l | l | l @{} }%\toprule
\thead{Format} & \thead{Sampling \\ frequency} &    \thead{Total bitrate \\ 4:2:2} & \thead{Active bitrate\\ 4:2:2} & \thead{Active  bitrate\\ 4:4:4} \\ \hline
$576p50$     &   $13.5~\mathrm{MHz}$        & $0.54~\mathrm{Gbit/s}$  & $0.41~\mathrm{Gbit/s}$ & $0.62~\mathrm{Gbit/s}$ \\
$720p60$     &   $74.25~\mathrm{MHz}$    	& $1.49~\mathrm{Gbit/s}$  & $1.11~\mathrm{Gbit/s}$ & $1.67~\mathrm{Gbit/s}$ \\
$1080i30$ 	 &   $74.25~\mathrm{MHz}$    	& $1.49~\mathrm{Gbit/s}$  & $1.24~\mathrm{Gbit/s}$ & $1.86~\mathrm{Gbit/s}$ \\
$1080p60$ 	 &   $148.5~\mathrm{MHz}$    	& $2.97~\mathrm{Gbit/s}$  & $2.49~\mathrm{Gbit/s}$ & $3.73~\mathrm{Gbit/s}$ \\
$2160p60$ 	 &   $297~\mathrm{MHz}$         &  $11.88~\mathrm{Gbit/s}$& $9.96~\mathrm{Gbit/s}$ & $14.93~\mathrm{Gbit/s}$
\end{tabular}
\end{center}
\end{table}
%
The uncompressed bit rate approximately equals for $720p$ and $1080i$ formats.
An advantage of the progressive format is it can be more efficiently compressed than interlaced video.
On the other hand interlaced format ensures high vertical resolution for still and slowly moving video content, although with the resolution degrading for moving reproduced objects.
Therefore, broadcasting operators with sport content in their main profile traditionally chose $720p50/60$ format, while operators, broadcasting mainly news and movies usually apply $1080i$ video format.

%TODO 8b/10b codin
% Fischer Digital Video and Audio broadcasting technology 227.o

%TODO HD ready
%TODO Gaumt coverage percentage, Surface colors
%TODO In typical production practice the encoding function of image sources is adjusted so that the final picture has the desired aesthetic look, as viewed on a reference monitor with a gamma of 2.4 (per ITU-R BT.1886) in a dim reference viewing environment (per ITU-R BT.2035).[10][11][12]
%TODO Rec. 2100, a standard for HDTV and UHDTV with high dynamic range

\subsection{The UHD format}

The original intention behind the introduction of HD format was to enhance the visual experience by increasing the listener's visual angle filled with video content.
The ultra high definition format aims at the further improvement of the reproduction quality by applying even larger display sizes---covering an even larger part of the field of view---and by increasing the displayed image quality.

Similarly to the HD format, the first notable initiation of UHD technology is connected to the Japanese NHK, capturing video data at the resolution of $7680 \times 4096$ by using an array of 16 HDTV recorders and four CCD sensors with the resolution $3840 \times 2048$, as early as 2003.
The first UHD standard was published in 2007 (SMPTE 2036), while the currently accepted UHD codification was introduced in 2012, entitled the \textbf{ITU-R BT. 2020}.
The standard specifies two UHD formats, the 4k with its spatial resolution being the double of the 1080p format both horizontally and vertically (meaning 4 times larger total pixel size), and the 8k, with a further  doubled resolution, compared to 4k.

\begin{figure}[]
	\centering
	\begin{overpic}[width = 0.85 \columnwidth ]{Figures/fov_formats.png}
	\small
	\end{overpic}
	\caption{Visual angle ensured by SD, HD and UHD formats, with the screen being watched from the optimal viewing distance.}
	\label{Fig:fov_formats}
\end{figure}

Theoretically, the goal of the 4k and 8k formats was to fill the visual angle of approximately $58^{\circ}$ and $96^{\circ}$ with content, respectively.
The visual angle of the different SD, HD and UHD formats are compared in Figure \ref{Fig:fov_formats}.

In order to ensure high quality video reproduction over a large visual angle the ITU-2020 improved the reproduction parameters in numerous aspect compared to HD:
\begin{itemize}
\item \textbf{Spatial resolution:} 
The standard specifies two resolution types: $3840 \times 2160$ termed as 4k and $7680 \times 4320$ as 8k resolutions.
Both 4k and 8k employs squared pixels, i.e. both the display and storage aspect ratios are 16:9.
%
\item \textbf{Scanning type:}
Unlike HD, ITU-2020 exclusively allows progressive scanning mode (i.e. the $p/i$ designation is no longer used).

\item \textbf{Frame rate:}
With the increased visual angle the UHD content already fills the peripheral vision of the observer with content.
Since the peripheral vision is dominated by rods with a much quicker response time than the cones of the central vision, thus, in order to ensure continuous motion and avoid flickering significantly higher frame rates are supported than the HD frame rates.
The standard allows the frame rates of $120, 119.88, 100, 60, 59.94, 50, 30, 29.97, 25, 24, 23.976~\mathrm{Hz}$.
\begin{figure}[]
	\centering
	\begin{overpic}[width = 0.75 \columnwidth ]{Figures/uhd_gamut.png}
	\small
	\end{overpic}
	\caption{Gamut of the ITU-2020 UHD color space compared with the gamut of SD and HD color representations.}
	\label{Fig:UHD_gamut}
\end{figure}

\item \textbf{Color space:} 
For the first time since the introduction of NTSC the ITU-2020 UHD standard applies new RGB primaries for color representation for the sake of an enlarged gamut.
The resulting gamut is shown in Figure \ref{Fig:UHD_gamut}: 
As the figure verifies it, the UHD color space applies spectral colors for the RGB primaries, described by the wavelengths of $\lambda_R = 630~\mathrm{nm}$, $\lambda_G =532~\mathrm{nm}$ and $\lambda_B =467~\mathrm{nm}$.
\footnote{
Obviously, this does not mean that UHD displays use spectral colors as RGB primaries, instead the color pixels of UHD video content are stored and transmitted in terms of spectral primary colors.
Monitors and TVs display colors by using the actual applied LCD or LED primary colors, first converting the ITU-2020 content into the RGB color space of the display.
The gamut of these color spaces are obviously smaller than the gamut of the UHD standard: as an example, JDI introduced a \href{https://www.displaydaily.com/?view=article&id=62235:jdi-may-have-commercial-problems-but-has-technical-highlights}{studio reference monitor} with the diameter of $14.3''$ in 2018, allowing the reproduction of $97\%$ of the ITU-2020 color space.}.
The ITU-2020 covers the $75.8\% of$ the CIE chromaticity diagram and the entire Pointer's gamut of real surface colors
\footnote{
The Pointer's gamut is the result of the series of measurements, containing the chromaticites (hues) of colors of reflective surfaces, occurring in the natural environment---opposed to the colors that can be produced by emissive surfaces, e.g. neon or colors of LED/LCD light sources.
The \href{https://cinepedia.com/picture/color-gamut/}{Pointer's gamut} is based on 4089 measurement samples, with the database published in 1980.
Since then, the Pointer's gamut is a de facto standard of qualifying \href{https://www.tftcentral.co.uk/articles/pointers_gamut.htm}{color spaces}.}.
The relative luminance (and the luma) can be calculated from the RGB coordinates as 
\begin{equation}
Y = 0.2627 \, R + 0.6780 \, G + 0.0593 \, B.
\label{eq:2020_Y}
\end{equation}
\item \textbf{Bit depth:} 
The increase of the range of the described colors requires increasing the bit depth of representation as well, in order to ensure the same quantization precision.
Since a larger color space increases the difference between colors an increase of 1-bit per sample is needed for Rec. 2020 to equal or exceed the color precision of ITU-709.
Thus, Rec. 2020 defines a bit depth of either 10 bits per sample for consumer or 12 bits per sample for studio applications.
\item \textbf{The opto-electronic transfer function:} 
The OETF of the ITU-2020 coincides with that of the SD and HD standards:
\begin{equation}
E = 
\begin{cases}
4.500 L, \hspace{20mm} \mathrm{ha}\, L < \beta \\
\alpha L^{0.45} - (\alpha - 1 ), \hspace{3mm} \mathrm{ha}\, L \geq \beta,
\end{cases}
\end{equation}
where $\alpha = 1.09929682680944$ and $\beta = 0.018053968510807$.
The only difference is the precision of the coefficients: for 12 bits bit depth the above coefficients should be evaluated with 5 digits precision.
%
\item \textbf{Chroma subsampling:} 
The standard specifies the subsampling schemes of 4:2:0, 4:2:2 and 4:4:4.
In the latter case instead of luma-chroma representation direct $R'G'B'$ representation is allows.
In case of 4:2:0 or 4:2:2 sampling, besides \ycbcr representation the so-called \textbf{contant luminance mode} is supported, resulting in the $Y C_{\mathrm{bc}} C_{\mathrm{rc}}$ components.
This constant luminance mode may be used when the top priority is the most accurate retention of luminance information.
The luma component in $Y C_{\mathrm{bc}} C_{\mathrm{rc}}$ is calculated using the same coefficient values as for \ycbcr, but it is calculated from linear RGB and then gamma corrected, rather than being calculated from gamma-corrected $R'G'B'$.
As a result $Y'$ in constant luminance mode is mathematically accurately describes the gamma-corrected relative luminance component (therefore, the resulting luma contains no color information and the resulting chroma contains no luminance information).
The prevents artifacts that arise in case of \ycbcr representation due to the subsampling of minor luminance information, present in the chroma components.
\end{itemize}

%TODO \paragraph{Konstans fénysűrűségű mód:\\}
\subsection{Digital interfaces}
The most common physical interface for SD, HD and UHD video transmission in video studiotechnologies is the SDI (Serial Digital Interface), and HDMI (High-Definition Multimedia Interface) in consumer applications.

The bit rates of most common video formats were summarized in Table \ref{tab:bitratet}.
As a comparison, the maximal digital bandwidth of the different HDMI interface versions are the following
\begin{itemize}
\item HDMI 1.0-1.2: 4.95 Gbit/s (3.96 Gbit/s effective)\footnote{
Due to undiscussed reasons the HDMI interface applies a so-called 8b/10b channel coding, transmitting 8 bits of data in 10 bit sequences.
Therefore, only the part of $\frac{8}{10}$ of the total bandwith can used for effective data transmission.}
\item HDMI 2.0: 18 Gbit/s (14.4 Gbit/s effective)
\item HDMI 2.1: 48 Gbit/s (38.4 Gbit/s effective)
\end{itemize}
Obviously, in order to find the HDMI version for the transmission of a given video format, the total number of lines and sample per lines have to be taken into consideration, since HDMI signals contains the horizontal and vertical blanking intervals as well.
As discussed earlier: the vertical blanking interval carries multichannel audio streams and other auxiliary data, presented simultaneously with the video data.

It can be concluded that HDMI 1.0 version was created mainly for the transmission of 1080p video format with 4:2:2 chroma subsampling scheme, and the bit depth of 10 or 12 bits (4:4:4 video can be only transmitted with this version represented on 8 bits).
On the other hand, HDMI 2.0 was developed for 4k, and the 2.1 version for 8k video transmission.

\begin{figure}[]
	\centering
	\begin{overpic}[width = 0.75 \columnwidth ]{Figures/optimal-viewing-distance-television-graph-size.png}
	\small
	\end{overpic}
	\caption{Optimal viewing distance as the function of display diameter.}
	\label{Fig:optimal_vd_2}
\end{figure}

\subsection{Optimal choice of display resolution}

As the conclusion of the present chapter the Lechner distance is revisited in order to arrive at the the optimal viewing distance for a given display size and for the optimal display size for a given observer distance.
Besides the Lechner distance, several other recommendations exist for the optimum viewing distance, including manufacturers, retail and THX recommendations.
\begin{itemize}
\item SMPTE 30: is the standardized codification of the Lechner distance, recommending the viewing distance of 1.6 times the diameter in case of a HD display with 1080 lines, resulting in a field of view of $30^{\circ}$.
This recommendation is very popular with the home theater enthusiast community, appearing in books on home theater design,
%
\item The recommendation of manufacturers, retails and several publications suggests the viewing distance of 2.5 times the diameter in case of a HD display, resulting in the field of view of $20^{\circ}$.
%
\item THX recommends that the ,,best seat-to-screen distance'' is one where the view angle approximates $40^{\circ}$, which according to the THX approximates the cinema experience the most.
This is achieved in case of HD format with the viewing distance being 1.2 times the display diameter.
\end{itemize}
Although there are slight discrepancies between these recommendations, they all agree in that for the viewing distance of the HD displays ,,the closer the better''.

The graph of optimal viewing distances can be visualized for the different video formats, as depicted in Figure \ref{Fig:optimal_vd_2}.
At a given display size, of course the viewing distance can be increased, the pixel structure of the display does not become visible.
Thus, at a fixed display size above a given line in the graph the display is applicable.
The graph, therefore, can be divided into regions, indicating the optimal display resolution for a given viewing distance and display size.

\vspace{3mm}
Based on statistics, conducted by Bernard J. Lechner the average domestic TV viewing distance is approximately $2.7~\mathrm{m}$.
At this distance displays above the diameter of 50'' should have the resolution of 1080p, while in order to exploit the advantages and quality of 4k resolution the display should have a diameter of at least 75'' ($\sim 1.9~\mathrm{m}$).
Displays with the diameter of 2 meters are extremely rarely applied in domestic use even today, suggesting that the capabilities of even 4k displays are not completely utilized nowadays.
However, the introduction of consumer 8k displays is already trending, with also experimental 8k broadcasting initiations beginning in the recent years.
As an example, the first dedicated satellite for broadcasting 8k content has been lunched (BSAT-4a), which was planned to broadcast the 2020 summer olympics, which has been however postponed to 2021 due to the outbreak of the COVID-19 pandemic.

%Periférikus látás:
%https://www.quora.com/What-is-the-aspect-ratio-of-human-vision

%TODO \subsection{A HDR kiterjesztés}

\vspace{2cm}
\noindent\rule{12cm}{0.4pt}

\subsection*{End-of-Chapter Questions}

\begin{itemize}
\item What were the reasons behind the introduction of the interlaced format?
What is the main idea of interlaced scanning?
%
\item How was the sampling frequency of the SD format chosen?
How was it extended in order to choose the sampling frequency of the HD format?
%
\item Calculate the optimal viewing distance for a 4k (2160p) display with the diameter of 65'' (the aspect ratio is 16:9)!
%
\item List some improvements of the UHD standard compared to the HD format!
%
\item Define the number of total and active resolution of the $2160p60$ format!
The number of inactive lines is 90 and the sampling frequency is $297~\mathrm{MHz}$.
%
\item Define the total bitrate of the video format of the previous example in case of a chroma subsampling scheme of 4:2:2, with 12 bits per sample representation!
Pick the minimal HDMI version which is capable of transmitting the exemplary video stream, if the HDMI transmits 8 bit data in 10 bits sequences (i.e. the effective bandwidth is $\frac{8}{10}$ times the total bandwidth), and the total bandwidth of the HDMI interface versions are
\begin{itemize}
\item HDMI 1.0-1.2: $4.95~\mathrm{Gbit/s}$
\item HDMI 1.3-1.4: $10.2~\mathrm{Gbit/s}$
\item HDMI 2.0-1.2: $18~\mathrm{Gbit/s}$
\item HDMI 2.1: $48~\mathrm{Gbit/s}$
\end{itemize}
\end{itemize} 

\chapter{Basics of image and video compression}
\label{sec:compression}
The previous chapter introduced the basic properties of consumer and professional, studio video parameters.

The active spatial resolution and the resulting bit rates of frequently used digital video formats are summarized in Table  \ref{tab:bitrate_2}.
%
\begin{table}[h!]
\caption{The active bitrate of frequently used video formats along with the size, required for storing 1 hour of video stream}
\renewcommand*{\arraystretch}{2}
\label{tab:bitrate_2}
\begin{center}
    \begin{tabular}[h!]{ @{}c | l | l | l | l | l @{} }%\toprule
\thead{Format} & \thead{Active\\ resolution} & \thead{Active bitrate\\4:2:2} & \thead{Active bitrate\\4:2:0}& \thead{Size of 1 hour\\ video} \\ \hline
SIF ($i59.54$)    & $352\times 240$ &   $40.6~\mathrm{Mbit/s}$  & $30.4~\mathrm{Mbit/s}$  & $13.7~\mathrm{Gbyte}$ \\
CIF ($i59.54$)   &  $352 \times 288$ & $48.6~\mathrm{Mbit/s}$   & $36.5~\mathrm{Mbit/s}$  & $16.4~\mathrm{Gbyte}$ \\
$576i50$    &  $576\times 720$  &   $199~\mathrm{Mbit/s}$       & $149.1~\mathrm{Mbit/s}$ & $67.1~\mathrm{Gbyte}$ \\
$720p60$   &  $1280\times 720$   &   $883~\mathrm{Mbit/s}$    	& $662.8~\mathrm{Mbit/s}$  & $298.3~\mathrm{Gbyte}$ \\
$1080i30$ 	&  $1920\times 1080$  &   $994~\mathrm{Mbit/s}$    	& $745.8~\mathrm{Mbit/s}$  & $335.6~\mathrm{Gbyte}$ \\
$1080p60$ 	 &  $1920\times 1080$ &   $1.99~\mathrm{Gbit/s}$    & $1.49~\mathrm{Gbit/s}$  & $671.2~\mathrm{Gbyte}$ \\
$2160p60$ (10 bits)	&  $3840\times 2160$  &   $9.95~\mathrm{Gbit/s}$   &  $7.47~\mathrm{Gbit/s}$& $3.36~\mathrm{Tbyte}$ \\
$4320p60$ (10 bits)	&  $7680 \times 4320$  &   $39.8~\mathrm{Gbit/s}$   &  $29.9~\mathrm{Gbit/s}$& $13.44~\mathrm{Tbyte}$ 
\end{tabular}
\end{center}
\end{table}
%

In the table SIF and CIF abbreviate Source Input Format and Common Intermediate Format respectively.
Both formats were introduced for the consumer digital representation of NTSC and PAL videos---with CIF being the default video format of the H.261 encoder and SIF being that for the MPEG-1 standard--- with a halved vertical resolution when compared to the professional ITU-601 studio standard.
	
As the table verifies it, the generated data rate of video formats---and thus the required storage space---grows exponentially with higher spatial and temporal resolution.
Modern studio and consumer interfaces---variants of the SDI interface for studio applications and HDMI or DisplayPort for consumer use---allow the transmission of the data rates of uncompressed video over short ranges, e.g. between local devices.
However, the storage and broadcasting of such high data rates is virtually impossible:
the compression of digital video data is indispensable.

\vspace{3mm}
Fortunately, real-life sequence of images contain significant amount of redundant information:
Statistically speaking within single frames the neighboring pixels are usually highly correlated.
Similarly, consequent frames are usually very similar to each other, even if they contain objects under motion.
In video signals, the redundancy can be classified as spatial, temporal, coding and psychovisual redundancies:
\begin{itemize}
\item Spatial redundancy (or intraframe/interpixel redundancy) is present in areas of images or video frames where pixel values vary only by small amounts.
\item Temporal redundancy (or interframe redundancy) is present in video signals when there is significant similarity between successive video frames.
\item Coding Redundancy is present if the symbols produced by the video encoder are inefficiently mapped to a binary bitstream. Typically, entropy coding techniques can be used in order to exploit the statistics of the output video data where some symbols occur with greater probability than others.
\item Psychovisual redundancy is present either in a video signal or a still image containing perceptually unimportant information:
The eye and the brain do not respond to all visual information with same sensitivity, some information is neglected during the processing by the brain. 
Elimination of this information does not affect the interpretation of the image by the brain and may lead to a significant compression.
Psychovisual redundancy is usually removed by appropriate requantization of the video data, so that the quantization noise remains under the threshold of visibility.
\end{itemize}
In order to achieve a high compression ratio, all the above redundancy types should be eliminated, being the basic goal of a 
\textbf{source encoder}.

%\begin{figure}[]
%	\centering
%	\begin{overpic}[width = 0.8\columnwidth ]{figures_en/digit_channel_modell.png}
%	\end{overpic}
%	\caption{Az azonos alapszínekkel dolgozó SD formátum, HD formátum és az sRGB színtér gamutja $xy$ és $uv$ diagramon ábrázolva.}
%	\label{Fig:digit_channel}
%\end{figure}
%Figure \ref{Fig:digit_channel} illustrates the general model of the digital video transmission channel.
%

\begin{figure}[]
	\centering
	\begin{overpic}[width = 0.8\columnwidth ]{figures_en/source_encoder.png}
	\end{overpic}
	\caption{Block scheme of a general video/audio source encoder.}
	\label{Fig:source_encoder}
\end{figure}
Generally speaking, the aim of source encoding is reducing the source redundancy by keeping only the relevant information, based on the properties of the source and the sink.
The source in this case is the video (or possibly audio) sequence, and the sink is the human visual system (or the auditory system for audio info).
The general structure of a source encoder, valid both for video or audio inputs is depicted in Figure \ref{Fig:source_encoder}.
The reduction of the different types of redundancy is performed by the following steps:
\begin{itemize}
\item Change of representation: in order to reduce spatial and temporal the input data is represented in a new data space containing less redundancy.
The change of representation can be performed by 
	\begin{itemize}
	\item Differential coding (DPCM: Differential Pulse Code Modulation)
	\item Transformation coding
	\item Sub-band coding
	\end{itemize}
\item Irreversible coding: the accuracy of representation is reduced by removing irrelevant information, hence, eliminating psychovisual redundancy.
Irreversible coding is achieved by 
	\begin{itemize}
	\item requantization of the data
	\item spatial and temporal subsampling
	\end{itemize}
\item Reversible coding: an efficient code-assignment is established reducing statistical redundancy.
Types of reversible entropy coding applied often in video, image and audio processing are
	\begin{itemize}
	\item Variable Length Coding (VLC)
	\item Run-Length Coding (RLC)
	\end{itemize}
\end{itemize}

In the following this chapter introduces the basic concepts of compression methods, based on differential coding and transformation coding.
The basic concepts are introduced for the generalized case of arbitrary one and two dimensional input signals, and later specialized to video signal inputs.

\section{Differential quantization}

Differential quantization is a compression technique, utilizing linear prediction along with the requantization of the predicted data (i.e. performing both a change of representation and irreversible coding):
instead of the direct quantization and transmission of the input signal, the actual input sample is predicted with an appropriately chosen prediction algorithm, and only the discrepancy between the actual and the estimated sample is further processed.
In the receiver the same prediction is performed as in the source side, and the output sample is obtained as the sum of the estimated signal and the error of estimation. 

The signal processing steps in a differential encoder and decoder are shown in Figure \ref{Fig:diff_quant} with the following notation:
\begin{itemize}
\item $\xi(n)$ is the input source sample
\item $\hat{\xi}(n)$ is the predicted input sample
\item $\delta(n)$ is the error of prediction/differential signal
\item $Q$ is the quantization of the signal
\item $Q^{-1}$ is the inverse quantization
\item $\delta'(n)$ is the quantized differential signal
\item $\xi'(n)$ is the quantized, reconstructed input sample
\end{itemize}
In the block diagram quantization is performed by rescaling the input signal to match the dynamic range of the quantizer, followed by the rounding of the signal level to the nearest integer.
Inverse quantizer, on the other hand scales back the quantized signal to the original dynamic range (obviously, information loss can not be reversed).

\begin{figure}[]
	\centering
	\begin{overpic}[width = 0.6\columnwidth ]{figures_en/diff_quant.png}
	\end{overpic}
	\caption{Block scheme of a general differential encoder and decoder.}
	\label{Fig:diff_quant}
\end{figure}

The basic idea behind differential quantization is the following:
Assuming an efficient prediction the dynamic range of the differential signal is significantly smaller than that of the original input signal.
Therefore, discretizing the error signal means the division of a smaller dynamic range to the same number of intervals ($2^N$ in case of $N$ bits representation) than in case of quantizing the input signal directly, resulting in an increased resolution, or mathematically speaking, in an increased signal-to-noise ratio.
Alternatively, the same signal-to-noise ratio may be achieved by using lower bit depths utilizing differential quantization.

\vspace{3mm}
In order to give a mathematical description on differential quantization and quantify the introduced quantities, first a brief summary of stochastic processes is given.

\subsection{Basic stochastic concepts}

A stochastic process is any process describing the evolution in time or space of a random phenomenon, given by an indexed sequence of random samples.
Each sample is a random variable with a given probability distribution, and with the probability usually depending on the previous samples.
For the sake of simplicity it is implied here that the process evolves over time, but all the following can be easily extended for e.g. spatially dependent processes.

Let $\xi$ denote a stochastic process, and the sample index denoted by $n$, hence for each index $\xi(n)$ is a random variable.
A stochastic process is fully described by its joint distribution function,  which is, however, rarely available either by measurement or analytically.
Instead, more often stochastic processes are characterized in a simplified manner by their \textbf{moments} (being the \textbf{mean value} its first and the \textbf{variance} its second moment) and the \textbf{autocorrelation function}.

\paragraph*{Wide-sense stationary processes:}
In the following only \textbf{stationary processes} are investigated, that's statistical properties do not change over time.
Strict stationary requires the entire joint distribution function of the process to be time invariant.
In most applications it is sufficient to require the process to be \textbf{wide-sense stationary (WSS)}, defined by the following properties:
\begin{itemize}
\item The mean/expected value of a WSS process is constant, invariant of $n$:
\begin{equation}
m_\xi(n) = m_\xi
\end{equation}
Once the above relation holds, the expected values of the process can be approximated as the average of a realization of length $N$ according to
\begin{equation}
m_\xi = \EX(\xi(n) ) = \frac{1}{N} \sum_{n = 1}^{N} \xi(n)
\end{equation}
\item For a general process the autocorrelation function can be defined for two distinct samples, i.e. it is a two-dimensional function 
\begin{equation}
r_\xi(n_1,n_2) = \EX( \xi(n_1) \cdot \tilde{\xi}(n_2) ),
\end{equation}
loosely speaking measuring the linear dependence between samples $\xi(n_1)$ and $\xi(n_2)$.
If two samples are uncorrelated---i.e. $r_\xi(n_1,n_2)=0$---it implies that no linear relation exists between them, however, higher order dependence may be present.
Therefore, uncorrelatedness does not imply independence (while independence strictly ensures uncorrelatednes).

For a WSS process this linear dependence is translation invariant
\begin{equation}
r_\xi(n_1,n_2) = r_\xi(n_1+ d , n_2 + d), \hspace{1cm} \forall d \in \mathcal{N}
\end{equation}
therefore autocorrelation depends only on the distance of the two samples (denoted now by $d$)
\begin{equation}
r_\xi(n_1 - n_2) = r_\xi(d).
\end{equation}
If the above relation holds, autocorrelation can be statistically approximated from a realization of the process as
\begin{equation}
r_\xi(d) = \EX( \xi(n) \cdot \tilde{\xi}(n + d) ) = \frac{1}{N} \sum_{n = 0}^N \xi(n)\tilde{\xi}(n+ d)
\end{equation}
\item As a further property for WSS process the auto-correlation function at zero lag ($d=0$) gives the mean value of the squared samples, i.e. the mean energy of the process, being obviously also time invariant
\begin{equation}
r_\xi(0) = E_\xi =  \EX( \xi(n)^2 ) = \frac{1}{N} \sum_{n = 0}^N \xi(n)^2.
\end{equation}
\end{itemize}

\paragraph*{Noise processes:}
As the most simple stochastic example, an uncorrelated random process is considered, meaning that linear relation exists between neighboring samples.
For such a process the autocorrelation is zero valued everywhere, except for zero lag ($d=0$), where the autocorrelation value is the energy of the random process.
The autocorrelation, therefore, is a Kronecker delta (discrete Dirac delta) function at the origin, given by
\begin{equation}
r_\xi(n) = E_\xi  \cdot \delta(n) = \begin{cases} 0, \hspace{7mm} \text{if} \hspace{2mm} n = 0 \\ E_\xi , \hspace{7mm} \text{elsewhere.} \end{cases} 
\end{equation}
Such a stochastic process is called \textbf{white noise}.
The distribution of the individual samples is arbitrary, most often the samples are drawn from uniform or Gaussian normal distribution.

\begin{figure}[]
	\centering
	\begin{overpic}[width = 0.85\columnwidth ]{figures_en/white_noise.png}
	\small
	\put(0,0){(a)}
	\end{overpic}
	\begin{overpic}[width = 0.85\columnwidth ]{figures_en/white_noise2.png}
	\small
	\put(0,0){(b)}
	\end{overpic}
	\caption{One dimensional and two dimensional white noise process (a) and correlated noise process (b).}
	\label{Fig:noise}
\end{figure}
The terminology originates from the \textbf{power spectral density}, defined as the Fourier transform of the autocorrelation function , describing the frequency content of the stochastic process.	
For white noise the spectral density function is constant, similarly to the spectrum of white light containing all lights with all the visible wavelengths equally.
A simple example realization of white noise process is depicted in Figure \ref{Fig:noise} (a) in one and two dimensions.

A correlated process can be most easily generated from white noise by linear filtering (e.g. FIR filtering):
since after filtering each output sample is produced as the linear combination of the previous samples, therefore neighboring samples become correlated, and the autocorrelation is described by the filtering coefficients themselves.
Correlated noise, obtained by filtering of the exemplary white noise realization is depicted in Figure \ref{Fig:noise}.

\subsection{The goal of differential quantization}

Having introduced basic stochastic concepts differential quantization can be discussed mathematically.

In the model applied the input signal $\xi(n)$ is assumed to be a wide sense stationary process.
The effect of quantization can be most easily modeled as an additive noise $\epsilon(n)$, added to the quantized signal.
Efficiency of quantization is usually described by the signal-to-quantization-noise ratio, defined as the ratio of the energy of the quantized signal and the quantization noise, written as
\begin{equation}
\mathrm{SQNR} = \frac{\EX(\xi(n)^2)}{\EX(\epsilon(n)^2)},
\end{equation}
assuming that the quantized signal is the input signal directly.

In an ideal case where the quantization error is uniformly distributed and the signal has a uniform distribution covering all quantization levels the quantization noise can be calculated as
\begin{equation}
\mathrm{SQNR} = 20 \mathrm{log}_{10} 2^N,
\end{equation}
where $N$ is the bit depth.
In case that differential quantization is applied, two statements can be made
\begin{itemize}
\item Assuming that in the receiver side the input signal can be regenerated from the quantized differential signal the final signal-to-noise ratio can be calculated as 
\begin{equation}
\mathrm{SNR} = \frac{\EX(\tilde{\xi}(n)^2)}{\EX(\epsilon(n)^2)},
\end{equation}
\item However, instead of the input signal, the differential signal is quantized, setting the quantization SNR to
\begin{equation}
\mathrm{SQNR} = \frac{\EX(\delta(n)^2)}{\EX(\epsilon(n)^2)} = 20 \mathrm{log}_{10} 2^N.
\end{equation}
\end{itemize}
Rewriting the above equations results in the total SNR of 
\begin{equation}
\mathrm{SNR} = \frac{\EX(\tilde{\xi}(n)^2)}{\EX(\epsilon(n)^2)} = \frac{\EX(\tilde{\xi}(n)^2)}{\EX(\delta(n)^2)} \cdot 
\underbrace{\frac{\EX(\delta(n)^2)}{\EX(\epsilon(n)^2)}}_{20 \mathrm{log}_{10} 2^N}
,
\end{equation}
revealing that compared to the direct quantization of the input signal the signal-to-noise ratio is increased by a factor of
\begin{equation}
\mathrm{G}_p = \frac{\EX(\tilde{\xi}(n)^2)}{\EX(\delta(n)^2)} 
\end{equation}
termed as the \textbf{prediction gain}, being a large number, assuming that the input signal can be predicted precisely.
This arises the question, how the actual input sample can be estimated based on the previous samples only.

\subsection{The optimal prediction coefficients}
As the most simple approach the actual input sample $\xi(n)$ can be predicted as the linear combination of the previous $N$ number of samples, written in the form of
\begin{equation}
\tilde{\xi}(n) = \sum_{m=1}^N a(m) \xi(n-m) = \mathbf{a}^{\mathrm{T}} \mathbf{\xi}_{n-1},
\label{Eq:lin_pred}
\end{equation}
written in a vectorial form.
In the expression vector $\mathbf{a} = [a(1), a(2),...,a(N)]^{\mathrm{T}}$ contains the weights of the previous input samples used for prediction, and vector $\mathbf{\xi}_{n-1} = [\xi(n-1),\xi(n-2),...,\xi(n-N)]^{\mathrm{T}}$ contains the previous $N$ number of the input samples.

The goal is to minimize the expected energy of the difference between the actual input sample $\xi(n)$ and the prediction $\hat{\xi}(n)$ by optimizing the prediction weights $\mathbf{a}^{\mathrm{T}}$ so that 
\begin{equation}
\arg\min\limits_{\mathbf{a}}: \EX\left( | \xi(n) - \tilde{\xi}(n) |^2 \right) = \arg\min\limits_{\mathbf{a}}: \EX\left( | \xi(n) - \mathbf{a}^{\mathrm{T}} \mathbf{\xi}_{n-1} |^2 \right) 
\end{equation}
holds.
The quadratic expression can be expounded to
\begin{multline}
\EX\left( | \xi(n)^2 - \mathbf{a}^{\mathrm{T}} \mathbf{\xi}_{n-1} |^2 \right) =
\EX\left( \xi(n) - 2\xi(n)\mathbf{a}^{\mathrm{T}} \mathbf{\xi}_{n-1}  + \mathbf{a}^{\mathrm{T}} \mathbf{\xi}_{n-1}  \mathbf{\xi}_{n-1}^{\mathrm{T}} \mathbf{a}\right) = \\
\EX\left( \xi(n)^2 \right) - 2 \mathbf{a}^{\mathrm{T}} \EX\left(  \xi(n) \mathbf{\xi}_{n-1} \right) + \mathbf{a}^{\mathrm{T}} \EX\left(  \mathbf{\xi}_{n-1}  \mathbf{\xi}_{n-1}^{\mathrm{T}} \right) \mathbf{a}
\end{multline}
with exploiting the linearity of expected value operator and collecting non-stochastic quantities outside of it.
The expected value of the scalar-vector product and the dyadic product terms of the expression can be recognized as the autocorrelation values of the input signals, rewritten in a matrix form as
\begin{equation}
\EX\left( | \xi(n)^2 - \mathbf{a}^{\mathrm{T}} \mathbf{\xi}_{n-1} |^2 \right) =
r_{\xi}(0) - 2 \mathbf{a}^{\mathrm{T}} \mathbf{r}_{\xi} + \mathbf{a}^{\mathrm{T}} \mathbf{R}_\xi \mathbf{a}
\label{Eq:quad_eq}
\end{equation}
with denoting the signal energy, and the autocorrelation vector and matrix as
\begin{align}
\begin{split}
r_{\xi}(0) = \EX\left( \xi(n)^2 \right), \hspace{15mm}
\mathbf{r}_{\xi} =  \begin{bmatrix}
       r_\xi(1) \\[0.3em]
       r_\xi(2) \\[0.3em]
       ... \\[0.3em]
       r_\xi(N) \end{bmatrix}, \\
\mathbf{R}_{\xi} =  \begin{bmatrix}
       r_\xi(0) & r_\xi(1) & ... & r_\xi(N-1) \\[0.3em]
       r_\xi(1) & r_\xi(0) & .... & r_\xi(N-2)\\[0.3em]
       ... \\[0.3em]
       r_\xi(N-1) & r_\xi(N-2) & ... & r_\xi(0)\end{bmatrix}.
\end{split}
\end{align}
Expression \ref{Eq:quad_eq} has to be minimized with respect to vector $\mathbf{a}^{\mathrm{T}}$.
The minimization can be performed by finding the zero of the derivative of the expression with respect to vector $\mathbf{a}^{\mathrm{T}}$, reading 
\begin{equation}
\frac{\partial}{\partial \mathbf{a}^{\mathrm{T}}}\EX\left( | \xi(n)^2 - \mathbf{a}^{\mathrm{T}} \mathbf{\xi}_{n-1} |^2 \right) =
- 2  \mathbf{r}_{\xi}  + 2\mathbf{R}_\xi \mathbf{a} = 0.
\end{equation}
Finally, from the above equation the optimal prediction coefficient vector can be expressed as
\begin{equation}
\mathbf{a}^{\mathrm{T}} =
\mathbf{R}_\xi^{-1} \mathbf{r}_{\xi}.
\end{equation}
The above coefficients are the so-called \textbf{Wiener filter} coefficients for the estimation of a stationary stochastic process.

From the form of the optimal prediction coefficients it is clear that the signal estimation is based on the measured correlation of the previous source samples, therefore prediction is efficient as long as neighboring samples are linearly related.
Hence the optimal prediction is often termed as \textbf{linear prediction}.
The above Wiener filter is, therefore, capable of the estimation of the correlated part of the input signal.

\subsection{Feedforward prediction and quantization}

It is important to note that linear prediction \eqref{Eq:lin_pred} describes the discrete linear convolution of vectors $\mathbf{a}$ and $\mathbf{\xi}_{n-1}$. 
This means that the estimation of the actual sample can be obtained by the simple FIR filtering\footnote{The term FIR (Finite Impulse Response) filtering refers to the fact that the applied filter contains no feedback, thus it is ensured that to an excitation with a finite extent the filter output is of finite extent.
The actual filter impulse response is described by the coefficient vector itself.} of the input stream with the coefficient vector $\mathbf{a}$.
The result of estimation is subtracted from the input sample, generating the differential signal, which therefore can be written as
\begin{equation}
\delta(n) = \xi(n) - \sum_{m=1}^{N} a(m) \xi(n-m),
\end{equation}
or, transforming the equation to the $z$-transform domain---by exploiting that delay by one sample is a multiplication by $z^{-1}$ in the $z$-domain---as
\begin{equation}
\delta(z) = \xi(z)\left( 1  - \sum_{m=1}^{N} A(z) \, z^{-m}\right).
\end{equation}

\begin{figure}[]
	\centering
	\begin{overpic}[width = 1\columnwidth ]{figures_en/lin_pred_fir.png}
	\small
	\end{overpic}
	\caption{One dimensional and two dimensional white noise process (a) and correlated noise process (b).}
	\label{Fig:noise}
\end{figure}

\begin{figure}[]
	\centering
	\begin{overpic}[width = 0.85\columnwidth ]{figures_en/fb_diff_quant.png}
	\small
	\put(0,0){(a)}
	\end{overpic}
	\begin{overpic}[width = 0.85\columnwidth ]{figures_en/fb_diff_quant2.png}
	\small
	\put(0,0){(b)}
	\end{overpic}
	\caption{One dimensional and two dimensional white noise process (a) and correlated noise process (b).}
	\label{Fig:noise}
\end{figure}


Once all the correlated part of the input signal is removed, by definition, in the remaining differential signal each sample is uncorrelated from the previous samples.
Thus, with optimal prediction differential coding \textbf{decorrelates} the input signal before quantization, and the differential signal is a white noise process. 


\end{document}

