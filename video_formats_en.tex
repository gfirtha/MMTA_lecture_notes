Az előző fejezet témája a színes képpontok ábrázolásának módja volt mind analóg, mind digitális esetben.
A jelen fejezet bemutatja, hogyan állíthatók össze ezen képpont-reprezentációkból az analóg és digitális videójelek, illetve az így létrehozott videóformátumok paraméterválasztásának kérdéseivel foglalkozik. 

\section{A videójel felépítése és jellemzői}

Elsőként a korai, normál felbontású analóg televíziós rendszerek képformátumát és paramétereinek megválasztásának kérdéseit tárgyaljuk, kezdve az NTSC és PAL rendszerben alkalmazott képformátum jellemzőivel, amelyet közvetlenül átvett az SD digitális formátum is.
Bár ezen analóg rendszerek már csak elvétve vannak használatban világszerte---Magyarországon például több éves digitális átállásra való előkészülés után 2013-ban szűnt meg az analóg műsorszórás---, mégis fontos tárgyalni főbb jellemzőit.
Ennek oka, egyrészt, hogy a paraméterek megválasztásának irányelvei ugyanúgy vonatkoznak a jelenlegi videóformátumokra is.
Másrészt a jelenleg alkalmazott képformátumok számos jellemzője ezen korai analóg formátumok hagyatéka.

\subsection{Az analóg videójel felépítése}

Az egyes videóparaméterek tárgyalása előtt fontos tisztázni, hogyan épül fel az analóg videójel, amely utána egyszerűen kiterjeszthető a digitális esetre is.
Az előző fejezetben látható volt a CRT megjelenítők működési elve: láthattuk, hogy az RGB alapszínek létrehozásáért a képernyőt bevonó foszforanyagok feleltek, amelyek gerjesztés hatására adott spektrális eloszlású fényt bocsájtanak ki.
Egy konkrét CRT megjelenítő által alkalmazott foszforanyagok jellemzőit a \ref{sec:CRT} szakasz tárgyalta részletesen, a CRT működési elvét is bemutatva.
A foszforok gerjesztését egy elektronágyú valósítja meg, amelynek elektronnyaláb-áramsűrűsége a vezérlőfeszültségének kb. 2.2 kitevőjű hatványa (ld. gamma-torzítás).

Megfelelően vezérelt mágneses eltérítők segítségével ez az elektronnyaláb sorról sorra járja be a kijelzőt\footnote{A bejárás megválasztása nem egyértelmű, a fekete-fehér szabvány létrehozása során vizsgálták az oszloponkénti bejárás, illetve a soronkénti váltott-irányú (oda-vissza) bejárás lehetőségét is, azonban az áramköri megvalósíthatóság szempontjából a bemutatott megoldás mellett esett a választás.}.
Sor végére érkezve horizontálisan visszafutva a következő TV-sor elejére, kép végére érkezve vertikálisan visszafutva a következő képernyő elejére.
Fontos alapelv az analóg TV-technikában, hogy a videójel vétele és kijelzése teljesen valós-időben valósul meg (a korai vevők esetében analóg tárolót nem lehetett beépíteni).
A videójel tartalma ezért maga a megjelenítendő kép sorainak tartalma---fekete-fehér esetben egyetlen $Y'$ jel, színes kép esetén az $R'G'B'$ intenzitások, vagy az ezekből képzett luma-chroma jelek---sorról sorra ábrázolva.
A sorról sorra történő kijelzés alapelve a \ref{Fig:TV_signal} (a) ábrán látható.
%
\begin{figure}[]
	\centering
	\begin{minipage}[c]{0.3\textwidth}
		\begin{overpic}[width = 1\columnwidth ]{figures/FormatJargon-1.png}	
		\small
		\put(0,0){(a)}		
		\end{overpic}
				\begin{overpic}[width = 1\columnwidth ]{figures/TV.png}	
		\small
		\put(0,0){(d)}		
		\end{overpic}
	\end{minipage} \hfill
	\begin{minipage}[c]{0.68\textwidth}
		\centering
		\begin{overpic}[width = 0.86\columnwidth ]{figures/FormatJargon-3.png}		\small
		\put(0,0){(b)}		\end{overpic}
		\begin{overpic}[width = 1\columnwidth ]{figures/FormatJargon-7.png}		\small
		\put(0,0){(c)}		\end{overpic}
	\end{minipage}
%
	\caption{Az analóg videójel felépítésének elve}
	\label{Fig:TV_signal}
\end{figure}
%

Természetesen az elektronnyalábnak véges idő alatt kell az egyes sorok végéről a következő sor elejére futnia, illetve a kép végére érve a következő kép kezdetére visszatérnie.
Ezalatt az idő alatt az elektronnyaláb kioltott állapotban kell, hogy legyen, különben a visszafutás látható nyomot hagyna a kijelzőn.
Ez a gyakorlatban azt jelenti, hogy a videójel ezekben az időszakokban nulla, vagy negatív értékű, azaz feketeszintű, vagy az alatti értékű.
Ezek az ún. \textbf{kioltási intervallumok (blanking interval)}, amelyek kihasználhatók információ-hordozásra is, ahogy azt a következőekben látni fogjuk.
Alapvető fontosságúak az ezekben az időintervallumokban elhelyezett ún. \textbf{szinkronjelek}, vagy \textbf{szinkronimpulzusok}.
Ezek az impulzusok az egyes sorok, illetve képek kezdetét jelzik a vevőnek annak érdekében, hogy az egyes videóminták a megfelelő pozícióban jelenjenek meg.

Az elmondottak alapján a sorfolytonos videójel jól megkülönböztethető időintervallumokra bontható:
\begin{itemize}
\item \textbf{Aktív soridőre}, amely egy adott TV-sor tényleges tartalmát tartalmazza.
Analóg formátumok esetén az intervallum hosszát $\mu \mathrm{s}$-ben, digitális esetben pixelben mérjük.
\item A \textbf{sorkioltási időre (horizontal blanking)}, vagy horizontális kioltási időre, amely idő alatt az elektronnyaláb kioltott állapotban a következő sor elejére fut.
A sorkioltási időben kerül elhelyezésre a \textbf{sorszinkron impulzus (HSYNC)}, amely gyakorlatilag a sorvisszafutás triggerjeleként fogható fel.
A sorszinkron impulzus előtti és utáni rövid intervallumok rendre az ún. előváll és utóváll (front porch és back porch).
Analóg formátumok esetén a kioltási intervallum hosszát $\mu \mathrm{s}$-ben, digitális esetben pixelben mérjük
\item A \textbf{képkioltási időre (vertical blanking)}, amely idő alatt az elektronnyaláb kioltott állapotban a képernyő végéről a képernyő első sorának elejére fut.
A vertikális kioltás hosszát jellemzően soridőben mérik (inaktív sorok száma).
\end{itemize}
A videójelben tehát egy teljes soridőben ($T_{\mathrm{H}}$, mint horizontal) mind aktív videótartalom, mind inaktív időintervallum található.
A videójel sorainak felépítése a \ref{Fig:TV_signal} (b) ábrán látható.
Hasonlóan, egy képidő ($T_{\mathrm{V}}$, mint vertical) felbontható aktív és inaktív sorokra.
Az aktív és inaktív sorok felépítése a \ref{Fig:TV_signal} (c) ábrán látható.

A teljes videójel és így egy teljes kép tehát általánosan felbontható aktív és inaktív területekre.
Ezt a \ref{Fig:TV_signal} (d) ábra illusztrálja.

A következőekben azt vizsgáljuk, hogy milyen elvek mentén kerültek megválasztásra a videójelben a különböző időzítések, pl. sorfrekvencia, képfrekvencia, illetve a felbontást meghatározó paraméterek.

\subsection{Az analóg videóformátumok paraméterei}

\subsubsection*{Képarány és képméret}
Elsőként fontos leszögezni, mekkora képméretre kell optimális formátum-paramétereket választani.
Az \ref{sec:HVS} fejezetben látható volt, hogy az emberi szemben a színlátás helye a sárgafolt, ezen belül is az éleslátásért a látógödörben (fovea centralis) elhelyezkedő receptorok felelnek.
A látógödör mérete alapján az éleslátásunk a teljes $\approx200$ fokos látószögünkből kb. 10-15 fokot \href{http://hyperphysics.phy-astr.gsu.edu/hbase/vision/retina.html}{fed le} a horizontális irányban.
A normál felbontású televíziós szabvány megalkotása során a cél ezen fő látószög tartalommal való kitöltése volt, vagyis a normál felbontású televízió kb. a látótérből 10 fokot kell, hogy kitöltsön (azaz a periférikus látásnak a képalkotásban nem volt szerepe).
Természetesen a konkrét képméret ezek után a nézőtávolság függvénye.
Adott pixelméret/sortávolság mellett az optimális nézőtávolság megválasztásával a későbbiekben foglalkozunk.

A kép mérete mellett fontos térbeli jellemző a kijelző horizontális és vertikális dimenziójának aránya, azaz az ún. \textbf{képarány (aspect ratio)}.
Az SD formátum alapjául szolgáló NTSC szabvány létrehozása az 1940-es évekig nyúlik vissza, és kidolgozása során nyilvánvaló törekvés volt a korabeli mozifilmek megjelenítésével való kompatibilitás biztosítása.
A mozi korai korszaka, így a teljes némafilm korszak (az anamorf lencsék megjelenése előtt) kizárólag 4:3 képarányt alkalmazott, azaz a horizontális és vertikális képhosszak aránya $1.3\dot{3}$ volt\footnote{A 4:3 képarány létrejötte egészen Thomas Alva Edison munkájáig vezethető vissza, aki az általa használt 35 mm széles filmen egy képkockát 4 perforációnyi magasságúra (19 mm) definiált. 
A perforációk közötti kihasználható szélességből (25.375 mm) így a hasznos terület épp 4:3-hoz képarányúra adódik. 
A 35 mm-es filmen 4 perforációnyi képméretet 1909-ben fogadták el általános szabványnak ("4-perf negative pulldown"), lehetővé téve a szabványos mozikamerák, mozigépek és így a mozi térhódítását.}.
Habár az 50-es években megjelentek az első szélesvásznú mozis formátumok, az NTSC szabvány ezt a \textbf{4:3} képarányt fogadta el a televízió szabványos képarányának, amely egészen a HD formátum bevezetésig használatban maradt.
%TODO anamorphic lenses?
% Forrás: https://www.shutterstock.com/blog/4-3-aspect-ratio
% https://www.cinematographers.nl/FORMATS1.html

\subsubsection*{Képfrissítési frekvencia és képfrekvencia}

Következő kérdésként vizsgáljuk a mozgókép temporális mintavételi frekvenciájának, azaz a másodpercenként felvillantott képelemek számának megválasztási szempontjait.
Ezzel kapcsolatban fontos megkülönböztetni néhány szorosan összefüggő fogalmat:
\begin{itemize}
\item Az $\mathbf{f_{\mathrm{r}}}$ \textbf{képfrissítési frekvencia (refresh rate)} a képernyő tartalmának másodpercenkénti újrarajzolásának száma (jellemzően $\mathrm{Hz}$-ben kifejezve)
\item Az $\mathbf{f_{\mathrm{V}}}$ \textbf{képfrekvencia (frame rate) (V: vertical)} a képernyő tartalmának másodpercenkénti változásának száma, vagyis a másodpercenként megjelenített új képek száma (jellemzően $\mathrm{fps}$-ben (frame per second) $\mathrm{Hz}$-ben kifejezve)
\item Az $\mathbf{f_{\frac{\mathrm{V}}{2}}}$ \textbf{félképfrekvencia (field rate)} váltott-soros letapogatás esetén értelmezhető mennyiség, ami a másodpercenként megjelenített félképek számát jelöli.
(jellemzően $\mathrm{fps}$-ben (frame per second) $\mathrm{Hz}$-ben kifejezve)
Általánosan igaz, hogy $f_{\frac{\mathrm{V}}{2}} = 2\cdot f_{\mathrm{V}}$.
\end{itemize}
Az első két fogalom között a különbség úgy értelmezhető, hogy a képfrekvencia a videóra jellemző mennyiség, a tárolt másodpercenkénti képkockák számát jelzi.
A képfrissítési frekvencia a megjelenítőre jellemző, azt jelzi, hányszor ,,villan fel'' a kijelzett képelem másodpercenként, függetlenül attól, hogy a kijelzett képtartalom változik-e.
Ebből kifolyólag általánosan igaz, hogy $f_{\mathrm{r}} \geq f_{\mathrm{V}}$.

\paragraph*{Alsó korlát a kép- és képfrissítési frekvenciára:}
A kép- és képfrissítési frekvencia megválasztásánál két szempontot szükséges figyelembe venni:
\begin{itemize}
\item Egyrészt mozgó objektumok képi reprodukciója során fontos, hogy elegendő mozgási fázist jelenítsünk meg ahhoz, hogy a megfigyelő folytonosnak érzékelje a képtartalom változását, amely szempont a képfrekvencia megválasztására ad kiindulópontot.
\item Emellett elegendően magas képfrissítési frekvenciát kell választani a képernyő újrarajzolásából származó \textbf{villogás (flickering)} elkerüléséhez.
\end{itemize}

Utóbbi szempont szigorúbb elvárásokat támaszt, azaz általánosan a képfrekvencia elegendő, ha alacsonyabb a képfrissítési frekvenciánál.
Ennek oka az ún. \textbf{béta mozgás (beta movement)} nevű optikai illúzió, amely a látás azon jellemzője, hogy egymás után vetített statikus képek sorozatát $~10-12~\mathrm{fps}$ változás fölött az emberi szem már folytonos, látszólagos mozgásként érzékeli \footnote{A béta mozgás magyarázata máig sem teljesen tisztázott, leggyakrabban a látóidegen terjedő ingerület létrejöttének gyakoriságával, terjedési tulajdonságaival magyarázzák.}.
A béta mozgás jelensége miatt tehát a folytonos mozgás biztosításához 
\begin{equation}
f_{\mathrm{frame}} > \sim20~\mathrm{Hz} 
\end{equation}
képfrekvencia már elegendő lenne 
\footnote{Érdemes megjegyezni, hogy ez a képfrissítési frekvencia csak ahhoz elegendő, hogy ténylegesen mozgásnak érzékeljük a képsorozatot, ettől még a mozgás gyakran ,,darabos'': a nagyobb---pl. 60 fps rögzített és vetített képek folytonosabbnak, ,,simábbnak'' fognak tűnni. 
Épp ezért számos modern kijelző, illetve számítógépes szoftver képes időbeli interpolációra, amely során az MPEG kódolóban is használatos mozgásbecslés alkalmazásával megpróbálják ''kitalálni'' az egyes képkockák közötti tartalmat.
Érdekes tény azonban, hogy a néző szeme már kellően hozzászokott a mozis 24 fps rögzítési frekvenciához, emiatt a magasabb fps-el rögzített, vagy interpolált videó természetellenesen hat.
Ennek a hatásnak a neve a szappanopera effektus (soap opera effect), amely elnevezés onnan származik, hogy a TV-s szappanoperákat---a klasszikus filmhez képest olcsón---közvetlenül digitális videóra rögzítették jellemzően $60 \mathrm{fps}$-el.},
ezt a képfrissítési frekvenciát azonban az átlagos néző még villogónak érzékelné.

Ahhoz, hogy ezt elkerüljük, a képfrissítési frekvenciának tehát magasabbnak kell lennie a fúziós frekvenciánál\footnote{
A fúziós frekvencia fényingerek változásának azon frekvenciája, amely fölött a fényinger változását az emberi szem már nem képes követni.
Különböző világosságú felületek váltakozása esetén a gyakorlatban efölött a megfigyelő csak egy ,,összeolvadt" átlagos világosságot érzékelni.
A fúziós frekvencia értéke számtalan tényezőtől függ. 
Többek között emberről emberre változik, függ az átlagos megvilágítási szinttől és színhőmérséklettől, az adaptációs állapottól, a váltakozó fényinger színétől (a frekvencia növelésével jellemzően 15-20 Hz környékén a színezetbeli fluktuáció megszűnik, és csak a világosságszintek közötti vibrálás érzékelhető) amplitúdójától, és a gerjesztés helytől a retinán: azaz, hogy a villogást a fő látóterünkben, vagy a periférikus látásunkkal érzékeljük-e.}.
Általánosan elmondható, hogy a fúziós frekvencia az embereknél 50-90 Hz közé esik: a fő látótérben, amelyet a csapok dominálnak a látás lassabb, itt a fúziós frekvencia ~50 Hz, míg a periférikus látás jóval gyorsabb, itt a fúziós frekvencia magasabb.
Mivel az analóg TV-formátum bevezetésénél a cél a fő látótér tartalommal való kitöltése volt, így célszerűen a képfrissítési frekvenciát 
\begin{equation}
f_{\mathrm{r}} > 50-60~\mathrm{Hz} 
\end{equation}
környékére kellett választani.
A konkrét érték megválasztását azonban már a katódsugárcsöves TV technológia egy hátránya határozta meg: a katódsugárcső tápfeszültségére rákerülő hálózati ,,brumm''.

\begin{figure}[]
	\centering
	\begin{overpic}[width = 1\columnwidth ]{figures/ripple.png}
	\end{overpic}
	\caption{Periodikus hálózati brumm megjelenése az egyenirányított tápfeszültségen egyutas egyenirányítás esetén}
	\label{Fig:ripple}
\end{figure}

\paragraph*{A hálózati frekvencia hatása a képre:}
A brumm (angolul ripple) a hálózati váltófeszültség egyenirányításának tökéletlenségéből származó periodikus zavarjel, ahogy az a \ref{Fig:ripple} ábrán látható.
A zavarjel frekvenciája a hálózati frekvenciával egyezik meg (egyutas egyenirányítás), vagy annak kétszerese (kétutas egyenirányítás esetén).
Magyar elnevezése a hangerősítők kimenetén megszólaló jellemzően 50 Hz-es mélyfrekvenciás zugásból származik.
Televízió esetében mivel ez a zavarjel közvetlenül hozzáadódik a katódsugárcső vezérlőjeléhez, ezért a zavarjel kirajzolódik a kijelzőn, így látható hibát okoz.

A fejezetben elején láthattuk a CRT kijelzők működési elvét és a videójel felépítését.
Vizsgáljuk meg, mi rajzolódik ki a képernyőn, ha az elektronágyú vezérlőjele, azaz maga a videó jel periodikus, legegyszerűbb esetben 0 és 1 között oszcilláló szinusz, azaz
\begin{equation}
Y(t) = \frac{1}{2} \sin 2 \pi f t + \frac{1}{2},
\end{equation}
ahol $f$ a vezérlőjel frekvenciája (most az egyszerűség kedvéért a kioltási időktől eltekintve).
A képernyőre ekkora sorról sorra kirajzolódik ez a szinuszos vezérlőjel.

Jelölje a képfrekvenciát $f_\mathrm{V}$, és a sorfrekvenciát $f_\mathrm{H}$ (mint horizontális frekvencia), köztünk természetesen fennáll az
\begin{equation}
f_\mathrm{H} = N_\mathrm{V} \cdot f_\mathrm{V}
\end{equation}
összefüggés, ahol $N_\mathrm{V}$ a képernyő sorainak száma!
Az egyszerűség kedvéért a képfrissítési frekvencia egyezzen meg a képfrekvenciával ($f_\mathrm{r} = f_\mathrm{V}$).
Könnyen belátható, hogy 
\begin{itemize}
\item $f = f_\mathrm{H}$ választással minden sor tartalma ugyanazon szinuszhullám, a hullám kezdőfázisa minden sor elején és minden kép elején azonos, így egy álló, horizontális hullámforma jelenik meg a képernyőn, ahogy az a \ref{Fig:ripple_display} (a) ábrán látható.
\item $f > f_\mathrm{H}$ választással a szinuszos jel fázisa sorról sorra lassan növekszik (mivel a periódushossza rövidebb, mint egy TV sor), így a hullámforma a horizontálishoz képest enyhe dőlést mutat.
Emellett már az első sorban is a hullám kezdőfázisa képről képre változik, így a teljes képtartalom lassan balra mozog.
Hasonlóképp a sorfrekvencia alatti választással lassan jobbra mozgó képet kapunk.
\item $f = f_\mathrm{V}$ választással a teljes szinuszhullám egy teljes kép kirajzolásának ideje alatt rajzolódik ki.
Mivel egy sor ideje alatt (megfelelően nagy $N_\mathrm{V}$ sorszám esetén) a jel értéke alig változik, ezért soronként állandónak tekintheő a tartalom.
Így tehát a teljes képidő alatt egy álló, vertikális szinuszhullám jelenik meg a kijelzőn, ahogy az az \ref{Fig:ripple_display} (b) ábrán látszik.
\item $f > f_\mathrm{V}$ választással a jelalak kezdőfázisa képről képre nő, így a hullámalak lassan felfelé mozdul.
Hasonlóképp $f < f_\mathrm{V}$ esetén a hullámalak lefelé mozog.
\end{itemize}
\begin{figure}[]
	\centering
	\begin{overpic}[width = 0.45\columnwidth ]{figures/horizontal_sine.png}
	\small
	\put(0,0){(a)}
	\end{overpic}
	\hspace{5mm}
	\begin{overpic}[width = 0.45\columnwidth ]{figures/vertical_sine.png}
	\small
	\put(0,0){(b)}
	\end{overpic}
	\caption{Periodikus jel képernyőn megjelenítve $f = f_\mathrm{H}$ (a) és $f = f_V$ (b) választással}
	\label{Fig:ripple_display}
\end{figure}

Belátható tehát, hogy periodikus jelek megjelenítése során megfelelő választással álló rajzolatot jeleníthetünk meg a kijelzőn.
Márpedig a hálózati brumm épp ilyen periodikus zavarjelként jelenik meg a képernyőn, frekvenciája pedig az adott régió hálózati frekvenciája.
A korai, fekete-fehér televíziós rendszer megalkotása során végzett megfigyelési tesztek egyértelműen kimutatták, hogy elektromos zavar esetén az álló zavarkép jóval kevésbé zavarja a nézőt, mintha a zavar mozgó rajzolatként jelenne meg.
Ennek megfelelően mind az amerikai, mind később, az európai rendszer esetében a képfrekvenciát a hálózati frekvenciának választották meg, így biztosítva, hogy az esetleges hálózati brumm a kijelzőn egy vertikális állóképként jelenik meg, amely a nézők számára alig észrevehető.
Így tehát a képfrissítési frekvencia értéke az amerikai és az európai rendszerben rendre
\begin{equation}
f_{\mathrm{r,USA}} = 60~\mathrm{Hz}, \hspace{1cm} f_{\mathrm{r,Eu}} = 50~\mathrm{Hz}
\end{equation}
lett \footnote{A helyzet a színes TV bevezetésével, azaz az NTSC megjelenésével Amerikában bonyolódott, mivel a színsegédvivő frekvenciáját nem lehetett megfelelően megválasztani.
Részletek nélkül: ennek eredményeképp mind a képfrekvenciát, mind a sorfrekvenciát $0.1~\%$-al csökkentették, így az amerikai rendszer képfrekvenciája $f_V = 60\cdot \frac{1000}{1001} = 59.94~\mathrm{Hz}$ lett végül. 
Ezt a változás szerencsére a megfelelő szinkronjeleknek köszönhetően a már létező TV vevőkészüléket nem befolyásolta.}.

Bár kézenfekvő lenne a képfrekvenciát a képfrissítési frekvenciával egyenlővé tenni, a gyakorlatban nem ezt a megoldást szokás alkalmazni.
Ennek okát és megoldását a következő bekezdés mutatja be.

\subsubsection*{A progresszív és interlaced letapogatás}

Az analóg videójel felépítésénél láthattuk, hogy az elektronnyaláb a teljes képernyő tartalmát bejárja egy képidő alatt.
A bejárás módja azonban nem egyértelmű, egy ügyes trükkel számottevő videójel-sávcsökkentés érhető el.
A következőekben a különböző \textbf{képernyő-letapogatási (raster scanning)} módokat vizsgáljuk\footnote{Természetesen a jelenlegi LCD kijelzők esetében értelmetlen a kijelző bejárásának módjáról beszélni, ezek a kijelzők egyszerre változtatják az összes pixelsor és oszlop tartalmát.
Azonban digitális esetben is értelmezhető a letapogatási mód átviteltechnika szempontjából, hiszen a videójel átvitele és felépítése a letapogatási módtól függ.}.

\paragraph{A progresszív letapogatás:}
A legkézenfekvőbb képernyő bejárási mód az ún. \textbf{progresszív letapogatás (progessive scanning)}, amely során a katódsugár egy képidő ($1/f_{\mathrm{V}}$) alatt sorról-sorra bejárja a képernyő összes sorát.
\begin{figure}[]
	\centering
	\begin{minipage}[c]{0.6\textwidth}
	\begin{overpic}[width = 1\columnwidth ]{Figures/progressive_scan.png}
	\end{overpic}   \end{minipage}\hfill
		\begin{minipage}[c]{0.3\textwidth}
	\caption{Progresszív letapogatás szemléltetése (az egyszerűség kedvéért 11 sorral ábrázolva), az aktív sortartalommal (1), sorvisszafutással (2) és képvisszafutással (3)}
	\label{Fig:progressive}  \end{minipage}
\end{figure}
A letapogatás módját a \ref{Fig:progressive} ábra szemlélteti.
Átviteltechnika szempontjából ez azt jelenti, hogy az adott interface-en (konzumer berendezések esetében jellegzetesen HDMI-n keresztül) a kijelzőn megjelenítendő adat sorról sorra érkezik, és természetesen a teljes kép adatait egy soridő alatt továbbítani kell.
A progresszív formátumot az alkalmazott sorszám utáni ,,p'' jelölés mutatja, lásd HD esetében 1080p.

Bár a progresszív letapogatás tűnik a legegyértelműbb, legkézenfekvőbb megoldásnak, mégis, egészen az UHDTV szabvány megjelenéséig nem ez volt az általánosan elfogadott megoldás.
Ennek okait a következőekben tárgyaljuk.

\vspace{3mm}
\paragraph{A váltottsoros letapogatás:}
Láthattuk, hogy a folytonos mozgás biztosításához már $20-25~\mathrm{Hz}$ képfrekvencia elegendő lenne, míg a villogás elkerüléséhez legalább $50-60~\mathrm{Hz}$ képfrissítési frekvencia szükséges.
Ez már bizonyos szintű tömörítést tesz lehetővé, hiszen a teljes képtartalom elegendő, ha lassabban változik, mint kijelző rajzolási frekvenciája.

%\begin{figure}  
%\small
%  \begin{minipage}[c]{0.64\textwidth}
%	\begin{overpic}[width = 1\columnwidth ]{Figures/triple_blade_shutter.png}
%	\end{overpic}   \end{minipage}\hfill
%	\begin{minipage}[c]{0.3\textwidth}
%    \caption{Triple blade shutter működése: \url{https://www.youtube.com/watch?v=jrSzRAch930} }
%\label{fig:triple_blade_shutter}  \end{minipage}
%\end{figure}

Ez a tömörítés már a korai mozitechnikában is megjelent: 
A korai, némafilmes korszakban számos képfrekvencia volt használatban $16-24~\mathrm{Hz}$ között.
Manapság mozitechnikában a szabványos rögzítési frekvenciát $24~\mathrm{fps}$-re rögzítették.
A villogás elkerüléséhez (tehát a képfrissítési frekvencia növeléséhez) speciális rekesszel látták el a vetítőgépet.
A fénynyaláb útjában forgó \href{https://www.youtube.com/watch?v=jrSzRAch930}{rekesz}, amelyen kettő, vagy három rés volt található (az ún. ''two'', vagy ''three blade shutter'') egy képkocka megjelenítése során tett meg egy teljes fordulatot, így a vetítőgép ugyanazt a képkockát kétszer, vagy háromszor villantja fel, mielőtt továbbhúzza a mozigép a szalagot.
Ezzel az egyszerű trükkel a $24~\mathrm{fps}$-en rögzített tartalmat $48\mathrm{fps}$, illetve manapság jellemzően $72~\mathrm{fps}$-en lehet megjeleníteni a mozikban.

Hasonlóan elven, a modern megjelenítők esetében a kijelző képfrissítési frekvenciája (pl. amivel egy LCD kijelző esetében a háttérvilágítás villog $200~\mathrm{Hz}$ körül\footnote{Az LCD technika természetesen folytonos háttérvilágítással is megoldható lenne, de egyszerűbb a háttérvilágítás dinamikus változásának PWM modulációval való megoldása.}) jóval a tényleges képtartalom frissítési frekvenciája fölött van.
A TV műsorszórás bevezetésének idején azonban a vevőkészülékek nem voltak képesek a képtartalom tárolására, a vett jel közvetlenül, valós időben rajzolódott ki a kijelzőre.
A feladat megoldásául, azaz a másodpercenként átvivendő képek számának csökkentésére, és így sávszélesség-takarékosságra az ún. \textbf{váltott-soros letapogatást (interlaced scanning)} vezették be.
A váltottsoros formátum jelzése a sorszám mögé illesztett ,,i'' jelzés (pl. 1080i).

\begin{figure}[]
	\centering
	\begin{overpic}[width = 0.85 \columnwidth ]{Figures/interlaced_scan.png}
	\end{overpic}
	\caption{Váltott-soros képbontás (TV sorok közbeszövése), a jobb áttekinthetőség kedvéért 21 sorral.
	A teljes képernyő pásztázásához (feltéve, hogy az elektronnyaláb az első félkép első sorának elejéről indul) az első félképnek fél sorban kell végződnie, míg a második félképnek félsorral kell kezdődnie.
	Ez csak páratlan teljes sorszám esetén teljesül (mindkét félkép $N_{\frac{V}{2}} + \frac{1}{2}$ sorból áll, a teljes sorszám $2 N_{\frac{V}{2}} + 1$, ami szükségszerűen páratlan) }
	\label{Fig:interlaced}
\end{figure}

A megoldás alapötlete--ahogy a \ref{Fig:interlaced} ábrán is látható---a következő:
Ahelyett, hogy a kijelző egy teljes képidő alatt az összes sort egymás után végigpásztázná, bontsuk a képernyőt páros és páratlan sorszámú sorokra, amelyek így egy páratlan és egy páros félképet alkotnak.
A teljes képet (angolul \textbf{frame}) tehát két \textbf{félképre} (angolul \textbf{field}) bontjuk.
A kijelző ezután a teljes képidő első felében a páratlan, a második felében a páros sorokat pásztázza végig.

Természetesen a képernyő tartalma a félképek frissítési frekvenciájával, az ún. $f_{\frac{\mathrm{V}}{2}}$\textbf{félképfrekvenciával} frissül, tehát ahhoz, hogy elkerüljük a villogást a félképfrekvenciának kell a fúziós frekvencia fölé esnie.
Az előző szakaszban tárgyalt megfontolásokból kiindulva így váltottsoros letapogatás esetén a félképfrekvencia lett az európai rendszerben $50$, valamint az amerikaiban $60~\mathrm{Hz}$-re (pontosabban $59.94~\mathrm{Hz}$-re) választva.
A teljes, effektív képfrekvencia pedig ezek felére, tehát $25~\mathrm{Hz}$, illetve $30~\mathrm{Hz}$-re ($29.97~\mathrm{Hz}$-re) adódik
A technikával tehát a mozis technikához hasonlóan, a képfrissítési frekvenciát elegendően magasra emelték, míg a tényleges, teljes ,,felbontású'' képtartalom ehhez képest fele sebességgel érkezik.
Összefoglalva tehát interlaced esetben az európai és amerikai rendszerre a 
\begin{align}
\begin{split}
f_{\mathrm{r,USA}} &= f_{\frac{\mathrm{V}}{2},\mathrm{USA}} = 60~\mathrm{Hz} = 2\cdot f_{\mathrm{V},\mathrm{USA}}, \hspace{1cm}  f_{\mathrm{V},\mathrm{USA}}= 30~\mathrm{Hz} \\
f_{\mathrm{r,Eu}} &= f_{\frac{\mathrm{V}}{2},\mathrm{Eu}} = 50~\mathrm{Hz} = 2\cdot f_{\mathrm{V},\mathrm{Eu}}, \hspace{1cm}  f_{\mathrm{V},\mathrm{Eu}}= 25~\mathrm{Hz}
\end{split}
\end{align}
összefüggések érvényesek


Fontos megjegyezni, hogy a félképek (field-ek) különböző időpillanatokban készülnek, azaz nem ugyanazon teljes képhez tartoznak (nem állítható elő egy teljes kép páros-páratlan sorra való felbontásával). 
Ez eltérés a mozis rendszerhez képest, amely ugyanazt a képkockát mutatta be többször.
Ez alapján a következő mondhatóak el a váltottsoros videóról:
\begin{itemize}
\item A váltott-soros letapogatás a progresszívhez képest 2:1 arányú tömörítést valósít meg, azaz a továbbítandó adatmennyiséget (és így a szükséges sávszélességet) lefelezi
\item Álló képtartalomnál a progresszív letapogatással megegyező vertikális felbontást valósít meg (hiszen a páros és páratlan félképek ugyanazt a képet egészítik ki)
\item Gyorsan mozgó képtartalom mellett a függőleges felbontás gyakorlatilag a progresszív formátum fele (hiszen a félkép tartalma folyamatosan változik)
\end{itemize}
Általánosan elmondhatjuk, hogy lassan változó képtartalom esetén (pl. filmek) a váltott-soros letapogatás megfelelően nagy vertikális felbontást és a progresszívnál folytonosabb mozgásreprodukciót biztosít, megfelelő tömörítés (sávszélesség-hatékonyság) mellett.
Gyors kameramozgások esetén, pl. sporttartalom már láthatóvá válhatnak a felezett vertikális felbontásból származó hatások.
\vspace{3mm}

Mivel a sávszélesség-hatékonyság az analóg műsorszórás bevezetésének idején központi kérdésnek számított, ezért a normálfelbontású analóg (majd később a digitális SD) formátum kizárólag váltottsoros letapogatási módot alkalmazott.
A HD szabvány bevezetésével már mind interlaced, mind progesszív formátumok léteznek, míg UHDTV esetén a szabványok már kizárólag progresszív formátumokat definiálnak.

\vspace{3mm}
\paragraph{Az interlaced letapogatás kérdései:}
Bár számos előnnyel rendelkezik, az interlaced technika számos kérdést, nehézséget is felvet egyszerűsége mellett.

Egyik példaképp: korábban láthattuk, hogy a térbeli mintavételi frekvencia megsértése térbeli átlapolódási jelenségekhez vezet, amelyek jellegzetesen térben periodikus képek esetében (pl. téglafal, ,,kockás'' ing) jól látható Moiré ábrák megjelenését okozza.
Mivel interlaced esetben a vertikális mintavételi frekvenciát lefelezzük, ezért félképeken ezek a Moiré ábrák erőteljesen megjelenhetnek, az egymás utáni átlapolódó félképek váltakozása pedig igen zavaró átlapolódási jelenségekhez, ún. interline twitter jelenséghez vezet már állókép megjelenítése esetén is.
A jelenségre egy szemléltető példa \href{https://en.wikipedia.org/wiki/File:Indian_Head_interlace.gif}{itt} található.
Minthogy az összes SD formátum interlaced letapogatást alkalmazott, épp az interline twitter jelensége volt a fő oka a TV felvételek során a négyzetrácsos, csíkos öltözékek elkerülésének.

\begin{figure}  
\small
  \begin{minipage}[c]{0.64\textwidth}
	\begin{overpic}[width = 1\columnwidth ]{Figures/Interlaced_video_frame_(car_wheel).jpg}
	\end{overpic}   \end{minipage}\hfill
	\begin{minipage}[c]{0.3\textwidth}
    \caption{Megfelelő deinterlacing technika nélkül váltottsoros formátum megjelenítése progresszív kijelzőn.}
\label{fig:deinterlacing}  \end{minipage}
\end{figure}

További érdekes kérdést vet fel az interlaced és progresszív formátum közötti konverzió.
Progresszívről interlaced formátumba a feladat viszonylag egyértelmű, a teljes kép páros és páratlan sorokra bontásával megoldható.
A váltottsoros formátumról progresszívre történő konverzió konzumer felhasználási szempontból gyakoribb, gondoljunk csak egy jellegzetesen váltottsoros formátumban rögzített DVD lemez jellemzően progresszív számítógép monitoron történő megjelenítésére.
Legegyszerűbb stratégiaként a monitor a szomszédos félképeket összeszőve alakít ki egy teljes felbontású képet.
Ez azonban gyors mozgások esetén ún. fésűsödési jelenségekhez vezet, amelyet a \ref{fig:deinterlacing} ábra szemléltet.
Épp ezért, a konverzióhoz kifinomultabb \textbf{deinterlacing} eljárás szükséges a félképek sorai közötti adatok interpolációjához.
A feladat létjogosultsága manapság is nagy, hiszen a jelenlegi LCD TV és számítógép monitorok már nem támogatnak natív interlaced megjelenítést, míg a HD műsorszórás még napjainkban is váltottsoros formátumot alkalmaz (jellemzően 1080i-t).


\section{Analóg videóformátumok}

Az előzőek alapján bevezethetjük a normál felbontású analóg televíziós műsorszórás képformátumát:

Ahogy azt már korábban láthattuk két analóg képformátum terjedt el a világon a színes műsorszórás kezdetével:
\begin{itemize}
\item Az Egyesült Államokban és Japánban alkalmazott NTSC képformátumot az FCC vezette be 1953-ban.
Az NTSC formátum a korabeli technológiának megfelelően $N_V = 525$ TV-sorból áll\footnote{
Az NTSC szabvány megalkotásának az idejében már több fekete-fehér TV adás létezett, pl. az RCA által létrehozott NBC csatorna, amely 441 sorral dolgozott, míg több gyártó 600-800 soros rendszerek bevezetését szorgalmazta.
Kompromisszumként 500 sor körüli sorszámot tűztek ki a szabványosítás során célul.
A konkrét érték végül a korabeli CRT technológia korlátjának hozadéka:
A TV vevőkben az időzítésekhez egy master oszcillátor hozott létre egy kétszeres sorfrekvenciájú jelet, amelynek frekvenciáját ($2\cdot 15750~\mathrm{Hz}$) leosztották a sorok számával, megkapva a félképfrekvenciát ($60~\mathrm{Hz}$).
Ezt a jelet összehasonlították a tényleges hálózati frekvenciával és ez alapján korrigálták a master oszcillátor értékét a hálózati zavarokból származó hibaképek elkerülésére.
A frekvenciaosztás a korabeli technológiával csak egymás után kötött multivibrátorokkal volt megvalósítható, a sorszám megfelelő faktorizációja alapján (szorzat alakra bontás).
Minthogy interlaced letapogatás esetén páratlan sorszám alkalmazása szükséges, ezért mindegyik multivibrátor szükségszerűen páratlan számmal kellett, hogy osszon (hiszen páratlan számok szorzata lesz páratlan).
Emellett ezek a szorzó-faktorok értéke aránylag kicsi kell, hogy legyen a frekvencia-drift elkerülésére.
Az 500-hoz legközelebbi kicsi páratlan számokra faktorizálható szám az $525 = 3\cdot 5\cdot 5 \cdot 7$.
A PAL esetében a sorszám megválasztásának az alapja ugyanez, ahol $625 = 5\cdot 5\cdot 5 \cdot 5$.
}, és váltott-soros letapogatást alkalmaz.
A korábban tárgyalt okokból kifolyólag a rendszer képfrissítési frekvenciája, azaz a félképfrekvencia $60~\mathrm{Hz}$ ($59.94~\mathrm{Hz}$), amelyből természetesen a képfrekvencia $30~\mathrm{Hz}$-re ($29.97~\mathrm{Hz}$-re) adódik.
\item Európában, Ausztráliában és Ázsiában a PAL rendszer került bevezetésre 1967-ben.
A PAL formátum sorszáma $N_V = 625$, váltott-soros letapogatással, míg a helyi hálózati frekvenciának megfelelően a félképfrekvencia $50~\mathrm{Hz}$ és így a képfrekvencia $25~\mathrm{Hz}$.
\end{itemize}
Az NTSC és PAL formátum fő jellemzőit a \ref{tab:sd_formats} táblázat tartalmazza.
\begin{table}[h!]
\caption{}
\renewcommand*{\arraystretch}{2.25}
\label{tab:sd_formats}
\begin{center}
    \begin{tabular}[h!]{ @{}c | | l | l @{} }%\toprule
				         &   NTSC  							       & PAL \\ \hline
    Összes sorok száma ($N_{\mathrm{V}}$):	 &  525   								   &  625 \\
    Aktív sorok száma ($N_{\mathrm{V,A}}$):   &  480   								   &  576 \\
    Képfrekvencia ($f_{\mathrm{V}}$):    &  $30~\mathrm{Hz}$ ($29.97~\mathrm{Hz}$) & $25~\mathrm{Hz}$ \\
    Félképfrekvencia ($f_{\frac{\mathrm{V}}{2}}$): &  $60~\mathrm{Hz}$ ($59.94~\mathrm{Hz}$) & $30~\mathrm{Hz}$ \\
    Sorfrekvencia ($f_{\mathrm{H}}$):    &  $525 \cdot 30 = 15750~\mathrm{Hz}$ ($15734~\mathrm{Hz}$) & $15625~\mathrm{Hz}$ \\
    Soridő ($T_{\mathrm{H}}$):           &  $63.49~\mathrm{\mu s}$ ($63.55~\mathrm{\mu s}$) & $64~\mathrm{\mu s}$ \\
    \end{tabular}
\end{center}
\end{table}

\begin{figure}[t!]
\captionsetup{singlelinecheck=off}
\small
  \begin{minipage}[c]{0.64\textwidth}
	\begin{overpic}[width = 1\columnwidth ]{Figures/Timing_PAL_FrameSignal.png}
	\end{overpic}
    \end{minipage} \hfill
	  \begin{minipage}[c]{0.3\textwidth}
    \caption[]{ Egy teljes kép felépítése váltottsoros letapogatás esetén egyetlen videokomponensre ábrázolva:
    \begin{itemize}
    \item aktív soridő: szürke
    \item sorkioltási idő: magenta, cián és sárga
    \item félképkioltási idő: zöld, narancssárga, fehér
    \end{itemize}
    }\label{fig:PAL_frame}
    \end{minipage}
\end{figure}
Egy teljes TV kép (azaz két egymás utáni félkép) felépítése a PAL rendszerben a \ref{fig:PAL_frame} ábrán látható.
Az ábra természetesen csak egyetlen video komponens felépítését szemlélteti.

Az ábrán jól megfigyelhetőek a a félkép és képkioltási idők, bennük pedig az ún. félképszinkron jelek (a zöld tartományban) és sorszinkron jelek (sárga tartomány).
Ismétlésként: Ezek a jelek a TV vevő (vagy általában a megjelenítőeszköz) szinkronizációját biztosítják a megfelelő megjelenítés érdekében.
A szinkronjelek hibás vétele esetén a kép vertikálisan (félképszinkron hiányában), vagy horizontális (sorszinkron hiányába) elmozdul.
Ezeket a jelenségeket ,,jitter''-nek, illetve ,,rolling''-nak nevezzük.

Bár a videójel felépítése az analóg CRT megjelenítők működési elvéből származik, mégis, ugyanez a felépítés jellemző a ma is alkalmazott digitális videójel esetén, pl. a HDMI, vagy DVI szabvány esetén.
A jelenleg elterjedt megjelenítők esetén a kioltási idők természetesen okafogyottá váltak:
A modern, főként stúdió célú CRT megjelenítők már jóval kisebb kioltási idő mellett is működőképesek, míg LCD megjelenítők esetén egyáltalán nincs szükség kioltási időre.
Ennek ellenére a kioltási idők a jelenlegi digitális szabványok esetén is ugyanúgy jelen vannak, így pl. a HDMI szabvány esetében is.
Ennek egyik, természetes oka az, hogy a technika fejlődésével megjelenő újabb és újabb szabványok mind a már létező, korábbi szabványokra épülnek.
Másrészt a kioltási idők lehetővé tették egyéb, kiegészítő adatok tárolását is ezekben az időszegmensekben.
Így a kioltási időkben továbbítható pl. a teletext adat, feliratok, és digitális esetben a video kísérő audio adat is.
Ezen adatok helyét az ITU-R BT.1364 és az SMPTE 291M szabványok definiálják. 
Megjegyezhető, hogy a szabványok a digitális hang átvitelét (pl. a HDMI szabvány esetén is) a sorkioltási időben írják elő \footnote{
Egy egyszerű példaként HDMI audio átvitelre:
1080p HD formátum esetén (összes sor: 1125, ld. később) $60~\mathrm{Hz}$-es képfrekvencia mellett a sorfrekvencia $f_V = 67.5~\mathrm{kHz}$.
$192~\mathrm{kHz}$ mintavételi frekvenciájú 8 csatornás audioanyag átvitele esetén az egy kép alatt átvivendő audiominták száma: $\frac{8 \cdot 192 000 }{60} = 25600~\mathrm{minta}$, azaz soronként kb. 23 minta átvitele szükséges.
Ez a HDMI 1.0 szabvány által megengedett audiosebesség. (Példa folytatása \href{https://www.sciencedirect.com/science/article/pii/B9780128016305000049}{itt} található.)
}.


\vspace{3mm}
% https://www.sciencedirect.com/topics/computer-science/blanking-interval
% https://www.sciencedirect.com/topics/computer-science/horizontal-blanking
Az eddigiek során a videójel felépítésének általános koncepcióját láthattuk, nem esett szó arról, hogy az aktív intervallumban mik a ténylegesen továbbított videójelek, illetve hogy pontosan hány érpáron történik a videójel átvitele.
Természetesen a színes képpontok ábrázolásához három videókomponens szükséges, amely videótechnikában leggyakrabban a luma-chroma jeleket jelenti.

Analóg formátumok esetén a továbbított jelek száma alapján a következő formátumokat különböztethetjük meg:
\begin{itemize}
\item \textbf{Komponens formátumok} esetén a három videójel-komponens egymástól függetlenül van tárolva, illetve párhuzamosan, külön érpárokon kerülnek továbbításra.
\item \textbf{Kompozit formátum} esetén a három videójel-komponens megfelelően multiplexálva egyetlen jellé alakítva van tárolva, illetve kerül egyetlen érpáron továbbításra.
\end{itemize}
\begin{figure}[]
	\centering
	\begin{overpic}[width = 0.90\columnwidth ]{figures/video_comp2.png}
	\end{overpic}
	\caption{A kompozit és komponens videójelek előállításának folyamatábrája}
	\label{Fig:video_components}
\end{figure}
A kompozit és komponens videójelek előállításának folyamatábrája a \ref{Fig:video_components} ábrán látható (kompozit esetben a PAL rendszer példáján).
Mint láthatjuk, mind a kompozit, mind a komponens jelek alapját a színpontok luma-chroma ábrázolása adja\footnote{Kompozit jelekre ez mindig igaz, komponens esetben ritkább esetben közvetlen az $R'G'B'$ jelek továbbítása történik.}.

\vspace{3mm}
\paragraph{A videójel sávszélessége:}
Természetesen mind a luma, mind a chroma jelek véges sávszélességgel kerülnek továbbításra.
Fontos látni, hogy mivel az időtartománybeli videójel valós-időben felrajzolódik a kijelzőre, ezért a jel időtartománybeli, azaz Hz-ben kifejezett sávszélessége a megjelenített kép részletességét határozza meg.
A kijelzett képet tehát véges részletességgel visszük át.
Ennek oka egyrészt a műsorszórás során a nyilvánvaló sávszélesség-takarékosság.
Másrészt a folytonos videójelet a CRT pixel-struktúrája a kijelzés során gyakorlatilag mintavételezi, amely mintavett képjel spektruma a mintavételi frekvencia egész számú többszörösein ismétlődni fog.
Ebből kifolyólag sávkorlátozás nélkül a megjelenítő hatására a videójel átlapolódhat, amely a képen Moiré-ábrák megjelenését okozza.

Az NTSC esetében láthattuk, hogy az aktív sorok száma $N_{V,A} = 480$, míg az aktív soridő $t_{H,A} \approx 52~\mu \mathrm{s}$.
Kiindulva abból, hogy a képarány 4:3, az egy sorban található pixelek száma kb. $N_{H,A} = \frac{4}{3} N_{V,A} = 640$, négyzet alakú pixeleket feltételezve.
A mintavételi tétel értelmében a legnagyobb frekvenciájú ábrázolható térbeli szinuszjelből $N_{H,A}$ periódus fér egy TV sor periódusidejébe (ez a pixelenként váltakozó fekete-fehér-fekete-fehér jel).
Ennek a szinuszjelnek a periódusidejére és frekvenciájára
\begin{equation}
T_{\mathrm{max}} = \frac{t_H}{N_{H,A}/2}, \hspace{1cm} f_{\mathrm{max}} = \frac{N_{H,A}}{2 t_H} = 6.15~\mathrm{MHz}
\end{equation}
adódik, amely tehát a Nyquist frekvencia, azaz az elméleti felső határ a megjeleníthető kép frekvenciájára.

A tapasztalat azonban azt mutatta, hogy a tényleges elméleti maximális felbontás a gyakorlatban nem használható ki teljesen, amely jelenséget az ún. \textbf{Kell-hatásnak} nevezzük.
A jelenség oka a CRT elektronnyalábának véges mérete, a mintavételi tétel alkalmazása során feltételezett végtelen keskeny nyaláb helyett (A nyaláb a képernyőre vetítve nem egyetlen pont, hanem egy Gauss-függvénnyel jól közelíthető intenzitás-profil).
\begin{figure}[]
	\centering
	\begin{minipage}[c]{0.65\textwidth}
	\begin{overpic}[width = 0.95\columnwidth ]{figures/kell.png}
	\end{overpic} \end{minipage}\hfill
	\begin{minipage}[c]{0.35\textwidth}	\caption{A Kell-hatás szemléltetése.}
	\label{Fig:Kell}  \end{minipage}
\end{figure}
Tekintsük példaképp a \ref{Fig:Kell} ábrán látható esetet, melynek során a megjelenítendő kép egyszerű fekete-fehér, soronként váltakozó vonalak, azaz egy épp vertikális Nyquist-frekvenciájú jel.
Abban az esetben ha a megjelenítendő vonalak épp a tényleges TV letapogatási sorra (scan line) esnek a kép tökéletesen visszaállítható.
Ha azonban a vonalak közepe épp két scan line közé esik, úgy a képernyőn a nyaláb a véges kiterjedése miatt a két sor átlagát jeleníti meg, azaz egy azonosan szürke képet kapunk.

Azt, hogy az elvi maximális felbontás (azaz a Nyquist frekvencia) hányadrésze a ténylegesen kihasználható képtartalom megjelenítésére a szubjektíven meghatározott ún. \textbf{Kell faktor} írja le.
NTSC esetében a korabeli CRT technológiához ezt kb. 0.7-nek adódott.

Az átlapolódás elkerüléséhez tehát a Kell hatás figyelembe vétele mellett a megjeleníthető képen a legnagyobb horizontális frekvenciájára az NTSC és PAL rendszerben
\begin{equation}
f^{\mathrm{NTSC}}_{\mathrm{max}} = \frac{N_{H,A} \cdot K}{2 t_H} = 4.3~\mathrm{MHz},
\hspace{3mm}
f^{\mathrm{PAL}}_{\mathrm{max}} = \frac{N_{H,A} \cdot K}{2 t_H} \approx 5.2 ~\mathrm{MHz}
\end{equation}
adódik, ahol $K = 0.7$ a Kell faktor.
Ennek megfelelően a luma jel (hasonlóan a korai fekete-fehér videójelhez) NTSC esetében $4.5~\mathrm{MHz}$-re, PAL esetében $5.6~\mathrm{MHz}$-re van sávkorlátozva.

Korábban láthattuk, hogy az emberi szem színezetre vett térbeli felbontóképessége jóval kisebb (kevesebb, mint fele) a világosságéhoz képest.
Ezt kihasználva a különböző videójel formátumokban közös, hogy a színkülönbségi (chroma) jeleket sávkorlátozva, azaz csökkentett felbontással reprezentálják.
Ez analóg formátumok esetén sávszélességet takarít meg, míg digitális esetben már hatékony kompressziós módszerként is felfogható, mint a következőekben látható lesz.

A következőekben a kompozit és komponens jelábrázolás részleteit vizsgáljuk.

\subsection{A kompozit videójel}
Analóg átviteltechnika szempontjából a legegyszerűbb megoldás a videójel továbbítására a 3 videókomponens egyetlen érpáron való átvitele.
Ebben az esetben a luma és chroma komponensekből egyetlen ún. \textbf{kompozit} jelet kell képzeni, hogy a vevő oldalon az eredeti három komponens különválasztható.
A feladat megoldására három---alapgondolatában azonos---módszer létezik, az NTSC, PAL és SECAM megoldások.
A rendszerek pontos működésétől eltekintve a következő bekezdés az NTSC és PAL kompozitjelek képzésének alapelvét mutatja be.

A kompozit formátum az NTSC rendszer bevezetésével került kidolgozásra a létező fekete-fehér TV-vevőkkel kompatibilis analóg színes műsorszórás megvalósítására.
A feladat a már létező műsorszóró rendszerben alapsávban továbbított luma jelhez (azaz a fekete-fehér jelhez) a színinformáció olyan módú hozzáadása volt, hogy a létező monokróm vevőkben a többletinformáció minimális látható hatást okozzon, míg a színes vevő megfelelően külön tudja választani a luma és chroma jeleket.
Tehát más szóval a visszafele-kompatibilitás miatt az új színes rendszerben a luma jelet változatlanul kellett átvinni. 
Minthogy az átvitelhez használt RF spektrum jelentős részét már elfoglalták a frekvenciaosztásban küldött egyes TV csatornák (a képinformáció, és az FM modulált hanginformáció), így a luma és chroma komponensek csak ugyanabban a frekvenciasávban kerülhetnek továbbításra.

Az alapsávi fekete-fehér TV jel felépítése egyszerű a már bemutatott \ref{fig:PAL_frame} ábrán látható felépítéssel megegyező:
Egymás után, soronként tartalmazza a CRT elektron-ágyú vezérlőfeszültségének időtörténetét, amely tehát így a műsor vételével teljesen valós időben rajzolja soronként a kijelző képernyőjére az $Y'(t)$ luma jel tartalmát.
Az egyes sorok és képek kijelzése között az elektron-ágyú kikapcsolt állapotban véges idő alatt fut vissza a következő sor, illetve kép elejére. 
Egy fekete-fehér TV sor felépítése az \ref{Fig:PAL_line} ábrán látható.

%
\begin{figure}[]
	\centering
	\begin{minipage}[c]{0.65\textwidth}
	\begin{overpic}[width = 0.95\columnwidth ]{figures/PAL_line.png}
	\end{overpic} \end{minipage}\hfill
	\begin{minipage}[c]{0.35\textwidth}	\caption{Egyetlen TV sor luma jele és szinkron jelei a PAL rendszer időzítései mellett. Az NTSC esetében a TV sor felépítse jellegere teljesen azonos, a PAL-tól eltérő időzítésekkel.}
	\label{Fig:PAL_line}  \end{minipage}
\end{figure}
%

A valós idejű átvitel/kijelzés elvéből látható, hogy a színinformáció átvitele időosztásban sem lehetséges, tehát a chroma jeleket a luma jelekkel azonos frekvenciasávban és időben szükséges átvinni.
A megoldás tárgyalása előtt vizsgáljuk külön a chroma jelek továbbításának módját.

\paragraph{A színsegédvivő bevezetése:}
A színformációt hordozó két chroma jel ($Y'(t)-R'(t), Y'(t)-B'(t)$) egyidőben történő átvitele során alapvető feladat a két analóg jel egyetlen jellé való átalakítása.
Erre az kvadratúra amplitúdómoduláció ad lehetőséget, amely egy olyan modulációs eljárás, ahol az információt részben a vivőhullám amplitúdójának változtatásával, részben annak fázisváltoztatásával kódoljuk (ezzel tehát két független jel vihető át egyszerre). 
Mind PAL, mind NTSC rendszer esetében az emberi látás színekre vett alacsony felbontását kihasználva a chroma jeleket erősen (PAL esetében pl. a luma jel ötödére, $1~\mathrm{MHz}$-re) sávkorlátozzák, ezzel az apró, nagyfrekvencián reprezentált részleteket kisimítják. 
Ezután a kvadratúramodulált chroma jeleket pl. PAL esetén
\begin{equation}
c^{\mathrm{PAL}}(t) = \underbrace{U'(t)}_{\left( B'- Y'\right) / 2.03} \cdot \sin \omega_c t + \underbrace{V'(t)}_{\left( R'- Y'\right) / 1.14}  \cdot \cos \omega_c t
\label{Eq:PAL_cr}
\end{equation}
alakban állíthatjuk elő, ahol $\sin \omega_c t$ az ún. \textbf{színsegédvivő}, $\omega_c$ a színsegédvivő frekvencia, $U'(t)$ az ún. fázisban lévő, $V'(t)$ pedig a kvadratúrakomponens.
A kvadratúramodulált színjelek tehát egyszerűen az átskálázott színkülönbségi jelek fázisban és kvadratúrában lévő színsegédvívővel való modulációjával állítható elő.

A színjelek demodulációja koherens (fázishelyes) vevővel egyszerű alapsávba való lekeveréssel és aluláteresztő szűréssel valósítható meg:
\begin{align}
\begin{split}
\sin x \cdot \sin x = \frac{1-\cos 2x}{2}&,\hspace{1cm}
\cos x \cdot \cos x = \frac{1+\cos 2x}{2} \\
\sin x &\cdot \cos x = \frac{1}{2}\sin 2x
\end{split}
\end{align}
trigonometrikus azonosságok alapján $U'(t)$ demodulációja
\begin{multline}
c^{\mathrm{PAL}}_{\mathrm{QAM}}(t)\cdot \sin \omega_c t = U'(t)\cdot \sin \omega_c t\cdot \sin \omega_c t + V'(t) \cdot \cos \omega_c t  \cdot	\sin \omega_c t = \\
\frac{1}{2} U'(t) -
\underbrace{ \xcancel{ \frac{1}{2} U'(t)\cos 2 \omega_c t  + V'(t) \cdot \frac{1}{2}\sin 2 \omega_c t }}_{\text{aluláteresztő szűrés}}
\end{multline}
szerint történik, míg $V'(t)$ demodulálása hasonlóan $\sin \omega_c t$ lekeverés szerint.
A megfelelő demodulációhoz tehát a vevőben a színsegédvivő fázishelye, koherens előállítása elengedhetetlen.
\begin{figure}[]
	\centering
	\hspace{4mm}
	\begin{overpic}[width = 0.70\columnwidth ]{figures/QAM_mod_demod.png}
	\end{overpic}
	\caption{QAM moduláció és demoduláció folyamatábrája}
	\label{Fig:QAM_mod_demod}
\end{figure}

Az NTSC rendszerben a PAL-hoz hasonlóan a színjelek
\begin{equation}
c^{\mathrm{NTSC}}_{\mathrm{QAM}}(t) = I'(t) \cdot \sin \omega_c t + Q'(t) \cdot \cos \omega_c t
\end{equation}
alakban kerültek átvitelre, ahol az in-phase és kvadratúra komponensek rendre
\begin{align}
\begin{split}
I'(t) &= k_1 (R'-Y') + k_2 (B'-Y) ,\\ 
Q'(t) &= k_3 (R'-Y') + k_4 (B'-Y).
\end{split}
\end{align}
A $k_{1-4}$ konstansokat úgy választották meg, hogy az in-phase és kvadratúra modulált jelek nem közvetlenül a kék és piros merőleges bázisvektorok (ld. \ref{Fig:ycbcr_gamut} ábra), hanem ezek kb. $+20^{\circ}$ elforgatottja.
Az így kapott új tengelyek a magenta-zöld és türkiz-narancssárga tengelyek a közvetlen modulálójelek.
Ennek oka, hogy úgy találták, az emberi látás felbontása jóval nagyobb türkiz-narancssárga közti változásokra, mint a magenta-zöld között.
Ezt kihasználva a magenta-zöld $Q'(t)$ színjeleket az $I'(t)$ jelhez képest is jobban sávkorlátozták, sávszélesség-takarékosság céljából.
A PAL rendszer bevezetésének idejére azonban kiderült, hogy ez rendszer felesleges túlbonyolítása, így az új rendszerben megmaradtak az eredeti színkülönbségi jelek modulációjánál.

\vspace{3mm}
Vizsgáljuk végül a modulált színjel fizikai jelentését, az egyszerűség kedvéért $c^{\mathrm{PAL}}(t)$ esetére (PAL rendszerben)!
Az \eqref{Eq:PAL_cr} egyenlet egyszerű trigonometrikus azonosságok alapján átírható a 
\begin{equation}
c^{\mathrm{PAL}}_{\mathrm{QAM}}(t) = \sqrt{U'(t)^2 + V'(t)^2} \, \sin \left( \omega_c t + \arctan \frac{V'(t)}{U'(t)} \right)
\end{equation}
polár alakra.
Minthogy a moduláló $U',V'$ jelek a színkülönbségi jelekkel arányosak, így a fenti kifejezést \eqref{eq:saturation_1} és \eqref{eq:hue}-val összehasonlítva megállapítható, hogy a QAM modulált jel egy olyan szinuszos vivő, amelynek pillanatnyi amplitúdója a továbbított színpont telítettségét, pillanatnyi fázisa a színpont színezetét adja meg.

\begin{figure}[]
	\centering
	\hspace{4mm}
	\begin{overpic}[width = 0.50\columnwidth ]{figures/SMPTE_Color_Bars.png}
\small
\put(-7	,0){(a)}
	\end{overpic} \hfill
	\begin{overpic}[width = 0.395\columnwidth ]{figures/vectorscope.png}
\small
\put(-10,0){(b)}
	\end{overpic}
	\caption{Egy gyakran alkalmazott vizsgálókép (SMPTE color bar) (a) és vektorszkóppal ábrázolva (b).}
	\label{Fig:bar_pattern_vscope}
\end{figure}

A színsegédvivő amplitúdójának és fázisának egyszerű értelmezhetősége miatt az NTSC és PAL jeleket gyakran vizsgálták ún. vektorszkóp segítségével jól meghatározott vizsgálóábrák megjelenítése mellett.
A vektorszkóp kijelzője gyakorlatilag a \ref{Fig:ycbcr_gamut} ábrán is látható $B'-Y', B'-Y'$ térben jeleníti meg a teljes képtartalom (azaz egyszerre az összes képpont) chroma jeleit, $Y'$-tól függetlenül a demodulált chroma-jelek megjelenítésével.
A vektorszkóp gyakorlatilag egy olyan oszcilloszkóp, amelynek $x$ kitérését a demodulált $B'-Y'$, $y$-kitérést a demodulált $R'-Y'$ jel vezérli, így a teljes képtartalom színezetét szinte egyszerre jeleníti meg az előre felrajzol vizsgálati rácson.
Egy tipikus vizsgáló ábra és annak vektorszkópos képe látható a \ref{Fig:bar_pattern_vscope} ábrákon.
A vektorszkóp alkalmazásának előnye, hogy az esetleges amplitúdó és fázishibából származó telítettség és színezethibák jól láthatóvá válnak a kijelzőn az egyes felvetített pontok ''összeszűkülése/tágulása'', illetve a teljes konstelláció elfordulásaként.
Megjegyezhető, hogy a mai digitális videojeleket is gyakran ábrázolják szoftveres vektorszkópon az egyes pixelek színezetének vizsgálatához.

\paragraph{A színsegédvivő frekvencia:}
Vizsgáljuk most, hogyan választható meg a színsegédvivő $\omega_c$ vivőfrekvenciája úgy, hogy a QAM modulált $c^{\mathrm{PAL}}(t)$ jelet a luma jelhez hozzáadva a vevő oldalon lehetséges legyen a vett $c^{\mathrm{PAL}}(t) + Y'(t)$ jelből az eredeti chroma és luma jelek szétválasztása!

A jelek vevőoldali szétválasztására a luma és chroma jelek spektruma ad lehetőséget:
Láthattuk, hogy a videójel az egyes TV sorokban megjelenítendő világosság és színinformáció sorfolytonos időtörténeteként fogható fel.
Természetes képeken a képtartalom sorról sorra csak lassan változik (természetesen a képtartalomban jelenlévő vízszintes éleket leszámítva), így mind a luma, mind a chroma jelek ún. kvázi-periodikusak, azaz közel periodikusak.
Jel- és rendszerelméleti ismereteink alapján tudjuk, hogy egy periodikus jel spektruma vonalas, a jelfrekvencia egész számú többszörösein tartalmaz csak komponenseket.
Ennek megfelelően mind a luma, mind a chroma jelek spektruma közel vonalas: az energiájuk a sorfrekvencia egész számú többszörösein csomósodik.
Természetesen a luma jel az alapsávban helyezkedik el ($0~\mathrm{Hz}$ környezetében), kb. $5.6~\mathrm{MHz}$ sávszélességben\footnote{Ez a sávszélesség eredményezi az azonos horizontális és vertikális képfelbontást.}.
A QAM modulált chroma jel spektruma a sávkorlátozás miatt keskenyebb ($1~\mathrm{MHz}$), és középpontját $\omega_c$ vivőfrekvencia határozza meg.
\begin{figure}[]
	\centering
	\hspace{4mm}
	\begin{overpic}[width = 0.80\columnwidth ]{figures/LC_interlace.png}
	\end{overpic} \hfill
	\caption{A luma és chroma jelek spektrális közbeszövésének alapelve a teljes spektrumokat ábrázolva (a) és a spektrális csomókat felnagyítva (b)}
	\label{Fig:YC_interlace}
\end{figure}

A luma-chroma jel összegzése ennek ismeretében egyszerű: 
Az $\omega_c$ vivőfrekvencia megfelelő megválasztásával elérhető, hogy a chroma jel spektrumvonalai (spektrumcsomói) éppen a luma jel spektrumvonalai közé essen, azaz a spektrumukat átlapolódás nélkül közbeszőhetjük.
Az eljárás alapötletét \ref{Fig:YC_interlace} ábra illusztrálja $f_{\mathrm{H}}$-val a sorfrekvenciát jelölve.
A szétválaszthatóság feltétele ekkor 
\begin{equation}
f_c = f_{\mathrm{H}} \cdot \left( \mathrm{n} + \frac{1}{2}\right), \hspace{1.5cm} \mathrm{n} \in \mathcal{N} 
\end{equation}
azaz a színsegédvivő frekvenciája a sorfrekvencia felének egész szűmú többszörösének kell, hogy legyen \footnote{Megjegyezhető, hogy PAL esetében az előre adott sorfrekvenciához egyszerű volt a színsegédvivő-frekvencia megválasztása, míg NTSC esetén bizonyos okok miatt a sorfrekvencia és ebből következően a képfrekvencia megváltoztatására volt szükség. 
Innen származnak a ma is használatos $59.94$ és $29.97~\mathrm{Hz}$ képfrekvenciák, amelyeket a következő fejezet tárgyal részletesen.}.

\paragraph{A CVBS kompozit videójel és luma-chroma szétválasztás:}
Ennek ismeretében végül a teljes kompozitjel a 
\begin{equation}
\text{CVBS}(t) = \mathrm{Sy}(t) + Y'(t) + c_{\mathrm{QAM}}(t)
\end{equation}
alakban áll elő, ahol $Y'$ a luma jel, $c_{\mathrm{QAM}}$ a QAM modulált chroma jelek és $\mathrm{S\!y}(t)$ a kioltási időben jelen lévő sorszinkron és képszinkron jelek.
A CVBS elnevezés gyakori szinoníma a kompozit videójelre, jelentése C: color, V: video (luma), B: blanking (azaz kioltás) és S: sync (azaz szinkronjelek).

Az így létrehozott videójel a fekete-fehér képhez képest csak a modulált színsegédvivőt tartalmazza többletinformációnak.
Egyszerű fekete-fehér vevőn a CVBS jelet megjelenítve a színinformáció nagyfrekvenciás zajként, pontozódásként (ún. \href{http://www.techmind.org/colrec/}{chroma dots}) jelenik csak meg a kijelzőn, így a visszafelé kompatibilitás biztosítva volt.
Színes vevőkben a CVBS jelből a luma és chromajel elméletileg fésűszűréssel szeparálható a sorfrekvencia felének egész számú többszöröseit elnyomva.
Ez ideálisan egy soridejű késleltetést igényel \footnote{A bizonyításhoz vizsgáljuk $h(t) = \delta(t) + \delta(t-t_{\mathrm{H}})$ szűrő átviteli karakterisztikáját, amely szűrő a jelből kivonja $t_H$-val késleltetett önmagát!}.
A fésűszűrős luma-chroma szeparáció lehetősége már az NTSC fejlesztésének idején ismert volt, azonban a szükséges soridejű késleltető nem állt rendelkezésre, ezért a korai NTSC vevők egyszerű alul/felüláteresztő szűrőkkel, vagy egyszerű chroma jelre állított lyukszűrőkkel szeparálták a luma-chroma jeleket.
Ennek eredményeképp még a színes vevőkben is a chroma jelen kisfrekvenciás tartalomként jelen lehetett a világosságinformáció látható \href{https://en.wikipedia.org/wiki/Dot_crawl}{hatással a megjelenített képre}.
A megfelelő analóg PAL fésűszűrő-tervezés még a 90-es években is aktív \href{https://www.renesas.com/in/en/www/doc/application-note/an9644.pdf}{K+F} alatt álló terület volt.

\begin{figure}[]
	\centering
	\begin{overpic}[width = 0.45\columnwidth ]{figures/ntsc_color_line.png}
	\end{overpic} \hfill
	\begin{overpic}[width = 0.48\columnwidth ]{figures/Waveform_monitor.jpg}
	\end{overpic} \hfill
	\caption{Az SMPTE color bar vizsgáló ábrának egy, illetve két sorának hullámformája sematikusan (a), és egy hulláforma monitoron (b) vizsgálva}
	\label{Fig:NTSC_line}
\end{figure}

Az elmondottak alapján az NTSC rendszerben a \ref{Fig:bar_pattern_vscope} ábrán látható vizsgálóábrának egy sorának kompozit ábrázolását a \ref{Fig:NTSC_line} mutatja be jellegre helyesen, és egy konkrét hullámforma monitoron mérve.
Az ábrán megfigyelhető az egyes oszlopokhoz tartozó hullámalak: látható, hogy a csökkenő világosságú oszlopokra (amelyek világosságát szaggatott vonal jelzi) hogyan ültették rá a QAM modulált chroma jeleket.
Az első és utolsó fehér, illetve fekete oszlop esetén a chroma jelek amplitúdója zérus (fehérpont), egyéb esetekben a szinuszos színsegédvivő amplitúdója az oszlopok színének telítettségével, fázisa a színezetükkel arányos.
Megjegyezzük, hogy a tényleges hullámforma már átskálázott chroma jeleket ábrázol, amely átskálázás épp azért történik, hogy a teljes CVBS jel beleférjen a fizikai interface dinamikatartományába (ez természetesen a nagy telítettségű színek esetén okozna problémát).
Ez magyarázza tehát az eddig figyelmen kívül hagyott 2.03 és 1.14 skálafaktorokat pl. \eqref{Eq:PAL_cr} esetében.


Az NTSC jel felépítése alapján egyértelmű, hogy a megfelelő színek helyreállításához a vevőben a színsegédvivő fázisának nagyon pontos ismerete szükséges.
Ahhoz, hogy ez biztosítva legyen a sorkioltási időben az ún. hátsó vállra (ld. \ref{Fig:PAL_line} ábra) beültetésre került néhány periódusnyi (9) képtartalom nélküli referenciavivő, az ún. color burst, vagy burst jel.
Ez a burst jel megfigyelhető a \ref{Fig:NTSC_line} ábrán is.

Ennek ellenére az NTSC rendszer továbbra is fázisérzékeny volt, hiszen fázishibát a vevőben is bármelyik alkatrész okozhatott.
A QAM moduláció jelege miatt már a legkisebb fázishiba is látható színezetváltozást okozott a megjelenített képen.
A PAL rendszer tervezésének egyik fő célja épp ezért a rendszer fázishibára vett érzékenységének csökkentése volt

\paragraph{A PAL rendszer:}
Míg az egyszerű NTSC rendszer már 1953-ban bevezetésre került Amerikában, addig Európában egészen az 1960-as éveikg vártak a színes műsorszórás bevezetésére.
Ennek oka, hogy az eltérő hálózati frekvencia miatt az NTSC-t nem lehetett egy az egyben átemelni Európába (ld. később).
Mire az európai rendszert kifejlesztették, az NTSC rendszer jó néhány gyengeségére fény derült, így az újonnan kifejlesztett PAL (Phase Alternate Lines) ezek kijavítását célozta főként meg.
Ennek eredményeképp a PAL rendszer más QAM modulációval dolgozik (a chroma jelek közvetlenül a modulálójelek), eltérő a színsegédvivő frekvencia, és legfontosabb újításként: egy egyszerű megoldással szinte érzéketlen a fázishibára.
\begin{figure}[]
	\centering
	\begin{overpic}[width = 0.45\columnwidth ]{figures/PAL1.png}
	\end{overpic} \hfill
	\begin{overpic}[width = 0.45\columnwidth ]{figures/PAL2.png}
	\end{overpic} \hfill
	\caption{Az SMPTE color bar vizsgáló ábrának egy, illetve két sorának hullámformája sematikusan (a), és egy hulláforma monitoron (b) vizsgálva}
	\label{Fig:PAL1}
\end{figure}

Láthattuk, hogy a vevő oldalán bármilyen fázishiba a színezet jól látható torzulását okozza.
Mivel a fázishiba gyakran elkerülhetetlen, ezért hatásának kiküszöbölésére a PAL rendszer a következő egyszerű megoldást alkalmazza:
\begin{itemize}
\item Az adó oldalon (a PAL jel létrehozása során) képezzük QAM moduláció során a V' chromajel előjelét minden második TV-sorban negáljuk meg, azaz sorról sorra fordított előjellel vigyük át (ez ekvivalens a sorról sorra változó $\pm \cos \omega_c t$ vivővel való modulációval)!
Az eljárás szemléltetésére tegyük fel, hogy két egymás utána sorban minden horizontális pozícióban a színinformáció azonos.
Ekkor egy adott pontra az n. és (n+1). sorban átvitt $U',V'$ jeleket a \ref{Fig:PAL1} (a) ábra szemlélteti pl egy lila képpont átvitele esetén.
\item Tegyük fel, hogy a vevő oldalon a vett jelhez $\Delta \alpha$ fázishiba adódik az átvitel és demoduláció során.
Természetesen a fázishiba hatására az így vett színvektor mind az n., mind az (n+1). sorban azonos irányba fordul az $U'-V'$ konstellációs diagramon (azaz a $R'-Y', B'-Y'$ síkon), ahogy az a \ref{Fig:PAL1} (b) ábrán látható.
\item A vevő oldalán forgassuk vissza minden második sorban a vett $V'$ komponens előjelét és képezzük az (n+1). sor és az n. sor átlagát.
Ezzel természetesen a színjelek vertikális felbontását csökkentjük (az átlagképzés az apró részleteket elsimítja), azonban ennek eredménye az emberi szem színezetre vett felbontása eredményeképp az információveszteség nem látható (a horizontális felbontás már egyébként is jelentősen lecsökkent az egyszerű sávkorlátozás hatására).
Könnyen belátható, hogy a két vektor átlagát képezve éppen az eredeti, hibamentes színvektort kapjuk eredményül.
Két sor esetén azonos sortartalom esetén tehát ezzel az egyszerű trükkel a fázishiba hatása teljesen kiküszöbölhető, míg levezethető, hogy változó sortartalom esetén a fázishiba az átlagvektor hosszának csökkenését okozza, tehát színezetváltozás helyett csak telítettségváltozást okoz.
\end{itemize}
A bemutatott módosított modulációs módszerrel még aránylag nagy fázishibák hatása is minimális hatással van a megjelenített képre.
Az ok, hogy mégis több, mint egy évtizedet kellett várni a PAL rendszer bevezetésére az volt, hogy a módszer alkalmazásához (az átlagolás elvégzéséhez) a videójel soridejű késleltetésére volt szükség.
Ez az 50-es években analóg módon nem megoldható probléma volt amely a PAL implementálását hátráltatta.

A PAL bevezetését végül az olcsón tömeggyártható ún akusztikus művonalak megjelenése tette lehetővé.
Ez az akusztikus művonal, vagy \href{https://www.google.com/search?q=PAL+delay+line&client=firefox-b-d&sxsrf=ALeKk03EUTzVwc7dkYJFnEK-nlEI_p3hng:1586379019108&source=lnms&tbm=isch&sa=X&ved=2ahUKEwi90Kav2tnoAhXJ-ioKHWz6AJcQ_AUoAXoECA0QAw&biw=1407&bih=675}{PAL delay line} egy egyszerű üvegtömb, amelyre egy piezo aktuátor és piezo vevő csatlakozik.
Az adó a TV chroma jelével arányos mechanikai rezgéseket (ultrahang) \href{https://www.youtube.com/watch?v=-qerYLM-eEg}{hoz létre}, amely többszörös visszaverődések után épp egy soridőnyi késleltetést szenvedve ér a vevő elektródához.
Az ultrahang alapú késleltetővonalak egészen a 90-es évek végéig a PAL dekóderek részét képezték.


\begin{figure}[]
	\centering
	\begin{overpic}[width = 0.82\columnwidth ]{figures/PAL_coder.png}
	\end{overpic} \hfill
	\caption{A PAL kódoló felépítése}
	\label{Fig:PAL_coder}
\end{figure}
Az egyszerű PAL kódoló felépítése az eddig elmondottak alapján a \ref{Fig:PAL_coder} ábrán látható.
Röviden összefoglalva, mind a PAL, mind NTSC esetén a kompozitjel létrehozása során a feladat a Gamma-torzított $R',G',B'$ jelekből az $Y',U',V$ (PAL) és $Y',I',Q'$ (NTSC) jelek létrehozása, majd az $U',V'$ és $I'Q'$ jelek megfelelő QAM modulációja. 
Az így létrehozott jeleket összeadva és a kioltási időben továbbított szinkronjelekkel ellátva előáll a CVBS kompozit jel.

\vspace{3mm}
A kompozit videójel fizikai interface megvalósítása szabványról szabványra változó.
Konzumer felhasználás (pl. kézikamerák, videólejátszók, DVD lejátszók) szempontjából a legelterjedtebb csatlakozó a sárga jelölésű RCA végződés, amely az esetleges kísérő hangtól szigetelve, külön érpáron továbbítja a kompozit videójelet.
\begin{figure}[]
	\centering
	\begin{minipage}[c]{0.6\textwidth}
	\begin{overpic}[width = 0.45\columnwidth ]{figures/Composite-video-cable.jpg}
	\end{overpic} 
		\begin{overpic}[width = 0.45\columnwidth ]{figures/s_video.jpg}
	\end{overpic} \end{minipage}\hfill
	\begin{minipage}[c]{0.4\textwidth}
	\caption{Konzumer alkalmazásokhoz használt sárga jelölésű RCA csatlakozó (a) és a luma-QAM chroma jeleket külön érpáron átvivő S-videó csatlakozó (b)}
	\label{Fig:composite_video}  \end{minipage}
\end{figure}

A kompozit és komponens jelek közti kompromisszumként az S-video formátum a luma és chroma jeleket külön érpáron viszi át.
Ezt leszámítva az interface jele teljesen a kompozit videóval azonosak, továbbíthat akár NTSC, akár PAL (akár SECAM) videókomponenseket:
A luma tehát változatlanul alapsávban, míg a chroma a színsegédvivővel modulálva kerül átvitelre.
A chroma jelek modulációja elkerülhetetlen, hiszen a két független színkülönbségi jel egy érpárra való ültetéséhez azokat legalább a sávszélességükkel megegyező frekvenciájú vivőjellel való moduláció szükséges az átlapolódás elkerüléséhez.
Az S-video szabvány csatlakozója a \ref{Fig:composite_video} (b) ábrán látható.


\subsection{A komponens videójel}

A videójel komponensenkét való tárolásának és továbbításának az ötlete kézenfekvő, annak ellenére, hogy a technológia ezt csak a kompozit formátum megjelenésénél jóval később tette lehetővé.
Komponens videó esetén a luma és chroma (ritkább esetben az $R'G'B'$ komponenseket) egymástól függetlenül, külön tároljuk és külön érpáron továbbítjuk, azaz a továbbított jelek gyakorlatilag az \ypbpr reprezentációk.

Az \ypbpr  ábrázolás esetén tehát a három átvitt jel a következő:
\begin{itemize}
\item $Y'$: az adott eszközfüggő színtérben ábrázolt luma komponens.
Megjegyezhető, hogy a szükséges szinkron impulzusok az \ypbpr rendszerben ugyanúgy jelen vannak az $Y'$ jelen.
Épp ezért pl. egy \ypbpr komponens bemenettel rendelkező megjelenítő $Y'$ bemenetére kompozit videójel szabadon ráköthető.
Ez esetben a fekete-fehér TV vevőhöz hasonlóan a TV képen a fekete-fehér képre zajként-pontozódásként (,,chroma dots'') érzékelve a megjelenített modulált színsegédvivőt.
\item $P'_{\mathrm{B}},P'_{\mathrm{R}}$: az aktuális analóg interface dinamikatartományához átskálázott (erősített) $B'-Y', R'-Y'$ chroma jelek.
Az interface-ek általában $\pm 0.5~\mathrm{V}$ feszültségszinteket definiálnak.
A $P$-rövidítés a kompozit videóból származik, ahol a színinformációt a színsegédvivő fázisa hordozza.
\end{itemize}

\begin{figure}[]
	\centering
	\begin{overpic}[width = 0.45\columnwidth ]{figures/1280px-Component-cables.jpg}
	\end{overpic} \hfill
	\begin{overpic}[width = 0.45\columnwidth ]{figures/YPBPR_signals.png}
	\end{overpic} \hfill
	\caption{A leggyakrabban alkalmazott $YP_bP_r$ komponens videójel továbbítására alkalmazott csatlakozó (a) és a komponens videójelek a korábbi színsáv-tesztábra esetén (b)}
	\label{Fig:comp_video}
\end{figure}
A külön kezelt luma-chroma információra egy példa a \ref{Fig:comp_video} (b) ábrán látható a már korábban is bemutatott SMPTE color bars tesztábra esetében.
A komponens videójel interface leggyakrabban a \ref{Fig:comp_video} (a) ábrán látható RCA csatlakozóhármas alkalmazásával került megvalósításra.

\hspace{3mm}
Az analóg interface-ek közül említést érdemel még elterjedtségük miatt két további videóinterface:
\begin{itemize}
\item A SCART (vagy EuroSCART, Syndicat des Constructeurs d'Appareils Radiorécepteurs et Téléviseurs rövidítése) interface mind kompozit videójelek, mind S-video, mind $RGB$ jelkomponensek kétirányú átvitelét lehetővé teszik sztereó hang és digitális jelzések továbbítása mellet.
A SCART csatlakozó jelentőségét az jelentette, hogy megjelenése előtt a TV vevők videó bemenetére nem létezett szabványos interface.
Gyakran a külső videójel-források jobb híján a TV vevő rádiófrekvenciás bemenetére csatlakoztak a kompozit videójelet valamilyen az adott jelforrásra előírt vivőfrekvenciára modulálva.
A SCART interface 1080p HD jel átvitelére is képes volt $YP_bPr$ formátumra a HDMI szabvány elterjedése előtt.
A tipikus 21-érintkezős SCART csatlakozó a \ref{Fig:scart_vga} (a) ábrán látható
\item A VGA (Video Graphics Array) a mai napig széles körben elterjedt interface videókártyák és megjelenítők (monitor, projektor, stb) közti analóg komponens videójel-átvitelre.
A VGA interface alapvetően az analóg megjelenítendő (legalábbis az adott videókártya színterében ábrázolt) $RGB$ komponenseket továbbítják a megjelenítő felé, külön ereken továbbítva a vertikális és horizontális (kép- és sor-) szinkronjeleket.
A VGA portot az utóbbi években már szinte teljesen leváltották a DVI, HDMI és DisplayPort digitális interface-ek.
\end{itemize}

\begin{figure}[]
	\centering
	\begin{minipage}[c]{0.63\textwidth}
	\begin{overpic}[width = 0.47\columnwidth ]{figures/scart.jpg}
	\end{overpic} \hfill
		\begin{overpic}[width = 0.4\columnwidth ]{figures/vga.jpg}
	\end{overpic} \end{minipage}\hfill
	\begin{minipage}[c]{0.35\textwidth}
	\caption{Analóg videótovábbításra alkalmazott SCART (a) és VGA (b) csatlakozók}
	\label{Fig:scart_vga}  \end{minipage}
\end{figure}

\section{Digitális videóformátumok}
	
\subsection{Az SD formátum}

Az első digitális videoformátumot a normálfelbontású, SD videót az ITU (akkoriban CCIR) alkotta meg 1982-ben a ITU-601 szabvány formájában \footnote{Munkájáért a CCIR 1983-ban tehcnikai Emmy díjat is kapott}.

Az SD formátum gyakorlatilag az eddig tárgyalt videojel komponensek digitális reprezentációjának tekinthető, azaz a \ref{fig:PAL_frame} látható videojel teljes egészében digitalizációra került kioltási intervallumokkal együtt, mind a luma és chroma komponensekre (más szóval az $Y'P_b'P_r'$ jelek közvetlen digitalizációjával kaphatjuk).
A digitalizált videojelek neve--ahogy arról már szó volt--$Y'C_b'C_r'$ jelek.
A jelek elvi előállítása az \ref{Fig:SD_production} ábrán látható.
\begin{figure}[]
	\centering
	\begin{overpic}[width = 0.8 \columnwidth ]{Figures/YCbCr_production.png}
	\end{overpic}
	\caption{Digitális SD jel előállítása a videokomponensek digitalizálásával}
	\label{Fig:SD_production}
\end{figure}


A digitalis reprezentáció egyes kérdéseit már a korábbiakban érintettük.
Nyitott kérdés még a soronkénti mintaszám meghatározása, amely a sorok számával együtt megadja az SD formátum felbontását (pixelszámát).
A feladat tehát az analóg videójel mintavételi frekvenciájának meghatározása.

A mintavételi frekvencia megválasztásánál a következő szempontokat vették figyelembe:
\begin{itemize}
\item Természetes törekvés volt, hogy a több évtizede egymás mellett létező NTSC és PAL rendszerre egyszerre alkalmazható legyen, azaz mind PAL, mind NTSC video digitális ábrázolását lehetővé tegye.
Emellett nyilvánvalóan a mintavételezést úgy kell végrehajtani, hogy mindkét rendszerben egy sorba egész számú mintavételi periódus (azaz pixel) férjen bele.
Ebből következik, hogy a mintavételi frekvencia a sorfrekvencia egész számú többszöröse kell, hogy legyen mind az NTSC, mind a PAL rendszerben, azaz
\begin{equation}
f_s = n \cdot f_H^{\mathrm{PAL}} = m \cdot f_H^{\mathrm{NTSC}},
\end{equation}
ahol $n, m$ egész számok.
Minthogy a sorfrekvenciák 
\begin{align}
f_H^{\mathrm{PAL}} &= 25 \cdot 625 = 15625~\mathrm{Hz} \\
f_H^{\mathrm{NTSC}} &= 30 \cdot \frac{1000}{1001} \cdot 525 = 15734.2~\mathrm{Hz} ,
\end{align}
ezek legkisebb közös többszöröse
\begin{equation}
144 \cdot f_H^{\mathrm{PAL}} = 143 \cdot f_H^{\mathrm{NTSC}} = 2.25~\mathrm{MHz}.
\end{equation}
A mintavételi frekvencia tehát $2.25~\mathrm{MHz}$ egész számú többszöröse.
\item Emellett a mintavételi tétel értelmében a mintavételi frekvencia az átlapolódás elkerülésének érdekében legalább a mintavett jel sávszélességének kétszerese kell, hogy legyen.
Korábban láttuk, hogy a luma jel sávszélessége $5.6~\mathrm{MHz}$, a chroma jeleké pedig ennek a fele.
\end{itemize}
A legkisebb frekvencia amire a két előbbi feltétel teljesül $13.5~\mathrm{MHz}$.
Ezt választották tehát a világosságjel mintavételi frekvenciájának, miközben a színkülönbségi jelek számára, figyelembe véve az emberi látás tulajdonságait, felezett mintavételi frekvenciát ($6.5~\mathrm{MHz}$) választottak.
Ez az európai rendszerben 1 sorra 864, az amerikaiban 858 teljes mintaszámot eredményez, amely a sorkioltási időt is tartalmazza.

\begin{figure}[]
	\centering
	\begin{overpic}[width = 0.65 \columnwidth ]{Figures/SD_formats.png}
	\end{overpic}
	\caption{Az SD formátum képmérete az amerikai és az európai rendszerben}
	\label{Fig:SD_format}
\end{figure}

A két rendszer további egységesítésének érdekében egy soron belül az aktív pixelek számát közösen 720 pixelre választották (amelyből csak 704 pixel tartalmaz tényleges képi adatot, a digitalizálás előtti analóg jel kezdetének bizonytalansága, szélekhez közeli torzításai, elmosódásai miatt).
Ezzel tehát megkaptuk az SD formátum tényleges képméretét, ahogy az az \ref{Fig:SD_format} ábrán látható.
Az aktív sorok száma alapján, és mivel mindkét rendszerben kizárólag interlaced videó definiált, a két formátum megjelölése \textbf{480i} és \textbf{576i}.
Könnyen belátható, hogy szabványos 4:3 képarány azonos horizontális pixelszám de különböző sorszám mellett csak úgy érhető el, ha az egyes képelemek (pixelek) nem négyzetalakúak (azaz a \textbf{pixel aspect ratio (PAR)} értéke 1-től különböző).
Számítástechnikában a monitorok ezzel szemben négyzetes pixelméretet definiáltak, így az elterjedt számítógépes SD formátum a jól ismert 640x480 pixelszám.
%TODO Pixel apsect ratio, storage aspect ratio, display aspect ratio kifejteni

\begin{figure}[]
	\centering
	\begin{overpic}[width = 0.4 \columnwidth ]{Figures/sd_gamut.png}
	\small
	\put(0,0){(a)}
	\end{overpic}
	\hspace{2mm}
	\begin{overpic}[width = 0.55 \columnwidth ]{Figures/sd_OETF.png}
	\small
	\put(0,0){(b)}
	\end{overpic}
	\caption{Az SD formátum gamutja(a) és Gamma-függvénye (b)}
	\label{Fig:SD_gamut}
\end{figure}

Röviden összefoglalva a jelen, és előző fejezetet a két SD formátum létrehozásának lépései és főbb tulajdonságai:
\begin{itemize}
\item A formátum primary színei és a színtér gamutja a \ref{Fig:SD_gamut} (a) ábrán látható.
A színtér fehérpontja D65 fehér.
A luma komponens számításának módja \footnote{Itt jegyezzük meg, hogy ,,matematikaiatlanul'', ezek a luma együtthatók az NTSC szabvány együtthatókból származnak, tradíció miatt a luma jel számítási módját nem változtatták meg az SD szabvány bevezetésével annak ellenére, hogy az alapszínek megváltoztak. 
Emiatt az $XYZ \rightarrow RGB$ mátrix második sora jelen esetben nem az itt bemutatott luma együtthatókat eredményezné.
Ez egy újabb példa arra, hogy a hasonló matematikai következetlenségek nem állnak távol a gyakorlatban alkalmazott videotechnikától.}
\begin{equation}
Y' = 0.299 R' + 0.587 G' + 0.112 B'
\end{equation}
\item A forrás RGB jelei a perceptuális kvantálás megvalósításának érdekében Gamma-torzításon mennek keresztül, ahol a Gamma-függvény, vagy Optoelectronic Transfer Function:
\begin{equation}
E = 
\begin{cases}
4.500 L, \hspace{20mm} \mathrm{ha}\, L < 0.018 \\
1.099 L^{0.45} - 0.099, \hspace{3mm} \mathrm{ha}\, L \geq 0.018,
\end{cases}
\end{equation}
ahol $L \in \{ R, G, B \}$.
A teljes görbe jól közelíthető egy $L^{0.5}$ függvénnyel
\item A formátum képaránya 4:3. 
A későbbiekben a HD megjelenése után ezt kiegészítették 16:9 képarányú formátummal is.
\item Kizárólag interlaced formátum definiált
\item A videókomponensek megfelelő sávkorlátozás után mintavételezésen és kvantáláson esnek át.
A kvantálás 8, vagy 10 biten történik.
\item A világosságjel mintavételi frekvenciája $f_s = 13.5~\mathrm{MHz}$.
Az 576i (625 soros, 50 félkép/s) rendszerben az aktív felbontás így 720x576 pixel, a 480i (525 soros, 60 félkép/s) rendszerben 720x480 px.
\item A színkülönbségi jel az eredeti, stúdióformátumban 4:2:2, tehát a horizontális színfelbontás a világosságjelének a fele.
Ezt később kiegészítették 4:2:0 struktúrával is konzumer célokra.
\end{itemize}

\subsection{A HD formátum}

%Whitaker 592. oldal
Az előzőekben részletesen tárgyaltuk as SD digitális videoformátum megalkotásának alapelveit.
A részletes vizsgálat oka, hogy ugyanezek az alapelvek, jelfeldolgozási lépések érvényesek a jelenlegi HD és UHD formátumok esetén is, valamint a jelenlegi SD műsorszórás Magyarországon is 576p (azaz már progresszív) formátumban történik.

Láthattuk, hogy az SD formátum megalkotásánál az egyes paramétereket úgy választották meg, hogy a kitűzött kb. 10 fokos látószögben minél élethűbb képi reprodukciót lehessen megvalósítani.
A HD és UHD formátumok tárgyalása előtt vizsgáljuk meg, hogy adott felbontás (pixelméret) mellett mekkora távolságból kell az adott kijelzőt megfigyelni, rávilágítva ezzel a HD formátum létrehozásának fő motivációjára.

\paragraph{Optimális nézőtávolság:\\}

\begin{figure}[]
	\centering
	\begin{overpic}[width = 0.67 \columnwidth ]{Figures/hd_pixel_angle_mod.png}
	\small
	\end{overpic}
	\caption{Geometria az optimális nézőtávolság származtatásához}
	\label{Fig:optimal_vd}
\end{figure}

Általánosan elmondható, hogy pixel alapú képi reprodukció során a fő szempont, hogy a szomszédos pixelekből érkező fénysugarak által bezárt szög az emberi szem felbontóképessége alá essen.
Ezzel biztosítva van, hogy a kijelző pixelstruktúrája nem látható (a kép nem ,,pixeles''), valamint az RGB alapszíneket alkalmazó reprodukció is lehetővé válik, hiszen az egyes alapszínek érzékelése helyett az additív színkeverés a szemben megvalósul.

Korábban láthattuk, hogy az emberi szem felbontása 1 szögperc (azaz $\frac{1}{60}^{\circ}$) (legalábbis a világosságjelre véve, segítségünkre van, hogy színezetre ennél is rosszabb).
Adott pixelméretre természetesen ebből már meghatározható az a minimális nézőtávolság, amelyre az előbbi feltétel teljesül.
Mivel jellemzően a kijelzőknek nem a pixelmérete van megadva, hanem a kijelző mérete és a vertikális, ill. horizontális pixelszám, ezért célszerű a fenti minimális nézőtávolságot ezek függvényében kifejezni.

Vizsgáljuk az \ref{Fig:optimal_vd} ábrán látható geometriát adott $H$ magasságú, $N_V$ sorszámú kijelző esetén.
A pixelméret ekkor természetesen $\frac{H}{N_V}$.
A kijelző a megfigyelőtől $D$ távolságra helyezkedik el.
A szomszédos (szemközti) pixelekből a megfigyelő szemébe érkező fénysugarakra felírható ekkor a 
\begin{equation}
\tan \frac{\Phi}{2} = \frac{H}{2 N_V D}
\end{equation}
egyenlőség.
Alkalmazzuk a tangens függvény kisargumentumú lineáris közelítését, azaz $\tan x \approx x$, ha $x \ll 1$.
Ekkor
\begin{equation}
\Phi = \frac{H}{N_V D} \hspace{3mm} \rightarrow \hspace{3mm} D = \frac{H}{N_V \Phi}
\end{equation}
érvényes.
Az emberi szem felbontását $\frac{1}{60}\cdot \frac{\pi}{180}~\mathrm{rad} = 2.9 \cdot 10^{-4}$ behelyettesítve adott felbontású és méretű kijelző esetén az optimális (minimális) nézőtávolságra
\begin{equation}
D = H \frac{1}{N_V \,  2.9 \cdot 10^{-4}}
\end{equation}
adódik. 
Ez a távolság az ún. Lechner-távolság, amely tehát megadja, hogy a tervezés során figyelembe vett képméret és felbontás mellett mekkora az optimális nézőtávolság adott képformátum esetén.
\begin{table}[h!]
\caption{Fontosabb SD és HD formátumok ideális nézőtávolsága és az így kitöltött horizontális látószög}
\renewcommand*{\arraystretch}{2.25}
\label{tab:viewing_dist}
\begin{center}
    \begin{tabular}[h!]{ @{}c | | l | l | l @{} }%\toprule
				         &   Amerikai 		   & Európai 				&	 HDTV \\ \hline
    TV-sor/képmagasság:	 &     480 	  		   &   576   				&	 1080\\
    Nézőtávolság:   &  7-szeres képmagasság    &  6-szoros képmagasság & 3-szoros képmagasság \\
    Nézőtávolság:       &  4.25-szörös képátló &  3.6-szoros képátló	& 1.5-szörös képátló \\
    Vízszintes látószög &  kb. 11 fok 		&    kb. 13 fok & kb. 32 fok\\
    \end{tabular}
\end{center}
\end{table}

Amennyiben az $a_r$ képátló ismert (SD esetén 4:3, HD esetén 16:9), az összefüggés kifejezhető a képszélesség függvényében is.
SD esetén ez
\begin{equation}
D = \frac{W}{a_r} \frac{1}{N_V \,2.9 \cdot 10^{-4}},
\end{equation}
ahol $W$ a kép szélessége.
Ekkor ha a kijelzőt az így kapott optimális távolságról nézzük, meghatározható a képernyő által bezárt vízszintes látószög:
\begin{equation}
\tan \frac{\Phi_H}{2} = \frac{W}{2 D} \hspace{1cm} \rightarrow \hspace{1cm} D = \frac{W}{2 \tan \frac{\Phi_H}{2}}, 
\end{equation}
és így adott felbontás mellett a horizontális látószög:
\begin{equation}
\Phi_H = 2\arctan \left( \frac{a_r \, N_V \, 2.9\cdot 10^{-4}}{2} \right).
\end{equation}

Az eredményeket az eddig bemutatott SD és a következőekben tárgyalt HD formátumokra kiszámítva a \ref{tab:viewing_dist} táblázat foglalja össze.
Láthatjuk, hogy az SD felbontást ideálisan a képmagasság 6-7-szereséről célszerű nézni.
Ekkor valóban, a formátum tervezésének kiindulási pontjába érünk vissza, azaz a kijelző a fő látóterünket, kb. 10-13 fokot tölti ki horizontálisan.

Ez már előrevetíti a HD formátum megalkotásának fő célját: a vizuális élmény fokozását nagyobb kitöltött látószög alkalmazásával.
A HD formátum célja tehát---ellentétben a közhiedelemmel---nem a pixelben kifejezett felbontás növelése, és így azonos felületre minél nagyobb számú képpont belezsúfolása, hanem az otthoni vizuális élmény növelése a tartalommal lefedett látótér megnövelésével.

\paragraph{Rövid HD történelem:\\}

A nagyfelbontású (High Definition) formátum létrehozása gyakorlatilag a televíziózás megjelenése óta a teljes XX. századon átívelt.
Noha manapság a kifejezés jellemzően digitális formátumra utal, már a korai analóg technika korában is léteztek HD kezdeményezések.
Olyannyira, hogy már 1949-ben, Franciaországban kísérleti műsorszórást kezdtek monokromatikus (fekete-fehér), de 819 sort alkalmazó analóg rendszerben (a műsorszórás ebben a formátumban egészen 1983-ig tartott).
A Szovjetunióban kísérleti jelleggel 1958-ban kifejlesztettek egy színes, 1125 sorból álló analóg rendszert, a hadászati célra létrehozott technikát azonban végül a gyakorlatban nem hasznosították.
Végül a japán NHK cég vezette be az első mai értelemben vett HD műsorszórást 1989-ben.
Rendszerük, az ún. Hi-Vision, vagy MUSE (Multiple sub-Nyquist Sampling Encoding) 5:3-as képarányú, 1125 soros analóg interlaced videó műsorszórására volt képes.
A japán HD műsorszórás erőteljesen ösztönözte a HD formátum szabványosítását, amely azonban az analóg rendszer hatalmas sávszélességigénye miatt egészen a 90-es évek elejéig nem valósult meg.

A HD szabvány létrejötte végül a digitális tömörítési módszerek, főként az MPEG-1 és MPEG-2 tömörítések megjelenésének köszönhető.
A szabványt 1990-ben tették közre az \href{https://www.itu.int/dms_pubrec/itu-r/rec/bt/R-REC-BT.709-6-201506-I!!PDF-E.pdf}{ITU-709} (Rec. 709) ajánlásban.

\paragraph{HD paraméterek:\\}

Az ITU-709 szabvány a következő HD paramétereket határozta meg:
\begin{itemize}
\item \textbf{Képarány:} az első szabványosított formátumjellemző a HD rendszer képaránya volt, amelyet 16:9 értékűre választottak.
A választás nem kézenfekvő, mivel a formátum létrehozásakor gyakorlatilag nem állt rendelkezésre ebben a képarányban nyersanyag: mind a mozifilmek, mind a korábbi SD formátumú anyagok ettől eltérő képformátumban kerültek rögzítésre.
A korabeli nyersanyagok jellegzetesen 4:3 (SD videó), 15:9, 1.85:1, 2.2:1, illetve 2,35:1 (mozis szabványok) voltak.
Azt találták, hogy amennyiben azonos területű téglalapokat vetítünk egymásra a fenti, gyakori oldalarányokkal, akkor az így kapott téglalap sokaság éppen egy 16:9 oldalarányú téglalapba rajzolható bele.
Továbbá az ezen téglalapok metszete szintén éppen egy 16:9 arányú téglalapot határoz meg.
A geometria a \ref{Fig:kerns_powers} ábrán látható.
\begin{figure}[]
	\centering
	\begin{overpic}[width = 0.9 \columnwidth ]{Figures/KernsPowers.png}
	\small
	\end{overpic}
	\caption{A 16:9 oldalarány, mint sok gyakori képarányú téglalap határoló-síkidoma és metszete}
	\label{Fig:kerns_powers}
\end{figure}
A gondolatmenet eredményeképp, annak érdekében, hogy a legtöbb létező képarányú korabeli nyersanyag optimálisan megjeleníthető legyen az új formátumban esett a választás a ma már jól ismert 16:9 képarányra.

\item \textbf{Képfrissítési frekvencia:} Az SD rendszer hagyatékaként a HD szabvány számos képfrissítési frekvenciát támogat, így a $24~\mathrm{Hz}$, $25~\mathrm{Hz}$, $30~\mathrm{Hz}$, $50~\mathrm{Hz}$ és $60~\mathrm{Hz}$ képfrekvenciákat, és ezek $\frac{1000}{1001}$-szeres módosításait.
Ezen kívül, ezek a frissítési frekvenciák mind progresszív, mind váltottsoros módban alkalmazhatóak.

\item \textbf{Mintavételi frekvencia:}
A mintavételi frekvencia esetében kiindulásul az SD mintavételi frekvencia szolgált.
A HD formátum célja SD-hez képest mind függőleges ($\times 2$), mind vízszintes ($\times 2$) felbontásduplázás, ezen felül a 4:3 képarány helyett 16:9 alkalmazása ($\times 4/3$), amely együttesen a mintaszám--és így a mintavételi frekvencia---legalább $2\times 2 \times \frac{4}{3} = 5.33$-szorozódását jelenti.
A legtöbb képfrekvenciához való kompatibilitás biztosítása érdekében (azaz egy sorba egész számú minta férjen) a mintavételi frekvencia $5.5 \cdot 13.5 = 74.25~\mathrm{MHz}$ lett, illetve törtszámú képfrekvencia esetén ennek az $\frac{1000}{1001}$-szerese.
Progresszív esetben $50-60~\mathrm{Hz}$ képfrekvencia esetén a mintavételi frekvencia ennek a duplája: $fs = 148.5~\mathrm{MHz}$.
%TODO : kibogarászni KovácsI jegyzetéből a mintavételi frek. megválasztását (VidStTech 01.pdf, 46.o)
\item \textbf{Felbontás:} Hosszas egyeztetések után\footnote{Az első HD szabvány a Japán rendszer nyomán 1035 sort definiált. Ezt később visszavonták} a szabvány 1080 aktív sort definiált 1125 teljes sorszámmal mind progresszív, mind váltottsoros letapogatás mellett (SMPTE 274M szabvány).
Emellett a különböző képfrekvenciák egységes kezelése érdekében a soronkénti mintaszámot fixen 1920 mintára választották, így a HD formátum felbontása 1920x1080 pixel.
A különböző képfrekvenciájú HD formátumok így tehát kizárólag a soronkénti inaktív pixelek számában különböznek egymástól (azaz a sorkioltási idő hosszában).

\begin{table}[h!]
\caption{Néhány HD formátum mintavételi frekvenciája, összes sor és oszlopszáma, felbontása. 
A soronkénti pixelek száma $S_{\mathrm{LT}} = $} alapján számítható.
\renewcommand*{\arraystretch}{2.25}
\label{tab:viewing_dist}
\begin{center}
    \begin{tabular}[h!]{ @{}c | l | l | l | l | l @{} }%\toprule
		Rendszer     &   $f_s \, [\mathrm{MHz}]$ 				& $S_{TL}$ & $L_T$ & Aktív pixelszám \\ \hline
		$720p50$     &   $74.25~\mathrm{MHz}$    				& 1980     &  750  & $1280 \times 720$ \\
		$720p59.94$  &$74.25\cdot\frac{1000}{1001}~\mathrm{MHz}$& 1650     &  750  & $1280 \times 720$ \\
		$1080i25$ 	 &   $74.25~\mathrm{MHz}$    				& 2640     & 1125  & $1920 \times 1080$ \\
		$1080i30$  	 &   $74.25~\mathrm{MHz}$    				& 2200     & 1125  & $1920 \times 1080$ \\
		$1080p50$ 	 &   $148.5~\mathrm{MHz}$    		    	& 2640     & 1125  & $1920 \times 1080$ \\
		$1080p59.94$ &$148.5\cdot\frac{1000}{1001}~\mathrm{MHz}$& 2200     & 1125  & $1920 \times 1080$ 
        \end{tabular}
\end{center}
\end{table}

A különböző HD formátumok jelölése a következő:
\begin{itemize}
\item Teljes aktív felbontás pixelben kifejezve.
Gyakran rövidítésképp csak a vertikális méretet jelölik meg.
\item képletapogatás módja: $p$ a progresszív, $i$ az interlaced letapogatást jelöli
\item képfrekvencia (frame rate) mind $p$, mind $i$ esetben.
Interlaced videó esetén gyakran a képfrekvencia helyett---hibásan---a félképfrekvenciát jelölik.
\end{itemize}
Így pl. az $1080i25$ a 1080 soros, váltottsoros $25~\mathrm{Hz}$ képfrekvenciájú, és $50~\mathrm{Hz}$ félkép-frekvenciájú formátumot jelöli.
\begin{figure}[]
	\centering
	\begin{overpic}[width = 1 \columnwidth ]{Figures/HD_format.png}
	\small
	\end{overpic}
	\caption{Az 1080 soros és 720 soros HD formátum szemléltetése}
	\label{Fig:HD_formats}
\end{figure}

A progresszív HD formátumok nagy adatsebesség-igénye miatt az ITU-709-et néhány évvel később kibővítették egy alacsonyabb felbontású formátummal, ez 720 aktív sort és 1280 aktív pixelt alkalmaz, és kizárólag progresszív letapogatással definiálták \footnote{Az 1080p bevezetése során az egyik kitűzött cél a legalább duplázott sorszám volt.
A 720p az SD és HD között félúton: másfélszeres sort alkalmaz így a sorszáma $\frac{3}{2} \cdot 480 = 720$-ból adódik, míg az oszlopszám a 16:9 képarányból, négyzetes pixelek mellett}.
Jelölése: \textbf{720p}.
A két HD formátumra egy-egy példa a \ref{Fig:HD_formats} ábrán látható.
\item A HD formátum alapszínei az SD ITU-601-el megegyezőek, így a gamutja is azonos.
A luma komponens számításának módja 
\begin{equation}
Y' = 0.2126 R' + 0.7152 G' + 0.0722B'.
\end{equation}
\begin{figure}[]
	\centering
	\begin{overpic}[width = 1\columnwidth ]{Figures/hd_line.png}
	\end{overpic} 
	\caption{1080 soros HD formátum egy sorának felépítése}
	\label{Fig:hd_line}
\end{figure}
Az együtthatók az SD-vel ellentétben már kolorimetriailag is helyesek, azaz a színtér alapszíneiből (és fehérpontjából) felírható $RGB \hspace{3mm} \rightarrow \hspace{3mm} XYZ$ transzformációs mátrix $Y$ sorából ugyanezeket a világosság-együtthatókat kapnánk.
\item A szabvány Gamma-karakterisztikája az SD-vel megegyező:
\begin{equation}
E = 
\begin{cases}
4.500 L, \hspace{20mm} \mathrm{ha}\, L < 0.018 \\
1.099 L^{0.45} - 0.099, \hspace{3mm} \mathrm{ha}\, L \geq 0.018,
\end{cases}
\end{equation}
ahol $L \in \{ R, G, B \}$.
\item A szabvány az SD-vel azonos 8 és 10 bites digitális reprezentációt ír elő.
\item A stúdió szabvány alapvetően 4:2:2 szín-mintavételezési struktúrát definiál.
\end{itemize}
A szabványos HD videójel ezek mellett teljesen az SD-vel azonos felépítésű, egy egyszerű példa a \ref{Fig:hd_line} ábrán látható.
Az egyetlen különbség gyakorlati megvalósítás szempontjából, hogy a kioltási időszakokban a szinkron impulzusok ún. 3 állapotúak ($0, \pm300~\mathrm{mV}$).

\paragraph{Tömörítetlen videó adatsebesség:\\}

Vizsgáljuk most néhány alapvető képformátum esetén a videó tömörítetlen adatsebességét!
Az aktív és teljes pixelszámokat a \ref{Fig:size} ábrán látható jelölésekkel jelölve a teljes adatsebesség
\begin{equation}
B\!R_T = \underbrace{S_{TL} \cdot L_T \cdot f_f \cdot n_{\mathrm{bit}}}_{\text{mintánkénti bitrate}} \cdot n_{\mathrm{CS}},
\end{equation}
ahol $f_f$ a képfrekvencia, $n_{\mathrm{bit}}$ a mintánkénti bitszám és $n_{\mathrm{CS}}$ a színkülönbségi jelek alulmintavételezését jelöli (komponens/minta).
Utóbbi értéke 4:4:4 struktúra esetén $n_{\mathrm{CS}} = 3$, 4:2:2 estén $n_{\mathrm{CS}}=2$, 4:2:0 esetén $n_{\mathrm{CS}} = 1.5$.

Hasonlóan, az aktív tartalom bitsebessége
\begin{equation}
B\!R_A = S_{TA} \cdot L_A \cdot f_f \cdot n_{\mathrm{bit}} \cdot n_{\mathrm{CS}}
\end{equation}
alapján számítható.

Néhány gyakran alkalmazott videóformátum teljes és aktív videósebessége a \ref{tab:bitrate} táblázatban látható\footnote{Az UHD formátum kioltási idejei \href{http://programmersought.com/article/7908103552/}{szabványosan} 90 inaktív sor és 560 inaktív pixel/sor}.
\begin{figure}[]
\captionsetup{singlelinecheck=off}
	\centering
	\begin{minipage}[c]{0.4\textwidth}
	\begin{overpic}[width = 1\columnwidth ]{Figures/size.png}
	\end{overpic}   \end{minipage}\hfill
		\begin{minipage}[c]{0.55\textwidth}
	\caption[]{Jelölés-konvenció az adatsebesség számításához
	\begin{itemize}
	\item $S_{TL}$: Összes mintaszám/sor
	\item $S_{TA}$: Aktív mintaszám/sor
	\item $L_T$: Összes sor/kép
	\item $L_A$: Aktív sor/kép
	\end{itemize}}
	\label{Fig:size}  \end{minipage}
\end{figure}
%
\begin{table}[h!]
\caption{Videóformátum mintavételi frekvenciája, összes sor és oszlopszáma, felbontása, $n_{\mathrm{bit}} = 10$ bitmélység esetén}
\renewcommand*{\arraystretch}{2.25}
\label{tab:bitratet}
\begin{center}
    \begin{tabular}[h!]{ @{}c | l | l | l | l | l @{} }%\toprule
\thead{Rendszer} & \thead{Mintavételi \\ frekvencia} &    \thead{Teljes bitrate \\ 4:2:2} & \thead{Aktív bitrate\\ 4:2:2} & \thead{Aktív bitrate\\ 4:4:4} \\ \hline
$576p50$     &   $13.5~\mathrm{MHz}$        & $0.54~\mathrm{Gbit/s}$  & $0.41~\mathrm{Gbit/s}$ & $0.62~\mathrm{Gbit/s}$ \\
$720p60$     &   $74.25~\mathrm{MHz}$    	& $1.49~\mathrm{Gbit/s}$  & $1.11~\mathrm{Gbit/s}$ & $1.67~\mathrm{Gbit/s}$ \\
$1080i30$ 	 &   $74.25~\mathrm{MHz}$    	& $1.49~\mathrm{Gbit/s}$  & $1.24~\mathrm{Gbit/s}$ & $1.86~\mathrm{Gbit/s}$ \\
$1080p60$ 	 &   $148.5~\mathrm{MHz}$    	& $2.97~\mathrm{Gbit/s}$  & $2.49~\mathrm{Gbit/s}$ & $3.73~\mathrm{Gbit/s}$ \\
$2160p60$ 	 &   $297~\mathrm{MHz}$         &  $11.88~\mathrm{Gbit/s}$& $9.96~\mathrm{Gbit/s}$ & $14.93~\mathrm{Gbit/s}$
\end{tabular}
\end{center}
\end{table}
%
Látható, hogy a $720p$ és $1080i$ formátumok tömörítetlenül azonos adatmennyiséget generálnak, ugyanakkor nagy előnye a progresszív formátumnak jóval hatékonyabban tömöríthetősége.
Ahogy korábban a váltott-soros formátum előnye ezzel szemben, hogy állóképekre a progresszívvel azonos vertikális felbontást biztosít, bár gyors mozgásokra ez a felbontás romlik.
Épp ezért, azon műsorszolgáltatók, amelyeknél a sport-tartalom elsődleges jellemzően a $720p50/60$ formátumot használ, míg a főként filmeket, hírműsorokat sugárzó operátorok jellemzően $1080i$-t alkalmaznak.
Magyarországon jelenleg a HD adások mellett minden szolgálgató biztosít SD felbontású verziót is, amely jellemzően $576p50$ formátumot alkalmaz.

Jelenleg stúdiótechnikában a legelterjedtebb digitális videóinterface az SDI (Serial Digital Interface), konzumer elektronikában pedig a HDMI (High-Definition Multimedia Interface).
A fenti adatsebességek kézzelfoghatóvá-tételéhez, néhány jelenleg is használt HDMI verzió a következő adatmennyiségek továbbítását teszi lehetővé:
\begin{itemize}
\item HDMI 1.0-1.2: 4.95 Gbit/s (3.96 Gbit/s tényleges)\footnote{A HDMI szabvány itt nem részletezett okokból ún. 8b/10b csatornakódolást alkalmaz, amely során 8 bitnyi adatot 10 biten visz át.
A 4.95 Gbit/s teljes sávszélességnek tehát csak $\frac{8}{10}$ része használható ki tényleges adatra.}
\item HDMI 2.0: 18 Gbit/s (14.4 Gbit/s tényleges)
\item HDMI 2.1: 48 Gbit/s (38.4 Gbit/s tényleges)
\end{itemize}
Természetesen a HDMI méretezéshez a fenti táblázatból a teljes adatsebességet kell tekinteni, hiszen a HDMI kábelen terjedő HD jel tartalmazza a kioltási időket is: mint korábban tárgyaltuk ezekben az időrésekben kerülnek az audio csatornák és egyéb járulékos adatok elhelyezésre.
Látható, hogy a HDMI 1.0 szabványt főként 1080p videó továbbítására fejlesztették ki 4:2:2 formátum 10, vagy 12 bit, míg 4:4:4 formátum már csak 8 biten továbbítható.
A HDMI 2.0 szabványt már 4k míg a 2.1 szabvány 8k UHD videó továbbítására fejlesztették ki.

%TODO 8b/10b codin
% Fischer Digital Video and Audio broadcasting technology 227.o

%TODO HD ready
%TODO Gaumt coverage percentage, Surface colors
%TODO In typical production practice the encoding function of image sources is adjusted so that the final picture has the desired aesthetic look, as viewed on a reference monitor with a gamma of 2.4 (per ITU-R BT.1886) in a dim reference viewing environment (per ITU-R BT.2035).[10][11][12]
%TODO Rec. 2100, a standard for HDTV and UHDTV with high dynamic range

\subsection{Az UHD formátum}

Láthattuk, hogy a HD formátum megalkotása során a fő cél a néző látóterének---az SD-hez képest---nagyobb részének tartalommal való kitöltése volt.
A HD-hez hasonlóan az UHD (Ultra High Definition) formátum a vizuális élmény továbbfokozását tűzte ki fő céljául a pixel-szám---és így a kitöltött látószög---további növelésével.

Hasonlóan a HD-hoz, a továbbemelt felbontású formátum fejlesztése a japán NHK nevéhez kötődik, akik már 2003-ban kísérleti UHDTV felvételt készítettek 16 darab HDTV rögzítővel és 4 darab 3840x2048 felbontású CCD szenzorral.
Az első UHDTV szabvány ezek után már 2007-ben megjelent (SMPTE 2036), majd a jelenleg is érvényben lévő \textbf{ITU-R BT.2020} szabványt 2012-ben fogadták.
A szabvány két új formátum paramétereit kodifikálja, a 4k formátumot, amely az 1080 soros HD formátumhoz képest mindkét dimenzió mentén duplázott felbontást (így 4-szeres teljes pixelszámot) alkalmaz, és a 8k-t, amely a 4k-hoz képest kétszeres felbontást definiál.
\begin{figure}[]
	\centering
	\begin{overpic}[width = 0.85 \columnwidth ]{Figures/fov_formats.png}
	\small
	\end{overpic}
	\caption{Az egyes formátumok által biztosított horizontális látószög, a kijelzőt ideális nézőtávolságból nézve.}
	\label{Fig:fov_formats}
\end{figure}

A szabvány célja szerint ideálisan a 4k formátum a néző horizontális látószögéből kb. $58^{\circ}$-ot, a 8k formátum $96^{\circ}$-ot tölt ki, azaz már a periférikus látás jelentős részét is tartalommal tölti ki.
Az egyes formátumok által ideálisan kitöltött látószög változását a \ref{Fig:fov_formats} ábra szemlélteti.

Ahhoz, hogy az ekkora látószögben megfelelő minőségű képi reprodukció valósuljon meg, az ITU-2020 szabvány számos szempontból javította a HD formátum alapparamétereit:
\begin{itemize}
\item \textbf{Felbontás: } A szabvány két felbontást deifiniál, ezek a 3840x2160 (4k) és a 7680x4320 (8k) aktív képméretek.
A HD szabványnak megfelelően mindkét formátum képaránya 16:9 és négyzetes pixelekből áll.
\item A HD-vel ellentétben az ITU-2020 szabvány már kizárólag csak progresszív letapogatást támogat (így a $p/i$ formátummegjelölés UHD esetén okafogyott.
Fontos, hogy a kijelző már optimális esetben a néző periférikus látásának jelentős részét is kitölti, ahol az érzékelést már a szem pálcikái dominálják.
Mivel ezek fúziós frekvenciája jóval nagyobb, mint a fő látóterünké, ezért a villogás elkerülése érdekében a képfrekvenciát az UHD esetében emelni kellett.
Ennek megfelelően a szabvány a 120, 119.88, 100, 60, 59.94, 50, 30, 29.97, 25, 24, 23.976 Hz képfrekvenciákat támogatja.
\begin{figure}[]
	\centering
	\begin{overpic}[width = 0.75 \columnwidth ]{Figures/uhd_gamut.png}
	\small
	\end{overpic}
	\caption{Az ITU-2020 szabvány gamutja, összehasonlítva a HD színtérrel}
	\label{Fig:UHD_gamut}
\end{figure}

\item \textbf{Színtér:} Az ITU-2020 szabvány az NTSC óta először új alapszíneket vezetett be a nagyobb ábrázolható színtartomány érdekében (és a világosság dinamikatartományának növelése érdekében, ld. köv. pont).
A szabvány színterének gamutja a \ref{Fig:UHD_gamut} ábrán látható.
Megfigyelhető, hogy a választott alapszínek spektrálszínek (azaz a színpatkó határán helyezkednek el), az RGB hullámhosszok rendre 630 nm, 532 nm és 467 nm
\footnote{Természetesen ez nem azt jelenti, hogy az UHD kijelzők RGB fényforrásai spektrálszínek lennének, a szabvány csak a tároláshoz és továbbításhoz használt színteret kodifikálja. 
A megjelenítés már az egyes kijelzők saját színterében történik, amit a ténylegesen alkalmazott alapszínek korlátoznak.
Szakmai körökön belül napjainkban is fontos hírnek számít, ha egy kijelző az ITU-2020 szabványos színterét közel egészében képes megjeleníteni.
Így pl. 2018-ban a JDI cég \href{https://www.displaydaily.com/?view=article&id=62235:jdi-may-have-commercial-problems-but-has-technical-highlights}{bemutatott} egy 17.3"-es kijelzőt, amely lézeres háttérvilágítással az ITU-2020-as szabvány gamutjának 97\%-át képes megjeleníteni.}.
Ezzel a CIE színpatkó 75.8\%-át lefedi (összehasonlításképp, a HD esetében ez 35.9\%), és a Pointer féle valós felületi színek egészét\footnote{
A Pointer féle valós felületi színek azon színek halmaza, amelyek a természetben előfordulnak, mint valamely felületről visszavert fény által keltett színérzet (ellentétben a természetben elő nem forduló színekkel, pl. neon, vagy monitor által kikevert színek).
A Pointer gamutot egy 4089 mintából álló mérési adatbázisból állították össze, és publikálták az eredményeket 1980-ban, azóta a Pointer gamut-lefedés az egyes színterek minősítésének de facto szabványa.
A Pointer féle gamut \href{https://cinepedia.com/picture/color-gamut/}{itt} található, illetve \href{https://www.tftcentral.co.uk/articles/pointers_gamut.htm}{itt} található egy összehasonlítás a gyakran alkalmazott színterek gamutjával.
}.
Az új alapszínekből a világosság a következőképp számítható:
\begin{equation}
Y = 0.2627 \, R + 0.6780 \, G + 0.0593 \, B.
\label{eq:2020_Y}
\end{equation}
\item \textbf{Mintánkénti bitszám:} A tárolt színek tartományának növelése természetesen magával vonja a világosságjel dinamikatartományának növekedését is (nyilván, hiszen adott szín világosságtartalma \ref{eq:2020_Y} alapján az RGB értékekből kiszámítható).
UHD esetében  tehát már nem csak az SD esetében ökölszabályként tárgyalt 100:1 dinamikatartomány ábrázolása volt a cél.
Ennek oka, hogy a 100:1 dinamikatartomány a látás fő látóterében, rögzített tekintet mellett értelmezendő.
Amennyiben egy jeleneten belül a nézőnek lehetősége van körülnézni, úgy a pupilla tágulása és összeszűkülése ezt a dinamikatartományt kb. 5-szörösére emeli.
Mivel rendeltetésszerűen egy UHD kijelző akkora részét tölti ki a látómezőnek, hogy ez a lokális adaptációs mechanizmus végbe tud menni, így a megfelelő reprodukált dinamikatartomány biztosításához a reprezentálandó dinamikatartományt is növelni kellett.
Részben ezt valósítja meg a szabvány növelt gamutja, illetve erre szolgál a jelenleg elterjedőben levő HDR kiegészítés is.
Az ábrázolt dinamikatartomány növelése természetesen magával vonja az ábrázolásra használt bitek számának növelését is, hogy az egyes szintekhez tartozó világosságértékek közti különbség továbbra is az emberi látás modulációs küszöbje alatt maradjon.
Kísérletek kimutatták, hogy a  ITU-2020 színterének perceptuális kvantálásához 11 bit elegendő, így a szabvány a HD szabvánnyal ellentétben már kizárólag 10 és 12 bit reprezentációt ír elő.
\item \textbf{Gamma függvény:} Az ITU-2020 szabvány Gamma-korrekciós függvénye az SD és HD szabványokkal megegyezik:
\begin{equation}
E = 
\begin{cases}
4.500 L, \hspace{20mm} \mathrm{ha}\, L < \beta \\
\alpha L^{0.45} - (\alpha - 1 ), \hspace{3mm} \mathrm{ha}\, L \geq \beta,
\end{cases}
\end{equation}
ahol $\alpha = 1.09929682680944$ és $\beta = 0.018053968510807$.
Különbségként látható: a szabvány a függvény paramétereit nagyobb pontossággal definiálja, és 12 bites ábrázolás esetén a paraméterek 5 tizedesjegy pontossággal számolandók.
\item \textbf{Mintavételi struktúra:} A szabvány 4:2:0, 4:2:2 és 4:4:4 chromamintavételi-struktúrákat engedélyez, utóbbi esetben természetesen luma/chroma jelek helyett közvetlenül RGB jelek kerülnek tárolásra és átvitelre.
4:2:0 és 4:2:2 esetében az $YC_bC_r$ ábrázolás mellett lehetőség van ún. konstans fénysűrűségű luma-chroma reprezentációra is.
\end{itemize}

%TODO \paragraph{Konstans fénysűrűségű mód:\\}
\begin{figure}[]
	\centering
	\begin{overpic}[width = 0.75 \columnwidth ]{Figures/optimal-viewing-distance-television-graph-size.png}
	\small
	\end{overpic}
	\caption{Optimális nézőtávolság a kijelzőátmérő függvényében}
	\label{Fig:optimal_vd_2}
\end{figure}

A fejezet zárógondolataként térjünk vissza a különböző sorszámú kijelzők ideális nézőtávolságához, az ún. Lechner távolsághoz.
Az optimális nézőtávolságra vonatkozóan számos ajánlás létezik, így léteznek gyártói, forgalmazói, illetve THX ajánlások.
\begin{itemize}
\item SMPTE 30: a már tárgyalt, Lechner távolságon alapuló ajánlás, amely HD felbontás esetén 1.6 x képátló nézőtávolságot ír elő, így a horizontális látószög kb. 30 fok.
Ez a házimozi közösségen belül legelfogadottabb, általános nézőtávolság
\item Gyártói és forgalmazói ajánlás: HD felbontás esetén 2.5 x képátló nézőtávolság, így a horizontális látószög kb. 20 fok.
\item THX ajánlás: HD felbontás esetén 1.2 x képátló nézőtávolság, így a horizontális látószög kb. 40 fok, amely a THX szerint a legjobban közelíti a mozikban biztosított élményt.
\end{itemize}
Bár a fenti ajánlások jelentősen eltérnek egymástól, abban azonban egyetértenek, hogy "minél közelebb, annál jobb".

Az optimális nézőtávolság ábrázolható a képernyőátmérő függvényében különböző videóformátumok mellett.
Az így kapott grafikont \ref{Fig:optimal_vd_2} mutatja.
Ekkor egy adott képernyő méret mellett a nézőtávolságot növelve természetesen a pixelstruktúra nem válik láthatóvá, így pl. az $1080p$ vonala fölött a HD kijelzők használhatók.
A gondolatmenet alapján a grafikon területekre oszthatók, amelyek megadják, egy adott kijelzőméret és nézőtávolság mellett milyen felbontású kijelző az optimális választás.

Statisztikák szerint (egy szintén Bernard J. Lechner nevéhez köthető felmérés szerint) a háztartásokban az átlagos nézőtávolság kb. 2.7 méterre adódott.
Ekkora nézőtávolság mellett 50" képátló felett már 1080p felbontású TV-t érdemes venni, míg az UHD tartalom adta előnyök kiélvezéséhez legalább 75"-es ($\sim 1.9~\mathrm{m}$) kijelzőre lenne szükség.
Jelenleg az ekkora kijelzők természetesen még nem elterjedtek, így ez az adat leginkább azt mutatja, hogy jelenleg a 4k tartalmak előnyeit sem használják ki teljes egészében a kijelzők, és a vásárlók nagyrésze.
Ennek ellenére mára sorra jelennek meg a konzumer felhasználásra szánt 8k felbontású kijelzők is, és a 8k műsorszórás is kísérleti jelleggel megkezdődött.
Többek között 2017-ben lőtték fel az első dedikáltan 8k tartalom közvetítésére szánt műholdat (BSAT-4a), amely tervek szerint a 2020-as Japánban tartandó nyári olimpiát hivatott 8k felbontással az ITU-2020 szabvány szerint közvetíteni (amelyet végül a koronavírus járvány miatt 2021 nyarára ütemeztek át).


%Periférikus látás:
%https://www.quora.com/What-is-the-aspect-ratio-of-human-vision

%TODO \subsection{A HDR kiterjesztés}

\vspace{2cm}
\noindent\rule{12cm}{0.4pt}

\subsection*{Ellenőrző kérdések}

\begin{itemize}
\item Mi volt az oka a váltottsoros formátum bevezetésének?
Mi volt a váltottsoros megoldás lényege?
\item Mik voltak az SD formátum mintavételi frekvenciájának szempontjai?
Hogy következett ebből a HD formátum mintavételi frekvenciája?
\item Határozza meg egy 65" átmérőjű 2160p kijelző (16:9 képarányú) ideális nézőtávolságát!
\item Sorolja fel az UHD szabvány néhány újdonságát az SD-hez és HD-hez képest!
\item Határozza meg $2160p60$ formátum esetén az aktív és teljes pixelszámot! 
Az inaktív sorok száma a szabvány szerint 90 sor.
A mintavételi frekvencia $297~\mathrm{MHz}$.
\item Határozza meg az előző feladat formátumára a teljes adatsebességet 4:2:2 mintavételi struktúra esetén, 12 bit/minta ábrázolás mellett!
Hányas HDMI verzió képes a videóadat továbbítására, ha a HDMI interface szabványosan 8 bitnyi adatot 10 biten ábrázol és visz át, és a különböző verziók sebességei a következők:
\begin{itemize}
\item HDMI 1.0-1.2: $4.95~\mathrm{Gbit/s}$
\item HDMI 1.3-1.4: $10.2~\mathrm{Gbit/s}$
\item HDMI 2.0-1.2: $18~\mathrm{Gbit/s}$
\item HDMI 2.1: $48~\mathrm{Gbit/s}$
\end{itemize}
\end{itemize}